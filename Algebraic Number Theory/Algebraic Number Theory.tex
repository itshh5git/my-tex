\documentclass[12pt, oneside]{book}

\usepackage{../mypackages}





\begin{document}
\frontmatter
\title{{\Huge{\textbf{Algebraic Number Theory}}}}
\maketitle

\dominitoc % 初始化minitoc
\pagenumbering{Roman}
\tableofcontents % 主目录


\mainmatter
\pagenumbering{arabic} % 正文编页码字体 



\chapter{Ring Extensions} % Extensions
\section{Integral Extensions} % Integral Extensions
\begin{definition}
    Let $S$ be a commutative ring with identity and $R$ a subring of $S$ containing $1_S$. Then $S$ is said to be an \textbf{extension ring} of $R$.
    \begin{enumerate}
        \item
              An element $s\in S$ is said to be \textbf{integral} over $R$ if
              $s$ is a root of  a monic polynomial in $R[x]$.
        \item
              If every element of $S$ is integral over $R$, $S$ is said to be an \textbf{integral extension} of $R$.
        \item
              The \textbf{integral closure} of $R$ in $S$ is the set of elements of $S$ that are integral over $R$.
        \item
              The ring $R$ is said to be \textbf{integrally closed} in $S$ if $R$ is equal to its integral closure in $S$.

              The integral closure of an integral domain $R$ in its field of fractions is called the \textbf{normalization} of $R$. An integral domain is called integrally closed or normal if it is integrally closed in its field of fractions.
    \end{enumerate}
    \begin{remark}
        If follows from \cref{cor: integral extensions} that the integral closure of $R$ in $S$ is a subring of $S$ containing $R$.
    \end{remark}
\end{definition}



\begin{theorem}
    \label{thm: integral}
    Let $S$ be an extension ring of $R$ and $s \in S$. Then the following conditions are equivalent.
    \begin{enumerate}
        \item
              $s$ is integral over $R$
        \item
              Subring $R[s]$ is a finitely generated $R$-module
        \item
              There is a subring $T$ that $R[s]\subset T\subset S$, which is finitely generated as an $R$-module;
        \item
              There is a faithful $R[s]$-submodule $M$ which is finitely generated as an $R$-module.
    \end{enumerate}
\end{theorem}


\begin{corollary}
    \label{cor: integral extensions}
    Let $S$ be an extension ring of $R$. Then
    \begin{enumerate}
        \item
              If $S$ is finitely generated as an $R$-module, then $S$ is an integral extension of $R$.
        \item
              If $s_1, \ldots, s_{t} \in S$ are integral over $R$, then $R\left[s_1, \ldots, s_{t}\right]$ is a finitely generated $R$-module and an integral extension ring of $R$.
        \item
              If $T$ is an integral extension ring of $S$ and $S$ is an integral extension ring of $R$, then $T$ is an integral extension ring of $R$ .
    \end{enumerate}
\end{corollary}




\begin{proposition}
    \begin{enumerate}
        \item
              Every unique factorization domain is integrally closed.
        \item
              In particular, the polynomial ring $F\left[x_1, \ldots, x_n\right]$ ( $F$ a field) is integrally closed in its quotient field $F\left(x_1, \ldots, x_n\right)$.
    \end{enumerate}
\end{proposition}


\begin{theorem}
    Let $S$ be a multiplicative subset of an integral domain $R$ such that $0 \notin S$.
    If $R$ is integrally closed, then $S^{-1} R$ is an integrally closed integral domain.
    \begin{proof}
        $S^{-1} R$ is an integral domain and $R$ may be identified with a subring of $S^{-1} R$ by \ref{thm: ring of quotients }.
        Extending this identification, the quotient field $Q(R)$ of $R$ may be considered as a subfield of the quotient field $Q\left(S^{-1} R\right)$ of $S^{-1} R$. Verify that $Q(R)=Q\left(S^{-1} R\right)$.

        Let $u \in Q\left(S^{-1} R\right)$ be integral over $S^{-1} R$; then for some $r_i \in R$ and $s_i \in S$,
        \begin{equation*}
            u^n+\left(r_{n-1} / s_{n-1}\right) u^{n-1}+\cdots+\left(r_1 / s_1\right) u+\left(r_0 / s_0\right)=0 .
        \end{equation*}
        Multiply through this equation by $s^n$, where $s=s_0 s_1 \cdots s_{n-1} \in S$, and conclude that $su$ is integral over $R$. Since $s u \in Q\left(S^{-1} R\right)=Q(R)$ and $R$ is integrally closed, $s u \in R$. Therefore, $u=s u / s \in S^{-1} R$, whence $S^{-1} R$ is integrally closed.
    \end{proof}
\end{theorem}



\begin{theorem}
    Let $S$ be an integral extension ring of $R$. Then the following statements hold.
    \begin{enumerate}
        \item
              Assume that $S$ is an integral domain. Then $R$ is a field if and only if $S$ is a field.
        \item
              Let $\mathfrak{p}$ be a prime ideal in $R$. Then there is a prime ideal $\mathfrak{q}$ in $S$ with $\mathfrak{p}=\mathfrak{q} \cap R$.

              Moreover, $\mathfrak{p}$ is maximal if and only if $\mathfrak{q}$ is maximal.
        \item
              (The Going-up Theorem)
              Let $\mathfrak{p}_1 \subseteq \mathfrak{p}_2 \subseteq \cdots \subseteq \mathfrak{p}_n$ be a chain of prime ideals in $R$ and suppose there are prime ideals $\mathfrak{q}_1 \subseteq \mathfrak{q}_2 \subseteq \cdots \subseteq \mathfrak{q}_m$ of $S$ with $\mathfrak{p}_i=\mathfrak{q}_i \cap R, 1 \leq i \leq m$ and $m<n$. Then the ascending chain of ideals can be completed: there are prime ideals $\mathfrak{q}_{m+1} \subseteq \cdots \subseteq \mathfrak{q}_n$ in $S$ such that $\mathfrak{p}_i=\mathfrak{q}_i \cap R$ for all $i$.
    \end{enumerate}
\end{theorem}


\begin{theorem} [The Going-down Theorem]
    Assume that $S$ is an integral domain and $R$ is integrally closed in $S$. Let $\mathfrak{p}_1 \supseteq \mathfrak{p}_2 \supseteq \cdots \supseteq \mathfrak{p}_n$ be a chain of prime ideals in $R$ and suppose there are prime ideals $\mathfrak{q}_1 \supseteq \mathfrak{q}_2 \supseteq \cdots \supseteq \mathfrak{q}_m$ of $S$ with $\mathfrak{p}_i=\mathfrak{q}_i \cap R, 1 \leq i \leq m$ and $m<n$. Then the descending chain of ideals can be completed: there are prime ideals $\mathfrak{q}_{m+1} \supseteq \cdots \supseteq \mathfrak{q}_n$ in $S$ such that $\mathfrak{p}_i=\mathfrak{q}_i \cap R$ for all $i$.
\end{theorem}















\begin{theorem}
    \label{thm: Let S be an integral extension ring of R and let q be a prime ideal in S which
        lies over a prime ideal p in R . Then q is maximal in S if and only if p is maximal in R .}
    Let $S$ be an integral extension ring of $R$ and let $\mathfrak{q}$ be a prime ideal in $S$ which lies over a prime ideal $\mathfrak{p}$ in $R$.
    Then $\mathfrak{q}$ is maximal in $S$ if and only if $\mathfrak{p}$ is maximal in $R$.

    \begin{proof}
        Suppose $\mathfrak{q}$ is maximal in $S$, there is a maximal ideal $\mathfrak{m}$ of $R$ that contains $\mathfrak{p}$ and $\mathfrak{m}$ is prime by \ref{thm: maximal ideal is prime}.
        By \ref{cor: going-up Theorem} there is a prime ideal $\mathfrak{q}^{\prime}$ in $S$ such that $\mathfrak{q} \subset \mathfrak{q}^{\prime}$ and $\mathfrak{q}^{\prime}$ lies over $\mathfrak{m}$. Since $\mathfrak{q}^{\prime}$ is prime, $\mathfrak{q}^{\prime} \neq S$.
        The maximality of $\mathfrak{q}$ implies that $\mathfrak{q}=\mathfrak{q}^{\prime}$, whence $\mathfrak{p}=\mathfrak{q} \cap R=\mathfrak{q}^{\prime} \cap R=\mathfrak{m}$. Therefore, $\mathfrak{p}$ is maximal in $R$.

        Conversely suppose $\mathfrak{p}$ is maximal in $R$. Since $\mathfrak{q}$ is prime in $S, \mathfrak{q} \neq S$ and there is a maximal ideal $N$ of $S$ containing $\mathfrak{q}$ and $N$ is prime, whence $1_R=1_S \notin N$. Since $\mathfrak{p}=R \cap \mathfrak{q} \subset R \cap N \subset R$, we must have $\mathfrak{p}=R \cap N$ by maximality. Thus $\mathfrak{q}$ and $N$ both lie over $\mathfrak{p}$ and $\mathfrak{q} \subset N$. Therefore, $\mathfrak{q}=N$ by \ref{thm: Let S be an integral extension ring of R and let q be a prime ideal in S which lies over a prime ideal p in R . Then q is maximal in S if and only if p is maximal in R .} .
    \end{proof}
\end{theorem}



\section{Discrete Valuation and Discrete Valuation Ring}

\begin{definition}
    Let $K$ be a field.
    A \textbf{discrete valuation} on $K$ is a nonzero group homomorphism $v: K^{\times} \rightarrow \mathbb{Z}$ such that $v(a+b) \geq \min (v(a), v(b))$.

    As $v$ is not the zero homomorphism, its image is a nonzero subgroup of $\mathbb{Z}$, and is therefore of the form $m \mathbb{Z}$ for some $m \in \mathbb{Z}$. If $m=1$, then $v: K^{\times} \rightarrow \mathbb{Z}$ is surjective, and $v$ is said to be \textbf{normalized}; otherwise, $x \mapsto m^{-1} \cdot v(x)$ will be a normalized discrete valuation.

    We extend $v$ to a map $K \rightarrow \mathbb{Z} \cup\{\infty\}$ by setting $v(0)=+\infty$, where $\infty$ is a symbol $\geq n$ for all $n \in \mathbb{Z}$.
    \begin{remark}
        We have
        \begin{enumerate}
            \item $v(\zeta)=0$ for some $\zeta \in K^{\times}$
            \item $v(-a)=v(a)$ for all $a \in K$;
            \item $v(a+b) = \max \left\{v(a),v(b)\right\}$ if $v(a) \neq v(b)$.
        \end{enumerate}
        We often use "ord" rather than " $v$ " to denote a discrete valuation.
    \end{remark}
\end{definition}

\begin{definition}
    The following conditions on a principal ideal domain are equivalent:
    \begin{enumerate}
        \item  $A$ has exactly one nonzero prime ideal;
        \item up to associates, $A$ has exactly one prime element;
        \item  $A$ is local and is not a field.
    \end{enumerate}
    A ring satisfying these conditions is called a \textbf{discrete valuation ring}.
\end{definition}

\begin{theorem}
    \label{thm:discrete valuation ring}
    Let $A$ be a domain ring. The following conditions are equivalent:
    \begin{enumerate}
        \item
              $A$ is a discrete valuation ring
        \item
              There is a discrete valuation $v$ on $K=\Frac(A)$ such that
              \begin{equation*}
                  A
                  =
                  \mathcal{O}_v
                  :=
                  \{a \in K \mid v(a) \geq 0\}
              \end{equation*}
              with unique maximal ideal $\mathfrak{m} = \{a \in K \mid v(a) > 0\}$.
        \item
              there exists a element $\pi \in A$ such that every nonzero ideal of $A$ is of the form $\left(\pi^n\right)$ for some $n \geq 0$.
        \item
              $A$ is a noetherian, integrally closed and has exactly one nonzero prime ideal.
    \end{enumerate}
\end{theorem}
We can associate discrete valuations to prime ideals in Dedekind domains.
\begin{definition}
    Let $A$ be a Dedekind domain and let $\mathfrak{p}$ be a prime ideal in $A$. For any $c \in K^{\times}$, let $v(c)$ be the exponent of $\mathfrak{p}$ in the factorization of
    $(c)$.
    Then $v$ is a normalized discrete valuation on $K$, called the \textbf{discrete valuation associated to} $\mathfrak{p}$, denoted by $\operatorname{ord}_{\mathfrak{p}}$.
\end{definition}

\begin{proposition}
    Let $x_1, \ldots, x_m$ be elements of a Dedekind domain $A$, and let $\mathfrak{p}_1, \ldots, \mathfrak{p}_m$ be distinct prime ideals of $A$. For every integer $n$, there is an $x \in A$ such that
    \begin{equation*}
        \operatorname{ord}_{\mathfrak{p}_i}\left(x-x_i\right)>n, \quad i=1,2, \ldots, m .
    \end{equation*}
\end{proposition}





\section{Dedekind Domain}
\begin{definition}
    A \textbf{Dedekind domain} is an integral domain $A$ satisfying
    \begin{enumerate}[label=(\roman*)]
        \item
              $A$ is noetherian;
        \item
              $A$ is integrally closed;
        \item
              $A$ has Krull dimension one, i.e., every nonzero prime ideal is maximal.
    \end{enumerate}

\end{definition}


\begin{proposition}
    Let $A$ be an integral domain, and let $S$ be a multiplicative subset of $A$.
    \begin{enumerate}
        \item
              If $A$ is noetherian, then so also is $S^{-1} A$.
        \item
              If $A$ is integrally closed, then so also is $S^{-1} A$.
        \item
              If $A$ has Krull dimension one, then so also does $A_\mathfrak{p}$ for any prime ideal $\mathfrak{p}$.
    \end{enumerate}
    \begin{remark}
        It follows that the localization $A_{\mathfrak{p}}$ of a Dedekind domain $A$ is local thus DVR.
    \end{remark}
\end{proposition}




\subsection{Fractional Ideals}
\begin{definition}
    Let $A$ be an integral domain with quotient field $K=\Frac(A)$.
    \begin{enumerate}
        \item
              A \textbf{fractional ideal} of $A$ is
              \begin{enumerate}[label=(\roman*)]
                  \item a nonzero $A$-submodule $I$ of $K$
                  \item there exists a nonzero $d \in A$ such that $d I \subset A$ i,e, $\left(A:I\right)\cap A \neq \varnothing$
              \end{enumerate}
        \item
              A fractional ideal $I$ of $A$ is said to be \textbf{integral} if $I \subset A$.
        \item
              A fractional ideal $I$ of $A$ is said to be \textbf{principal} if $I=Ax$ for some nonzero $x \in K$.
        \item
              the \textbf{ideal quotient} of two fractional ideals $I$ and $J$ is defined as
              \begin{equation*}
                  (I: J)
                  :=
                  \{x \in K \mid x J \subset I\} .
              \end{equation*}
        \item
              the \textbf{inverse} of a fractional ideal $I$ is defined as
              \begin{equation*}
                  I^{-1}
                  :=
                  (A: I) .
              \end{equation*}
              thus $II^{-1} \subset A$.
        \item
              A fractional ideal $I$ is called \textbf{invertible} if there is a fractional ideal $J$ such that $I J=A$.
    \end{enumerate}
    \begin{remark}
        Let $I$ be a fractional ideal of $A$, $\mathfrak{p}$ a prime ideal of $A$ and $S=A-\mathfrak{p}$.
        Then the localization $I_{\mathfrak{p}}:=IA_{\mathfrak{p}}=S^{-1} I=\left\{x/s:x\in I,s\in S\right\}$ is a fractional ideal of $A_{\mathfrak{p}}$.

        We may assume that all rings and ideals are contained in $K=\operatorname{Frac}(A)$.
    \end{remark}
\end{definition}

\begin{lemma}
    Let $A$ be a integral domain and fractional ideal $I,J$, then
    \begin{itemize}
        \item $I+J$
        \item $IJ$
        \item $I\cap J$
        \item $\left(I:J\right)$
    \end{itemize}
    are both ideal fractional ideal. And
    \begin{enumerate}
        \item $IJ\subset I\cap J$
        \item $H+(I+J)=I+(H+J):=H+I+J$
        \item $IJ=JI$
        \item $H(IJ)=(HI)J:=HIJ$
        \item $H(I+J)=HI+HJ$
    \end{enumerate}
\end{lemma}


\begin{proposition}
    Let $A$ be an integral domain, $K=\operatorname{Frac}(A)$ and $I$ a fractional ideal. Then the following statements hold:
    \begin{enumerate}
        \item
              $I I^{-1} \subseteq A$.
        \item
              $I$ is invertible $\Leftrightarrow I I^{-1}=A$.
        \item
              Let $J$ be an invertible ideal. Then $(I: J)=I J^{-1}$.
        \item  If $0 \neq i \in I$ such that $i^{-1} \in I^{-1}$, then $I=(i)$.
    \end{enumerate}
\end{proposition}
\begin{corollary}
    Let $A$ be an integral domain. The set $\mathcal{I}(A)$ of invertible fractional ideals forms an abelian group with respect to multiplication, with $A$ being the identity element, and the inverse of $I \in \mathcal{I}(A)$ being $I^{-1}$.
\end{corollary}

\begin{definition}
    Let $A$ be an integral domain.
    One calls $\mathcal{I}(A)$ the group of invertible fractional ideal and $\mathcal{P}(R)$ the subgroup of principal invertible fractional ideal.
    The quotient group $\operatorname{Pic}(R):=\mathcal{I}(R) / \mathcal{P}(R)$ is called the \textbf{Picard group} of $A$.

    If $K$ is a number field and $\mathbb{Z}_K$ its ring of
    integers, one also writes $\operatorname{CL}(K):=\operatorname{Pic}\left(\mathbb{Z}_K\right)$, and calls it the \textbf{ideal class group} of $K$.
    \begin{remark}
        Then we have the exact sequence of abelian groups
        \begin{equation*}
            1 \rightarrow A^{\times} \rightarrow K^{\times} \xrightarrow{\mathrm{prin}} \mathcal{I}(A) \xrightarrow{\mathrm{proj}} \operatorname{Pic}(A) \rightarrow 1,
        \end{equation*}
        where $f(x)$ is the principal fractional $R$-ideal $x R$.
    \end{remark}
\end{definition}

Invertibility is a local property:
\begin{proposition}
    For a fractional ideal $I$ in integral domain $A$, the following are equivalent:
    \begin{enumerate}
        \item
              $I$ is invertible;
        \item
              $I$ is finitely generated and, for each prime ideal $\mathfrak{p}, I_{\mathfrak{p}}$ is invertible:
        \item
              $I$ is finitely generated and, for each maximal ideal $\mathfrak{m}, I_{\mathfrak{m}}$ is invertible.
    \end{enumerate}
\end{proposition}




\subsection{Unique factorization of fractional ideals} % Unique factorization of fractional ideal
\begin{theorem}
    Let $A$ be a Dedekind domain.
    Every fractional ideal $I$ of $A$ can be written uniquely in the form
    \begin{equation*}
        I=\prod_{\mathfrak{p} \text{ prime }} \mathfrak{p}^{v_\mathfrak{p}(I)}
    \end{equation*}
    where discrete valuation $v_\mathfrak{p}(I)$.

    The set $\mathcal{I}(A)$ of fractional ideals is a group; in fact, it is the free abelian group on the set of nonzero prime ideals.
    \begin{proof}
        In order to show that $\mathcal{I}(A)$ is a group, it remains to show that inverses exist.
        Let $\mathfrak{a}$ be a nonzero integral ideal, there is an ideal $\mathfrak{a}^*$ and an $a \in A$ such that $\mathfrak{a} \mathfrak{a}^*=(a)$. Clearly $\mathfrak{a} \cdot\left(a^{-1} \mathfrak{a}^*\right)=A$, and so $a^{-1} \mathfrak{a}^*$ is an inverse of $\mathfrak{a}$. If $\mathfrak{a}$ is a fractional ideal, then $d \mathfrak{a}$ is an integral ideal for some $d$, and $d \cdot(d \mathfrak{a})^{-1}$ will be an inverse for $\mathfrak{a}$.

        It remains to show that the group $\operatorname{Id}(A)$ is freely generated by the prime ideals, i.e., that each fractional ideal can be expressed in a unique way as a product of powers of prime ideals. Let $\mathfrak{a}$ be a fractional ideal. Then $d \mathfrak{a}$ is an integral ideal for some $d \in A$, and we can write
        \begin{equation*}
            d \mathfrak{a}=\mathfrak{p}_1^{r_1} \cdots \mathfrak{p}_m^{r_m}, \quad(d)=\mathfrak{p}_1^{s_1} \cdots \mathfrak{p}_m^{s_m} .
        \end{equation*}
        Thus $\mathfrak{a}=\mathfrak{p}_1^{r_1-s_1} \cdots \mathfrak{p}_m^{r_m-s_m}$. The uniqueness follows from the uniqueness of the factorization for integral ideals.
    \end{proof}
\end{theorem}


\subsection{Proof of factorization} % Proof of factorization
We prove the theorem in several steps. Assume that $A$ is a commutative ring throughout.
\begin{lemma}
    \label{lem:1}
    Let $A$ be a noetherian ring; then every ideal $\mathfrak{a}$ in $A$ contains a product of nonzero prime ideals.
    \begin{proof}
        Suppose that the statement is false for $A$, and choose a maximal counterexample $\mathfrak{a}$ by Noetherian property.
        Then $\mathfrak{a}$ itself cannot be prime, and so there exist elements $x$ and $y$ of $A$ such that $x y \in \mathfrak{a}$ but neither $x$ nor $y \in \mathfrak{a}$.

        The ideals $\mathfrak{a}+(x)$ and $\mathfrak{a}+(y)$ strictly contain $\mathfrak{a}$ and contain a product of prime ideals respectively, but their product is contained in $\mathfrak{a}$.
        It follows that $\mathfrak{a}$ contains a product of prime ideals.
    \end{proof}
\end{lemma}

\begin{lemma}
    \label{lem:2}
    Let $A$ be a ring, and let $\mathfrak{a}$ and $\mathfrak{b}$ be relatively prime ideals in $A$;
    \begin{enumerate}
        \item
              for all $m, n \in \mathbb{N}$, $\mathfrak{a}^m$ and $\mathfrak{b}^n$ are relatively prime.
        \item $I\cap J=IJ$
        \item $A/(IJ)\cong A/(I)\times A/(J)$
        \item if $IJ=H^n$ for some ideal $H$ and some $n \in \mathbb{N}$, then there exist ideals $I_1:=I+H$ and $J_1:=J+H$ such that $I=I_1^n$, $J=J_1^n$ and $I_1J_1=H$
    \end{enumerate}
    \begin{proof}
        If $\mathfrak{a}^m$ and $\mathfrak{b}^n$ are not relatively prime, then they are both contained in some prime (even maximal) ideal $\mathfrak{p}$.
        Thus $\mathfrak{a}$ and $\mathfrak{b}$ are both contained in $\mathfrak{p}$, which contradicts the hypothesis.
    \end{proof}
\end{lemma}

\begin{lemma}
    Let $\mathfrak{p}$ be a maximal ideal of an integral domain $A$, and let $\mathfrak{q}=\mathfrak{p}^e=\mathfrak{p} A_{\mathfrak{p}}$ be the ideal in $A_{\mathfrak{p}}$. The map
    \begin{equation*}
        a+\mathfrak{p}^m \mapsto a+\mathfrak{q}^m: A / \mathfrak{p}^m \rightarrow A_{\mathfrak{p}} / \mathfrak{q}^m
    \end{equation*}
    is an isomorphism for all $m \in \mathbb{N}$.
    \begin{proof}
        Let $S$ be the $A-\mathfrak{p}$. The map is clearly a homomorphism of rings, so we have to prove that it is bijective.


        We first show that the map is injective. For this we have to show that $\mathfrak{q}^m \cap A=\mathfrak{p}^m$.
        But $\mathfrak{q}^m=S^{-1} \mathfrak{p}^m$, and so we have to show that $\mathfrak{p}^m=\left(S^{-1} \mathfrak{p}^m\right) \cap A$. An element of $\left(S^{-1} \mathfrak{p}^m\right) \cap A$ can be written $a=b / s$ with $b \in \mathfrak{p}^m, s \in S$, and $a \in A$. Then $s a \in \mathfrak{p}^m$, and so $s a=0$ in $A / \mathfrak{p}^m$. The only maximal ideal containing $\mathfrak{p}^m$ is $\mathfrak{p}$ (because $\mathfrak{m} \supset \mathfrak{p}^m \Rightarrow \mathfrak{m} \supset \mathfrak{p}$ ),
        and so the only maximal ideal in $A / \mathfrak{p}^m$ is $\mathfrak{p} / \mathfrak{p}^m$; in particular, $A / \mathfrak{p}^m$ is a local ring. As $s+\mathfrak{p}^m$ is not in $\mathfrak{p} / \mathfrak{p}^m$, it is a unit in $A / \mathfrak{p}^m$, and so $s a=0$ in $A / \mathfrak{p}^m \Rightarrow a=0$ in $A / \mathfrak{p}^m$, i.e., $a \in \mathfrak{p}^m$.

        We now prove that the map is surjective. Let $\frac{a}{s} \in A_{\mathfrak{p}}$. Because $s \notin \mathfrak{p}$ and $\mathfrak{p}$ is maximal, we have that $(s)+\mathfrak{p}=A$, i.e., $\left(s\right)$ and $\mathfrak{p}$ are relatively prime.
        Therefore $\left(s\right)$ and $\mathfrak{p}^m$ are relatively prime by \cref{lem:2}, and so there exist $b \in A$ and $q \in \mathfrak{p}^m$ such that $b s+q=1$. Then $b$ maps to $s^{-1}$ in $A_{\mathfrak{p}} / \mathfrak{q}^m$ and so $b a$ maps to $\frac{a}{s}$. Thus the map is surjective.
    \end{proof}
\end{lemma}





\begin{proof}
    [Proof of factorization]
    We now prove that a nonzero ideal $\mathfrak{a}$ of Dedekind domain $A$ can be factored into a product of prime ideals. According to \ref{lem:1} applied to $A$, the ideal $\mathfrak{a}$ contains a product of nonzero prime ideals,

    \begin{equation*}
        \mathfrak{b}=\mathfrak{p}_1^{r_1} \cdots \mathfrak{p}_m^{r_m}
    \end{equation*}
    We may suppose that the $\mathfrak{p}_i$ are distinct. Then
    \begin{equation*}
        A / \mathfrak{b} \simeq A / \mathfrak{p}_1^{r_1} \times \cdots \times A / \mathfrak{p}_m^{r_m} \simeq A_{\mathfrak{p}_1} / \mathfrak{q}_1^{r_1} \times \cdots \times A_{\mathfrak{p}_m} / \mathfrak{q}_m^{r_m},
    \end{equation*}
    where $\mathfrak{q}_i=\mathfrak{p}_i A_{\mathfrak{p}_i}$ is the maximal ideal of $A_{\mathfrak{p}_i}$.
    Under this isomorphism,
    \begin{equation*}
        A\rightarrow A / \mathfrak{b} \simeq A_{\mathfrak{p}_1} / \mathfrak{q}_1^{r_1} \times \cdots \times A_{\mathfrak{p}_m} / \mathfrak{q}_m^{r_m}
    \end{equation*}

    $\mathfrak{a} / \mathfrak{b}$ in $A/\mathfrak{b}$ corresponds to $\mathfrak{q}_1^{s_1} / \mathfrak{q}_1^{r_1} \times \cdots \times \mathfrak{q}_m^{s_m} / \mathfrak{q}_m^{r_m}$ for some $s_i \leq r_i$ (recall that the rings $A_{\mathfrak{p}_i}$ are all discrete valuation rings). Since this ideal is also the image of $\mathfrak{p}_1^{s_1} \cdots \mathfrak{p}_m^{s_m}$ under the isomorphism, we see that

    \begin{equation*}
        \mathfrak{a}=\mathfrak{p}_1^{s_1} \cdots \mathfrak{p}_m^{s_m} \text { in } A / \mathfrak{b} .
    \end{equation*}
    Both of these ideals contain $\mathfrak{b}$, and so this implies that
    \begin{equation*}
        \mathfrak{a}=\mathfrak{p}_1^{s_1} \cdots \mathfrak{p}_m^{s_m}
    \end{equation*}
    in $A$.

    To complete the proof, we have to prove that the above factorization is unique. Suppose that we have two factorizations of the ideal $\mathfrak{a}$. After adding factors with zero exponent, we may suppose that the same primes occur in each factorization, so that
    \begin{equation*}
        \mathfrak{p}_1^{s_1} \cdots \mathfrak{p}_m^{s_m}=\mathfrak{a}=\mathfrak{p}_1^{t_1} \cdots \mathfrak{p}_m^{t_m}
    \end{equation*}
    In the course of the above proof, we showed that
    \begin{equation*}
        \mathfrak{q}_i^{s_i}=\mathfrak{a} A_{\mathfrak{p}_i}=\mathfrak{q}_i^{t_i},
    \end{equation*}
    where $\mathfrak{q}_i=\mathfrak{p}_iA_{\mathfrak{p}_i}$ the maximal ideal in $A_{\mathfrak{p}_i}$. Therefore $s_i=t_i$ for all $i$.
\end{proof}


\begin{corollary}
    Let $\mathfrak{a} \supset \mathfrak{b} \neq 0$ be two ideals in a Dedekind domain; then $\mathfrak{a}=\mathfrak{b}+(a)$ for some $a \in A$.
    \begin{proof}
        Let $\mathfrak{b}=\mathfrak{p}_1^{r_1} \cdots \mathfrak{p}_m^{r_m}$ and $\mathfrak{a}=\mathfrak{p}_1^{s_1} \cdots \mathfrak{p}_m^{s_m}$ with $r_i, s_j \geq 0$. Because $\mathfrak{b} \subset \mathfrak{a}, s_i \leq r_i$ for all $i$. For $1 \leq i \leq m$, choose an $x_i \in A$ such that $x_i \in \mathfrak{p}_i^{s_i}, x_i \notin \mathfrak{p}_i^{s_i+1}$. By the Chinese Remainder Theorem, there is an $a \in A$ such that

        \begin{equation*}
            a \equiv x_i \quad \bmod \mathfrak{p}_i^{r_i}, \text { for all } i .
        \end{equation*}
        Now one sees that $\mathfrak{b}+(a)=\mathfrak{a}$ by looking at the ideals they generate in $A_{\mathfrak{p}}$ for all $\mathfrak{p}$.
    \end{proof}
\end{corollary}

\begin{corollary}
    Let $\mathfrak{a}$ be an ideal in a Dedekind domain, and let $a$ be any nonzero element of $\mathfrak{a}$; then there exists $b \in \mathfrak{a}$ such that $\mathfrak{a}=(a, b)$.
\end{corollary}

\begin{corollary}
    Let $\mathfrak{a}$ be a nonzero ideal in a Dedekind domain; then there exists a nonzero ideal $\mathfrak{a}^*$ in $A$ such that $\mathfrak{a} \mathfrak{a}^*$ is principal. Moreover, $\mathfrak{a}^*$ can be chosen to be relatively prime to any particular ideal $\mathfrak{c}$, and it can be chosen so that $\mathfrak{a} \mathfrak{a}^*=(a)$ with $a$ any particular element of $\mathfrak{a}$ (but not both).
    \begin{proof}
        Let $a \in \mathfrak{a}, a \neq 0$; then $\mathfrak{a} \supset(a)$, and so we have
        \begin{equation*}
            (a)=\mathfrak{p}_1^{r_1} \cdots \mathfrak{p}_m^{r_m} \text { and } \mathfrak{a}=\mathfrak{p}_1^{s_1} \cdots \mathfrak{p}_m^{s_m}, \quad s_i \leq r_i .
        \end{equation*}
        If $\mathfrak{a}^*=\mathfrak{p}_1^{r_1-s_1} \cdots \mathfrak{p}_m^{r_m-s_m}$, then $\mathfrak{a} \mathfrak{a}^*=(a)$.

        We now show that $\mathfrak{a}^*$ can be chosen to be prime to $\mathfrak{c}$.
        We have $\mathfrak{a} \supset \mathfrak{a} \mathfrak{c}$, and so (by 3.15) there exists an $a \in \mathfrak{a}$ such that $\mathfrak{a}=\mathfrak{a} \mathfrak{c}+(a)$. As $\mathfrak{a} \supset(a)$, we have $(a)=\mathfrak{a} \cdot \mathfrak{a}^*$ for some ideal $\mathfrak{a}^*$ (by the above argument); now, $\mathfrak{a} \mathfrak{c}+\mathfrak{a} \mathfrak{a}^*=\mathfrak{a}$, and so $\mathfrak{c}+\mathfrak{a}^*=A$. (Otherwise $\mathfrak{c}+\mathfrak{a}^* \subset \mathfrak{p}$ some prime ideal, and $\mathfrak{a c}+\mathfrak{a} \mathfrak{a}^*=\mathfrak{a}\left(\mathfrak{c}+\mathfrak{a}^*\right) \subset \mathfrak{a} \mathfrak{p} \neq \mathfrak{a}$.)
    \end{proof}
\end{corollary}
































\chapter{}

\section{Number Fields and Rings of Integers} % Number Fields and Rings of Integers
\begin{definition}
    A \textbf{number field} $K$ is a finite extension of the field of rational numbers $\mathbb{Q}$.
    \begin{remark}
        As $\char\mathbb{Q}=0$, $K/\mathbb{Q}$ is separable, then $K=\mathbb{Q}\left(\alpha\right)$ for some primitive element $\alpha$.
    \end{remark}
\end{definition}

\begin{definition}
    Let $K$ be a number field.
    The \textbf{ring of integers} of $K$ is the integral closure of $\mathbb{Z}$ in $K$, denoted by $\mathcal{O}_K$ or $\mathbb{Z}_K$; its elements are called the \textbf{algebraic integers} in $K$.
\end{definition}

\section{}

\begin{definition}
    Let $A=\mathbb{Z}$ and $M$ be a free $A$-module of rank $n$, the \textbf{index} of $N := \mathbb{Z} f_1+\mathbb{Z} f_2+\cdots+\mathbb{Z} f_n$ in $M$ is
    \begin{equation*}
        (M: N):=\left|\operatorname{det}\left(a_{i j}\right)\right|
    \end{equation*}
    where $\left(f_k\right)=\left(a_{ij}\right)\left(e_k\right)$ for some $A$-basis $\left\{e_k\right\}$.
\end{definition}



\section{Trace and Norm} % Trace and Norm
\begin{definition}
    Let $B/A$ be a ring extension such that $B$ is a free $A$-module of rank $n$. Then every $\beta \in B$ defines an $A$-linear map
    \begin{equation*}
        T_\beta : B \rightarrow B,\quad x \mapsto \beta x
    \end{equation*}
    and the trace and determinant of this map are well-defined.
    We call them the \textbf{trace} $\operatorname{Tr}_{B / A} \beta$ and \textbf{norm} $\operatorname{N}_{B / A} \beta$ of $\beta$ in the extension $B / A$.
\end{definition}



\begin{proposition}
    Let $L / K$ be an separable extension of fields of degree $n$, $\overline{K}$ an algebraic closure of $K$ containing $L$.
    Let
    \begin{equation*}
        \left\{\sigma_1,\ldots,\sigma_n\right\}
        =
        \Hom_{K}\left( L, \overline{K}\right)
    \end{equation*}
    Then the following statements hold for any $a \in L$:
    \begin{enumerate}
        \item
              $\chi_a=m_a^e$ where $e=[L: K(a)]=n/\deg m_a$.
        \item
              $\chi_a(X)=\prod_{\sigma \in \operatorname{Hom}_K(L, \overline{K})}(X-\sigma(a))$,
        \item
              $\operatorname{Tr}_{L / K}(a)=\sum_{\sigma} \sigma(a)$, and
              $\operatorname{N}_{L / K}(a)=\prod_{\sigma} \sigma(a)$.
    \end{enumerate}
    \begin{proof}
        Let $F=K(a)$, then $\overline{K}/F$ is Galois thus separable
        \begin{equation*}
            m_a(X)
            :=
            \prod_{\overline{\sigma}_\alpha \in K^\prime/F^\prime}\left(X-\overline{\sigma}_\alpha(a)\right) .
        \end{equation*}
        Then by the preceding proposition and $e=\left[L:F\right]=\left[F^\prime:L^\prime\right]$
        \begin{equation*}
            \prod_{\overline{\sigma}_\alpha \in K^\prime/F^\prime}\left(X-\overline{\sigma}_\alpha(a)\right)^e
            =
            \prod_{\overline{\sigma}_\alpha \in K^\prime/F^\prime}
            \prod_{\overline{\sigma}_\beta \in F^\prime/L^\prime}
            \left(X-\overline{\sigma}_\alpha \circ \overline{\sigma}_\beta(a)\right)=\prod_{\sigma \in \operatorname{Hom}_K(L, \overline{K})}(X-\sigma(a)) .
        \end{equation*}
    \end{proof}
\end{proposition}

\begin{corollary}
    Let $L / F / K$  be finite separable field extensions. Then
    \begin{equation*}
        \operatorname{Tr}_{L / K}=\operatorname{Tr}_{F / K} \circ \operatorname{Tr}_{L / F} \text { and } \operatorname{Norm}_{L / K}=\operatorname{Norm}_{F / K} \circ \operatorname{Norm}_{L / F}
    \end{equation*}

\end{corollary}





\section{Discriminant} % Discriminant
\subsection{}
Let $A$ be an integral domain with fraction field $K=\operatorname{Frac}(A)$ and $L/K$ be finite separable field extension. Let $B:=A_L$ be the integral closure of $A$ in $L$.
\begin{proposition}
    \label{pro: pro of integral closure}
    Then the following statements hold:
    \begin{enumerate}
        \item
              Every $a \in L$ can be written as $a=\frac{s}{r}$ with $s \in B$ and $0 \neq r \in A$.
        \item
              $L=\operatorname{Frac}(B)$ and $B$ is integrally closed.
    \end{enumerate}
    If $A$ is integrally closed
    \begin{enumerate}[resume]
        \item
              For any $K$-basis $\alpha_1, \ldots, \alpha_n$ of $L$, there is an element $r \in A \backslash\{0\}$ such that $r \alpha_i \in B$ for all $i=1, \ldots, n$.
              Clearly, $\left\{r\alpha_i\right\}_{i=1}^n\subset B$ is also a $K$-basis of $L$.
        \item
              $B \cap K=A$.
    \end{enumerate}
\end{proposition}

\begin{lemma}
    Let $\alpha_1, \ldots, \alpha_n$ be a basis of $L/K$ which is contained in $B$, of discriminant $d=d\left(\alpha_1, \ldots, \alpha_n\right)$. Then one has
    \begin{equation*}
        d B \subseteq A \alpha_1+\cdots+A \alpha_n
    \end{equation*}
\end{lemma}

\begin{proposition}
    ,
    \begin{enumerate}
        \item
              There exists free $A$-submodules $M$ and $M^{\prime}$ of $L$ such that
              \begin{equation*}
                  M \subset B \subset M^{\prime} .
              \end{equation*}
        \item
              Therefore $B$ is a finitely generated $A$-module if $A$ is noetherian,
        \item
              If $A$ is a principal ideal domain, then every finitely generated $B$-submodule $M \neq 0$ of $L$ is a free $A$-module of rank $n$.
              In particular, $B$ admits an integral basis over $A$.
    \end{enumerate}
    \begin{remark}
        When $A$ is a principal ideal domain, a basis for $B$ as an $A$-module is called an \textbf{integral basis} of $B$ over $A$ (is also a $K$-basis of $L$).
    \end{remark}
    \begin{proof}
        Let $\left\{\alpha_1, \ldots, \alpha_n\right\}\subset B$ be a basis for $L$ over $K$.
        Because the trace pairing is nondegenerate, there is a dual basis $\left\{\alpha_1^{\prime}, \ldots, \alpha_n^{\prime}\right\}$ of $L$ over $K$ such that $\operatorname{Tr}\left(\alpha_i \cdot \alpha_j^{\prime}\right)=\delta_{i j}$.
        We shall show that
        \begin{equation*}
            A \alpha_1+A \alpha_2+\cdots+A \alpha_n \subset B \subset A \alpha_1^{\prime}+A \alpha_2^{\prime}+\cdots+A \alpha_n^{\prime} .
        \end{equation*}
        The first inclusion is clear because the $\alpha_i$ are in $B$.

        To show the second inclusion, let $b \in B$ and $b$ can be written uniquely as a linear combination $b=\sum k_j \alpha_j^{\prime}$ of the $\alpha_j^{\prime}$ with coefficients $k_j \in K$.
        As $\alpha_i$ and $b$ are in $B$, so also is $b \cdot \alpha_i$, and so $\operatorname{Tr}\left(b \cdot \alpha_i\right) \in A$.
        But
        \begin{equation*}
            \operatorname{Tr}\left(b \cdot \alpha_i\right)=\operatorname{Tr}\left(\sum_j k_j \alpha_j^{\prime} \cdot \alpha_i\right)=\sum_j k_j \operatorname{Tr}\left(\alpha_j^{\prime} \cdot \alpha_i\right)=\sum_j k_j \cdot \delta_{i j}=k_i .
        \end{equation*}
        Hence $k_i \in A\cap K= A$, proving the second inclusion.
    \end{proof}
\end{proposition}






\subsection{Discriminant} % Discriminant

\begin{definition}
    Let $B/A$ be a ring extension, and assume that $B$ is free of rank $n$ as an $A$-module.

    \begin{enumerate}
        \item
              Let $\alpha_1, \ldots, \alpha_n$ be $A$-basis of $B$.
              We define their \textbf{discriminant} to be
              \begin{equation*}
                  \disc_{B/A}\left(\alpha_1, \ldots, \alpha_n\right)
                  :=
                  \operatorname{det}\left(\operatorname{Tr}_{B / A}\left(\alpha_i \alpha_j\right)\right)_{1 \leq i, j \leq n} .
              \end{equation*}
        \item
              The \textbf{trace pairing} on $B/A$ is the bilinear pairing
              \begin{equation*}
                  B \times B \rightarrow A, \quad(x, y) \mapsto \operatorname{Tr}_{B / A}(x y)
              \end{equation*}
              with Gram matrix $\left(\operatorname{Tr}_{B / A}\left(\alpha_i \alpha_j\right)\right)_{1 \leq i, j \leq n}$ with respect to the basis $\left\{\alpha_1, \ldots, \alpha_n\right\}$ of $B$.
        \item
              If two basis $\left(\beta_1,\ldots,\beta_n\right)=\left(\alpha_1,\ldots,\alpha_n\right)\left(a_{ij}\right)_{1 \leq i,j \leq n}$ where $\left(a_{ij}\right)\in M_n(A)$, then
              \begin{equation*}
                  \operatorname{disc}\left(\beta_1, \ldots, \beta_n\right)=\operatorname{det}(a_{ij})^2 \operatorname{disc}\left(\alpha_1, \ldots, \alpha_n\right) .
              \end{equation*}
              Thus the discriminant of a basis of $B$ is well-defined up to multiplication by the square of a unit in $A$.
              The ideal generated by the discriminant, or $\disc\left(\alpha_1, \ldots, \alpha_n\right)$ itself regarded as an element of $A / A^{\times 2}$, is called the \textbf{discriminant} of $B$ over $A$, denoted $\operatorname{disc}(B / A)$.
              \begin{remark}
                  Then elements $\gamma_1, \ldots, \gamma_n$ form a basis for $B$ as an $A$-module if and only if
                  \begin{equation*}
                      \left(\disc\left(\gamma_1, \ldots, \gamma_n\right)\right)=(\disc(B / A)) \quad(\text { as ideals in } A) .
                  \end{equation*}
              \end{remark}
    \end{enumerate}
\end{definition}






\begin{definition}
    By proposition, every finitely generated $\mathcal{O}_K$-submodule $\mathfrak{a}$ of $K$ admits a $\mathbb{Z}$-basis $\alpha_1, \ldots, \alpha_n$.
    The \textbf{discriminant} of ideal $\mathfrak{a}$ is defined as
    \begin{equation*}
        d(\mathfrak{a})
        :=
        \disc_{\mathfrak{a}/\mathbb{Z}}\left(\alpha_1, \ldots, \alpha_n\right)
    \end{equation*}
    is independent of the choice of a $\mathbb{Z}$-basis. ($\mathbb{Z}^{\times 2}=\{1\}$)

    In the special case of an integral basis $\omega_1, \ldots, \omega_n$ of $\mathcal{O}_K$ we obtain the discriminant of the algebraic number field $K$,
    \begin{equation*}
        d_K:=d\left(\mathcal{O}_K\right)=d\left(\omega_1, \ldots, \omega_n\right)
    \end{equation*}
    \begin{remark}
        Note that $d_K=\disc_{\mathcal{O}_K/\mathbb{Z}}\left(\omega_1, \ldots, \omega_n\right)=\disc_{K/\mathbb{Q}}\left(\omega_1, \ldots, \omega_n\right)$.
    \end{remark}
\end{definition}

\begin{proposition}
    If $\mathfrak{a} \subseteq \mathfrak{a}^{\prime}$ are two nonzero finitely generated $\mathcal{O}_K$-submodules of $K$, then the index $\left(\mathfrak{a}^{\prime}: \mathfrak{a}\right)$ is finite and satisfies

    \begin{equation*}
        d(\mathfrak{a})=\left(\mathfrak{a}^{\prime}: \mathfrak{a}\right)^2 d\left(\mathfrak{a}^{\prime}\right) .
    \end{equation*}

\end{proposition}



\begin{proposition}
    Let $L / K$ be a finite separable field extension of degree $n$, $\left\{\alpha_i\right\}$ a $K$-basis of $L$ and $\operatorname{Hom}_K(L, \overline{K})=\left\{\sigma_1, \ldots, \sigma_n\right\}$.
    Then let matrix $D=D\left(\alpha_1,\ldots,\alpha_n\right):=\left(\sigma_i\left(\alpha_j\right)\right)_{1 \leq i, j \leq n}$, the following statements hold:
    \begin{enumerate}
        \item
              Then $D^{\operatorname{tr}} D$ is the Gram matrix of the $\operatorname{Tr}_{L / K}(- \cdot -)$ with respect to $\left\{\alpha_i\right\}$.
              That is,
              \begin{equation*}
                  D^{\operatorname{tr}} D=\left(\operatorname{Tr}_{L / K}\left(\alpha_i \alpha_j\right)\right)_{1 \leq i, j \leq n}
              \end{equation*}
              Consequently, $\operatorname{det}\left(\operatorname{Tr}_{L / K}\left(\alpha_i \alpha_j\right)\right)_{1 \leq i, j \leq n}=\left(\operatorname{det} D\left(\alpha_1, \ldots, \alpha_n\right)\right)^2$.

        \item
              Let $L=K(a)$ for some primitive element $a$, then
              \begin{equation*}
                  \operatorname{disc}\left(1, a, \ldots, a^{n-1}\right)
                  =
                  \det\left(\sigma_i(a)^{k-1}\right)_{1 \leq i, k \leq n}
                  =
                  \prod_{1 \leq i<j \leq n}\left(\sigma_j(a)-\sigma_i(a)\right)^2\neq 0 .
              \end{equation*}


        \item
              Therefore $\operatorname{disc}\left(L/K\right)$ is non-zero and the trace pairing on $L / K$ is non-degenerate.
    \end{enumerate}
\end{proposition}



\begin{corollary}
    $d_{\mathcal{O}_K}\neq 0$
\end{corollary}








\section{Ramification}
In this section, let $A$ be a Dedekind domain with field of fractions $K$, and let $B$ be the integral closure of $A$ in a finite separable extension $L$ of $K$.
\begin{theorem}
    $B$ is also a Dedekind domain.
\end{theorem}


\begin{definition}
    A prime ideal $\mathfrak{p}$ of $A$ will factor in $B$,
    \begin{equation*}
        \mathfrak{p}B
        =
        \mathfrak{P}_1^{e_1} \cdots \mathfrak{P}_g^{e_g}
    \end{equation*}
    wher $\mathfrak{P}$ are distinct prime ideals in $B$ and $e_i\geq 1$,
    \begin{enumerate}
        \item
              We say $\mathfrak{P}$ divides $\mathfrak{p}$,written $\mathfrak{P} \mid \mathfrak{p}$, if $\mathfrak{P}$ occurs in the factorization of $\mathfrak{p}$ in $B$.
              The number $e_i$ is called the \textbf{ramification index} of $\mathfrak{P}_i$ over $\mathfrak{p}$.
        \item
              If any of the numbers is $>1$, then we say that $\mathfrak{p}$ is \textbf{ramified}.

    \end{enumerate}
    We then write $e(\mathfrak{P} / \mathfrak{p})$ for the ramification index and $f(\mathfrak{P} / \mathfrak{p})$ for the degree of the field extension $[B / \mathfrak{P}: A / \mathfrak{p}]$ (called the \textbf{residue class degree}).
    \begin{enumerate}[resume]
        \item
              $\mathfrak{p}$ is said to \textbf{split} (or split completely) in $L$ if $e_i=f_i=1$ for all $i$

        \item
              $\mathfrak{p}$ is said to be \textbf{inert} in $L$ if $\mathfrak{p}$ is a prime ideal in $B$ (so $g=1=e$ ).
    \end{enumerate}
\end{definition}


\begin{theorem}
    Let $m$ be the degree of $L$ over $K$, and let $\mathfrak{P}_1, \ldots, \mathfrak{P}_g$ be the prime ideals dividing $\mathfrak{p}$; then
    \begin{equation*}
        \sum_{i=1}^g e_i f_i=m
    \end{equation*}
    where $e_i=e\left(\mathfrak{P}_i / \mathfrak{p}\right)$ and $f_i=f\left(\mathfrak{P}_i / \mathfrak{p}\right)$. If $L$ is Galois over $K$, then all the ramification numbers are equal, and all the residue class degrees are equal, and so

    \begin{equation*}
        e f g=m .
    \end{equation*}
\end{theorem}


Again $A$ is a Dedekind domain with field of fractions $K$, and $B$ is the integral closure of $A$ in a finite separable extension $L$ of $K$.
\begin{theorem}
    Assume that $B$ is a free $A$-module.
    Then a prime $\mathfrak{p}$ ramifies in $L$ if and only if $\mathfrak{p} \mid \operatorname{disc}(B / A)$. In particular, only finitely many prime ideals ramify.
\end{theorem}


\begin{theorem}[Dedekind-Kummer]
    Suppose that $B=A[\alpha]$, and let $f(X)$ be the minimal polynomial of $\alpha$ over $K$. Let $\mathfrak{p}$ be a prime ideal in $A$ and reducible $f(X) =\prod g_i(X)^{e_i}$ in $\left(A/\mathfrak{p}\right)[X]$. Then
    \begin{equation*}
        \mathfrak{p} B=\prod\left(\mathfrak{p}, g_i(\alpha)\right)^{e_i}
    \end{equation*}
    is the factorization of $\mathfrak{p} B$ into a product of powers of distinct prime ideals.

    Moreover, the residue field $B /\left(\mathfrak{p}, g_i(\alpha)\right) \simeq(A / \mathfrak{p})[X] /\left(\bar{g}_i\right)$, and so the residue class degree $f_i$ is equal to the degree of $g_i$.
\end{theorem}
















\chapter{Dirichlet Unit Theorem}
\begin{theorem} [Dirichlet]
    Let $K$ be a number field of degree $n=r_1+2 r_2$. Then there is a group isomorphism
    \begin{equation*}
        \mathcal{O}_K^{\times} \simeq \mu_K \times \mathbb{Z}^{r_1+r_2-1},
    \end{equation*}
    where $\mu_K$ is the torsion subgroup of $\mathcal{O}_K^{\times}$ (the finite cyclic subgroup consisting of roots of unity)
\end{theorem}










\end{document}