\usepackage{amsmath}
\usepackage{amsthm}
\usepackage{amssymb}
\usepackage{bm}
\usepackage{graphicx}
\usepackage{mathrsfs}
\usepackage{esint}
\usepackage{hyperref}
\usepackage{enumerate}
\usepackage{tikz-cd}
\usepackage{commutative-diagrams}
\usepackage{geometry}
\usepackage{titlesec}       % 自定义章节标题
\usepackage{titletoc}       % 自定义目录样式
\usepackage{fancyhdr}       % 自定义页眉页脚
\usepackage{setspace}       % 行间距设置
\usepackage{fontspec}       % 字体设置






 % 设置正文字体




% ========== 行间距设置 ==========
\onehalfspacing % 1.5倍行距

% ========== 目录样式 ==========
\setcounter{tocdepth}{3} % 目录显示到 subsection
\setcounter{secnumdepth}{2} % 编号深度subsection

% ========== 自定义命令 ==========
\newtheorem{theorem}{Theorem}[section] 
% 定义定理类环境 跟section的计数器
\newtheorem{definition}[theorem]{Definition}
\newtheorem{lemma}[theorem]{Lemma}
\newtheorem{corollary}[theorem]{Corollary}
\newtheorem{example}[theorem]{example}
\newtheorem{proposition}[theorem]{Proposition}
% 与theorem环境共享计数器 


\newcommand{\Gal}{\mathrm{Gal}}
\newcommand{\Aut}{\mathrm{Aut}}
\newcommand{\Ker}{\mathrm{Ker}}
\newcommand{\norm}[1]{\left\| \emph{#1} \right\|}
\newcommand{\rank}{\mathrm{rank}}
\newcommand{\Int}{\mathrm{Int}}


\renewcommand{\char}{\mathrm{char}}
\renewcommand{\Im}{\mathrm{Im}}















% ========== 封面 ==========
\author{HHH}