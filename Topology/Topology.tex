\documentclass[12pt, a4paper, oneside]{book}


\usepackage{../mypackages}





\begin{document}
\frontmatter
\title{{\Huge{\textbf{Abstract Algebra}}}}
\maketitle

\dominitoc % 初始化minitoc
\pagenumbering{Roman}
\tableofcontents % 主目录


\mainmatter
\pagenumbering{arabic} % 正文编页码字体 



\chapter{Topological Space}
\section{Topological Space}
\subsection{Basic Definition}
\begin{definition}
    A \textbf{topology} on a set $X$ is a collection $\mathcal{T}$ of subsets of $X$ having the following properties:
    \begin{enumerate}[label=(\roman*)]
        \item $\varnothing$ and $X$ are in $\mathcal{T}$.
        \item  The union of the elements of any subcollection of $\mathcal{T}$ is in $\mathcal{T}$.
        \item The intersection of the elements of any finite subcollection of $\mathcal{T}$ is in $\mathcal{T}$.
    \end{enumerate}
    A set $X$ for which a topology $\mathcal{T}$ has been specified is called a \textbf{topological space}.
\end{definition}

\begin{proposition}
    Let $X$ be a topological space. Then the following conditions hold:

    (1) $\varnothing$ and $X$ are closed.

    (2) Arbitrary intersections of closed sets are closed.

    (3) Finite unions of closed sets are closed.
\end{proposition}


\begin{definition}
    Suppose that $\mathcal{T}$ and $\mathcal{T}^{\prime}$ are two topologies on a given set $X$. If $\mathcal{T}^{\prime} \supset \mathcal{T}$, we say that $\mathcal{T}^{\prime}$ is \textbf{finer} than $\mathcal{T}$; if $\mathcal{T}^{\prime}$ properly contains $\mathcal{T}$, we say that $\mathcal{T}^{\prime}$ is \textbf{strictly finer} than $\mathcal{T}$.

    We also say that $\mathcal{T}$ is \textbf{coarser} than $\mathcal{T}^{\prime}$, or strictly coarser, in these two respective situations.

    We say $\mathcal{T}$ is \textbf{comparable} with $\mathcal{T}^{\prime}$ if either $\mathcal{T}^{\prime} \supset \mathcal{T}$ or $\mathcal{T} \supset \mathcal{T}^{\prime}$.
\end{definition}
\subsection{Limit Points, Closure and Interior}


\begin{definition}
    Given a subset $A$ of a topological space $\left(X,\mathscr{T}\right)$, the \textbf{interior} of $A$ is defined as the union of all open sets contained in $A$,denoted by $\operatorname{Int} A$.
    And the \textbf{closure} of $A$ is defined as the intersection of all closed sets containing $A$, denoted by $\bar{A}$.

    Obviously Int $A$ is an open set and $\bar{A}$ is a closed set; furthermore,
    \begin{equation*}
        \operatorname{Int} A \subset A \subset \bar{A}
    \end{equation*}
\end{definition}

\begin{definition}
    If $A$ is a subset of the topological space $X$ and if $x$ is a point of $X$, we say that $x$ is a \textbf{limit point} of $A$ if every neighborhood of $x$ intersects $A$ in some point other than $x$ itself.

    The sets $A^\prime$ consisting of all limit points of $A$ is called \textbf{derived set} of $A$.
\end{definition}

\begin{theorem}
    Let $A$ be a subset of the topological space $X$.

    (1) Then $x \in \bar{A}$ if and only if every open set $U$ containing $x$ intersects $A$.

    (2) $x$ is a limit point of $A$ if and only if it belongs to the closure of $A-\{x\}$.

    (3) $\bar{A}=A\cup A^\prime$
\end{theorem}

\begin{corollary}
    A subset of a topological space is closed if and only if it contains all its limit points.
\end{corollary}


\subsection{Basis For a Topological Space}
\begin{definition}
    If $X$ is a set,

    (1) A \textbf{basis} for a topology on $X$ is a collection $\mathscr{B}$ of subsets of $X$ (called basis elements) such that
    \begin{enumerate}[label=(\roman*)]
        \item For each $x \in X$, there is at least one basis element $B$ containing $x$.
        \item If $x$ belongs to the intersection of two basis elements $B_1$ and $B_2$, then there is a basis element $B_3$ containing $x$ such that $B_3 \subset B_1 \cap B_2$.
    \end{enumerate}

    (2) If $\mathscr{B}$ satisfies these two conditions, then we define the topology \textbf{$\mathcal{T}$ generated by basis $\mathcal{B}$} as follows: $A$ subset $U$ of $X$ is said to be open in $X$ if for each $x \in U$, there is a basis element $B \in \mathscr{B}$ such that $x \in B$ and $B \subset U$.
\end{definition}

\begin{proposition}
    Let $X$ be a set; let $\mathscr{B}$ be a basis for a topology $\mathcal{T}$ on $X$. Then $\mathcal{T}$ equals the collection of all unions of elements of $\mathscr{B}$.
\end{proposition}


\begin{definition}
    A \textbf{subbasis} $\mathcal{S}$ for a topology on $X$ is a collection of subsets of $X$ whose union equals $X$.
    The topology generated by the subbasis $S$ is defined to be the collection $\mathcal{T}$ of all unions of finite intersections of elements of $\mathcal{S}$.
\end{definition}

\subsection{Subspace Topology} %% Subspace Topology
\begin{definition}
    Let $X$ be a topological space with topology $\mathcal{T}$. If $Y$ is a subset of $X$, the collection
    \begin{equation*}
        \mathcal{T}_Y=\{Y \cap U \mid U \in \mathcal{T}\}
    \end{equation*}
    is a topology on $Y$, called the \textbf{subspace topology}. With this topology, $Y$ is called a subspace of $X$; its open sets consist of all intersections of open sets of $X$ with $Y$.
\end{definition}

\begin{theorem}
    Let $X$ be a topological space and $Y$ a subset of $X$

    (1) If $\mathscr{B}$ is a basis for the topology of $X$ then the collection
    \begin{equation*}
        \mathscr{B}_Y=\{B \cap Y \mid B \in \mathscr{B}\}
    \end{equation*}
    is a basis for the subspace topology on $Y$.

    (2) Then a set $A$ is closed in $Y$ if and only if it equals the intersection of a closed set of $X$ with $Y$.

    (3) If $A$ is closed in $Y$ and $Y$ is closed in $X$, then $A$ is closed in $X$.
    \begin{proof}
        (1)

        (2)
        Assume that $A=C \cap Y$, where $C$ is closed in $X$. Then $X-C$ is open in $X$, so that the complement of $A$ in $Y$, $Y-A=(X-C) \cap Y$, is open in $Y$, then so $A$ is closed in $Y$.

        Conversely, assume that $A$ is closed in $Y$. Then $Y-A$ is open in $Y$, so that it equals the intersection of an open set $U$ of $X$ with $Y$. The set $X-U$ is closed in $X$, and $A=Y \cap(X-U)$, so that $A$ equals the intersection of a closed set of $X$ with $Y$, as desired.
    \end{proof}
\end{theorem}



\chapter{Countability} %% Countability

\section{The Countability Axioms} %% The Countability Axioms
\begin{definition}
    A space $X$ is said to have a countable basis at $x$ if there is a countable collection $\mathscr{B}$ of neighborhoods of $x$ such that each neighborhood of $x$ contains at least one of the elements of $\mathscr{B}$.

    A space that has a countable basis at each of its points is said to satisfy the first countability axiom, or to be \textbf{first-countable}.
\end{definition}

\begin{theorem}
    Let $X$ be a topological space.

    (1) Let $A$ be a subset of $X$. If there is a sequence of points of $A$ converging to $x$, then $x \in \bar{A}$; the converse holds if $X$ is first-countable.

    (2) Let $f: X \rightarrow Y$. If $f$ is continuous, then for every convergent sequence $x_n \rightarrow x$ in $X$, the sequence $f\left(x_n\right)$ converges to $f(x)$. The converse holds if $X$ is first countable.
\end{theorem}

\begin{definition}
    If a space $X$ has a countable basis for its topology, then $X$ is said to satisfy the second countability axiom, or to be \textbf{second-countable}.
\end{definition}

\begin{definition}
    A subset $A$ of a space $X$ is said to be \textbf{dense} in $X$ if $\bar{A}=X$.
    A space having a countable dense subset is often said to be \textbf{separable}.
\end{definition}



\begin{theorem}[Hereditarity and Countable Multiplicativity]
    A subspace of a first-countable space is first-countable, and a countable product of first-countable spaces is first-countable.

    A subspace of a secondcountable space is second-countable, and a countable product of second-countable spaces is second-countable.
\end{theorem}

\begin{definition}
    A space for which every open coverng contains a countable subcovering is called a \textbf{Lindelöf} space.
\end{definition}

\begin{theorem}
    Suppose that $X$ has a countable basis. Then:

    (1) $X$ is a Lindelöf space.

    (2) $X$ is separable.
\end{theorem}



\section{The Separation Axioms}  %% The Separation Axioms

\begin{definition}
    A space $X$ is said to be \textbf{Frechét} (or $T_1$) if for each pair of distinct point $x$ and
    $y$, there exist neighborhood of $x$ which not contains $y$.

    A space $X$ is said to be \textbf{Hausdorff} (or $T_2$) if for each pair consisting of distinct point $x$ and $y$, there exist disjoint open sets containing $x$ and $y$, respectively.


    A space $X$ is said to be \textbf{regular} (or $T_3$) if for each pair consisting of a point $x$ and a closed set $B$ disjoint from $x$, there exist disjoint open sets containing $x$ and $B$, respectively.

    The space $X$ is said to be \textbf{normal} (or $T_4$) if for each pair $A, B$ of disjoint closed sets of $X$, there exist disjoint open sets containing $A$ and $B$, respectively.
\end{definition}


\begin{theorem}
    Let $\left(X,\tau\right)$ be topology space

    (1) $X$ is $T_1$ if and only if single point set $\left\{x\right\}$ is closed.

    (2) $X$ is $T_2$ if and only if $\Delta=\left\{\left(x,x\right): x\in X\right\}$ is closed in $X\times X$

    (3) $X$ is $T_3$ if and only if for every $x\in X$ and open set $U$ containing $x$, there exists open $V$ that $x \in V \subset \overline{V} \subset U$.

    (4) $X$ is $T_4$ if and only if for every closed  $A\subset X$ and open set $U$ containing $A$, there exists open $V$ that $A \subset V \subset \overline{V} \subset U$.
\end{theorem}


\begin{proposition}
    Let $X$ be a space. Then

    (1) $T_2 \Rightarrow T_1$

    (2) Compact $+T_2 \Rightarrow T_3$

    (3) Compact $+T_3 \Rightarrow T_4$; thus Compact $+T_2 \Rightarrow T_4$

    (4) Lindelof $+T_3 \Rightarrow T_4$; Thus $A_2 + T_3 \Rightarrow T_4$ and $\sigma$-compact $+T_3 \Rightarrow T_4$
\end{proposition}


\section{Urysohn Lemma}  %% Urysohn lemma

\begin{theorem}[Urysohn lemma]
    Let $X$ be a normal ($T_4$) space, let $A$ and $B$ be disjoint closed subsets of $X$. Let $[a, b]$ be a closed interval in the real line. Then there exists a continuous map
    \begin{equation*}
        f: X \longrightarrow[a, b]
    \end{equation*}
    such that $f(A)=a$ and $f(B)=b$.
    \begin{proof}
        Put $r_1=0, r_2=1$, and let $r_3, r_4, r_3, \ldots$ be an enumeration of the rationals in $( 0,1 )$. By Theorem 2.7, we can find open sets $V_0$ and then $V_1$ such that $\overline{V}_0$ is compact and
        \begin{equation*}
            A \subset V_1 \subset \overline{V}_1 \subset V_0 \subset \overline{V}_0 \subset B^c
        \end{equation*}
        Suppose $n \geq 2$ and $V_{r_1}, \ldots, V_{r_n}$ have been chosen in such a manner that $r_i<r_j$ implies $\overline{V}_{r_j} \subset V_{r_1}$. Then one of the numbers $r_1, \ldots, r_n$, say $r_i$, will be the largest one which is smaller than $r_{n+1}$, and another, say $r_j$, will be the smallest one larger than $r_{n+1}$. Using Theorem 2.7 again, we can find $V_{r_{n+1}}$ so that
        \begin{equation*}
            \overline{V}_{r_ j} \subset V_{r_{n+1}} \subset \overline{V}_{r_{n+1}} \subset V_{r_j}
        \end{equation*}
        Continuing, we obtain a collection $\left\{V_r\right\}$ of open sets, one for every rational $r \in[0,1]$, with the following properties: $A \subset V_1, \overline{V}_0 \subset B^c$, each $\overline{V}_r$, and
        \begin{equation*}
            s>r \text { implies } \overline{V}_s \subset V_r
        \end{equation*}
        Define
        \begin{equation*}
            f_r(x)
            =
            r \chi_{V_r}
            \quad
            g_s(x)
            =
            s + (1-s)\chi_{\overline{V}_s}
        \end{equation*}
        and
        \begin{equation*}
            f=\sup _r f_r, \quad g=\inf _s g_s
        \end{equation*}
        then $f$ is lower semicontinuous and that $g$ is upper semicontinuous. It is clear that $0 \leq f \leq 1$, that
        $f(A)=1$ and that $f(B)=0$.
        The proof will be completed by showing that $f=g$.

        The inequality $f_r(x)>g_s(x)$ is possible only if $r>s, x \in V_r$, and $x \notin \bar{V}_s$. But $r>s$ implies $V_r \subset V_s$. Hence $f_r \leq g$, for all $r$ and $s$, so $f \leq g$.
        Suppose $f(x)<g(x)$ for some $x$. Then there are rationals $r$ and $s$ such that $f(x)<r<s<g(x)$. Since $f(x)<r$, we have $x \notin V_r$; since $g(x)>s$, we have $x \in \overline{V}_s$. By (3), this is a contradiction. Hence $f=g$.
    \end{proof}
\end{theorem}

\begin{corollary}
    Suppoes that $X$ is a $T_4$ space, then there exists $f\in C(X)$ that $f^{-1}(0)=A$ if and only if $A$ is a closed $G_\delta$'s.

    Proof.
    1.
    \begin{equation*}
        A =f^{-1}(0)
        =
        \bigcup_n f^{-1}\left((-\frac{1}{n},\frac{1}{n})\right)
    \end{equation*}

    2. Assume that $A$ is a closed $G_\delta$'s, there exists open sers $U_i$ that $A=\bigcap U_i$. By Urysohn lemma, there is $g_n : X \rightarrow \left[0,1\right]$ that
    \begin{equation*}
        A \subset g_n^{-1}(0), \quad U_n^c \subset g_n^{-1}(1)
    \end{equation*}
    Define
    \begin{equation*}
        f=\sum_{n=1}^\infty \frac{g_n(x)}{2^n}
    \end{equation*}
    thus $f^{-1}(0)=A$
\end{corollary}


\begin{theorem}[Strong Urysohn lemma]
    Let $X$ be a normal ($T_4$) space, let $A$ and $B$ be disjoint closed $G_\delta$'s subsets of $X$. Let $[a, b]$ be a closed interval in the real line. Then there exists a continuous map $f: X \rightarrow[a, b]$ such that\begin{equation*}
        f^{-1}(a)=A, \quad f^{-1}(b)=B
    \end{equation*}
\end{theorem}

\section{Tietze Extension Theorem} %% Tietze extension theorem

\begin{theorem}[Tietze extension theorem]
    Let $X$ be a normal $T_4$ space; let $A$ be a closed subspace of $X$.
    Then any continuous map $f:A \rightarrow \mathbb{R}$ may be extended to a continuous map $F : X\rightarrow \mathbb{R}$ that $\left.F\right|_A=f$
\end{theorem}

\section{Urysohn Metrization Theorem} %% Urysohn Metrization Theorem
\begin{proposition}
    The topology space $\left(X,\tau\right)$ is metrizable if and only if $X$ can be topologically embedded into a metric space.
\end{proposition}

\begin{theorem}
    If $X$ is $T_1,T_4$ and $A_2$ (thus $T_i,A_j$), then $X$ can be imbedding into $\left[0,1\right]^\omega$ or $\mathbb{R}^\omega$; thus metrizable.
\end{theorem}



\chapter{Connectness}  %% Connectness
\section{Connected Space} %% Connected Space

\begin{definition}
    Let $X$ be a topological space. A \textbf{separation} of $X$ is a pair $U, V$ of disjoint nonempty open subsets of $X$ whose union is $X$.
    The space $X$ is said to be \textbf{connected} if there does not exist a separation of $X$.
\end{definition}

\begin{theorem}
    If $Y$ is a subspace of $X$, a separation of $Y$ is a pair of disjoint nonempty sets $A$ and $B$ whose union is $Y$, neither of which contains a limit point of the other. The space $Y$ is connected if there exists no separation of $Y$.
    \begin{proof}
        Suppose first that $A$ and $B$ form a separation of $Y$. Then $A$ is both open and closed in $Y$. The closure of $A$ in $Y$ is the set $\bar{A} \cap Y$ (where $\bar{A}$ as usual denotes the closure of $A$ in $X$ ). Since $A$ is closed in $Y, A=\bar{A} \cap Y$; or to say the same thing, $\bar{A} \cap B=\varnothing$. Since $\bar{A}$ is the union of $A$ and its limt points, $B$ contains no limit points of $A$. A similar argument shows that $A$ contains no limt points of $B$.

        Conversely, suppose that $A$ and $B$ are disjoint nonempty sets whose union is $Y$, neither of which contains a limt point of the other. Then $\bar{A} \cap B=\varnothing$ and $A \cap \bar{B}=\varnothing$;
        therefore, we conclude that $\bar{A} \cap Y=A$ and $\bar{B} \cap Y=B$. Thus both $A$ and $B$ are closed in $Y$, and since $A=Y-B$ and $B=Y-A$, they are open in $Y$ as well.
    \end{proof}
\end{theorem}


\begin{theorem}
    Let $\left(X,\tau\right)$ be a topology space.

    (1) If the sets $C$ and $D$ form a separation of $X$, and if $Y$ is a connected subspace of $X$, then $Y$ lies entirely within either $C$ or $D$.

    (2) The union of a collection of connected subspaces of $X$ that have a point in common is connected.

    (3) Let $A$ be a connected subspace of $X$. If $A \subset B \subset \bar{A}$, then $B$ is also connected.
    \begin{proof}
        We only prove (3).
        Let $A$ be connected and let $A \subset B \subset \bar{A}$. Suppose that $B=C \cup D$ is a separation of $B$, then $A$ must lie entirely in $C$ or in $D$; suppose that $A \subset C$. Then $\bar{A} \subset \bar{C}$; since $\bar{C}$ and $D$ are disjoint, $B$ cannot intersect $D$. This contradicts the fact that $D$ is a nonempty subset of $B$.
    \end{proof}
\end{theorem}

\begin{theorem}
    A finite cartesian product of connected spaces is connected.
    \begin{proof}
        We prove the theorem first for the product of two connected spaces $X$ and $Y$.
        Choose a "base point" $a \times b$ in the product $X \times Y$. Note that the "horizontal slice" $X \times b$ is connected, and each "vertical slice" $x \times Y$ is connected.
        As a result, each "T-shaped" space
        \begin{equation*}
            T_x=(X \times b) \cup(x \times Y)
        \end{equation*}
        is connected, being the union of two connected spaces that have the point $x \times b$ in common.
        Now the union
        \begin{equation*}
            \bigcup_{x\in X} T_x =X\times Y
        \end{equation*}
        is connected because it is the union of a collection of connected spaces that have the point $a \times b$ in common.

        The proof for any finite product of connected spaces follows by induction, using the fact that $X_1 \times \cdots \times X_n$ is homeomorphic with $\left(X_1 \times \cdots \times X_{n-1}\right) \times$ $X_n$.
    \end{proof}
\end{theorem}



\begin{theorem}
    The image of a connected space under a continuous map is connected.
\end{theorem}


\section{}
\begin{lemma}
    Given $X$, define an equivalence relation on $X$ by setting $x \sim y$ if there is a connected subspace of $X$ containing both $x$ and $y$. The equivalence classes are called the \textbf{connected components} of $X$.
\end{lemma}

\begin{theorem}
    The components of $X$ are connected disjoint subspaces of $X$ whose union is $X$, such that each nonempty connected subspace of $X$ intersects only one of them.
\end{theorem}

\begin{definition}
    We define another equivalence relation on the space $X$ by defining $x \sim y$ if there is a path in $X$ from $x$ to $y$. The equivalence classes are called the \textbf{path components} of $X$.
\end{definition}

\begin{theorem}
    The path components of $X$ are path-connected disjoint subspaces of $X$ whose union is $X$, such that each nonempty path-connected subspace of $X$ intersects only one of them.
\end{theorem}



\begin{definition}
    A space $X$ is said to be \textbf{locally connected at $x$}if for every neighborhood $U$ of $x$, there is a connected neighborhood $V$ of $x$ contained in $U$.
    If $X$ is locally connected at each of its points, it is said simply to be \textbf{locally connected}.

    Similarly, a space $X$ is said to be locally path connected at $x$ if for every neighborhood $U$ of $x$, there is a path-connected neighborhood $V$ of $x$ contained in $U$. If $X$ is locally path connected at each of its points, then it is said to be \textbf{locally path connected}.
\end{definition}

\begin{theorem}
    A space $X$ is locally connected if and only if for every open set $U$ of $X$, each connected component of $U$ is open in $X$.

    \begin{proof}
        Suppose that $X$ is locally connected; let $U$ be an open set in $X$; let $C$ be a component of $U$ If $x$ is a point of $C$, we can choose a connected neighborhood $V$ of $x$ such that $V \subset U$. Since $V$ is connected, it must lie entirely in the component $C$ of $U$. Therefore, $C$ is open in $X$.

        Conversely, suppose that components of open sets in $X$ are open. Given a point $x$ of $X$ and a neighborhood $U$ of $x$, let $C$ be the component of $U$ containing $x$. Now $C$ is connected; since it is open in $X$ by hypothesis, $X$ is locally connected at $x$.
    \end{proof}
\end{theorem}


\begin{theorem}
    A space $X$ is locally path connected if and only if for every open set $U$ of $X$, each path connected component of $U$ is open in $X$.
\end{theorem}

\begin{theorem}
    Let $\left(X, \tau\right)$ be a topology space.

    (1) If $X$ is path connected, then $X$ is connected.

    (2) Each path component of $X$ lies in a component of $X$

    (3) If $X$ is locally path connected, then the components and the path components of $X$ are the same.
\end{theorem}


\chapter{Compactness}
\section{Basic Definition} %% Basic Definition}
\begin{definition}
    A collection $\mathcal{A}$ of subsets of a space $X$ is said to cover $X$, or to be a covering of $X$, if the union of the elements of $\mathcal{A}$ is equal to $X$. It is called an open covering of $X$ if its elements are open subsets of $X$.
\end{definition}



\begin{definition}
    A space $X$ is said to be \textbf{compact} if every open covering $\mathcal{A}$ of $X$ contains a finite subcollection that also covers $X$.

    A space $X$ is said to be \textbf{sequentially compact} if every sequence of $X$ contains a convergent subsequence.
\end{definition}

\begin{theorem}[Criterion for compact sets]
    Let $Y$ be a subspace of $X$. Then $Y$ is compact if and only if every covering of $Y$ by sets open in $X$ contains a finite subcollection covering $Y$.
    \begin{proof}
        Suppose that $Y$ is compact and $\mathcal{A}=\left\{A_\alpha\right\}_{\alpha \in J}$ is a covering of $Y$ by sets open in $X$. Then the collection
        \begin{equation*}
            \left\{A_\alpha \cap Y \mid \alpha \in J\right\}
        \end{equation*}
        is a covering of $Y$ by sets open in $Y$; hence a finite subcollection
        \begin{equation*}
            \left\{\boldsymbol{A}_{\boldsymbol{\alpha}_1} \cap Y, \ldots, \boldsymbol{A}_{\boldsymbol{\alpha}_n} \cap Y\right\}
        \end{equation*}
        covers $Y$. Then $\left\{A_{\alpha_1}, \ldots, A_{\alpha_n}\right\}$ is a subcollection of $\mathscr{A}$ that covers $Y$.

        Conversely, suppose the given condition holds; we wish to prove $Y$ compact. Let $\mathcal{A}^{\prime}=\left\{A_\alpha^{\prime}\right\}$ be a covering of $Y$ by sets open in $Y$. For each $\alpha$, choose a set $A_\alpha$ open in $X$ such that
        \begin{equation*}
            A_\alpha^{\prime}=A_\alpha \cap Y .
        \end{equation*}
        The collection $\mathcal{A}=\left\{A_\alpha\right\}$ is a covering of $Y$ by sets open in $X$ By hypothesis, some finite subcollection $\left\{A_{\alpha_1}, \ldots, A_{\alpha_n}\right\}$ covers $Y$. Then $\left\{A_{\alpha_1}^{\prime}, \ldots, A_{\alpha_n}^{\prime}\right\}$ is a subcollection of $\mathcal{A}^{\prime}$ that covers $Y$
    \end{proof}
\end{theorem}

\begin{theorem}
    Let $X$ be a topology space.

    (1) If $X$ is compact, then closed set of $X$ is compact.

    (2) Every compact subspace of a Hausdorff space is closed.

    (3) If $Y$ is a compact subspace of the $T_2$ space $X$ and $x_0 \notin Y$, then there exist disjoint open sets $U$ and $V$ of $X$ containing $x_0$ and $Y$, respectively.

    (4) The image of a compact space under a continuous map is compact.

    (5) The product of finitely many compact spaces is compact.
\end{theorem}

\begin{lemma}[The tube lemma]
    Consider the product space $X \times Y$, where $Y$ is compact. If $N$ is an open set of $X \times Y$ containing the slice $x_0 \times Y$, then $N$ contains some tube $W \times Y$, where $W$ is a neighborhood of $x_0$ in $X$.
\end{lemma}






\begin{definition}
    A collection $\mathcal{C}$ of subsets of $X$ is sad to have the \textbf{finite intersection property} if for every finite subcollection
    \begin{equation*}
        \left\{C_1, \ldots, C_n\right\}
    \end{equation*}
    of $\mathcal{C}$, the intersection $C_1 \cap \cdots \cap C_n$ is nonempty.


\end{definition}



\begin{theorem}
    Let $X$ be a topological space. Then $X$ is compact if and only if for every collection $\mathcal{C}$ of closed sets in $X$ having the finite intersection property, the intersection $\bigcap_{C \in \mathcal{C}} C$  is nonempty.
\end{theorem}

\section{Tychonoff Theorem} %%  Tychonoff Theorem

\begin{theorem}[Tychonoff theorem]
    An arbitrary product of compact spaces is compact in the product topology
    \begin{proof}
        Let
        \begin{equation*}
            X=\prod_{\alpha \in J} X_\alpha
        \end{equation*}
        where each space $X_\alpha$ is compact. Let $\mathcal{A}$ be a collection of subsets of $X$ having the finite intersection property. We prove that the intersection
        is nonempty.

        Compactness of $X$ follows.
        Applying Lemma 37.1, choose a collection $\mathcal{D}$ of subsets of $X$ such that $\mathcal{D} \supset \mathcal{A}$ and $D$ is maximal with respect to the finite intersection property It will suffice to show that the intersection $\bigcap_{D \in D} \bar{D}$ is nonempty.

        Given $\alpha \in J$, let $\pi_\alpha: X \rightarrow X_\alpha$ be the projection map, as usual Consider the collection
        \begin{equation*}
            \left\{\pi_\alpha(D) \mid D \in D\right\}
        \end{equation*}
        §37
        The Tychonoff Theorem
        235
        of subsets of $X_\alpha$. This collection has the finite intersection property because $\mathcal{D}$ does. By compactness of $X_\alpha$, we can for each $\alpha$ choose a point $x_\alpha$ of $X_\alpha$ such that
        \begin{equation*}
            x_\alpha \in \bigcap_{D \in D} \overline{\pi_\alpha(D)}
        \end{equation*}

        Let x be the point $\left(x_\alpha\right)_{\alpha \in J}$ of $X$. We shall show that $\mathrm{x} \in \bar{D}$ for every $D \in D$; then our proof will be finished.

        First we show that if $\pi_\beta^{-1}\left(U_\beta\right)$ is any subbasis element (for the product topology on $X$ ) containing x , then $\pi_\beta^{-1}\left(U_\beta\right)$ intersects every element of $\mathcal{D}$. The set $U_\beta$ is a neighborhood of $x_\beta$ in $X_\beta$. Since $x_\beta \in \overline{\pi_\beta(D)}$ by definition, $U_\beta$ intersects $\pi_\beta(D)$ in some point $\pi_\beta(\mathbf{y})$, where $\mathbf{y} \in D$ Then it follows that $\mathbf{y} \in \pi_\beta^{-1}\left(U_\beta\right) \cap D$.

        It follows from (b) of Lemma 37.2 that every subbasis element containing $\mathbf{x}$ belongs to $\mathfrak{D}$. And then it follows from (a) of the same lemma that every basis element containing $\mathbf{x}$ belongs to $\mathcal{D}$. Since $\mathcal{D}$ has the finite intersection property, this means that every basis element containing $\mathbf{x}$ intersects every element of $\mathcal{D}$; hence $\mathbf{x} \in \bar{D}$ for every $D \in D$ as desired.
    \end{proof}
\end{theorem}


\section{Compactness in Metric Space}

\begin{theorem}
    If $X$ is a metric space, then the following
    \begin{enumerate}
        \item
              $X$ is compact

        \item  $X$ is sequentially compact, that is, every sequence $\left\{x_n\right\}_{n=1}^\infty$ in $X$ has a convergent subsequence.

        \item  $X$ is complete and totally bounded.

        \item
              Bolzano-Weierstrass property. Every infinite subset $A\subset X$ has a limit point at least.
    \end{enumerate}
\end{theorem}

\begin{corollary}
    Let $X$ be a metric space and a subset $A\subset X$.
    \begin{enumerate}
        \item
              $A$ is precompact, i.e., its closure $\overline{A}$ is compact.
        \item
              every sequence in $A$ has a Cauchy subsequence.
        \item
              $A$ is totally bounded.

        \item
              Bolzano-Weierstrass property. Every infinite subset of $A$ has a limit point (may be not in $\overline{A}$) at least.
    \end{enumerate}
\end{corollary}

























































\chapter{The Fundamental Group}

\section{Homotopy of Paths}
\begin{definition}
    If $f$ and $f^{\prime}$ are continuous maps of the space $X$ into the space $Y$, we say that $f$ is homotopic to $f^{\prime}$ if there is a continuous map $F : X \times I \rightarrow Y$ such that
    \begin{equation*}
        F(x, 0)=f(x) \quad \text { and } \quad F(x, 1)=f^{\prime}(x)
    \end{equation*}
    for each $x$. (Here $I=[0,1]$.) The map $F$ is called a \textbf{homotopy} between $f$ and $f^{\prime}$.
    If $f$ is homotopic to $f^{\prime}$, we write $f \simeq f^{\prime}$. If $f \simeq f^{\prime}$ and $f^{\prime}$ is a constant map, we say that $f$ is \textbf{nulhomotopic}.

    Two paths $f$ and $f^{\prime}$, mapping the interval $I=[0,1]$ into $X$, are said to be \textbf{path homotopic} if they have the same initial point $x_0$ and the same final point $x_1$, and if there is a continuous map $F: I \times I \rightarrow X$ such that
    \begin{equation*}
        \begin{aligned}
             & F(s, 0)=f(s) \quad \text { and } \quad F(s, 1)=f^{\prime}(s) \\
             & F(0, t)=x_0 \quad \text { and } \quad F(1, t)=x_1
        \end{aligned}
    \end{equation*}
    for each $s \in I$ and each $t \in I$. We call $F$ a \textbf{path homotopy} between $f$ and $f^{\prime}$. If $f$ is path homotopic to $f^{\prime}$, we write $f \simeq_p f^{\prime}$.
\end{definition}


\begin{lemma}
    The relations $\simeq$ and $\simeq_p$ are equivalence relations.
\end{lemma}

\begin{definition}
    If $f$ is a path in $X$ from $x_0$ to $x_1$, and if $g$ is a path in $X$ from $x_1$ to $x_2$, we define the product $f * g$ of $f$ and $g$ to be the path $h$ given by the equations
    \begin{equation*}
        h(s)= \begin{cases}f(2 s) & \text { for } s \in\left[0, \frac{1}{2}\right] \\ g(2 s-1) & \text { for } s \in\left[\frac{1}{2}, 1\right]\end{cases}
    \end{equation*}
    The function $h$ is well-defined and continuous, by the pasting lemma; it is a path in $X$ from $x_0$ to $x_2$.


\end{definition}



\begin{theorem}
    The operation $*$ has the following properties:

    (1) The product operation on paths induces a well-defined operation on path-homotopy classes, defined by the equation
    \begin{equation*}
        [f] *[g]=[f * g]
    \end{equation*}

    (2) Associativity. If $[f] *([g] *[h])$ is defined, so is $([f] *[g]) *[h]$, and they are equal.


    (3) Right and left identities. Given $x \in X$, let $e_x$ denote the constant path $e_x : I \rightarrow$ $X$ carrying all of $I$ to the point $x$. If $f$ is a path in $X$ from $x_0$ to $x_1$, then
    \begin{equation*}
        [f] *\left[e_{x_1}\right]=[f] \quad \text { and } \quad\left[e_{x_0}\right] *[f]=[f]
    \end{equation*}

    (4) Inverse. Given the path $f$ in $X$ from $x_0$ to $x_1$, let $\bar{f}$ be the path defined by $\bar{f}(s)=f(1-s)$. It is called the reverse of $f$ Then
    \begin{equation*}
        [f] *[\bar{f}]=\left[e_{x_0}\right] \quad \text { and } \quad[\bar{f}] *[f]=\left[e_{x_1}\right]
    \end{equation*}
\end{theorem}


\begin{definition}
    Let $X$ be a space; let $x_0$ be a point of $X$.
    A path in $X$ that begins and ends at $x_0$ is called a \textbf{loop based at $x_0$}.
    The set of path homotopy classes of loops based at $x_0$, with the operation $*$, is called the \textbf{fundamental group of $X$ relative to the base point $x_0$}. It is denoted by $\pi_1\left(X, x_0\right)$.
\end{definition}



\begin{theorem}
    Let $\alpha$ be a path in $X$ from $x_0$ to $x_1$. Then $\alpha$ induces a group isomorphism by
    \begin{equation*}
        \hat{\alpha}: \pi_1\left(X, x_0\right) \longrightarrow \pi_1\left(X, x_1\right)
    \end{equation*}
    by the equation
    \begin{equation*}
        \hat{\alpha}([f])
        =
        [\bar{\alpha}] *[f] *[\alpha]
        =
        [\bar{\alpha}]^{-1} *[f] *[\alpha]
    \end{equation*}
\end{theorem}

\begin{proposition}
    Let $\alpha$ be a path in $X$ from $x_0$ to $x_1$; let $\beta$ be a path in $X$ from $x_1$ to $x_2$. Show that if $v=\alpha * \beta$, then $\hat{v}=\hat{\beta} \circ \hat{\alpha}$.
\end{proposition}


\begin{corollary}
    If $X$ is path connected and $x_0$ and $x_1$ are two points of $X$, then $\pi_1\left(X, x_0\right)$ is isomorphic to $\pi_1\left(X, x_1\right)$.
\end{corollary}


\begin{definition}
    A space $X$ is said to be \textbf{simply connected} if it is a path-connected space and if $\pi_1\left(X, x_0\right)$ is the trivial (one-element) group for some $x_0 \in X$, and hence for every $x_0 \in X$. We often express the fact that $\pi_1\left(X, x_0\right)$ is the trivial group by writing $\pi_1\left(X, x_0\right)=0$.
\end{definition}

\begin{definition}
    Let $h : \left(X, x_0\right) \rightarrow\left(Y, y_0\right)$ be a continuous map. Define
    \begin{equation*}
        h_* :\pi_1\left(X, x_0\right) \longrightarrow \pi_1\left(Y, y_0\right)
    \end{equation*}
    by the equation
    \begin{equation*}
        h_*([f])=[h \circ f]
    \end{equation*}
    The map $h_*$ is called the \textbf{homomorphism induced by $h$}, relative to the base point $x_0$.
\end{definition}

\begin{theorem}
    If $h:\left(X, x_0\right) \rightarrow\left(Y, y_0\right)$ and $k:\left(Y, y_0\right) \rightarrow\left(Z, z_0\right)$ are continuous, then $(k \circ h)_*=k_* \circ h_*$ If $i:\left(X, x_0\right) \rightarrow\left(X, x_0\right)$ is the identity map, then $i_*$ is the identity homomorphism.
\end{theorem}

\begin{corollary}
    If $h:\left(X, x_0\right) \rightarrow\left(Y, y_0\right)$ is a homeomorphism of $X$ with $Y$, then $h_*$ is an isomorphism of $\pi_1\left(X, x_0\right)$ with $\pi_1\left(Y, y_0\right)$.
\end{corollary}




\section{Covering Space} %% Covering Space
\begin{definition}
    Let $p : E \rightarrow B$ be a continuous surjective map. The open set $U$ of $B$ is said to be \textbf{evenly covered by $p$} if the inverse image
    $p^{-1}(U)$
    can be written as the union of disjoint open sets $V_\alpha$ in $E$ such that for each $\alpha$, the restriction of $p$ to $V_\alpha$ is a homeomorphism of $V_\alpha$ onto $U$.
    The collection $\left\{V_\alpha\right\}$ will be called a partition of $p^{-1}(U)$ into slices

    If every point $b$ of $B$ has a neighborhood $U$ that is evenly covered by $p$, then $p$ is called a \textbf{covering map}, and $E$ is said to be a \textbf{covering space} of $B$
\end{definition}



\begin{theorem}[Hereditarity]
    Let $p: E \rightarrow B$ be a coverng map. If $B_0$ is a subspace of $B$, and if $E_0=p^{-1}\left(B_0\right)$, then the map $p_0: E_0 \rightarrow B_0$ obtained by restructing $p$ is a covering map.
    \begin{proof}
        Given $b_0 \in B_0$, let $U$ be an open set in $B$ containing $b_0$ that is evenly covered by $p$; let $\left\{V_\alpha\right\}$ be a partition of $p^{-1}(U)$ into slices. Then $U \cap B_0$ is a neighborhood of $b_0$ in $B_0$, and the sets $V_\alpha \cap E_0$ are disjoint open sets in $E_0$ whose union is $p^{-1}\left(U \cap B_0\right)$, and each is mapped homeomorphically onto $U \cap B_0$ by $p$.
    \end{proof}
\end{theorem}


\begin{theorem}[Finite Multiplicativity]
    It $p: E \rightarrow B$ and $p^{\prime}: E^{\prime} \rightarrow B^{\prime}$ are coverng maps, then
    \begin{equation*}
        p \times p^{\prime}: E \times E^{\prime} \rightarrow B \times B^{\prime}
    \end{equation*}
    is a covering map.

    \begin{proof}
        Given $b \in B$ and $b^{\prime} \in B^{\prime}$, let $U$ and $U^{\prime}$ be neighborhoods of $b$ and $b^{\prime}$, respectively, that are evenly covered by $p$ and $p^{\prime}$, respectively. Let $\left\{V_\alpha\right\}$ and $\left\{V_\beta^{\prime}\right\}$ be partitions of $p^{-1}(U)$ and $\left(p^{\prime}\right)^{-1}\left(U^{\prime}\right)$, respectively, into slices. Then the inverse image under $p \times p^{\prime}$ of the open set $U \times U^{\prime}$ is the union of all the sets $V_\alpha \times V_\beta^{\prime}$. These are disjoint open sets of $E \times E^{\prime}$, and each is mapped homeomorphically onto $U \times U^{\prime}$ by $p \times p^{\prime}$.
    \end{proof}
\end{theorem}











\section{Lifting Lemma}

\begin{definition}
    Let $p :E \rightarrow B$ be a map. If $f$ is a continuous mapping of some space $X$ into $B$, a \textbf{lifting} of $f$ is a map $\bar{f}: X \rightarrow E$ such that $p \circ \bar{f}=f$.
    \begin{equation*}
        \begin{tikzcd}
            X \arrow[rr, "\bar{f}"] \arrow[rrdd, "f"'] &  & E \arrow[dd, "p"] \\
            &  &    \\
            &  & B
        \end{tikzcd}
    \end{equation*}
\end{definition}

\begin{theorem}[Path-lifting lemma]
    Let $p: \left(E,e_0\right) \rightarrow \left(B,b_0\right)$ be a covering map.
    Any path $f: [0,1] \rightarrow B$ beginning at $b_0$ has a unique lifting to a path $\bar{f}$ in $E$ beginning at $e_0$.
\end{theorem}

\begin{theorem}
    Let $p: \left(E,e_0\right) \rightarrow \left(B,b_0\right)$ be a covering map.
    Let the map $F: I \times I \rightarrow B$ be continuous, with $F(0,0)=b_0$. There is a unique lifting of $F$ to a continuous map
    \begin{equation*}
        \tilde{F}: I \times I \rightarrow E
    \end{equation*}
    such that $\bar{F}(0,0)=e_0$. If $F$ is a path homotopy, then $\tilde{F}$ is a path homotopy.
\end{theorem}

\newpage

\begin{definition}
    Let $p: \left(E,e_0\right) \rightarrow \left(B,b_0\right)$ be a covering map.
    Given an element $[f]$ of $\pi_1\left(B, b_0\right)$, let
    $\bar{f}$ be the lifting of $f$ to a path in $E$ that begins at $e_0$. Let
    \begin{equation*}
        \phi([f]) =\tilde{f}(1)
    \end{equation*}
    Then $\phi$ is a well-defined set map
    \begin{equation*}
        \phi: \pi_1\left(B, b_0\right) \rightarrow p^{-1}\left(b_0\right)
    \end{equation*}
    We call $\phi$ the \textbf{lifting correspondence derived from the covering map $p$}.
\end{definition}





\section{}

\begin{definition}
    If $A \subset X$, a \textbf{retraction} of $X$ onto $A$ is a continuous map $r: X \rightarrow A$ such that $\left.r\right|_A$ is the identity map of $A$. If such a map $r$ exists, we say that $A$ is a \textbf{retract} of $X$.
\end{definition}

\begin{proposition}
    If $A$ is a retract of $X$, then the homomorphism of fundamental groups induced by inclusion $j: A \rightarrow X$ is injective.

    Proof:
    If $r: X \rightarrow A$ is a retraction, then the composite map $r \circ j$ equals the identity map of $A$. It follows that $r_* \circ j_*$ is the identity map of $\pi_1(A, a)$, so that $j_*$ must be injective.
\end{proposition}

\begin{theorem}[No-retraction theorem]
    There is no retraction of $B^2$ onto $S^1$.

    Proof:
    If $S^1$ were a retract of $B^2$, then the homomorphism induced by inclusion $j: S^1 \rightarrow B^2$ would be injective. But the fundamental group of $S^1$ is nontrivial and the fundamental group of $B^2$ is trivial.
\end{theorem}

\begin{lemma}
    Let $h: S^1 \rightarrow X$ be a continuous map Then the following conditions are equivalent:

    (1) $h$ is nulhomotopic.

    (2) $h$ extends to a continuous map $k: B^2 \rightarrow X$.

    (3) $h_*$ is the trivial homomorphism of fundamental groups.

    Proof:
    (1) $\Rightarrow$ (2). Let $H: S^1 \times I \rightarrow X$ be a homotopy between $h$ and a constant map. Let $\pi: S^1 \times I \rightarrow B^2$ be the map
    \begin{equation*}
        \pi(x, t)=(1-t) x
    \end{equation*}
    Then $\pi$ is continuous, closed and surjective, so it is a quotient map; it collapses $S^1 \times 1$ to the point 0 and is otherwise injective. Because $H$ is constant on $S^1 \times 1$, it induces, via the quotient map $\pi$, a continuous map $k: B^2 \rightarrow X$ that is an extension of $h$.

    (2) $\Rightarrow$ (3). If $j: S^1 \rightarrow B^2$ is the inclusion map, then $h$ equals the composite $k \circ j$. Hence $h_*=k_* \circ j_*$. But
    \begin{equation*}
        j_*: \pi_1\left(S^1, b_0\right) \rightarrow \pi_1\left(B^2, b_0\right)
    \end{equation*}
    is trivial because the fundamental group of $B^2$ is trivial. Therefore $h_*$ is trivial.
    (3) $\Rightarrow$ (1). Let $p: \mathbb{R} \rightarrow S^1$ be the standard covering map, and let $p_0: I \rightarrow S^1$ be its restriction to the unit interval. Then $\left[p_0\right]$ generates $\pi_1\left(S^1, b_0\right)$ because $p_0$ is a loop in $S^1$ whose lift to $\mathbb{R}$ begins at 0 and ends at 1 .

    Let $x_0=h\left(b_0\right)$. Because $h_*$ is trivial, the loop $f=h \circ p_0$ represents the identity element of $\pi_1\left(X, x_0\right)$. Therefore, there is a path homotopy $F$ in $X$ between $f$ and the constant path at $x_0$. The map $p_0 \times$ id $\cdot I \times I \rightarrow S^1 \times I$ is a quotient map, being continuous, closed, and surjective; it maps $0 \times t$ and $1 \times t$ to $b_0 \times t$ for each $t$, but is otherwise injective. The path homotopy $F$ maps $0 \times I$ and $1 \times I$ and $I \times 1$ to the point $x_0$ of $X$, so it induces a continuous map $H: S^1 \times I \rightarrow X$ that is a homotopy between $h$ and a constant map. See Figure 55.2.
\end{lemma}
























































































\end{document}