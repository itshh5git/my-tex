\chapter{The Structure of Rings}
\minitoc


\section{Simplicity and Primitivity} % Simplicity
\begin{definition}
    A ring $R$ is said to be \textbf{simple} if $R$ has no proper two-sided ideals.
\end{definition}

\begin{definition}
    A left module $M$ over a ring $R$ is said to be \textbf{simple} (or \textbf{irreducible}) if $M$ has no proper submodules.
    \begin{remark}
        A left ideal $\mathfrak{a}$ of $R$ is a simple left $R$-module if and only if $\mathfrak{a}$ is a minimal left ideal of $R$.
        In this case, we call $\mathfrak{a}$ the \textbf{simple left ideal} of $R$.
    \end{remark}
\end{definition}
\begin{proposition}
    \label{pro: Proposition of simple module}
    Let $R$ be a ring and $M$ be a simple $R$-module, then
    \begin{enumerate}
        \item
              $M=Rm$ for every $0 \neq m \in M$.
        \item
              If $0 \neq u \in M$, then $M \cong R / \Ann(u)$, thus $\Ann\left( u \right)$ is a left maximal ideal.

              Conversely, if $\mathfrak{m}$ is left maximal in $R$, then $R/\mathfrak{m}$ is a simple $R$-module with $\Ann\left( R/\mathfrak{m} \right)=\mathfrak{m}$
        \item
              If $R$ is not a division ring, then $M$ is a torsion module.$m\in M$.
    \end{enumerate}
\end{proposition}

\begin{lemma}[Schur]
    \label{lem: Schur's lemma}
    Let $M$ be a simple module over a ring $R$ and let $N_i$ be any $R$-module.
    \begin{enumerate}
        \item
              Every nonzero $R$-module homomorphism $f: M \rightarrow N_1$ is a monomorphism;
        \item
              Every nonzero $R$-module homomorphism $g: N_2 \rightarrow M$ is an epimorphism;

        \item
              The endomorphism ring $D=\operatorname{Hom}_{R}(M,M)$ is a division ring,
              then $M$ is a vector space over $\operatorname{Hom}_R(M,M)$ with $f a=f(a)$
    \end{enumerate}
\end{lemma}





\subsection{Primitivity} % Primitivity
\begin{definition}
    Let $R$ be a ring.
    \begin{enumerate}
        \item
              A ring $R$ is said to be \textbf{left [resp. right] primitive} if there exists a simple faithful left [resp. right] $R$-module.
        \item
              An ideal $\mathfrak{a}$ of a ring $R$ is said to be \textbf{left [resp. right] primitive} if the quotient ring $R / \mathfrak{a}$ is a left [resp. right] primitive ring.
    \end{enumerate}
    \begin{remark}
        If $M$ is a simple left $R$-module, then $R/\Ann\left( M \right)$ is left primitive with faithful simple left $R/\Ann\left( M \right)$-module $M$.
    \end{remark}
\end{definition}


\begin{proposition}
    Let $R$ be a ring.
    \begin{enumerate}
        \item
              A simple ring $R$ is primitive.

        \item
              A commutative ring $R$ is primitive if and only if $R$ is a field.
    \end{enumerate}

    \begin{proof}
        1.
        Since $R$ has an identity, $R$ contains a maximal left ideal $\mathfrak{m}$  by \ref{thm: existence of maximal ideal}, whence $R / \mathfrak{m}$ is a simple $R$-module.
        Since $\Ann(R / \mathfrak{m})$ is an ideal of $R$ that does not contain $1_R$, $\Ann(R / \mathfrak{m})=0$ by simplicity of $R$. Therefore $R /  \mathfrak{m}$ is a faithful $R$-module.

        2.
        Conversely, let $M$ be a faithful simple left $R$-module. Then $M \cong R / I$ for some maximal ideal $I$ of $R$. Therefore, $0=\Ann\left( M \right)=\Ann\left( R/I \right)\supset I$. Then $I=0$ is a maximal ideal of $R$, thus $R$ is a field.
    \end{proof}
\end{proposition}

\subsection{Jacobson Density Theorem} % Jacobson Density Theorem
\begin{definition}
    Let $V$ be a vector space over a division ring $D$.
    A subring $R$ of $\Hom_{D}(V, V)$ is called a \textbf{dense ring of endomorphisms} of $V$ if for every positive integer $n$, every linearly independent subset $\left\{u_1, \ldots, u_{n}\right\}$ of $V$ and every arbitrary subset $\left\{v_1, \ldots, v_{n}\right\}$ of $V$ , there exists $f \in R$ such that $f\left(u_{i}\right)=v_{i}, (i=1,2, \ldots, n)$.
\end{definition}


\begin{theorem}
    \label{thm: Dense ring of endomorphisms of a finite dimensional vector space}
    Let $R$ be a dense ring of endomorphisms of a vector space $V$ over a division ring $D$.
    Then $R$ is Artinian if and only if $\operatorname{dim}_{D} V$ is finite, in which case $R=\Hom_{D}(V, V)\cong M_n\left(D\right)$.

    \begin{proof}
        If $R$ is Artinian and $\operatorname{dim}_D V$ is infinite, then there exists an infinite linearly independent subset $\left\{u_1, u_2, \ldots\right\}$ of $V$. By  $V$ is a left $\mathrm{Hom}_D(V, V)$-module and hence a left $R$-module. For each $n$ let $I_n=\Ann\left\{u_1, \ldots, u_n\right\}$
        Then $I_1 \supset I_2 \supset \cdots$ is a descending chain of left ideals of $R$ and hence $I_1\supsetneq I_2\supsetneq\cdots$ is a properly descending chain, which is a contradiction. Hence $\dim_D V$ is finite.

        Conversely if $\operatorname{dim}_D V$ is finite, then $V$ has a finite basis $\left\{v_1, \ldots, v_m\right\}$. Then $R= \operatorname{Hom}_D(V, V)\cong M_n\left(D\right) $ is Artinian.
    \end{proof}
\end{theorem}


\begin{lemma}
    \label{lem: Lemma of Jacobson Density Theorem}
    Let $M$ be a simple module over a ring $R$. Consider $M$ as a vector space over the division ring $D=\operatorname{Hom}_{R}(M, M)$ by \ref{lem: Schur's lemma}.
    If $V$ is a finite dimensional $D$-subspace of $M$ and $a \in M-V$, then there exists $r \in R$ such that $ra \neq 0$ and $rV=0$.
    \begin{remark}
        In other words, the element $r\in \Ann_R\left( V \right)$ only annihilates $D$-subspace $V$.
    \end{remark}
    \begin{proof}
        The proof is by induction on $n=\operatorname{dim}_D V$. If $n=0$, then $V=0$ and $a \neq 0$. Since $M$ is simple, $M=R a$. Consequently, there exists $r \in R$ such that $r a=a \neq 0$ and $r V=r 0=0$.

        Suppose now $\operatorname{dim}_D V=n>0$ and the theorem is true for dimensions less than $n$.
        Let $\left\{u_1, \ldots, u_{n-1}, u\right\}$ be a $D$-basis of $V$ and let $W=\span\left\{u_1, \ldots, u_{n-1}\right\}$ ( $W=0$ if $n=1$ ).
        Then $V=W \oplus D u$ (vector space direct sum, $W$ may not be an $R$-submodule of $M$) the left annihilator $I=\Ann_R(W)$ is a left ideal of $R$.

        Consequently, $I u$ is an $R$-submodule of $M$. Since $u \in M-W$, the induction hypothesis implies that there exists $r \in R$ such that $r u \neq 0$ and $r W=0$. Consequently $r\in I$ and $0 \neq r u \in I u$, whence $I u \neq 0$.
        Therefore $M=I u$ by simplicity.

        We must find $r \in R$ such that $r a \neq 0$ and $r V=0$. If no such $r$ exists, $\Ann\left( a \right)\subset\Ann\left( V\right)$, then we can define a map
        $\theta: M \rightarrow M$
        as follows.
        For $r u \in I u=M$ let $\theta(r u)=r a \in M$.
        We claim that $\theta$ is well defined.
        If $r_1 u=r_2 u\left(r_i \in I\right)$, then $\left(r_1-r_2\right) u=0$, whence $\left(r_1-r_2\right) V$ $=\left(r_1-r_2\right)(W \oplus D u)=0$. Consequently by hypothesis $\left(r_1-r_2\right) a=0$.
        Therefore, $\theta\left(r_1 u\right)=r_1 a=r_2 a=\theta\left(r_2 u\right)$.
        Verify that $\theta \in \operatorname{Hom}_R(M, M)=D$.
        Then for every $r \in I$,
        \begin{equation*}
            0=\theta(r u)-r a=r \theta(u)-r a=r(\theta(u)-a)
        \end{equation*}
        Therefore $\theta(u)-a \in W$ by induction hypothesis.
        Consequently
        \begin{equation*}
            a=\theta u-(\theta u-a) \in D u+W=V,
        \end{equation*}
        which contradicts the fact that $a \not \in V$.
        Therefore, there exists $r \in R$ such that $r a \neq 0$ and $r V=0$.
    \end{proof}
\end{lemma}

\begin{theorem}[Classic Jacobson Density Theorem]
    \label{thm: Jacobson Density Theorem}
    Let $R$ be a primitive ring with a faithful simple $R$-module $M$ and division ring $D=\End_{R}(M)$.
    Then $R$ is a dense ring of endomorphisms of the $D$-vector space $M$ (viewed $\alpha :R \hookrightarrow \Hom_{R}\left( M,M \right)$ by $ r\mapsto \alpha_r$ where $\alpha_r: m\mapsto rm$ in $M$).
    \begin{remark}
        If $R$ is not primitive, then $R$ is not a subring of $\Hom_{R}\left( M,M \right)$.
        But $R/\Ann\left( M \right)$ is primitive with faithful simple left $R/\Ann\left( M \right)$-module $M$ with the action of $\bar{r}$ on $M$ which is same as that of $r$ on $M$, so we also can say that $R$ acts on simple $M$ densely i.e. for every positive integer $n$, every linearly independent subset $\left\{u_1, \ldots, u_n\right\}$ and every arbitrary subset $\left\{v_1, \ldots, v_n\right\}$, there exists $r \in R$ such that $r u_i=v_i, (i=1,2, \ldots, n)$.
    \end{remark}
    \begin{proof}
        It clear that $\alpha: R \rightarrow \operatorname{Hom}_D(M, M)$ is a ring monomorphism since $M$ is faithful.
        Let $\left\{u_1, u_2, \ldots, u_n\right\}$ be a $D$-linearly independent subset and $\left\{v_1, v_2, \ldots, v_n\right\}$ be an arbitrary subset.
        For each $i$ let
        \begin{equation*}
            V_i
            =
            \span\left\{u_1, \ldots, u_{i-1}, u_{i+1}, \ldots, u_n\right\}.
        \end{equation*}
        Since $U$ is $D$-linearly independent, $u_i \notin V_i$. Consequently, by \ref{lem: Lemma of Jacobson Density Theorem} there exists $r_i \in R$ such that
        \begin{center}
            $r_i u_i \neq 0$ and $r_i V_i=0$
        \end{center}
        whence $R r_i u_i=M$ by simplicity. Therefore exists $t_i \in R$ such that $t_i r_i u_i=v_i$. Let
        \begin{equation*}
            r=t_1 r_1+t_2 r_2+\cdots+t_n r_n \in R .
        \end{equation*}
        Consequently for each $i=1,2, \ldots, n$
        \begin{equation*}
            \alpha_r\left(u_i\right)=\left(t_1 r_1+\cdots+t_n r_n\right) u_i
            =
            v_i
        \end{equation*}
        Therefore $\operatorname{Im} \alpha$ is a dense ring of endomorphisms of the $D$-vector space $M$.
    \end{proof}
\end{theorem}



\section{Simple ring} % Simple ring
\begin{corollary}
    If $R$ is a nonzero simple ring, then
    \begin{equation*}
        R\cong \alpha(R) \subseteq \End_D(V)
    \end{equation*}
    for some faithful simple $R$-module $V$ and division ring $D=\End_R(V)$.
\end{corollary}










\begin{corollary}
    \label{cor: Structure of primitive rings}
    If $R$ is a primitive ring, then for some division ring $D$ either $R$ is isomorphic to the endomorphism ring of a finite dimensional vector space over $D$
    or
    for every positive integer $m$ there is a subring $R_{m}$ of $R$ and an epimorphism of rings $R_{m} \rightarrow \Hom_{D}\left(V_{m}, V_{m}\right)$, where $V_{m}$ is an $m$-dimensional vector space over $D$.

    \begin{proof}
        In the notation of \ref{thm: Jacobson Density Theorem},
        \begin{equation*}
            \alpha: R \rightarrow \operatorname{Hom}_D(M, M)
        \end{equation*}
        is a monomorphism such that $R\cong\operatorname{Im} \alpha$ and $\operatorname{Im} \alpha$ is dense in $\operatorname{Hom}_D(M, M)$.
        If $\operatorname{dim}_D M=n$ is finite, then $R\cong \operatorname{Im} \alpha=\operatorname{Hom}_D(M, M)$ by \ref{thm: Dense ring of endomorphisms of a finite dimensional vector space}.
        If $\operatorname{dim}_D A$ is infinite and $\left\{u_1, u_2, \ldots\right\}$ is an infinite linearly independent set, let $V_m$ be the $m$-dimensional $D$-subspace of $A$ spanned by $\left\{u_1, \ldots, u_m\right\}$.
        Verify that $R_m=\left\{r \in R \mid r V_m \subset V_m\right\}$ is a subring of $R$.
        Use the density of $R \cong \operatorname{Im} \alpha$ in $\operatorname{Hom}_D(M, M)$ to show that the map $R_m \rightarrow \operatorname{Hom}_D\left(V_m, V_m\right)$ given by $r \mapsto \left. \alpha_r \right|_{V_m}$ is a well-defined ring epimorphism.
    \end{proof}
\end{corollary}




\begin{theorem}[Wedderburn-Artin]
    \label{thm: Wedderburn-Artin theorem}
    The following conditions on a ring $R$ are equivalent.
    \begin{enumerate}
        \item
              $R$ is simple Artinian.

        \item
              $R$ is primitive Artinian.

        \item
              $R$ is isomorphic to $M_n\left(D\right)$ for some positive integer $n$ and some division ring $D$.
    \end{enumerate}
    In this case, $D$ is isomorphic to $\operatorname{Hom}_R(M, M)$ for any simple left $R$-module $M$ and $n=\operatorname{dim}_D M$.
    \begin{proof}
        (1) $\Rightarrow$ (2)
        This is clear since a simple ring is primitive.

        (2) $\Rightarrow$ (3)
        Let $M$ be a faithful simple left $R$-module.
        By \cref{thm: Jacobson Density Theorem}, $R$ is isomorphic to a dense ring of endomorphisms of the $D$-vector space $M$, where $D=\operatorname{Hom}_R(M, M)$.
        Since $R$ is left Artinian, $\operatorname{dim}_D M$ is finite by \cref{thm: Dense ring of endomorphisms of a finite dimensional vector space}.
        Therefore $R \cong \operatorname{Hom}_D(M, M) \cong M_n\left(D\right)$, where $n=\operatorname{dim}_D M$.

        (3) $\Rightarrow$ (1)
        Since $M_n\left(D\right)$ is left Artinian, it suffices to show that $M_n\left(D\right)$ is simple.
        Let $\mathfrak{a}$ be a nonzero two-sided ideal of $M_n\left(D\right)$ and let $0 \neq A=\left(a_{i j}\right) \in \mathfrak{a}$.
        Then there exist indices $p, q$ such that $a_{p q} \neq 0$.
        For any indices $i, j$, let $E_{i j}$ be the matrix unit whose $(i, j)$-entry is 1 and all other entries are 0.
        Then
        \begin{equation*}
            E_{i p} A E_{q j}=a
            _{p q} E_{i j} \in \mathfrak{a}
        \end{equation*}
    \end{proof}
\end{theorem}

\begin{lemma}
    \label{lem: }
    Let $V$ be a nonzero vector space over a division ring $D$.
    If $g: V \rightarrow V$ is a homomorphism of additive groups such that $gf=fg$ for all $f \in \Hom_{D}\left( V,V \right)$, then there exists $\lambda \in D$ such that $g(x)=\lambda x$ for all $x\in V$.
\end{lemma}

\begin{lemma}
    Let $V$ be a finite dimensional vector space over a division ring $D$.
    If $M$ and $N$ are simple faithful modules over $R=\operatorname{Hom}_{D}(V, V)$, then $M$ and $N$ are isomorphic $R$-modules.
    \begin{proof}
        Since $M$ and $N$ are simple and faithful, we have $\Ann(M)=0$ and $\Ann(N)=0$.
        By \cref{lem: Schur's lemma}, we have $\Hom_{R}(M, N) \cong \Hom_{D}(V, V)$, which is a division ring. Thus $M$ and $N$ are isomorphic as $R$-modules.
    \end{proof}
\end{lemma}

\begin{proposition}
    For $i=1,2$ let $V_{i}$ be a vector space of finite dimension $n_{i}$ over the division ring $D_{i}$.
    \begin{enumerate}
        \item
              If there is an isomorphism of rings $\operatorname{Hom}_{D_1}\left(V_1, V_1\right) \cong \operatorname{Hom}_{D_2}\left(V_2, V_2\right)$, then $\operatorname{dim}_{D_1} V_1=\operatorname{dim}_{D_2} V_2$ and $D_1$ is isomorphic to $D_2$.
        \item
              If there is an isomorphism of rings $M_{n_1} \left(D_1\right) \cong M_{n_2} \left(D_2\right)$, then $n_1=n_2$ and $D_1$ is isomorphic to $D_2$.
    \end{enumerate}
\end{proposition}






\chapter{Semisimplicity} % Semirings
\section{Definitions}

\begin{theorem}
    \label{thm: Semisimple modules}
    Let $R$ be a ring and $M$ a left $R$-module.
    The following conditions on $M$ are equivalent:
    \begin{enumerate}
        \item
              $M$ is the sum of a family of simple submodules.
        \item

              $M$ is the direct sum of a family of simple submodules.
        \item
              Every submodule $N$ is a direct summand of $M$.
    \end{enumerate}
    \begin{proof}
        (1) $\Rightarrow$ (2)
        Let $\mathcal{S}$ be the set of all families $\mathcal{F}$ of simple submodules of $M$ such that the sum of the members of $\mathcal{F}$ is direct.
        Since $M$ is the sum of a family of simple submodules, $\mathcal{S}$ is nonempty.
        Partially order $\mathcal{S}$ by inclusion and let $\mathcal{C}$ be a chain in $\mathcal{S}$.
        Then $\mathcal{U}=\bigcup_{\mathcal{F} \in \mathcal{C}} \mathcal{F}$ is an upper bound for $\mathcal{C}$ in $\mathcal{S}$.
        By Zorn's lemma there exists a maximal element $\mathcal{F}_0$ in $\mathcal{S}$.
        We claim that $M=\bigoplus_{N \in \mathcal{F}_0} N$.
        If not, there exists a simple submodule $K$ of $M$ such that
        \begin{equation*}
            K \cap\left(\bigoplus_{N \in \mathcal{F}_0} N\right)=0 .
        \end{equation*}
        Consequently, $\mathcal{F}_0 \cup\{K\} \in S$, contradicting the maximality of $\mathcal{F}_0$.

        (2) $\Rightarrow$ (3)
        Let $M=\bigoplus_{i \in I} N_i$, where each $N_i$ is a simple submodule of $M$, and let $N$ be a submodule of $M$.
        For each $i \in I$, either $N_i \subset N$ or $N_i \cap N=0$ by simplicity.
        Let $J=\left\{i \in I \mid N_i \subset N\right\}$ and $K=I-J$. Then
        \begin{equation*}
            M=N \oplus\left(\bigoplus_{i \in K} N_i\right) .
        \end{equation*}

        (3) $\Rightarrow$ (1)
        Let $N$ be the sum of all simple submodules of $M$. By hypothesis, $M=N \oplus P$ for some submodule
    \end{proof}

    \begin{remark}
        A left $R$-module $M$ satisfying the three conditions is said to be \textbf{semisimple}.
        Similarly one defines a right semisimple module.
    \end{remark}
\end{theorem}


\begin{proposition}
    Every submodule and every factor module of a left semisimple module is left semisimple.
\end{proposition}


\section{Structure of semisimple rings} % Structure of semisimple rings

\begin{definition}
    A ring $R$ is called \textbf{semisimple} if $1\neq 0$, and if $R$ is semisimple as a left $R$-module.
\end{definition}




\begin{proposition}
    If $R$ is a semisimple ring, then $R$ is Artinian.
    \begin{proof}
        Since ${}_R R $ is semisimple module, there exist simple left ideals $L_j$ of $R$ such that
        \begin{equation*}
            {}_R R
            \cong
            \bigoplus_{j \in J} L_j
        \end{equation*}
        with projection maps $\pi_j: R \rightarrow L_j$ for each $j \in J$.
        Since the sum is direct, $\left\{j_k: \pi_{j_k}(1_R) \neq 0\right\}$ is finite.
        Then
        $J=\left\{j_k: \pi_{j_k}(1_R) \neq 0\right\}$ is finite and
        \begin{equation*}
            {}_R R
            \cong
            L_{1} \oplus L_{2} \oplus \cdots \oplus L_{n}
        \end{equation*}
        Therefore ring $R$ is Artinian.
    \end{proof}
    \begin{remark}
        It is clear that any simple ideal $L$ of $R$ is isomorphic to some $L_j$, thus $R$ has finitely many minimal left ideals up to isomorphism.
    \end{remark}
\end{proposition}



\begin{lemma}
    \label{lem:lemma of Wedderburn}
    If $L$ and $L^{\prime}$ are minimal left ideals in a ring $R$, then each of the following statements implies the one below it:
    \begin{enumerate}
        \item
              $L L^{\prime} \neq(0)$.
        \item
              $\operatorname{Hom}_R\left(L, L^{\prime}\right) \neq\{0\}$ and there exists $b^{\prime} \in L^{\prime}$ with $L^{\prime}=L b^{\prime}$.
        \item
              $L \cong L^{\prime}$ as left $R$-modules.
    \end{enumerate}
    If also $L^2 \neq(0)$, then (iii) implies (i) and the three statements are equivalent.
\end{lemma}
\begin{theorem}[Wedderburn-Artin]
    If $R$ is a semisimple ring, then we have ring isomorphisms
    \begin{equation*}
        R
        \cong
        \prod_{i=1}^t  R_i
        \cong \prod_{i=1}^t M_{n_i}\left(D_i\right)
    \end{equation*}
    where $R_i\cong M_{n_i}\left(D_i\right)$ is a simple ring for each $i \in I$.
    \begin{proof}
        Let $\left\{L_{j_i}\right\}_{i=1}^t$ be all distinct simple left ideals of $R$ up to isomorphism and define $R_i$ be the sum
        \begin{equation*}
            {}_R R_i
            :=
            \bigoplus_{L_j \cong L_i} L_j
        \end{equation*}
        Thus, as left $R$-modules,
        \[
            {}_R R \cong \bigoplus_{i=1}^t {}_R R_i.
        \]
        Then $R_i$ is a two-sided ideal of $R$ by \cref{lem:lemma of Wedderburn} and hence a ring with identity element $e_i=\pi_i(1_R)$ ($\pi_i$ the projection from $R$ to $R_i$).
        The projection from $R$ to $R_i$ induce a ring homomorphism
        \begin{equation*}
            R\cong \prod_{i=1}^t R_i
        \end{equation*}


        It remains to show that each $R_i$ is simple.
        Let $I$ be a nonzero two-sided ideal of $R_i$, then $I$ is a left ideal of $R$.
        Since $R$ is semisimple, there exists a left ideal $J$ of $R$ such that
        \begin{equation*}
            R=I \oplus J
        \end{equation*}
        Then
        \begin{equation*}
            R_i=e_i R=e_i I+e_i J \subset I
        \end{equation*}
        Thus $R_i$ is simple.
    \end{proof}
\end{theorem}

\section{Structure of modules over semisimple rings} % Structure of modules over semisimple rings

\subsection{Modules over $M_n(D)$}
\begin{proposition}
    Let $R=M_n(D)$ for some division ring $D$ and positive integer $n$.
    \begin{enumerate}
        \item
              If $S$ is a left $R$-module, then
              \begin{equation*}
                  {}_R S   \cong {}_R D^n
              \end{equation*}
        \item
              every left $R$-module is isomorphic to$\left({}_R D^n\right)^{(I)}$ for some index set $I$.
    \end{enumerate}
\end{proposition}

\begin{corollary}
    There is, up to isomorphism, only one simple left module over a Artinian simple ring $R$; namely, the left $R$-module $R e$, where $e$ is any primitive idempotent of $R$.
\end{corollary}

\subsection{Modules over semisimple rings}

\begin{theorem}
    Let semisimple ring
    \begin{equation*}
        R\cong \prod_{i=1}^t R_i
    \end{equation*}
    where each $R_i\cong M_{n_i}(D_i)$.
    Then every left $R$-module $M$ is isomorphic to
    \begin{equation*}
        \bigoplus_{i=1}^t e_iM
        \cong
        \bigoplus_{i=1}^t S_i^{\left(\kappa_i\right)}
    \end{equation*}
    where $e_iM\cong S_i$ is the unique simple left $R_i$-module and $\kappa_i$ is a cardinal number.
    And the decomposition is unique in the sense that if
    \begin{equation*}
        M \cong \bigoplus_{i=1}^t S_i^{\left(\kappa_i\right)} \cong \bigoplus_{i=1}^t S_i^{\left(\kappa_i^{\prime}\right)}
    \end{equation*}
    then $\kappa_i=\kappa_i^{\prime}$ for each $i$
    \begin{remark}
        If $M$ is finitely generated, then each $\kappa_i$ is finite.
    \end{remark}
\end{theorem}


\section{Jacobson Radical} % Jacobson Radical

\begin{definition}
    Let $R$ be a ring. A element $x \in R$ is said to be \textbf{right quasi-regular} if there exists $y \in R$ such that $x+y-xy=0$, $y$ is called a \textbf{right quasi-inverse} of $x$.
    \begin{remark}
        That is, $1-x$ has a right inverse $1-y$.
    \end{remark}

\end{definition}
\begin{definition}
    A element $a\in R$ is said \textbf{right quasi-nilpotent element} if for every $r \in R$, $r a$ is right quasi-regular.
    \begin{remark}
        That is, $1-ra$ has a right inverse for every $r \in R$.
    \end{remark}
\end{definition}

\subsection{Jacobson radical}
\begin{theorem}
    \label{thm: Jacobson radical}
    Let $R$ be a ring, then the following sets are equal:
    \begin{enumerate}
        \item
              the intersection of all maximal left ideals of $R$
        \item [1']
              the intersection of all maximal right ideals of $R$

        \item
              the intersection of all the annihilators of simple left $R$-modules;
        \item [2']
              the intersection of all the annihilators of simple right $R$-modules;
        \item
              $\left\{x\in R: x \text{ is right quasi-nilpotent element}\right\}$
        \item [3']
              $\left\{x\in R: x \text{ is left quasi-nilpotent element}\right\}$
        \item
              Nakayama's
              $\left\{x\in R: \text{ for any finitely generated module } M, xM=M\Rightarrow M=0\right\}$
        \item
              the ideal $J$ such that $R/J$ is semisimple.
    \end{enumerate}
    the two-sided ideal is called the \textbf{Jacobson radical} of $R$ and is denoted by $J(R)$.
\end{theorem}


\begin{corollary}
    \begin{enumerate}
        \item
              $J\left(\prod_{i \in I} R_i\right)=\prod_{i \in I} J\left(R_i\right)$.
        \item
              $J\left(J\left(R\right)\right)=J\left(R\right)$.
        \item
              $R/J(R)$ is semisimple.
    \end{enumerate}
\end{corollary}

\begin{proposition}
    If $R$ is a Artinian ring, then the radical $J(R)$ is a nilpotent ideal.
    Thus $J(R)=\Rad\left( R \right)$
    \begin{proof}
        Since $R$ is Artinian, the descending chain of ideals
        \begin{equation*}
            J(R) \supseteq J(R)^2 \supseteq J(R)^3 \supseteq \cdots
        \end{equation*}
        stabilizes, say $J(R)^n=J(R)^{n+1}$ for some positive integer $n$.
        On the other hand, ${}_R R$ is Noetherian thus $J(R)^n$ is finitely gengerated submodule of ${}_R R$ and $J(R)^n =0$ by Nakayama's lemma.
    \end{proof}
\end{proposition}




\begin{proposition}
    Let $R$ be a ring. $J\left(M_n R\right)=M_n \left(J(R)\right)$.
    \begin{proof}
        (a) If $A$ is a left $R$-module, consider the elements of $A^n=A \oplus A \oplus \cdots \oplus A$ ( $n$ summands) as column vectors; then $A^n$ is a left $\left(\operatorname{Mat}_n R\right)$-module (under ordinary matrix multiplication).

        (b) If $A$ is a simple $R$-module, $A^n$ is a simple $\left(\operatorname{Mat}_n R\right)$-module.

        (c) $J\left(\operatorname{Mat}_n R\right) \subset \operatorname{Mat}_n J(R)$.

        (d) $\operatorname{Mat}_n J(R) \subset J\left(\operatorname{Mat}_n R\right)$. [Hint: prove that $\operatorname{Mat}_n J(R)$ is a left quasi-regular ideal of $\operatorname{Mat}_n R$ as follows. For each $k=1,2, \ldots, n$ let $K_k$ consist of all matrices $\left(a_{i j}\right)$ such that $a_{i j} \varepsilon J(R)$ and $a_{i j}=0$ if $j \neq k$. Show that $K_k$ is a left quasi-regular left ideal of $\operatorname{Mat}_n R$ and observe that $K_1+K_2+\cdots+K_n=\operatorname{Mat}_n J(R)$. $]$
    \end{proof}
\end{proposition}


\begin{corollary}
    Let $R$ be a ring. If matrix $A\in M_n\left(J( R)\right)$, then $I-A$ is invertible in $M_n(R)$.
\end{corollary}


\subsection{Nakayama's Lemma} % Noetherian Rings and Modules


\begin{theorem}[Nakayama's lemma]
    \label{pro: Nakayama's lemma}
    Let $M$ be a finitely generated left $R$-module and a subset $X \subset J(R)$.
    Then $\mathfrak{a} M=M$ implies $M=0$.
\end{theorem}


\begin{corollary}
    Let $M$ be a finitely generated left $R$-module, $N$ a submodule of $M$, left ideal $\mathfrak{a} \subset J(R)$.
    Then $M=\mathfrak{a} M+N \Rightarrow M=N$.
\end{corollary}




\begin{corollary}
    If $R$ is a Noetherian local ring with maximal ideal $\mathfrak{m}$, then $\bigcap_{n=1}^{\infty} \mathfrak{m}^n=0$.
\end{corollary}

\begin{proposition}
    If $R$ is a local ring, then every finitely generated projective $R$-module is free.
    \begin{proof}
        If $P$ is a finitely generated projective $R$-module, then by \ref{cor: Every left module $M$ over a ring $R$ is the homomorphic image of a free $R$-module $F$.}
        there exists a free $R$-module $F$ with a finite basis and an epimorphism $\pi: F \rightarrow P$. Among all the free $R$-modules $F$ with this property choose one with a basis $\left\{x_1, x_2, \ldots, x_n\right\}$ that has a minimal number of elements. Since $\pi$ is an epimorphism $\left\{\pi\left(x_1\right), \ldots, \pi\left(x_n\right)\right\}$ necessarily generate $P$.

        We shall first show that $K=\operatorname{Ker} \pi$ is contained in $\mathfrak{m} F$, where $\mathfrak{m}$ is the unique maximal ideal of $R$.

        If $K \not \subset \mathfrak{m} F$, then there exists $k \in K$ with $k \notin \mathfrak{m} F$. Now $k=r_1 x_1+r_2 x_2+\cdots+r_n x_n$ with $r_i \in R$ uniquely determined. Since $k \notin \mathfrak{m} F$, some $r_i$, say $r_1$, is not an element of $\mathfrak{m}$, thus $r_1$ is a unit, whence $x_1-r_1^{-1} k=-r_1^{-1} r_2 x_2-\cdots-r_1^{-1} r_n x_n$.

        Consequently, since $k \in \operatorname{Ker} \pi, \pi\left(x_1\right)
            =
            \pi\left(x_1-r_1^{-1} k\right)
            =
            \sum_{i=2}^n-r_1 r_i \pi\left(x_i\right)$. Therefore, $\left\{\pi\left(x_2\right), \ldots, \pi\left(x_n\right)\right\}$ generates $P$. Thus if $F^{\prime}$ is the free submodule of $F$ with basis $\left\{x_2, \ldots, x_n\right\}$ and $\pi^{\prime}: F^{\prime} \rightarrow P$ the restriction of $\pi$ to $F^{\prime}$, then $\pi^{\prime}$ is an epimorphism. This contradicts the choice of $F$ as having a basis of minimal cardinality. Hence $K \subset \mathfrak{m} F$.

        Since $0 \rightarrow K \xrightarrow{\subset} F \xrightarrow{\pi} P \rightarrow 0$ is exact and $P$ is projective $K \oplus P \cong F$ by \ref{thm: Equivalent conditions of projective}.
        Under this isomorphism $(k, 0) \mapsto k$ for all $k \in K$ (see the proof of Theorem IV.1.18), whence $F$ is the internal direct sum $F=K \oplus P^{\prime}$ with $P^{\prime} \cong P$. Thus $F=K+P^{\prime} \subset \mathfrak{m} F+P^{\prime}$. If $u \in F$, then $u=\sum_i m_i v_i+p_i$ with $m_i \in \mathfrak{m}, v_i \in F, p_i \in P^{\prime}$. Consequently, in the $R$-module $F / P^{\prime}$,
        \begin{equation*}
            u+P^{\prime}=\sum_i m_i v_i+P^{\prime}=\sum_i m_i\left(v_i+P^{\prime}\right) \in \mathfrak{m}\left(F / P^{\prime}\right)
        \end{equation*}
        whence $\mathfrak{m}\left(F / P^{\prime}\right)=F / P^{\prime}$. Since $F$ is finitely generated, so is $F / P^{\prime}$. Therefore $K \cong F / P^{\prime}=0$ by \ref{lem: Nakayama lemma}.
        Thus $P \cong P^{\prime}=F$ and $P$ is free.
    \end{proof}
\end{proposition}





\section{Characterizations of semisimple rings}
\begin{theorem}
    The following conditions on a ring $R$ are equivalent:
    \begin{enumerate}
        \item
              $R$ is a semisimple ring.
        \item
              Every left $R$-module is a semisimple module.
        \item
              Every left $R$-module is injective.
        \item
              Every left $R$-module is projective.
        \item
              Every short exact sequence of left $R$-modules splits.
        \item
              $R$ is Artinian and $J(R)=0$.
    \end{enumerate}
\end{theorem}















\section{Algebra}
\begin{definition}
    Let $A$ be an algebra over a commutatuve ring $K$ with identity.
    \begin{enumerate}
        \item
              A \textbf{left algebra $A$-module} is a left $K$-module $M$ such that $M$ is a left module over the ring A and $k(am)=(ka) m=a(km )$ for all $k \in K, a \in A, m \in M$.
              Indeed,
              \begin{equation*}
                  \begin{cases*}
                      \left(k_1a_1+k_2a_2\right)\left(m_1+m_2\right)
                      =
                      k_1a_1m_1+k_1a_1m_2+k_2a_2m_1+k_2a_2m_2 \\
                      k(am)=(ka) m=a(km)                      \\
                      1_K m=m ,1_K a=a
                  \end{cases*}
              \end{equation*}
              for all $k \in K, a \in A, m \in M$

        \item
              A left algebra $A$-submodule of $M$ is a subset of $M$ which is itself an left algebra $A$-module.

        \item
              A left algebra $A$-module $M$ is \textbf{simple} (or \textbf{irreducible}) if $M$ has no proper $A$-submodules.

        \item
              A homomorphism $f: M \rightarrow N$ of algebra $A$-modules is a map that is both a $K$-module and an $A$-module homomorphism.
    \end{enumerate}
    \begin{remark}

    \end{remark}
\end{definition}

\begin{theorem}
    Let $A$ be a $K$-algebra.
    The Jacobson radical of the ring $A$ coincides with the Jacobson radical of the algebra $A$.
    In particular $A$ is a semisimple ring if and only if $A$ is a semisimple algebra.
\end{theorem}



\begin{theorem}
    Let $A$ be a $K$-algebra.

    (1)
    Every simple algebra $A$-module is a simple module over the ring $A$.

    (2)
    Every simple module $M$ over the ring $A$ can be given a unique $K$-module structure in such a way that $M$ is a simple algebra $A$-module.
\end{theorem}






