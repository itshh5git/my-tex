\chapter{}

\section{Ideals quotient} % Ideals quotient
\begin{definition}
    If $\mathfrak{a}, \mathfrak{b}$ are ideals in a commutative ring $R$, their \textbf{ideal quotient} is
    \begin{equation*}
        (\mathfrak{a}: \mathfrak{b})
        =
        \{x \in R: x \mathfrak{b} \subseteq \mathfrak{a}\}
    \end{equation*}
    which is an ideal.
    If $\mathfrak{b}$ is a principal ideal $(x)$, we shall write $(\mathfrak{a}: x)$ in place of $(\mathfrak{a}:(x))$.
\end{definition}

\begin{proposition}
    Let $R$ be a commutative ring. Then
    \begin{enumerate}
        \item
              $\mathfrak{a} \subseteq(\mathfrak{a}: \mathfrak{b})$

        \item
              $(\mathfrak{a}: \mathfrak{b}) \mathfrak{b} \subseteq \mathfrak{a}$

        \item
              $((\mathfrak{a}: \mathfrak{b}): \mathfrak{c})=(\mathfrak{a}: \mathfrak{b} \mathfrak{c})=((\mathfrak{a}: \mathfrak{c}): \mathfrak{b})$

        \item
              $\left(\bigcap_i \mathfrak{a}_i: \mathfrak{b}\right)=\bigcap_i\left(\mathfrak{a}_i: \mathfrak{b}\right)$

        \item
              $\left(\mathfrak{a}: \sum_i \mathfrak{b}_i\right)=\bigcap_i\left(\mathfrak{a}: \mathfrak{b}_i\right)$.
    \end{enumerate}
\end{proposition}



\section{Primary Decomposition} % Primary Decomposition
Throught this section, $R$ be a commutative ring with identity
\subsection{Primary (Submodule) Ideals} % Primary Decomposition
\begin{definition}
    Let $A$ be a commutative ring with identity and $M$ an $A$-module.
    \begin{enumerate}
        \item An ideal $\mathfrak{q}$ in $A$ is \textbf{primary} if $\mathfrak{q} \neq A$ and if
              \begin{equation*}
                  x y \in \mathfrak{q} ,x\notin \mathfrak{q}\Rightarrow  y^n \in \mathfrak{q} \text { for some } n>0 \text {. }
              \end{equation*}
              In other words,
              $\mathfrak{q}$ is primary $\Leftrightarrow R / \mathfrak{q} \neq 0$ and every zero-divisor in $R / \mathfrak{q}$ is nilpotent.
        \item
              A submodule $Q$ of $M$ is primary if $Q\neq M$ and if
              \begin{center}
                  $r \in R, m \in M-Q$ and $rm \in Q \Rightarrow r^n M \subset Q$ for some positive integer $n$
              \end{center}
              It is equivalent that
              \begin{itemize}
                  \item $\left(Q:M\right)=\Ann\left(M/Q\right)$ is a primary ideal in $R$
                  \item principal homomorphism $a_{M/Q}$ is injective or nilpotent for each $a \in R$
              \end{itemize}
    \end{enumerate}
    \begin{remark}
        If we view $A$ as itself $A$-module, the two definition are equivalent for $A$.
    \end{remark}
\end{definition}


\begin{proposition}
    Let $A$ be a commutative ring and $M$ an $A$-module.
    \begin{enumerate}
        \item
              If $\mathfrak{q}$ is a primary ideal in $A$, ideal $\mathfrak{p}=\Rad(\mathfrak{q})$ is a prime ideal containing $\mathfrak{q}$.
              The radical $\mathfrak{p}$ is called the \textbf{associated prime ideal of $\mathfrak{q}$} or that $\mathfrak{q}$ is \textbf{$\mathfrak{p}$-primary}.

        \item
              If $N$ is a primary submodule of $M$, $\left(N:M\right)=
                  \left\{r\in A \mid rM\subset N\right\}$
              is a primary ideal in $A$.
              Thus $\mathfrak{p}
                  =
                  \Rad\left(N:M\right)
                  =
                  \left\{r \in A \mid r^n M \subset N \text{ for some } n>0\right\}$ is a prime ideal in $A$ .
              The primary submodule $N$ of a module $M$ is said to \textbf{belong to a prime ideal $\mathfrak{p}$} or to be a \textbf{$\mathfrak{p}$-primary submodule} of $M$.
    \end{enumerate}
\end{proposition}


\begin{theorem}
    \label{thm: Equivalent condition about associated prime ideal of primary ideals}
    Let $A$ be a commutative ring, $\mathfrak{q}$ and $\mathfrak{p}$ be ideals in $A$.
    Then $\mathfrak{q}$ is primary for $\mathfrak{p}$ if and only if
    \begin{enumerate}[label=(\roman*)]
        \item
              $\mathfrak{q} \subset \mathfrak{p}\subset \Rad \left(\mathfrak{q}\right)$
        \item
              if $ab \in \mathfrak{q}$ and $a \notin \mathfrak{q}$, then $b \in \mathfrak{p}$.
    \end{enumerate}
    \begin{proof}
        Suppose (i) and (ii) hold.
        If $a b \in \mathfrak{q}$ with $a \not\in \mathfrak{q}$, then $b \in \mathfrak{p} \subset \operatorname{Rad} \mathfrak{q}$, whence $b^n \in \mathfrak{q}$ for some $n>0$. Therefore $\mathfrak{q}$ is primary.

        To show that $\mathfrak{q}$ is primary for $\mathfrak{p}$ we need only show $\mathfrak{p}=\operatorname{Rad} \mathfrak{q}$.
        By (i), $\mathfrak{p} \subset \operatorname{Rad} \mathfrak{q}$. If $b \in \operatorname{Rad} \mathfrak{q}$, let $n$ be the least integer such that $b^n \in \mathfrak{q}$. If $n=1, b \in \mathfrak{q} \subset \mathfrak{p}$. If $n>1$, then $b^{n-1} b=b^n \in \mathfrak{q}$, with $b^{n-1} \notin \mathfrak{q}$ by the minimality of $n$.
        By (ii), $b \in \mathfrak{p}$. Thus $b \in \operatorname{Rad} \mathfrak{q}$ implies $b \in \mathfrak{p}$, whence $\operatorname{Rad} \mathfrak{q} \subset \mathfrak{p}$.
    \end{proof}
\end{theorem}


\begin{corollary}
    \label{cor: intersection of primary ideal}
    Let $A$ be a commutative ring with identity, if $\mathfrak{q}_1, \mathfrak{q}_2, \ldots, \mathfrak{q}_{n}$ are $\mathfrak{p}$-primary, then $\bigcap_{i=1}^n \mathfrak{q}_i$ is also $\mathfrak{p}$-primary.
    \begin{proof}
        Let $\mathfrak{q}=\bigcap_{i=1}^n \mathfrak{q}_i$.
        Then by \ref{pro: operations of radical}, $\operatorname{Rad} \mathfrak{q}=\bigcap_{i=1}^n \operatorname{Rad} \mathfrak{q}_i$ $=\bigcap_{i=1}^n \mathfrak{p}=\mathfrak{p}$; in particular, $\mathfrak{q} \subset \mathfrak{p} \subset \operatorname{Rad} \mathfrak{q}$.
        If $a b \in \mathfrak{q}$ and $a \notin \mathfrak{q}$, then $a b \in \mathfrak{q}_i$ and $a \not \in \mathfrak{q}_i$ for some $i$. Since $\mathfrak{q}_i$ is $\mathfrak{p}$-primary, $b \in \mathfrak{p}$ by \ref{thm: Equivalent condition about associated prime ideal of primary ideals}.
        Consequently, $\mathfrak{q}$ itself is $\mathfrak{p}$-primary by \ref{thm: Equivalent condition about associated prime ideal of primary ideals}.
    \end{proof}
\end{corollary}

\begin{proposition}
    Clearly every prime ideal is primary. Also the contraction of a primary ideal is primary, for if $f: A \rightarrow B$ and if $\mathfrak{q}$ is a primary ideal in $B$, then $A / \mathfrak{q}^c$ is isomorphic to a subring of $B / q$.
\end{proposition}



\begin{proposition}
    If $\Rad(\mathfrak{a})$ is maximal, then $\mathfrak{a}$ is primary. In particular, the powers of a maximal ideal $\mathfrak{m}$ are $\mathfrak{m}$-primary.
    \begin{proof}
        Let $\Rad(\mathfrak{a})=\mathfrak{m}$. The image of $\mathfrak{m}$ in $A / \mathfrak{a}$ is the nilradical of $A / \mathfrak{a}$, hence $A / \mathfrak{a}$ has only one prime ideal $\pi(\mathfrak{m})$, by (1.8). Hence every element of $A / \mathfrak{a}$ is either a unit or nilpotent, and so every zero-divisor in $A / a$ is nilpotent.
    \end{proof}
\end{proposition}

\subsection{Primary Decomposition} % Primary Decomposition
\begin{definition}
    Let $R$ be a commutative ring with identity and $M$ an unitary $R$-module.
    \begin{enumerate}
        \item
              An ideal $\mathfrak{a}$ of $R$ has a \textbf{primary decomposition} if
              \begin{enumerate}[label=(\roman*)]
                  \item
                        $\mathfrak{a}=\mathfrak{q}_1 \cap \mathfrak{q}_2 \cap \cdots \cap \mathfrak{q}_n$ with each $\mathfrak{q}_i$ primary
              \end{enumerate}
              then the primary decomposition is said to be \textbf{reduced (or irredundant)} if
              \begin{enumerate}[label=(\roman*),resume]
                  \item
                        no $\mathfrak{q}_i$ contains $\mathfrak{q}_1 \cap \cdots \cap \mathfrak{q}_{i-1} \cap \mathfrak{q}_{i+1} \cap \cdots \cap \mathfrak{q}_{n}$ and the $\mathfrak{p}_i=\Rad\mathfrak{q}_i$ are all distinct,
              \end{enumerate}

        \item
              A submodule $N$ of $N$ has a \textbf{primary decomposition} if
              \begin{enumerate}[label=(\roman*)]
                  \item
                        $N
                            =
                            Q_1 \cap Q_2 \cap \cdots \cap Q_n$, with each $Q_i$ a $\mathfrak{p}_i$-primary submodule of $N$ for some prime ideal $\mathfrak{p}_i$ of $R$.
              \end{enumerate}
              then the primary decomposition is said to be \textbf{reduced}. if
              \begin{enumerate}[label=(\roman*),resume]
                  \item
                        no $Q_i$ contains $Q_1 \cap \ldots \cap Q_{i-1} \cap Q_{i+1} \cap \ldots \cap Q_n$ and the ideals $\mathfrak{p}_i, \ldots, \mathfrak{p}_i$ are all distinct,
              \end{enumerate}
    \end{enumerate}
    If $\mathfrak{p}_j \not \subset \mathfrak{p}_i$ for all $j \neq i$, then $\mathfrak{p}_i$ is said to be an \textbf{isolated prime} ideal of $N$.
    In other words, $\mathfrak{p}_i$ is isolated if it is minimal in the set $\left\{\mathfrak{p}_1, \ldots, \mathfrak{p}_n\right\}$. If $\mathfrak{p}_i$ is not isolated it is said to be \textbf{embedded}.

\end{definition}


\begin{theorem}
    \label{thm: Existence of reduced primary decomposition}
    Let $R$ be a commutative ring with identity and $M$ a unitary module.
    \begin{enumerate}
        \item
              If an ideal $\mathfrak{a}$ of $R$ has a primary decomposition, then $\mathfrak{a}$ has a reduced primary decomposition.

        \item
              If a submodule $N$ has a primary decomposition, then $N$ has a reduced primary decomposition.
    \end{enumerate}
    \begin{proof}
        1.
        If $\mathfrak{a}=\mathfrak{q}_1 \cap \ldots \cap \mathfrak{q}_n\left(\mathfrak{q}_i\right.$ primary $)$ and some $\mathfrak{q}_i$ contains $\mathfrak{q}_1 \cap \ldots \cap$ $\mathfrak{q}_{i-1} \cap \mathfrak{q}_{i+1} \cap \cdots \cap \mathfrak{q}_n$, then $\mathfrak{a}=\mathfrak{q}_1 \cap \cdots \cap \mathfrak{q}_{i-1} \cap \mathfrak{q}_{i+1} \cap \cdots \cap \mathfrak{q}_n$ is also a primary decomposition. By thus eliminating the superfluous $\mathfrak{q}_i$ (and reindexing) we have $\mathfrak{a}=\mathfrak{q}_1 \cap \cdots \cap \mathfrak{q}_k$ with no $\mathfrak{q}_i$ containing the intersection of the other $\mathfrak{q}_j$.

        Let $\mathfrak{p}_1, \ldots, \mathfrak{p}_r$ be the distinct prime ideals in the set $\left\{\operatorname{Rad} \mathfrak{q}_1, \ldots, \operatorname{Rad} \mathfrak{q}_k\right\}$. Let $\mathfrak{q}_i^{\prime}(1 \leq i \leq r)$ be the intersection of all the $\mathfrak{q}$ 's that belong to the prime $\mathfrak{p}_i$, that is,
        \begin{equation*}
            \mathfrak{q}^\prime_i
            =
            \bigcap_{\Rad\left( \mathfrak{q_j} \right)=\mathfrak{p_i}} \mathfrak{q}_j
        \end{equation*}
        By \ref{cor: intersection of primary ideal} each $\mathfrak{q}_i^{\prime}$ is primary for $\mathfrak{p}_i$. Clearly no $\mathfrak{q}_i^{\prime}$ contains the intersection of all the other $\mathfrak{q}_i^{\prime}$. Therefore, $\mathfrak{a}=\bigcap_{i=1}^k \mathfrak{q}_i=\bigcap_{i=1}^r \mathfrak{q}_i^{\prime}$, whence $\mathfrak{a}$ has a reduced primary decomposition.

        2.
        It is similar to 1.
        Note that $\left(\bigcap Q_i:M\right) = \bigcap \left(Q_i:M\right)$.
    \end{proof}
\end{theorem}


\begin{theorem}
    \label{thm: Uniqueness of reduced primary decomposition}
    Let $R$ be a commutative ring with identity.
    If $M$ is an unitary $R$-module and $N$ is a proper submodule of $M$ with two reduced primary decompositions,
    \begin{equation*}
        Q_1 \cap Q_2 \cap \cdots \cap Q_k=N
        =
        Q_1^{\prime} \cap Q_2^{\prime} \cap \cdots \cap Q_{s}^{\prime}
    \end{equation*}
    where $Q_i$ is $\mathfrak{p}_i$-primary and $Q_{j}^{\prime}$ is $\mathfrak{p}_j^{\prime}$-primary.
    Then $k=s$ and (after reordering if necessary) $\mathfrak{p}_i=\mathfrak{p}_i^{\prime}$ for all $i=1,2, \ldots, k$.
    Furthermore if $Q_{i}$ and $Q_i^{\prime}$ both are $\mathfrak{p}_i$-primary and $\mathfrak{p}_i$ is an isolated prime, then $Q_{i}=Q_{i}^{\prime}$.
\end{theorem}

\begin{theorem}
    \label{thm: primary decomposition in Noetherian unitary R-module}
    Let $R$ be a commutative ring with identity and $M$ an Noetherian unitary $R$-module.
    Then every submodule $N\neq M$ has a reduced primary decomposition.
    \begin{proof}
        Let $\mathcal{S}$ be the set of all submodules of $M$ that do not have a primary decomposition.
        Clearly no primary submodule is in $\mathcal{S}$.
        We must show that $\mathcal{S}$ is actually empty. If $\mathcal{S}$ is nonempty, then $\mathcal{S}$ contains a maximal element $C$ by Theorem 1.4.

        Since $C$ is not primary, there exist $r \in R$ and $b \in M-C$ such that $r b \in C$ but $r^n M \not \subset C$ for all $n>0$.
        Let $M_n=\left(C:r^n\right)=\left\{x \in M \mid r^n x \in C\right\}$.
        Then each $M_n$ is a submodule of $M$ and $M_1 \subset M_2 \subset M_3 \subset \cdots$. By hypothesis there exists $k>0$ such that $M_i=M_k$ for $i \geq k$.
        Let $D$ be the submodule $r^kM+C=\left\{x \in M \mid x=r^k y+c\right.$ for some $\left.y \in M, c \in C\right\}$. Clearly $C \subset M_k \cap D$.

        Conversely, if $x \in M_k \cap D$, then $x=r^k y+c$ and $r^k x \in C$, whence $r^{2 k} v=r^k\left(r^k y\right)=r^k(x-c)=r^k x-r^k c \in C$. Therefore, $y\in M_{2 k}=M_k$. Consequently, $r^k y \in C$ and hence $x=r^k y+c \in C$.
        Therefore $M_k \cap D \subset C$, whence $M_k \cap D=C$. Now $C \neq M_k \neq M$ and $C \neq D \neq M$ since $b \in M_k-C$ and $r^k M \not \subset C$. By the maximality of $C$ in $\mathcal{S}, M_k$ and $D$ must have primary decompositions. Thus $C$ has a primary decomposition, which is a contradiction. Therefore $\mathcal{S}$ is empty and every submodule has a primary decomposition. Consequently, every submodule has a reduced primary decomposition by \ref{thm: Existence of reduced primary decomposition}.
    \end{proof}
\end{theorem}

\begin{corollary}
    If $R$ is a commutative Noetherian ring with identity and $M$ is a finitely generated unitary $R$-module.
    Then every submodule $N(\neq M)$ of $M$ has a reduced primary decomposition.
    \begin{proof}
        This is an immediate consequence of \ref{thm: finitely generated left unitary R-module over left Noetherian [resp. Artinian] ring} \ref{thm: equivalent conditions of Noetherian modules} and \ref{thm: primary decomposition in Noetherian unitary R-module}
    \end{proof}
\end{corollary}


\section{Contraction and extension of ideal} % Contraction and extension
\begin{definition}
    Let $R$ be a ring and $f:A \rightarrow B$ be a ring homomorphism, \begin{enumerate}
        \item
              the \textbf{extension} of ideal $\mathfrak{a}$ of $A$ is the ideal generated by $f\left(\mathfrak{a}\right)$ in $B$, denoted by $\mathfrak{a}^e$.

        \item
              the \textbf{contraction} of $\mathfrak{b}$ is $f^{-1}\left(\mathfrak{b}\right)$, denoted by $\mathfrak{b}^c$.
    \end{enumerate}
    Especially if $A$ be a subring of $B$ and $i: A\rightarrow B$, the contraction of ideal of $\mathfrak{b}$ of $B$ is $A\cap \mathfrak{b}$.
\end{definition}

\begin{proposition}
    \label{pro: Contraction and extension}
    .
    \begin{enumerate}
        \item $\mathfrak{a} \subseteq \mathfrak{a}^{e c}, \mathfrak{b} \supseteq \mathfrak{b}^{c e}$;
        \item  $\mathfrak{b}^c=\mathfrak{b}^{\text {cec }}, \mathfrak{a}^e=\mathfrak{a}^{\text {ece }}$;
        \item  If $\mathcal{C}$ is the set of all contracted ideals in $A$ and if $\mathcal{E}$ is the set of all extended ideals in $B$, then $\mathcal{C}=\left\{\mathfrak{a} \mid \mathfrak{a}^{e c}=\mathfrak{a}\right\}, \mathcal{E}=\left\{\mathfrak{b} \mid \mathfrak{b}^{c e}=\mathfrak{b}\right\}$, and $\mathfrak{a} \mapsto \mathfrak{a}^e$ is a bijective map, whose inverse is $\mathfrak{b} \mapsto \mathfrak{b}^c$.
    \end{enumerate}
\end{proposition}


\begin{proposition}
    \label{pro: Contraction and extension and other operations}
    .
    \begin{equation*}
        \begin{array}{cc}
            \left(\mathfrak{a}_1+\mathfrak{a}_2\right)^e=\mathfrak{a}_1^e+\mathfrak{a}_2^e                         & \left(\mathfrak{b}_1+\mathfrak{b}_2\right)^c \supseteq \mathfrak{b}_1^c+\mathfrak{b}_2^c,              \\
            \left(\mathfrak{a}_1 \cap \mathfrak{a}_2\right)^e \subseteq \mathfrak{a}_1^e \cap \mathfrak{a}_2^e     & \left(\mathfrak{b}_1 \cap \mathfrak{b}_2\right)^c=\mathfrak{b}_1^c \cap \mathfrak{b}_2^c,              \\
            \left(\mathfrak{a}_1 \mathfrak{a}_2\right)^e=\mathfrak{a}_1^e \mathfrak{a}_2^e                         & \left(\mathfrak{b}_1 \mathfrak{b}_2\right)^c \supseteq \mathfrak{b}_1^c \mathfrak{b}_2^c,              \\

            \left(\mathfrak{a}_1: \mathfrak{a}_2\right)^e \subseteq\left(\mathfrak{a}_1^e: \mathfrak{a}_2^e\right) & \left(\mathfrak{b}_1: \mathfrak{b}_2\right)^c \subseteq\left(\mathfrak{b}_1^c: \mathfrak{b}_2^c\right) \\
            \Rad(\mathfrak{a})^e \subseteq \Rad\left(\mathfrak{a}^e\right)                                         & \Rad(\mathfrak{b})^c=\Rad\left(\mathfrak{b}^c\right)
        \end{array}
    \end{equation*}

    if $\mathfrak{b}$ is a prime ideal in $B$, then so $\mathfrak{b}^c$.

    The set $\mathcal{C}$ is closed under the other three operations, and $\mathcal{E}$ is closed under sum and product.




\end{proposition}

\section{Nil and nilpotent ideals} % Nil and nilpotent ideals

\begin{definition}
    Let $R$ be a ring and $\mathfrak{a}$ be an (left, right, two-sided) ideal of $R$.
    \begin{enumerate}
        \item
              $\mathfrak{a}$ of $R$ is \textbf{nil} if every element of $\mathfrak{a}$ is a nilpotent element;
        \item
              $\mathfrak{a}$ is \textbf{nilpotent} if $\mathfrak{a}^n=0$ for some integer $n$.
    \end{enumerate}
\end{definition}


\begin{theorem}
    \label{thm: nil ideal}
    Let $R$ be a ring.
    \begin{enumerate}
        \item
              If $a\in R$ is nilpotent, $a$ is both left and right quasiregula with quasi inverse $r=-a+a^2-a^3+\cdots+(-1)^{n-1} a^{n-1}$
        \item
              Every nil left ideal is contained in $J(R)$.
        \item
              Thus every nil ring is a radical ring.
    \end{enumerate}
\end{theorem}

\begin{proposition}
    If $R$ is a left [resp. right] Noetherian ring, then every nil ideal is nilpotent (Exercise 16).
\end{proposition}




\chapter{Radical Ideals}% Radical Ideals
\minitoc

$R$ is a commutative ring (with identity) throughout this chapter unless otherwise stated.

\section{Radical and Nilradical} % Radical and Nilradical 
\begin{definition}
    Let $R$ be a commutative ring.
    If $\mathfrak{a}$ is any ideal of $R$, the ideal
    \begin{equation*}
        \Rad(\mathfrak{a})
        =
        \left\{x \in R: x^n \in \mathfrak{a} \text { for some } n \in \mathbb{Z}_{\geq 1}\right\}
    \end{equation*}
    is called \textbf{radical} of $\mathfrak{a}$, sometimes denoted by $\sqrt{\mathfrak{a}}$.
    The radical of $0$ (the set of all nilpotent elements in $R$) is called \textbf{nilradical} of $R$, denoted by $\Nil\left(R\right)$.
\end{definition}
\begin{proposition}
    \label{pro: operations of radical}
    Let $R$ be a commutative ring. Then
    \begin{enumerate}
        \item
              $\mathfrak{a} \subset r(\mathfrak{a})  $

        \item

              $r(r(\mathfrak{a}))=r(\mathfrak{a})$

        \item
              $r(\mathfrak{a b})=r(\mathfrak{a} \cap \mathfrak{b})=r(\mathfrak{a}) \cap r(\mathfrak{b})$

        \item

              thus $r\left(\mathfrak{a}_1\mathfrak{a}_2\cdots \mathfrak{a}_n\right)
                  =
                  r\left(\bigcap \mathfrak{a}_i\right)
                  =\bigcap r\left(\mathfrak{a}_i\right)
              $ and
              $r(\mathfrak{a}^n)=r(\mathfrak{a})$

        \item

              $r(\mathfrak{a}+\mathfrak{b})=r(r(\mathfrak{a})+r(\mathfrak{b}))$

        \item

              if $\mathfrak{p}$ is prime in $R$,
              $r\left(\mathfrak{p}^n\right)
                  =r\left(\mathfrak{p}\right)
                  =
                  \mathfrak{p}$ for all $n>0$.
        \item
              $r(\mathfrak{a})=(1) \Leftrightarrow \mathfrak{a}=(1)$
    \end{enumerate}
\end{proposition}


\begin{proposition}
    \label{pro: The separability between ideal and multiplicative subset}
    Let $R$ be a commutative ring.
    If $S$ is a multiplicative subset which is disjoint from an ideal $\mathfrak{a}$, then there exists a prime ideal $\mathfrak{p}$ which is maximal in $\mathcal{S}=\left\{\mathfrak{b}: \mathfrak{a}\subset  \mathfrak{b} \text{ and }  \mathfrak{b}\cap S =\varnothing\right\}$.
    \begin{proof}
        Since $S \neq \varnothing$ and every ideal in $\mathcal{S}$ is properly contained in $R$, set $\mathcal{S}$ is partially ordered by inclusion.
        By Zorn's Lemma there is an ideal $\mathfrak{p}$ which is maximal in $\mathcal{S}$.

        Let $\mathfrak{a}_1,\mathfrak{a}_2$ be ideals of $R$ such that $\mathfrak{a}_1\mathfrak{a}_2 \subset \mathfrak{p}$.
        If $\mathfrak{a}_1 \not \subset \mathfrak{p}$ and $\mathfrak{a}_2 \not \subset \mathfrak{p}$, then each of the ideals $\mathfrak{p}+\mathfrak{a}_1$ and $\mathfrak{p}+\mathfrak{a}_2$ properly contains $\mathfrak{p}$ and hence must meet $S$.
        Consequently, for some $p_i \in \mathfrak{p}, a_i \in \mathfrak{a}_i$.
        \begin{equation*}
            p_1+a_1=s_1 \in S \quad \text { and } \quad p_2+a_2=s_2 \in S
        \end{equation*}
        Thus $s_1 s_2=p_1 p_2+p_1 a_2+a_1 p_2+a_1 a_2 \in \mathfrak{p}+\mathfrak{a}_1 \mathfrak{a}_2 \subset \mathfrak{p}$. This is a contradiction since $s_1 s_2 \in S$ and $S \cap \mathfrak{p}=\varnothing$.
        Therefore $\mathfrak{a}_1 \subset \mathfrak{p}$ or $\mathfrak{a}_2 \subset \mathfrak{p}$, whence $\mathfrak{p}$ is prime.
    \end{proof}
\end{proposition}


\begin{theorem}
    \label{thm: radical}
    Let $R$ be a commutative rng and an ideal $\mathfrak{a}$.
    \begin{enumerate}
        \item
              If $\pi: R \rightarrow R / \mathfrak{a}$ is the canonical projection, then
              $\Rad(\mathfrak{a})
                  =
                  \pi^{-1}\left(\Nil\left(R / \mathfrak{a}\right)\right)$

        \item
              The radical of an ideal $\mathfrak{a}$ is the intersection of the prime ideals which contain $\mathfrak{a}$, that is,
              \begin{equation*}
                  \Rad\left(\mathfrak{a}\right)
                  =
                  \bigcap_{\substack{\mathfrak{a}\subset\mathfrak{p}
                          \\
                          \mathfrak{p} \text{ is prime}}}
                  \mathfrak{p}
              \end{equation*}
    \end{enumerate}
    \begin{proof}
        It is clear that
        \begin{equation*}
            \Rad\left(\mathfrak{a}\right)
            \subset
            \bigcap_{\substack{\mathfrak{a}\subset\mathfrak{p}
                    \\
                    \mathfrak{p} \text{ is prime}}}
            \mathfrak{p}
            :=
            \tilde{\mathfrak{p}}
        \end{equation*}
        by \cref{cor: Prime ideal}.
        If $ S = \tilde{\mathfrak{p}}-\Rad\left(\mathfrak{a}\right)$ is nonempty, whence is a multiplicative subset of $R$ (verify that $x,y\in S \Rightarrow xy \in S$) and disjoint from $\Rad\left(\mathfrak{a}\right)$, there exist a prime ideal $\mathfrak{p}^\prime$ that contains $\Rad\left(\mathfrak{a}\right)$ and disjoint from $S$ by \ref{pro: The separability between ideal and multiplicative subset}. But $\tilde{\mathfrak{p}}\subset \mathfrak{p}^\prime$ by the definition of $\tilde{\mathfrak{p}}$, this is a contradiction.
    \end{proof}
\end{theorem}

\begin{proposition}
    If $R$ is a commutative ring with identity $\neq 0$, then $R^{\times}+\Nil\left(R\right)\subset R^{\times}$.
\end{proposition}






\chapter{Fractions and Localization}% Fractions and Localization

\minitoc



\section{Rings of Quotient} % Rings of Quotient
$A$ is a commutative ring with identity throughout this section unless otherwise stated.

\begin{definition}
    Let $A$ be a commutative ring. A subset $S$ called a \textbf{multiplicative subset} of $A$ if $S$ is a submonoid of $\left(A, \times \right)$.
    \begin{remark}
        In general, we always assume that $0 \notin S$.
    \end{remark}
\end{definition}


\begin{definition}
    \label{thm: ring of quotients }
    Let $S$ be a multiplicative subset of $A$ and
    \begin{enumerate}
        \item
              The relation defined on the set ${A} \times {S}$ by
              \begin{equation*}
                  (a, s) \sim \left(a^{\prime}, s^{\prime}\right) \Leftrightarrow s_1\left(as^{\prime}-a^{\prime}s\right)=0 \text { for some } s_1 \in S
              \end{equation*}
              is an equivalence relation and the equivalence class containing the element $(a, s)$ is denoted by ${a} / {s}$.
        \item
              ${S}^{-1} {R}$ is a commutative ring with identity $1/1$, where addition and multiplication are defined by
              \begin{equation*}
                  {r} / {s}+{r}^{\prime} / {s}^{\prime}=\left({rs}^{\prime}+{r}^{\prime} {s}\right) / {ss}^{\prime} \quad \text { and } \quad({r} / {s})\left({r}^{\prime} / {s}^{\prime}\right)={rr}^{\prime} / {ss}^{\prime}
              \end{equation*}
              is called the \textbf{ring of fractions} of $R$ by $S$.
    \end{enumerate}
    \begin{remark}
        The map $\varphi_{S}: A \rightarrow {S}^{-1} {A}$ given by $a\mapsto a / 1$ is a well-defined homomorphism of rings and $\varphi\left(S\right) \subset \left({S}^{-1} {A}\right)^{\times}$.
    \end{remark}
\end{definition}
\begin{theorem}[Universal property]
    Let $\mathcal{C}$ be the category
    \begin{itemize}
        \item
              whose objects are ring-homomorphisms (commutative rings with identity)
              \begin{equation*}
                  f: A \rightarrow B
              \end{equation*}
              such that for every $s \in S$, the element $f(s)$ is invertible in $B$.
        \item
              If $f: A \rightarrow B$ and $f^{\prime}: A \rightarrow B^{\prime}$ are two objects of $\mathcal{C}$, a morphism $g$ of $f$ into $f^{\prime}$ is a homomorphism
              \begin{equation*}
                  g: B \rightarrow B^{\prime}
              \end{equation*}
              making the diagram commutative:
    \end{itemize}
    We have that $\varphi_S: A \rightarrow S^{-1}A$ is a universal object in this category $\mathcal{C}$.
\end{theorem}





\begin{theorem}
    Let $S$ be a multiplicative subset of $A$.
    \begin{enumerate}
        \item
              If $S$ contains no zero divisors, then $\varphi_{{S}}$ is a monomorphism.
        \item
              If $A$ has no zero divisors and $0 \notin {S}$, then ${S}^{-1} {A}$ is an integral domain.

        \item
              If $S\subset A^\times$, then $\varphi_{{S}}$ is an isomorphism.
    \end{enumerate}
\end{theorem}




\begin{definition}
    Let $A$ be a commutative ring and $S$ be the set of all nonzero elements of $A$ that are not zero divisors, then $S^{-1} A$ is called the \textbf{complete ring of quotients} of the ring $A$.

    The complete ring of quotients of an integral domain $A$ is its \textbf{quotient field}, denoted by $\Frac(A)$.
\end{definition}


\section{Extensions and Contractions in ring of fractions} % Extensions and Contractions in ring of fractions
Let $A$ be a commutative with identity $1_A$ and $S$ be a multiplicative subset with $\varphi_S:A\rightarrow S^{-1}A$

\begin{proposition}
    \begin{enumerate}
        \item
              If $\mathfrak{a}$ is an ideal in $A$, then $S^{-1} \mathfrak{a}=\left\{a / s \mid a \in \mathfrak{a} ; s \in S\right\}=\mathfrak{a} S^{-1}A=\mathfrak{a}^e$.
        \item
              If $\mathfrak{b}$ is an ideal in $S^{-1}A$, then $\varphi_S^{-1}\left(\mathfrak{b}\right)$ coincides with $\mathfrak{b}^c$
        \item
              let $\mathfrak{a}$ be an ideal of $A$, then  $S^{-1} \mathfrak{a}=S^{-1} A$ if and only if $S \cap \mathfrak{a} \neq \varnothing$.
    \end{enumerate}
\end{proposition}

\begin{corollary}
    \begin{equation*}
        S^{-1}\left(\mathfrak{a}+\mathfrak{b}\right)=S^{-1}\mathfrak{a}+S^{-1}\mathfrak{b}
    \end{equation*}
    \begin{equation*}
        S^{-1}\left(\mathfrak{a}\mathfrak{b}\right)=\left(S^{-1}\mathfrak{a}\right)\left(S^{-1}\mathfrak{b}\right)
    \end{equation*}
    \begin{equation*}
        S^{-1}\left(\mathfrak{a}\cap \mathfrak{b}\right)=S^{-1}\mathfrak{a}\cap S^{-1}\mathfrak{b}
    \end{equation*}
    \begin{equation*}
        S^{-1}\Rad(\mathfrak{a})=\Rad\left(S^{-1}\mathfrak{a}\right)
    \end{equation*}
\end{corollary}

\begin{theorem}.
    \begin{enumerate}
        \item
              $\mathfrak{b}^{ce}= \mathfrak{b}$ for all ideals $\mathfrak{b}$ of $S^{-1}A$.
              In other words every ideal in $S^{-1} A$ is of the $S^{-1} \mathfrak{a}=\mathfrak{a}^e$ for some ideal $\mathfrak{a}$ in $A$ by \cref{pro: Contraction and extension}.
        \item $\mathfrak{a}^{ec}=\bigcup_{s\in S} \left(\mathfrak{a}:s\right)$

        \item
              If $\mathfrak{p}$ is a prime ideal in $A$ and  $S \cap \mathfrak{p}=\varnothing$, then $S^{-1} \mathfrak{p}$ is a prime ideal in $S^{-1} A$
        \item
              there is a one-to-one correspondence between the set  $\mathcal{U}=\left\{\mathfrak{p}: \mathfrak{p} \text{ is prime and disjoint from } S\right\}$ and the set $\mathcal{V}=\left\{S^{-1}\mathfrak{p}: S^{-1}\mathfrak{p} \text{ is prime in } S^{-1} R\right\}$, given by $\mathfrak{p} \mapsto S^{-1} \mathfrak{p}$.
    \end{enumerate}
    \begin{proof}
        Let $I=\varphi_S^{-1}(J)$, then $I^e=J^{ce}\subset J$, whence $S^{-1} I \subset J$.
        Conversely, if $r / s \in J$, then $\varphi_S(r)=r s / s =(r / s)\left(s^2 / s\right) \in J$, whence $r \in \varphi_S^{-1}(J)=I$.
        Thus $r / s \in S^{-1} I$ and hence $J \subset S^{-1} I$.
    \end{proof}
\end{theorem}










\section{Fractions of modules} % Fractions of modules
Let $A$ be a commutative ring with identity, $M$ be a $A$-module and $S$ be a multiplicative subset of $A$.
\begin{definition}
    The \textbf{module of fractions} of $M$ with respect to $S$ is the set
    \begin{equation*}
        S^{-1} M=\left\{m / s \mid m \in M, s \in S\right\}
    \end{equation*}
    of equivalence classes of the relation on $M \times S$ defined by
    \begin{equation*}
        (m, s) \sim\left(m^{\prime}, s^{\prime}\right) \Leftrightarrow s_1\left(s^{\prime} m-s m^{\prime}\right)=0 \text { for some } s_1 \in S
    \end{equation*}
    with addition and scalar multiplication defined by
    \begin{equation*}
        m / s+m^{\prime} / s^{\prime}=\left(s^{\prime} m+s m^{\prime}\right) / s s^{\prime} \quad \text { and } \quad(a / t)(m / s)=(a m) / t s
    \end{equation*}
    for all $m, m^{\prime} \in M, s, s^{\prime}, t \in S$ and $a \in A$.
\end{definition}


\begin{lemma}
    \begin{enumerate}
        \item
              $S^{-1}(N+P)=S^{-1}(N)+S^{-1}(P)$
        \item
              $S^{-1}(N \cap P)=S^{-1}(N) \cap S^{-1}(P)$
    \end{enumerate}
\end{lemma}

\begin{proposition}.
    \begin{enumerate}
        \item
              The functor $S^{-1}:{}_A\Mod \rightarrow {}_{S^{-1}A}\Mod$ is exact
        \item
              $S^{-1}\square$ is additive
        \item
              $S^{-1}\square\cong S^{-1}A\otimes_A\square$ in $\left[{}_A\Mod,{}_{S^{-1} A}\Mod\right]$
    \end{enumerate}
\end{proposition}

\begin{corollary}.
    \begin{enumerate}
        \item
              $S^{-1}(M / N)\cong \left(S^{-1} M\right) /\left(S^{-1} N\right)$ in ${}_{S^{-1} A}\Mod$
        \item
              $S^{-1}M\otimes_{S^{-1}A} S^{-1}N \cong S^{-1}(M \oplus_A N)$ in ${}_{S^{-1} A}\Mod$
        \item  $S^{-1}A$ is a flat $A$-module
    \end{enumerate}
\end{corollary}






\section{Localization and Local rings}
% Localization and Local rings
\begin{definition}
    Let $A$ be a commutative ring with identity, $\mathfrak{p}$ a prime ideal of $A$ and multiplicative subset $S=A-\mathfrak{p}$.
    The ring of quotients $S^{-1} A$ is called the \textbf{localization of $A$ at $\mathfrak{p}$} and is denoted $A_\mathfrak{p}$.
    If $\mathfrak{a}$ is an ideal in $A$, then the ideal $\mathfrak{a}^e=S^{-1} \mathfrak{a}$ in $A _\mathfrak{p}$.
    \begin{remark}
        We always identify $A$ with its image $\varphi_S(A)$ in $A_{\mathfrak{p}}$ thus $A$ can be considered as a subring of $A_{\mathfrak{p}}$. In this case, the extension ideal $\mathfrak{a}^e= S^{-1}\mathfrak{a}=\mathfrak{a}A_\mathfrak{p}$.
    \end{remark}
\end{definition}

\begin{theorem}
    Let $\mathfrak{p}$ be a prime ideal in a commutative ring $A$ with identity and localization $A_{\mathfrak{p}}$.
    \begin{enumerate}
        \item
              There is a one-to-one correspondence between the set $\left\{\mathfrak{q}: \mathfrak{q} \text{ is prime and contained in }\mathfrak{p} \right\}$ and the set $\left\{S^{-1}\mathfrak{q}: S^{-1}\mathfrak{q} \text{ is prime in } A_\mathfrak{p}\right\}$, given by $\mathfrak{q} \mapsto S^{-1}\mathfrak{q}$;
        \item
              The ideal $S^{-1}{\mathfrak{p}}$ is the unique maximal ideal of $A_{\mathfrak{p}}$.
    \end{enumerate}
\end{theorem}


\begin{definition}
    A \textbf{local ring} is a commutative ring with identity which has a unique maximal ideal.
\end{definition}

\begin{theorem}
    The following conditions are equivalent.
    \begin{enumerate}
        \item
              $R$ is a local ring;
        \item
              all nonunits of $R$ are contained in some ideal $M \neq R$;
        \item
              the nonunits of $R$ form an ideal.

        \item for all $r, s \in R$, $r + s = 1_R$ implies $r$ or $s$ is a unit.
    \end{enumerate}
\end{theorem}

\begin{proposition}
    Every nonzero homomorphic image of a local ring is local.
\end{proposition}
\section{Local properties}
\begin{definition}
    A property $P$ of a ring $A$ (or of an $A$-module $M$ ) is said to be a \textbf{local property} if the following is true:
    $A$ (or $M$ ) has $P \Leftrightarrow A_{\mathfrak{p}}$ (or $M_{\mathfrak{p}}$ ) has $P$, for each prime ideal $\mathfrak{p}$ of $A$.
\end{definition}

\begin{proposition}
    The following are local properties:
    \begin{enumerate}
        \item $A$-module $M$ is zero
        \item $A$-module homomorphism $\phi:M\rightarrow N$ is injective [resp. surjective, bijective]
        \item $A$-module $M$ is flat
    \end{enumerate}
\end{proposition}

\chapter{Chain Condition} % Chain Condition
\label{Chain Condition}
\section{}
\begin{definition}
    Let $R$ be a ring.
    \begin{enumerate}
        \item
              A $R$-module $M$ is said to satisfy the \textbf{ascending chain condition (ACC) on submodules} (or to be Noetherian) if for every chain $M_1 \subset M_2 \subset M_3 \subset \cdots$ of submodules of $M$, there is an integer n such that $M_{i}=M_{n}$ for all $i \geq n$.

        \item
              The ring $R$ is \textbf{left [resp. right] Noetherian} if $R$ satisfies ACC on submodules as a left [resp. right] $R$-module.
              It equivalent that $R$ satisfies the ascending chain condition on left [resp. right] ideals.
              $R$ is said to be \textbf{Noetherian} if $R$ is both left and right Noetherian.

        \item
              A module $N$ is said to satisfy, the \textbf{descending chain condition (DCC) on submodules} (or to be Artinian) if for every chain $N_1 \supset N_2 \supset N_3 \supset \cdots$ of submodules of $N$, there is an integer $m$ such that $N_{i}=N_m$ for all $i \geq m$.

        \item
              $R$ is \textbf{left [resp. right] Artinian } if $R$ satisfies DCC on submodules as a left [resp. right] $R$-module.
              It equivalent that $R$ satisfies the descending chain condition on left [resp. right] ideals.
              $R$ is said to be \textbf{Artinian} if $R$ is both left and right Artinian.
    \end{enumerate}
\end{definition}

\begin{definition}
    Let $R$ be a ring, A module $M$ is said to satisfy the \textbf{maximum condition} [resp. minimum condition] on submodules if every nonempty set of submodules of $M$ contains a maximal [resp. minimal] element (with respect to set theoretic inclusion).
\end{definition}





\subsection{Equivalent Condition of Chain Condition}
% Equivalent Condition of Chain Condition
\begin{theorem}
    \label{thm: chain condition and m condition}
    A module $A$ satisfies the ascending [resp. descending] chain condition on submodules if and only if $A$ satisfies the maximal [resp. minimal] condition on submodules.
\end{theorem}

\begin{theorem}
    Let $R$ be a ring and $0 \rightarrow A \xrightarrow{f} B \xrightarrow{g} C \rightarrow 0$ be a short exact sequence of $R$-modules.
    Then $B$ satisfies the ACC [resp. DCC] on submodules if and only if $A$ and $C$ satisfy it.
    \begin{proof}
        Sufficiency.
        If $B$ satisfies the ascending chain condition, then so does its submodule $f(A)$. By exactness $A$ is isomorphic to $f(A)$, whence $A$ satisfies the ascending chain condition. If $C_1 \subset C_2 \subset \cdots$ is a chain of submodules of $C$, then $g^{-1}\left(C_1\right) \subset g^{-1}\left(C_2\right) \subset \cdots$ is a chain of submodules of $B$. Therefore, there is an $n$ such that $g^{-1}\left(C_i\right)=g^{-1}\left(C_n\right)$ for all $i \geq n$. Since $g$ is an epimorphism by exactness, it follows that $C_i=C_n$ for all $i \geq n$. Therefore, $C$ satisfies the ascending chain condition.

        Necessity.
        Suppose $A$ and $C$ satisfy the ascending chain condition and $B_1 \subset B_2 \subset \cdots$ is a chain of submodules of $B$. For each $i$ let
        \begin{equation*}
            A_i=f^{-1}\left(f(A) \cap B_i\right) \quad \text { and } \quad C_i=g\left(B_i\right)
        \end{equation*}
        Let $f_i=\left.f\right|_{A_i}$ and $g_i=\left.g\right|_{B_i}$.
        Verify that for each $i$ the following sequence is exact:
        \begin{equation*}
            0 \rightarrow A_i \xrightarrow{f_i} B_i \xrightarrow{g_i} C_i \rightarrow 0 .
        \end{equation*}
        Verify that $A_1 \subset A_2 \subset \cdots$ and $C_1 \subset C_2 \subset \cdots$. By hypothesis there exists an integer $n$ such that $A_i=A_n$ and $C_i=C_n$ for all $i \geq n$. For each $i \geq n$ there is a commutative diagram with exact rows:
        \begin{equation*}
            \begin{tikzcd}
                0 \arrow[r,] & A_n\arrow[r,"f_n"] \arrow[dd,"\id"]& B_n \arrow[r,"g_n"]\arrow[dd,"i"]& C_n\arrow[r]\arrow[dd,"\id"]& 0\\
                &&&&\\
                0 \arrow[r,]& A_i\arrow[r,"f_i"] & B_i \arrow[r,"g_i"]& C_i\arrow[r,]& 0\\
            \end{tikzcd}
        \end{equation*}
        The Short Five Lemma implies that the inclusion map $i$ is a isomorphism, thus be the identity map, whence $B$ satisfies the ascending chain condition.
    \end{proof}
\end{theorem}

\begin{corollary}
    \label{cor: chain condition on submodules of direct sum }
    Let $R$ be a ring, we have
    \begin{enumerate}
        \item
              If $M_1$ is a submodule of a module $M$ , then $M$ satisfies the ascending [resp. descending] chain condition if and only if $M_1$ and $M / M_1$ satisfiy it.

        \item
              If $M_1, \ldots, M_n$ are modules, then the direct sum $M_1 \oplus M_2 \oplus \cdots \oplus M_n$ satisfies the ascending [resp. descending] chain condition on submodules if and only if each $A_i$ satisfies it.
    \end{enumerate}
\end{corollary}


\begin{theorem}
    \label{thm: finitely generated left unitary R-module over left Noetherian [resp. Artinian] ring}
    If $R$ is a left [resp. right] Noetherian [resp. Artinian] ring, then every finitely generated left [resp. right]  $R$-module $M$ is Noetherian [resp. Artinian].
    \begin{proof}
        If $M$ is finitely generated, then by \ref{cor: Every unitary module $M$ over a ring $R$ with identity is the homomorphic image of a free unitary $R$-module $F$.}
        there is a free unitary $R$-module $F$ with a finite basis and an epimorphism $\pi: F \rightarrow M$. Since $F$ is a direct sum of a finite number of copies of $R$ by \ref{thm: Free unitary R-module}, $F$ is left Noetherian [resp. Artinian], whence $M \cong F / \Ker \pi$ is Noetherian [resp. Artinian] by \ref{cor: chain condition on submodules of direct sum }.
    \end{proof}
\end{theorem}






\section{Normal series and Composition Series of Modules} % Normal series and Composition Series of Modules
\begin{definition}
    Let $R$ be a ring and a $R$-module $A$.
    \begin{enumerate}
        \item
              A \textbf{normal series} for $A$ is a chain of submodules: $A=A_0 \supset A_1 \supset$ $A_2 \supset \cdots \supset A_n$. The factors of the series are the quotient modules
              \begin{equation*}
                  A_i / A_{i+1} \quad(i=0,1, \ldots, n-1) .
              \end{equation*}
              The \textbf{length} of the normal series is the number of proper inclusions ( $=$ number of nontrivial factors).

        \item
              A \textbf{refinement} of the normal series $A=A_0 \supset A_1 \supset \cdots \supset A_n$ is a normal series obtained by inserting a finite number of additional submodules between the given ones. A \textbf{proper refinement} is one which has length larger than the original series.

        \item
              Two normal series are \textbf{equivalent} if there is a one-to-one correspondence between the nontrivial factors such that corresponding factors are isomorphic modules. Thus equivalent series necessarily have the same length.

        \item
              A \textbf{composition series} for $A$ is a normal series $A=A_0 \supset A_1 \supset A_2 \supset \cdots \supset A_n=0$ such that each factor $A_k / A_{k+1}$ $(k=0,1, \ldots, n-1)$ is a module with no proper nonempty submodules.
    \end{enumerate}
\end{definition}
\begin{theorem}
    \label{thm: Uniqueness of composition series of modules}
    Any two normal series of a module $A$ have refinements that are equivalent.
    Any two composition series of $A$ are equivalent.
\end{theorem}

\begin{theorem}
    \label{thm: Existence of composition series of modules}
    A nonzero module  $M$ has a composition series if and only if $M$ satisfies both the ACC and DCC on submodules.
    \begin{proof}
        $(\Rightarrow)$ Suppose $A$ has a composition series $S$ of length $n$. If either chain condition fails to hold, one, can find submodules
        \begin{equation*}
            A=A_0 \supsetneq A_1 \supsetneq A_2 \supsetneq \cdots \supsetneq A_n \supsetneq A_{n+1}
        \end{equation*}
        which form a normal series $T$ of length $n+1$. By \ref{thm: Uniqueness of composition series of modules}, $S$ and $T$ have refinements that are equivalent. This is a contradiction since equivalent series have equal length. For every refinement of the composition series $S$ has the same length $n$ as $S$, but every refinement of $T$ necessarily has length at least $n+1$. Therefore, $A$ satisfies both chain conditions.

        $(\Leftarrow)$ If $B$ is a nonzero submodule of $A$, let $S(B)$ be the set of all proper submodules $C$ of $B$.
        Also define $S(0)=\{0\}$.
        For each $B$ there is a maximal element $B^{\prime}$ of $S(B)$ by \ref{thm: chain condition and m condition}. Let $S$ be the set of all submodules of $A$ and define a map $f: S \rightarrow S$ by $f(B)=B^{\prime}$

        Let $A_i=f^{(i)}\left(A\right)$, then $A \supset A_1 \supset A_2 \supset \cdots$ is a descending chain by construction, whence for some $n, A_i=A_n$ for all $i \geq n$.
        Since $A_{n+1}=f\left(A_n\right)$, the definition of $f$ shows that $A_{n+1}=A_n$ only if $A_n=0=A_{n+1}$. Let $m$ be the smallest integer such that $A_m=0$. Then $m \leq n$ and $A_k \neq 0$ for all $k<m$. Furthermore for each $k<m$, $A_{k+1}$ is a maximal submodule of $A_k$ such that $A_k \supsetneq A_{k+1}$. Consequently, each $A_k / A_{k+1}$ is nonzero and has no proper submodules by \ref{thm: The Fourth or Lattice Isomorphism Theorem of Modules}. Therefore, $A \supset A_1 \supset \cdots \supset A_m=0$ is a composition series for $A$.
    \end{proof}
\end{theorem}




\chapter{Noetherian Modules and Rings} % Noetherian Modules and Rings
\minitoc

\section{Artinian Rings} % Artinian Rings

\subsection{}
\begin{theorem}[Hopkins-Levitzki]
    Let $R$ be a ring.
    Then
    \begin{center}
        left Artinian $\Leftrightarrow$ right Artinian $\Rightarrow$ left Noetherian $+$ right Noetherian
    \end{center}.
\end{theorem}

\subsection{}
In this subsection we shall study commutative Artinian rings.
\begin{theorem}
    Let $A$ be a commutative ring with identity. Then the following conditions are equivalent:
    \begin{enumerate}
        \item
              $A$ is Artinian
        \item
              $\Spec \left(A\right)$ is finite and $\dim A=0$
        \item
              $A$ is Noetherian and $\dim A=0$.
        \item
              $A$ is uniquely (up to isomorphism) a finite direct product of Artin local rings.
    \end{enumerate}
\end{theorem}

\begin{corollary}
    In a commutative Artin ring the nilradical is equal to the Jacobson radical.
\end{corollary}

\begin{proposition}
    Let $A$ be an Artin local ring. Then the following are equivalent:
    \begin{enumerate}
        \item
              every ideal in $A$ is principal;
        \item
              the maximal ideal $\mathfrak{m}$ is principal;
        \item
              $\operatorname{dim}_k\left(\mathfrak{m} / \mathfrak{m}^2\right) \leqslant 1$.
    \end{enumerate}
\end{proposition}


\section{Noetherian Modules and Rings} % properties of Noetherian Modules and Rings


\begin{theorem}
    \label{thm: equivalent conditions of Noetherian modules}
    A module
    $M$ is Noetherian if and only if every submodule of $M$ is finitely generated.
\end{theorem}

\begin{theorem}
    Recall that a module $M$ is Noetherian.

    \ref{thm: chain condition and m condition}
    A module $M$ satisfies the ascending [resp. descending] chain condition on submodules if and only if $M$ satisfies the maximal [resp. minimal] condition on submodules.

    \ref{thm: equivalent conditions of Noetherian modules}
    A module
    $M$ satisfies the ACC on submodules if and only if every submodule of $M$ is finitely generated.
    In particular, a commutative ring $R$ is Noetherian if and only if every ideal of $R$ is finitely generated.

    \ref{thm: finitely generated left unitary R-module over left Noetherian [resp. Artinian] ring}
    If $R$ is a left [resp. right] Noetherian [resp. Artinian] ring with identity, then every finitely generated unitary left [resp. right]  $R$-module $A$ satisfies the ACC [resp. DCC] on submodules.
\end{theorem}


\begin{proposition}
    Let $A$ be Noetherian ring
\end{proposition}


\begin{proposition}
    If $A$ is Noetherian and $\phi$ is a homomorphism, then $B=\phi(A)$ is Noetherian.
\end{proposition}

\begin{proposition}
    Let $A$ be a subring of $B$; suppose that $A$ is Noetherian and that $B$ is finitely generated as an $A$-module. Then $B$ is Noetherian (as a ring).
\end{proposition}




\begin{proposition}
    If $A$ is Noetherian and $S$ is any multiplicatively closed subset of $A$, then $S^{-1} A$ is Noetherian.
\end{proposition}



\begin{theorem}
    If $R$ is a commutative Noetherian ring with identity, then so is $R\left[x_1, \ldots, x_{n}\right]$ and $R\left[\left[x\right]\right]$
\end{theorem}

\begin{proposition}
    If $R$ is a commutative ring with identity and $\mathfrak{p}$ is an ideal which is maximal in the set of all ideals of $R$ which are not finitely generated, then $\mathfrak{p}$ is prime.

    \begin{proof}
        Suppose $a b \in \mathfrak{p}$ but $a \notin \mathfrak{p}$ and $b \notin \mathfrak{p}$. Then $\mathfrak{p}+(a)$ and $\mathfrak{p}+(b)$ are ideals properly containing $\mathfrak{p}$ and therefore finitely generated by maximality of $\mathfrak{p}$. Consequently $\mathfrak{p}+(a)=\left(p_1+r_1 a, \ldots, p_n+r_n a\right)$
        and
        $\mathfrak{p}+(b)=\left(p_1^{\prime}+r_1^{\prime} b, \ldots p_m^{\prime}+r_m^{\prime} b\right)$ for some $p_i, p_i^{\prime} \in \mathfrak{p}$ and $r_i, r_i^{\prime} \in R$.

        If $J=\left(\mathfrak{p}:a\right)=\{r \in R \mid r a \in \mathfrak{p}\}$, then $J$ is an ideal. Since $a b \in \mathfrak{p},\left(p_i{ }^{\prime}+r_i{ }^{\prime} b\right) a=p_i{ }^{\prime} a+r_i{ }^{\prime} a b \in \mathfrak{p}$ for all $i$, whence $\mathfrak{p} \subset\mathfrak{p}+(b) \subset J$. By maximality, $J$ is finitely generated, say $J=$ $\left(j_1, \ldots, j_k\right)$.

        If $x \in \mathfrak{p}$, then $x \in \mathfrak{p}+(a)$ and hence for some $s_i \in R, x=\sum_{i=1}^n s_i\left(p_i+r_i a\right)$ $=\sum_{i=1}^n s_i p_i+\sum_{i=1}^n s_i r_i a$. Consequently, $\left(\sum_i s_i r_i\right) a=x-\sum_i s_i p_i \in \mathfrak{p}$, whence $\sum_i s_i r_i \in J$. Thus for some $t_i \in R, \sum_{i=1}^n s_i r_i=\sum_{i=1}^k t_i j_i$ and $x=\sum_{i=1}^n s_i p_i+\sum_{i=1}^k t_i j_i a$. Therefore, $\mathfrak{p}$ is generated by $p_1, \ldots, p_n, j_1 a, \ldots, j_k a$, which is a contradiction. Thus $a \in \mathfrak{p}$ or $b \in \mathfrak{p}$ and $\mathfrak{p}$ is prime by
        \ref{thm: Prime ideal}
    \end{proof}
\end{proposition}


\begin{theorem} [I.S.Cohen]
    A commutative ring $R$ with identity is Noetherian if and only if every prime ideal of $R$ is finitely generated.

    \begin{proof}
        Let $\mathcal{S}$ be the set of all ideals of $R$ which are not finitely generated.
        If $\mathcal{S}$ is nonempty, then use Zorn's Lemma to find a maximal element $P$ of $S . P$ is prime by Proposition 2.4 and hence finitely generated by hypothesis.

        This is a contradiction unless $\delta=\varnothing$. Therefore, $R$ is Noetherian by Theorem 1.9.
    \end{proof}
\end{theorem}

















\chapter{Integral}
\section{Rings Extensions} % Rings Extensions
\begin{definition}
    Let $S$ be a commutative ring with identity and $R$ a subring of $S$ containing $1_S$.

    \begin{enumerate}
        \item
              Then $S$ is said to be an \textbf{extension ring} of $R$.
        \item
              An element $s\in S$ is said to be \textbf{integral} over $R$ if
              $s$ is a root of  a monic polynomial in $R[x]$.
        \item
              If every element of $S$ is integral over $R$, $S$ is said to be an \textbf{integral extension} of $R$.
        \item
              The \textbf{integral closure} of $R$ in $S$ is the set of elements of $S$ that are integral over $R$.
        \item
              The ring $R$ is said to be \textbf{integrally closed} in $S$ if $R$ is equal to its integral closure in $S$.

              The integral closure of an integral domain $R$ in its field of fractions is called the \textbf{normalization} of $R$. An integral domain is called integrally closed or normal if it is integrally closed in its field of fractions.
    \end{enumerate}
    \begin{remark}
        If follows from \cref{cor: integral extensions} that the integral closure of $R$ in $S$ is a subring of $S$ containing $R$.
    \end{remark}
\end{definition}



\begin{theorem}
    \label{thm: integral}
    Let $S$ be an extension ring of $R$ and $s \in S$. Then the following conditions are equivalent.
    \begin{enumerate}
        \item
              $s$ is integral over $R$
        \item
              Subring $R[s]$ is a finitely generated $R$-module
        \item
              There is a subring $T$ that $R[s]\subset T\subset S$, which is finitely generated as an $R$-module;
        \item
              There is a faithful $R[s]$-submodule $M$ which is finitely generated as an $R$-module.
    \end{enumerate}
\end{theorem}


\begin{corollary}
    \label{cor: integral extensions}
    Let $S$ be an extension ring of $R$. Then
    \begin{enumerate}
        \item
              If $S$ is finitely generated as an $R$-module, then $S$ is an integral extension of $R$.
        \item
              If $s_1, \ldots, s_{t} \in S$ are integral over $R$, then $R\left[s_1, \ldots, s_{t}\right]$ is a finitely generated $R$-module and an integral extension ring of $R$.
        \item
              If $T$ is an integral extension ring of $S$ and $S$ is an integral extension ring of $R$, then $T$ is an integral extension ring of $R$ .
    \end{enumerate}
    \begin{proof}
        It immediately follows from \ref{thm: integral}
    \end{proof}
    \begin{proof}
        We have a tower of extension rings:
        \begin{equation*}
            R \subset R\left[s_1\right] \subset R\left[s_1, s_2\right] \subset \cdots \subset R\left[s_1, \ldots, s_t\right]
        \end{equation*}
        For each $i, s_i$ is integral over $R$ and hence integral over $R\left[s_1, \ldots, s_{i-1}\right]$. Since $R\left[s_1, \ldots, s_i\right]=R\left[s_1, \ldots, s_{i-1}\right]\left[s_i\right], R\left[s_1, \ldots, s_i\right]$ is a finitely generated module over $R\left[s_1, \ldots, s_{i-1}\right]$ by \ref{thm: integral}.
        Thus $R\left[s_1, \ldots, s_n\right]$ is a finitely generated $R$-module, then $R\left[s_1, \ldots, s_n\right]$ is an integral extension ring of $R$ by 1.
    \end{proof}
    \begin{proof}
        $T$ is obviously an extension ring of $R$.
        If $t \in T$, then $t$ is integral over $S$ and therefore the root of some monic polynomial $f \in S[x]$, say $f=\sum_{i=0}^n s_i x^i$. Since $f$ is also a polynomial over the ring $R\left[s_0, s_1, \ldots, s_{n-1}\right], t$ is integral over $R\left[s_0, \ldots, s_{n-1}\right]$.

        By \ref{thm: integral} $R\left[s_0, \ldots, s_{n-1}\right][t]$ is a finitely generated $R\left[s_0, \ldots, s_{n-1}\right]$-module.
        But since $S$ is integral over $R, R\left[s_0, \ldots, s_{n-1}\right]$ is a finitely generated $R$-module by 2. Then
        \begin{equation*}
            R\left[s_0, \ldots, s_{n-1}\right][t]=R\left[s_0, \ldots, s_{n-1}, t\right]
        \end{equation*}
        is a finitely generated $R$-module. Since $R[t] \subset R\left[s_0, \ldots, s_{n-1}, t\right], t$ is integral over $R$ by \ref{thm: integral}.
    \end{proof}
\end{corollary}




\begin{proposition}
    \begin{enumerate}
        \item
              Every unique factorization domain is integrally closed.

        \item
              In particular, the polynomial ring $F\left[x_1, \ldots, x_n\right]$ ( $F$ a field) is integrally closed in its quotient field $F\left(x_1, \ldots, x_n\right)$.
    \end{enumerate}
\end{proposition}

\subsection{integral extension} % integral extension
\begin{theorem}
    Let $T$ be a multiplicative subset of an integral domain $R$ such that $0 \notin T$.
    If $R$ is integrally closed, then $T^{-1} R$ is an integrally closed integral domain.
    \begin{proof}
        $T^{-1} R$ is an integral domain and $R$ may be identified with a subring of $T^{-1} R$ by \ref{thm: ring of quotients }.
        Extending this identification, the quotient field $Q(R)$ of $R$ may be considered as a subfield of the quotient field $Q\left(T^{-1} R\right)$ of $T^{-1} R$. Verify that $Q(R)=Q\left(T^{-1} R\right)$.

        Let $u \in Q\left(T^{-1} R\right)$ be integral over $T^{-1} R$; then for some $r_i \in R$ and $s_i \in T$,
        \begin{equation*}
            u^n+\left(r_{n-1} / s_{n-1}\right) u^{n-1}+\cdots+\left(r_1 / s_1\right) u+\left(r_0 / s_0\right)=0 .
        \end{equation*}
        Multiply through this equation by $s^n$, where $s=s_0 s_1 \cdots s_{n-1} \in T$, and conclude that $su$ is integral over $R$. Since $s u \in Q\left(T^{-1} R\right)=Q(R)$ and $R$ is integrally closed, $s u \in R$. Therefore, $u=s u / s \in T^{-1} R$, whence $T^{-1} R$ is integrally closed.
    \end{proof}
\end{theorem}



\begin{theorem}
    Let $S$ be an integral extension ring of $R$. Then the following statements hold.
    \begin{enumerate}
        \item
              Assume that $S$ is an integral domain. Then $R$ is a field if and only if $S$ is a field.
        \item
              Let $\mathfrak{p}$ be a prime ideal in $R$. Then there is a prime ideal $\mathfrak{q}$ in $S$ with $\mathfrak{p}=\mathfrak{q} \cap R$.

              Moreover, $\mathfrak{p}$ is maximal if and only if $\mathfrak{q}$ is maximal.
        \item
              (The Going-up Theorem)
              Let $\mathfrak{p}_1 \subseteq \mathfrak{p}_2 \subseteq \cdots \subseteq \mathfrak{p}_n$ be a chain of prime ideals in $R$ and suppose there are prime ideals $\mathfrak{q}_1 \subseteq \mathfrak{q}_2 \subseteq \cdots \subseteq \mathfrak{q}_m$ of $S$ with $\mathfrak{p}_i=\mathfrak{q}_i \cap R, 1 \leq i \leq m$ and $m<n$. Then the ascending chain of ideals can be completed: there are prime ideals $\mathfrak{q}_{m+1} \subseteq \cdots \subseteq \mathfrak{q}_n$ in $S$ such that $\mathfrak{p}_i=\mathfrak{q}_i \cap R$ for all $i$.
    \end{enumerate}
\end{theorem}


\begin{theorem} [The Going-down Theorem]
    Assume that $S$ is an integral domain and $R$ is integrally closed in $S$. Let $\mathfrak{p}_1 \supseteq \mathfrak{p}_2 \supseteq \cdots \supseteq \mathfrak{p}_n$ be a chain of prime ideals in $R$ and suppose there are prime ideals $\mathfrak{q}_1 \supseteq \mathfrak{q}_2 \supseteq \cdots \supseteq \mathfrak{q}_m$ of $S$ with $\mathfrak{p}_i=\mathfrak{q}_i \cap R, 1 \leq i \leq m$ and $m<n$. Then the descending chain of ideals can be completed: there are prime ideals $\mathfrak{q}_{m+1} \supseteq \cdots \supseteq \mathfrak{q}_n$ in $S$ such that $\mathfrak{p}_i=\mathfrak{q}_i \cap R$ for all $i$.
\end{theorem}















\begin{theorem}
    \label{thm: Let S be an integral extension ring of R and let q be a prime ideal in S which
        lies over a prime ideal p in R . Then q is maximal in S if and only if p is maximal in R .}
    Let $S$ be an integral extension ring of $R$ and let $\mathfrak{q}$ be a prime ideal in $S$ which lies over a prime ideal $\mathfrak{p}$ in $R$ . Then $\mathfrak{q}$ is maximal in $S$ if and only if $\mathfrak{p}$ is maximal in $R$ .

    \begin{proof}
        Suppose $\mathfrak{q}$ is maximal in $S$. By \ref{thm: existence of maximal ideal} there is a maximal ideal $\mathfrak{m}$ of $R$ that contains $\mathfrak{p}$ and $\mathfrak{m}$ is prime by \ref{thm: maximal ideal is prime}.
        By \ref{cor: going-up Theorem} there is a prime ideal $\mathfrak{q}^{\prime}$ in $S$ such that $\mathfrak{q} \subset \mathfrak{q}^{\prime}$ and $\mathfrak{q}^{\prime}$ lies over $\mathfrak{m}$. Since $\mathfrak{q}^{\prime}$ is prime, $\mathfrak{q}^{\prime} \neq S$.
        The maximality of $\mathfrak{q}$ implies that $\mathfrak{q}=\mathfrak{q}^{\prime}$, whence $\mathfrak{p}=\mathfrak{q} \cap R=\mathfrak{q}^{\prime} \cap R=\mathfrak{m}$. Therefore, $\mathfrak{p}$ is maximal in $R$.

        Conversely suppose $\mathfrak{p}$ is maximal in $R$. Since $\mathfrak{q}$ is prime in $S, \mathfrak{q} \neq S$ and there is a maximal ideal $N$ of $S$ containing $\mathfrak{q}$ and $N$ is prime, whence $1_R=1_S \notin N$. Since $\mathfrak{p}=R \cap \mathfrak{q} \subset R \cap N \subset R$, we must have $\mathfrak{p}=R \cap N$ by maximality. Thus $\mathfrak{q}$ and $N$ both lie over $\mathfrak{p}$ and $\mathfrak{q} \subset N$. Therefore, $\mathfrak{q}=N$ by \ref{thm: Let S be an integral extension ring of R and let q be a prime ideal in S which lies over a prime ideal p in R . Then q is maximal in S if and only if p is maximal in R .} .
    \end{proof}
\end{theorem}



\section{Discrete Valuation Ring} % Discrete Valuation Ring
\begin{definition}
    The following conditions on a principal ideal domain are equivalent:
    \begin{enumerate}
        \item  $A$ has exactly one nonzero prime ideal;
        \item up to associates, $A$ has exactly one prime element;
        \item  $A$ is local and is not a field.
    \end{enumerate}
    A ring satisfying these conditions is called a \textbf{discrete valuation ring}.
\end{definition}

\begin{theorem}
    An integral domain $A$ is a discrete valuation ring if and only if
    \begin{enumerate}[label=(\roman*)]
        \item  $A$ is noetherian,
        \item  $A$ is integrally closed, and
        \item  $A$ has exactly one nonzero prime ideal.
    \end{enumerate}
\end{theorem}




\section{Dedekind Domain}
\begin{definition}
    A \textbf{Dedekind domain} is an integral domain $R$ satisfying the following equivalent conditions:
    \begin{enumerate}
        \item
              $R$ is Noetherian, integrally closed and has Krull dimension one (Every nonzero prime ideal of $R$ is maximal).
        \item
              Every nonzero ideal of $R$ is invertible
        \item
              Every finitely generated torsion-free $R$-module is free.
        \item
              the localization $R_{\mathfrak{p}}$ at each prime ideal $\mathfrak{p}$ of $R$ is a discrete valuation ring.
        \item
              Every nonzero proper ideal of $R$ can be written as a product of prime ideals of $R$, and this factorization is unique up to the order of the factors.
    \end{enumerate}
\end{definition}


\begin{proposition}
    Let $A$ be an integral domain, and let $S$ be a multiplicative subset of $A$.
    \begin{enumerate}
        \item If $A$ is noetherian, then so also is $S^{-1} A$.
        \item If $A$ is integrally closed, then so also is $S^{-1} A$.
        \item If $A$ has Krull dimension one, then so also does $S^{-1} A$.
        \item If $A$ is a Dedekind domain, then so also is $S^{-1} A$.
    \end{enumerate}
\end{proposition}

\begin{proposition}
    A noetherian integral domain $A$ is a Dedekind domain if and only if, for every nonzero prime ideal $\mathfrak{p}$ in $A$, the localization $A_{\mathfrak{p}}$ is a discrete valuation ring.
\end{proposition}



\subsection{Unique factorization of ideals} % Unique factorization of ideals
\begin{theorem}
    Let $A$ be a Dedekind domain.
    Every proper nonzero ideal $\mathfrak{a}$ of $A$ can be written in the form
    \begin{equation*}
        \mathfrak{a}
        =
        \mathfrak{p}_1^{r_1} \cdots \mathfrak{p}_n^{r_n}
    \end{equation*}
    with the $\mathfrak{p}_i$ distinct prime ideals and the $r_i>0$; the $\mathfrak{p}_i$ and the $r_i$ are uniquely determined.
\end{theorem}

\subsection{The ideal class group} % The ideal class group
\begin{definition}
    Let $R$ be an integral domain with quotient field $K$.
    A \textbf{fractional ideal} of $R$ is
    \begin{enumerate}[label=(\roman*)]
        \item a nonzero $R$-submodule $I$ of $K$
        \item there exists a nonzero $d \in R$ such that $d I \subset R$ i,e, $\left(R:I\right)\cap R \neq \varnothing$
    \end{enumerate}
\end{definition}




\begin{definition}
    If $R$ is an integral domain with quotient field $K$, then the set of all
    fractional ideals of $R$ forms a commutative monoid, with identity $R$ and multiplication given by
    \begin{equation*}
        I J
        =
        \left\{\sum_{i=1}^n a_i b_i \mid a_i \in I ; b_i \in J ; n \in \mathbb{Z}_{\geq 1}\right\}
    \end{equation*}
    A fractional ideal $I$ of an integral domain $R$ is said to be \textbf{invertible} if $I J=R$ for some fractional ideal $J$ of $R$.
\end{definition}


\begin{theorem}
    Let $A$ be a Dedekind domain.
    The set $\operatorname{Id}(A)$ of fractional ideals is a group; in fact, it is the free abelian group on the set of nonzero prime ideals.
\end{theorem}



\begin{definition}
    We define the \textbf{ideal class group} $\mathrm{Cl}(A)$ of $A$ to be the quotient $\mathrm{Cl}(A)=\operatorname{Id}(A) / \mathrm{P}(A)$ of $\operatorname{Id}(A)$ by the subgroup of principal ideals.
    The \textbf{class number} of $A$ is the order of $\operatorname{Cl}(A)$ (when finite).

    In the case that $A$ is the ring of integers $\mathcal{O}_K$ in a number field $K$, we often refer to $\mathrm{Cl}\left(\mathcal{O}_K\right)$ as the ideal class group of $K$, and its order as the class number of $K$.
\end{definition}

\begin{proposition}
    Let $R$ be an integral domain with quotient field $K$.
    \begin{enumerate}
        \item
              Indeed for any fractional ideal $I$ the set $I^{-1}=\{a \in K \mid a I \subset R\}$ is easily seen to be a fractional ideal such that $I^{-1} I=I I^{-1} \subset R$.
        \item
              The inverse of an invertible fractional ideal $I$ is unique and is $I^{-1}=\left\{a \in K \mid a I \subset R\right\}$.
              If $I$ is invertible and $I J=J I=R$, then clearly $J \subset I^{-1}$.
              Conversely, since $I^{-1}$ and $J$ are $R$-submodules of $K, I^{-1}=R I^{-1}=(J I) I^{-1}=J\left(I I^{-1}\right) \subset J R=R J \subset J$, whence $J=I^{-1}$.
        \item
              If $I, A, B$ are fractional ideals of $R$ such that $I A=I B$ and $I$ is invertible, then $A=R A=\left(I^{-1} I\right) A=I^{-1}(I B)=R B=B$.
        \item
              If $I$ is an ordinary ideal in $R$, then $R \subset I^{-1}$.
    \end{enumerate}
\end{proposition}

\begin{lemma}
    Let $I, I_1, I_2, \ldots, I_{n}$ be ideals in an integral domain R .
    \begin{enumerate}
        \item
              The ideal $I_1 I_2 \cdots I_{n}$ is invertible if and only if each $I_{j}$ is invertible.

        \item
              If $\mathfrak{p}_1 \cdots \mathfrak{p}_{m}=I=\mathfrak{q}_1 \cdots \mathfrak{q}_{n}$, where the $\mathfrak{p}_{i}$ and $\mathfrak{q}_{j}$ are prime ideals in $R$ and every $\mathfrak{p}_{i}$ is invertible, then $m=n$ and (after reindexing) $\mathfrak{p}_{i}=\mathfrak{q}_{i}$ for each $i=1, \ldots, m$.
    \end{enumerate}
\end{lemma}






\begin{lemma}
    If $I$ is a fractional ideal of an integral domain $R$ with quotient field $K$ and $f\in \operatorname{Hom}_{R}(I, R)$, then for all $a, b \in I: af(b)=bf(a)$.
\end{lemma}

\begin{lemma}
    Every invertible fractional ideal of an integral domain $R$ with quotient field $K$ is a finitely generated $R$-module.
\end{lemma}

\begin{theorem}
    Let $R$ be an integral domain and $I$ a fractional ideal of $R$.
    Then $I$ is invertible if and only if $I$ is a projective $R$-module.
\end{theorem}



\section{Discrete valuations}
\begin{definition}
    Let $K$ be a field.
    A \textbf{discrete valuation} on $K$ is a nonzero homomorphism $v: K^{\times} \rightarrow \mathbb{Z}$ such that $v(a+b) \geq \min (v(a), v(b))$.

    As $v$ is not the zero homomorphism, its image is a nonzero subgroup of $\mathbb{Z}$, and is therefore of the form $m \mathbb{Z}$ for some $m \in \mathbb{Z}$. If $m=1$, then $v: K^{\times} \rightarrow \mathbb{Z}$ is surjective, and $v$ is said to be \textbf{normalized}; otherwise, $x \mapsto m^{-1} \cdot v(x)$ will be a normalized discrete valuation.

    We extend $v$ to a map $K \rightarrow \mathbb{Z} \cup\{\infty\}$ by setting $v(0)=\infty$, where $\infty$ is a symbol $\geq n$ for all $n \in \mathbb{Z}$.
    \begin{remark}
        We have
        \begin{enumerate}
            \item $v(\zeta)=0$ for some $\zeta \in K^{\times}$
            \item $v(-a)=v(a)$ for all $a \in K$;
            \item $v(a+b) = \max \left\{v(a),v(b)\right\}$ if $v(a) \neq v(b)$.
        \end{enumerate}
        We often use "ord" rather than " $v$ " to denote a discrete valuation.
    \end{remark}
\end{definition}

\begin{proposition}
    Let $v$ be a discrete valuation on $K$, then
    \begin{equation*}
        A
        :=
        \{a \in K \mid v(a) \geq 0\}
    \end{equation*}
    is a principal ideal domain with maximal ideal
    \begin{equation*}
        \mathfrak{m}
        =
        \{a \in K \mid v(a)>0\}
    \end{equation*}
    If $v\left(K^{\times}\right)=m \mathbb{Z}$, then the ideal $\mathfrak{m}$ is generated by every element of $v^{-1}\left(m\right)$.
\end{proposition}


\begin{definition}
    Let $A$ be a Dedekind domain and let $\mathfrak{p}$ be a prime ideal in $A$. For any $c \in K^{\times}$, let $v(c)$ be the exponent of $\mathfrak{p}$ in the factorization of
    $(c)$.
    Then $v$ is a normalized discrete valuation on $K$, called the \textbf{discrete valuation associated to} $\mathfrak{p}$, denoted by $\operatorname{ord}_{\mathfrak{p}}$.
\end{definition}

\begin{proposition}
    Let $x_1, \ldots, x_m$ be elements of a Dedekind domain $A$, and let $\mathfrak{p}_1, \ldots, \mathfrak{p}_m$ be distinct prime ideals of $A$. For every integer $n$, there is an $x \in A$ such that
    \begin{equation*}
        \operatorname{ord}_{\mathfrak{p}_i}\left(x-x_i\right)>n, \quad i=1,2, \ldots, m .
    \end{equation*}
\end{proposition}


\section{Integral closures of Dedekind domains}

\begin{theorem}
    Let $A$ be a Dedekind domain with field of fractions $K$ and $L/K$ be a finite separable extension, then the integral closure of $A$ in $L$ is Dedekind domain.
\end{theorem}


\begin{definition}
    Let $A$ be a Dedekind domain with field of fractions $K$, and let $B$ be the integral closure of $A$ in a finite separable extension $L$ of $K$.
    A prime ideal $\mathfrak{p}$ of $A$ will factor in $B$,
    \begin{equation*}
        \mathfrak{p} B=\mathfrak{P}_1^{e_1} \cdots \mathfrak{P}_g^{e_g}
    \end{equation*}
    wher $\mathfrak{P}$ are distinct prime ideals in $B$ and $e_i\geq 1$,
    \begin{enumerate}
        \item
              If any of the numbers is $>1$, then we say that $\mathfrak{p}$ is \textbf{ramified} in $B$ (or $L$ ).
              The number $e_i$ is called the \textbf{ramification index} .

        \item
              We say $\mathfrak{P}$ divides $\mathfrak{p}$,written $\mathfrak{P} \mid \mathfrak{p}$, if $\mathfrak{P}$ occurs in the factorization of $\mathfrak{p}$ in $B$.
    \end{enumerate}
    We then write $e(\mathfrak{P} / \mathfrak{p})$ for the ramification index and $f(\mathfrak{P} / \mathfrak{p})$ for the degree of the field extension $[B / \mathfrak{P}: A / \mathfrak{p}]$ (called the \textbf{residue class degree}).
    \begin{enumerate}[resume]
        \item
              $\mathfrak{p}$ is said to \textbf{split} (or split completely) in $L$ if $e_i=f_i=1$ for all $i$

        \item
              $\mathfrak{p}$ is said to be \textbf{inert} in $L$ if $\mathfrak{p} B$ is a prime ideal (so $g=1=e$ ).
    \end{enumerate}
\end{definition}


\begin{theorem}
    Let $m$ be the degree of $L$ over $K$, and let $\mathfrak{P}_1, \ldots, \mathfrak{P}_g$ be the prime ideals dividing $\mathfrak{p}$; then
    \begin{equation*}
        \sum_{i=1}^g e_i f_i=m
    \end{equation*}
    where $e_i=e\left(\mathfrak{P}_i / \mathfrak{p}\right)$ and $f_i=f\left(\mathfrak{P}_i / \mathfrak{p}\right)$. If $L$ is Galois over $K$, then all the ramification numbers are equal, and all the residue class degrees are equal, and so

    \begin{equation*}
        e f g=m .
    \end{equation*}
\end{theorem}























\chapter{Completions} % Completions
\minitoc

\section{Basic Definitions}
\subsection{Graded ring} % Graded
\begin{definition}
    A \textbf{graded ring} is a ring $A$ together with a family $\left\{A_n\right\}_{n \geq 0}$ of subgroups of $A$, such that
    \begin{enumerate}[label=(\roman*)]
        \item
              $ A=\bigoplus_{n=0}^{\infty} A_n$ in $\Ab$
        \item
              and $A_m A_n \subseteq A_{m+n}$ for all $m, n \geq 0$.
    \end{enumerate}
    Thus $A_0$ must be a subring of $A$ containing $1_A$, and $A$ is a $A_0$-algebra.
    $A$ is saied to be \textbf{standard graded ring} if
    \begin{enumerate}[label=(\roman*), resume]
        \item $A=A_0[A_1]$ as a $A_0$-algebra.
    \end{enumerate}
\end{definition}

\begin{proposition}
    \label{pro:Noetherian of graded ring}
    Let $A=\bigoplus_{n\geq 0} A_n$ be a graded ring. Then the following statements are equivalent:
    \begin{enumerate}
        \item
              $A$ is a Notetherian ring.
        \item
              $A_0$ is Noetherian and $A$ is finitely generated as $A_0$-algebra ($A_0$-ring)
    \end{enumerate}
\end{proposition}


\begin{definition}
    Let $A=\bigoplus A_n$ be a graded ring.
    \begin{enumerate}
        \item
              A ideal (left, right, two-sided) $I$ of $A$ is called \textbf{graded ideal} if $I=\bigoplus_{n\geq 0} (A_n \cap I)$ in $\Ab$.
        \item
              If $I$ is a two-sided graded ideal of $A$, then quotient graded ring $A/I=\bigoplus_{i\geq 0} A_i/\left(A_i\cap I\right)$ is well-defined, called the \textbf{quotient graded ring} of $A$ by $I$.
    \end{enumerate}
    \begin{remark}
        It clear that a ideal $I$ is graded if and only if it is generated by homogeneous elements.
    \end{remark}
\end{definition}

\begin{definition}
    Let $A$ and $B$ be graded rings.
    \begin{enumerate}
        \item
              A \textbf{graded ring homomorphism} from $A$ to $B$ is a ring homomorphism $f: A \rightarrow B$ such that $f\left(A_n\right) \subseteq B_n$ for all $n \geqslant 0$.
        \item
              the \textbf{kernel} of a graded ring homomorphism is a two-sided graded ideal.
    \end{enumerate}
\end{definition}


\subsection{Graded modules}
\begin{definition}
    Let $A=\bigoplus_{n\geq 0} A_n$ be a graded ring.
    A \textbf{graded left $A$-module} is a left $A$-module $M$ together with a family $\left\{M_n\right\}_{n \geqslant 0}$ of subgroups of $M$ such that
    \begin{enumerate}[label=(\roman*)]
        \item
              $M=\bigoplus_{n=0}^{\infty} M_n$ as an abelian group
        \item
              $A_m M_n \subseteq M_{m+n}$ for all $m, n \geqslant 0$.
    \end{enumerate}
    Elements of $M_n$ are called \textbf{homogeneous elements of degree $n$}.
    \begin{remark}
        Thus each $M_n$ is an $A_0$-module.
    \end{remark}
\end{definition}


\begin{definition}
    Let $A$ be a graded ring and $M, N$ graded $A$-modules.
    \begin{enumerate}
        \item
              A \textbf{graded submodule} of $M$ is a submodule $N$ of $M$ such that $N=\bigoplus_{n \geq 0}\left(N \cap M_n\right)$ in $\Ab$.
        \item
              then the $M / N=\bigoplus_{n \geq 0} M_n /\left(N \cap M_n\right)$ is well-defined, called the \textbf{quotient graded module} of $M$ by $N$.
        \item
              A \textbf{graded $A$-homomorphism} from $M$ to $N$ is an $A$-module homomorphism $f: M \rightarrow N$ such that $f\left(M_n\right) \subseteq N_n$ for all $n \geqslant 0$.
    \end{enumerate}
\end{definition}

\subsection{Graded algebra} % Graded Algebra
\begin{definition}
    Let $k$ be a commutative ring (usually a field).
    A \textbf{graded $k$-algebra} is $k$-algebra $A$ such that
    \begin{enumerate}[label=(\roman*)]
        \item
              $A=\bigoplus_{n \geq 0} A_n$ is a graded ring.
        \item
              the structure morphism $\varphi : k \rightarrow A$ with $\Im(\varphi)\subset A_0$ (scalar action preserves degrees).
    \end{enumerate}
    \begin{remark}
        In this case, $A_0$ is naturally a $k$-algebra, and $A$ may be viewed as a graded $A_0$-algebra with base ring $A_0$ instead of $k$.
        In many situations one further assumes that the base ring $k=A_0$..
    \end{remark}
\end{definition}


We give a graded ring of particular importance in commutative algebra.
\begin{definition}
    Let $k$ be commutative ring (usually a field).
    A \textbf{standard graded $k$-algebra} is a graded $k$-algebra $A=\bigoplus_{n \geq 0} A_n$ such that
    $A$ is generated by $A_1$ as a $k$-algebra.
\end{definition}



\section{Filtration and Associated Graded Rings and Modules} % Filtration and Associated Graded Rings and Modules
\subsection{Increasing filtration} % Filtered and Graded Modules

\begin{definition}
    Let $A$ be a ring.
    A \textbf{increasing filtration} of $A$ is a sequence of subgroups $\left\{F_iA\right\}$ of $A$ such that
    \begin{enumerate}[label=(\roman*)]
        \item
              $F_0A \subset F_1A \subset F_2A \subset \cdots \subset F_nA \subset \cdots \subset A$
        \item
              $\bigcup_{i\geq 0} F_iA=A$.
        \item
              $F_iA \cdot F_jA \subseteq F_{i+j}A$ for all $i,j$.
    \end{enumerate}
    $A$ is called a \textbf{filtered ring} if it has an increasing filtration.
\end{definition}

\begin{proposition}
    Let $A$ be a ring.
    \begin{enumerate}
        \item
              If $A=\bigoplus_{i\geq 0} A_i $ is a graded ring, then the sequence of subgroups $\left\{F_kA:=\bigoplus_{i=0}^k A_i\right\}$ is an increasing filtration of $A$.
        \item
              If $\mathcal{F}=\left\{F_iA\right\}$ is a filtration of $A$, then
              \begin{equation*}
                  \operatorname{gr}_{\mathcal{F}}A
                  :=
                  \bigoplus_{i\geq 0} F_{i+1}A / F_iA
              \end{equation*}
              is a graded ring (multiplication follows from $A$), called the \textbf{associated graded ring of $A$ associated with the filtration $\mathcal{F}$} .
    \end{enumerate}
\end{proposition}


\begin{definition}
    Let $A$ be a filtered ring with increasing filtration $\mathcal{F}=\left\{F_iA\right\}$. A left $A$-module $M$ is called a \textbf{filtered left $A$-module} if it has a sequence of subgroups $\Gamma=\left\{\Gamma_iM\right\}$ such that
    \begin{enumerate}[label=(\roman*)]
        \item
              $\Gamma_0M \subset \Gamma_1M \subset \Gamma_2M \subset \cdots \subset \Gamma_nM \subset \cdots \subset M$
        \item
              $\bigcup_{i\geq 0} \Gamma_iM = M$
        \item
              $F_iA  \cdot \Gamma_jM \subseteq \Gamma_{i+j}M$ for all $i,j$.
    \end{enumerate}
\end{definition}



\begin{proposition}
    Let $M$ be a left $A$-module where $A$ is a filtered ring with increasing filtration $\mathcal{F}=\left\{F_iA\right\}$.
    \begin{enumerate}
        \item
              If $M=\bigoplus_{i\geq 0} M_i$ is a graded $A$-module, then the sequence of subgroups $\Gamma_kM:=\bigoplus_{i=0}^k M_i$ is an increasing filtration of $M$.
        \item
              If $\Gamma=\left\{\Gamma_iM\right\}$ is a filtration of $M$, then
              The \textbf{graded module of $M$ associated with the filtration $\Gamma$} is defined by
              \begin{equation*}
                  \operatorname{gr}_{\Gamma}M
                  =
                  \bigoplus_{i\geq 0} \Gamma_{i+1}M / \Gamma_i M
              \end{equation*}
              which is a graded $\operatorname{gr}_{\mathcal{F}}A$-module.
    \end{enumerate}
    \begin{remark}
        In (1), the  $\operatorname{gr}_{\mathcal{F}}M\cong M$ (as graded modules).

        But in (2), let $A=\mathbb{Z}$ with trivial filtration and $M=\mathbb{Z}_{p^2}$ with filtration $F_0M=0, F_1M=p \mathbb{Z}_{p^2}, F_2M=M,\ldots$, then $\operatorname{gr}_{\mathcal{F}}M \cong \mathbb{Z}_p^2\not \cong M$ in ${}_\mathbb{Z}\Mod$.

        The functor $\operatorname{gr}(-):  {}_R \mathbf{FiltMod}\rightarrow {}_R\mathbf{GrMod}$ is not faithful
    \end{remark}
\end{proposition}


\section{Good filtration}
\begin{definition}
    Let $A$ be a filtered ring with an increasing filtration $\left\{F_nA\right\}$, and $M$ a filtered $A$-module with an increasing filtration $\left\{F_nM\right\}$.
    The filtration of $M$ is called a \textbf{good filtration} if $\operatorname{gr}(M)$ is finitely generated over $\operatorname{gr}(A)$.
\end{definition}
\begin{theorem}
    A
\end{theorem}

\section{Induced filtrations}
\begin{definition}
    Let $A$ be a filtered ring with filtration $\mathcal{F}=\left\{F_iA\right\}$ and $M$ be a left filtered $A$-module with filtration $\Gamma=\left\{\Gamma_i\right\}$.
    \begin{enumerate}
        \item
              If $N$ is a submodule of $M$, then the \textbf{induced filtration} on $N$ is a filtration $\Gamma^\prime=\left\{\Gamma_iN\right\}$ defined by
              \begin{equation*}
                  \Gamma_i^\prime 
                  :=
                  N \cap \Gamma_i .
              \end{equation*}
              \begin{remark}
                  associated graded module $\operatorname{gr}_{\Gamma^\prime}N$ of $N$ and graded module homomorphism $i_k:N \cap \Gamma_{k}/N \cap \Gamma_{k-1}\rightarrow \Gamma_k/\Gamma_{k-1}$
              \end{remark}
        \item
              the \textbf{quotient filtration} $\Gamma^{\prime\prime}=\left\{\Gamma_i^{\prime\prime}\right\}$ on $M / N$ is defined by
              \begin{equation*}
                  \Gamma_i^{\prime\prime}
                  :=
                  \Gamma_i /(N \cap \Gamma_i) .
              \end{equation*}
              \begin{remark}
                  associated graded module $\operatorname{gr}_{\Gamma^{\prime\prime}}(M / N)$ of $M / N$ and graded module homomorphism $\pi_k:\Gamma_kM/\Gamma_{k-1}M \rightarrow \left(\Gamma_kM /(N \cap \Gamma_kM)\right) /\left(\Gamma_{k-1}M /(N \cap \Gamma_{k-1}M)\right) $ (note that the right hand side is isomorphic to $\Gamma_kM /\left(\Gamma_{k-1}M+\Gamma_kM\cap N\right)$, $\left(A_1/B_1\right)/\left(A_2/B_2\right)\cong A_1/\left(B_1+A_2\right)$)
              \end{remark}
    \end{enumerate}
\end{definition}


\begin{proposition}
    Then there exists a short exact sequence of graded $\operatorname{gr}_{\mathcal{F}}R$-modules
    \begin{equation*}
        0 \longrightarrow \operatorname{gr}_{\Gamma^\prime}N \xrightarrow{i} \operatorname{gr}_{\Gamma}M \xrightarrow{\pi} \operatorname{gr}_{\Gamma^{\prime\prime}}(M / N) \longrightarrow 0 .
    \end{equation*}
    is exact.
    \begin{proof}
        It is clear that the each
        \begin{equation*}
            \begin{tikzcd}
                0 \arrow[r]&  \Gamma_k^\prime  \cap \Gamma_{k-1}^\prime \arrow[r,"i_k"] & \Gamma_k / \Gamma_{k-1} \arrow[r,"\pi_k"] & \Gamma_k^{\prime\prime} /\Gamma_{k-1}^{\prime\prime} \arrow[r]& 0
            \end{tikzcd}
        \end{equation*}
        is exact in $\Ab$. Thus the proposition follows.
    \end{proof}
\end{proposition}





\section{General definitions}
\subsection{}
\begin{definition}
    Let $A$ be a ring, $\mathfrak{a}$ an ideal of $A$ and $M$ an $A$-module.
    \begin{enumerate}
        \item
              A \textbf{(increasing) filtration} of $M$ one means an sequence of submodules $\mathcal{F}=\left\{F_iM\right\}$
              \begin{equation*}
                  \cdots \subset F_{n+1}M\subset F_{n}M \subset F_{n+1}M \subset \cdots \subset M
              \end{equation*}

              \begin{remark}
                  A descending filtration of $M$ one means a sequence of submodules
                  \begin{equation*}
                      M=F^0M \supset F^1M \supset F^2M \supset \cdots \supset F^nM \supset \cdots
                  \end{equation*}
                  A increasing filtration of $M$ one means a sequence of submodules
                  \begin{equation*}
                      F_0M \subset F_1M \subset F_2M \subset \cdots \subset F_nM \subset \cdots \subset M
                  \end{equation*}
                  with union $\bigcup_{n=0}^{\infty} F_nM=M$.
              \end{remark}
    \end{enumerate}
    In this chapter, we shall only consider descending filtrations.
    \begin{enumerate}[resume]
        \item
              We say that it is an \textbf{$\mathfrak{a}$-filtration} if $\mathfrak{a} F^nM \subset F^{n+1}M$ for all $n$.


        \item
              We say that an $\mathfrak{a}$-filtration is \textbf{$\mathfrak{a}$-stable} if we have $\mathfrak{a} F^nM=F^{n+1}M$ for all $n$ sufficiently large.
    \end{enumerate}
\end{definition}


\begin{lemma}
    If $\left(M_n\right),\left(M_n^{\prime}\right)$ are stable a-filtrations of $M$, then they have bounded difference: that is, there exists an integer $n_0$ such that $M_{n+n_0} \subseteq M_n^{\prime}$ and $M_{n+n_0}^{\prime} \subseteq M_n$ for all $n \geqslant 0$.
    \begin{proof}
        Enough to take $M_n^{\prime}=\mathfrak{a}^n M$. Since $\mathfrak{a} M_n \subseteq M_{n+1}$ for all $n$, we have $\mathfrak{a}^n M \subseteq M_n$; also $\mathfrak{a} M_n=M_{n+1}$ for all $n \geqslant n_0$ say, hence $M_{n+n_0}=\mathfrak{a}^n M_{n_0} \subseteq \mathfrak{a}^n M$.
    \end{proof}
\end{lemma}



\subsection{}


\begin{definition}
    Let $A$ be a ring, ideal $\mathfrak{a}$ and $A$-module $M$ filtered by $\mathfrak{a}$-filtration $\left\{M_n\right\}$.
    \begin{enumerate}
        \item
              We can form a \textbf{first associated graded ring}
              \begin{equation*}
                  S=S_{\mathfrak{a}}\left(A\right)
                  =
                  \bigoplus_{n=0}^{\infty} \mathfrak{a}^n
              \end{equation*}
              $\mathfrak{a}^0=A$.
              \begin{remark}
                  It is also a $A$-algebra called \textbf{Rees algebra}, with the homomorphism $A \rightarrow S_{\mathfrak{a}}\left(A\right)$ defined by $a \mapsto \left(a,0,0,\ldots\right)$.
              \end{remark}

        \item
              Then $M_S=\bigoplus_n M_n$ is a graded $S_{\mathfrak{a}}\left(A\right)$-module.
    \end{enumerate}
    \begin{remark}
        If $A$ is Noetherian, $\mathfrak{a}$ is generated by $x_1, \ldots, x_r$; then $S_{\mathfrak{a}}\left(A\right)= A\left[x_1, \ldots, x_r\right]$ and is Noetherian.
    \end{remark}
\end{definition}


\begin{lemma}
    Let $A$ be a Noetherian ring, ideal $\mathfrak{a}$, and $M$ a finitely generated module, with an $\mathfrak{a}$-filtration.
    Then $M_S$ is finite over $S_{\mathfrak{a}}\left(A\right)$ if and only if the filtration of $M$ is $\mathfrak{a}$-stable.
\end{lemma}

\begin{theorem}[Artin-Rees]
    Let $A$ be a Noetherian ring, $\mathfrak{a}$ an ideal, $M$ a finite $A$-module with a stable $\mathfrak{a}$-filtration.
    Let $N$ be a submodule, and let $N_n=N \cap M_n$. Then $\left\{N_n\right\}$ is a stable $\mathfrak{a}$-filtration of $N$.
\end{theorem}

\begin{corollary}
    Let $A$ be a Noetherian ring, $M$ a finite $A$-module, and $N$ a submodule.
    Let $\mathfrak{a}$ be an ideal. There exists an integer $s$ such that for all integers $n \geq s$ we have
    \begin{equation*}
        \mathfrak{a}^n M \cap N=\mathfrak{a}^{n-s}\left(\mathfrak{a}^s M \cap N\right)
    \end{equation*}
\end{corollary}

\subsection{Second associated graded ring}
\begin{definition}
    Let $A$ be a ring, $\mathfrak{a}$ an ideal and $M$ an $A$-module with an $\mathfrak{a}$-filtration $\left\{M_n\right\}$.
    \begin{enumerate}
        \item
              We define the \textbf{second associated graded ring}
              \begin{equation*}
                  \operatorname{gr}_{\mathfrak{a}}(A)=\bigoplus_{n=0}^{\infty} \mathfrak{a}^n / \mathfrak{a}^{n+1} .
              \end{equation*}
              where $\mathfrak{a}^0=A$.
        \item
              We define
              \begin{equation*}
                  \operatorname{gr}(M)=\bigoplus_{n=0}^{\infty} M_n / M_{n+1}
              \end{equation*}
              then $\operatorname{gr}(M)$ is a graded $\operatorname{gr}_{\mathfrak{a}}(A)$-module.
    \end{enumerate}
\end{definition}

\begin{proposition}
    Let $A$ be a Noetherian ring and $\mathfrak{a}$ an ideal. Then
    \begin{enumerate}
        \item
              $\operatorname{gr}_{\mathfrak{a}}(A)$ is Noetherian.
        \item
              If $M$ is a finitely generated $A$-module with a stable $\mathfrak{a}$-filtration, then $\operatorname{gr}(M)$ is a finitely generated graded $\operatorname{gr}_{\mathfrak{a}}(A)$-module.
    \end{enumerate}
\end{proposition}



\chapter{Dimension}


\section{Hilbert Functions} % Hilbert Function

\begin{definition}
    Let $\mathcal{A}$ be a abelian category. A \textbf{additive function} $\lambda: \mathcal{A} \rightarrow \Ab$ is a covariant functor such that for every short exact sequence
    \begin{equation*}
        0 \rightarrow M^\prime \rightarrow M \rightarrow M^{\prime\prime} \rightarrow 0
    \end{equation*}
    in $\mathcal{A}$, we have
    \begin{equation*}
        \lambda(M)=\lambda(M^\prime)+\lambda(M^{\prime\prime})
    \end{equation*}
\end{definition}

In the following, $\lambda(N)$ is a finitely generated abelian group and we always view $\lambda(N)$ as the rank of $\lambda(N)$.

\begin{definition}% Hilbert Function
    Let $A=\bigoplus_{n\geq 0} A_n$ be a commutative Noetherian graded ring, $M=\bigoplus_{n\geq 0} M_n$ a finitely generated graded $A$-module and assume that each $M_n$ is a finitely generated $A_0$-module with a additive function $\lambda$.
    \begin{enumerate}
        \item
              The \textbf{Hilbert function} of $M$ with respect to $\lambda$ is the function defined by
              \begin{equation*}
                  h_M^\lambda(n)
                  :=
                  \lambda\left(M_n\right):\mathbb{N} \rightarrow \mathbb{N}
              \end{equation*}
        \item
              the \textbf{Hilbert-Poincare series} of $M$
              \begin{equation*}
                  P_M^\lambda(t)
                  :=
                  \sum_{n=0}^{\infty} h_M^\lambda(n) t^n
              \end{equation*}
    \end{enumerate}
    \begin{remark}
        If $R_0$ is Artinian ring, $\lambda(N)=\ell_{R_0}(N)$. If $R_0$ is a field, $\lambda(N)=\dim_{R_0}(N)$.
    \end{remark}
\end{definition}



\begin{theorem}[Hilbert-Serre]
    The Poincare series $P_M^\lambda(t)$ is a rational function of the form
    \begin{equation*}
        \frac{f(t)}{\prod_{i=1}^s\left(1-t^{k_i}\right)}
    \end{equation*}
    , where $f(t) \in \mathbb{Z}[t]$ and $s$ be the number of generators of ${}_{R_0} R$  and $k_i$ are the degrees of the homomgeneous generators $x_i$ respectively.
    \begin{proof}
        By induction on $s$, the number of generators of $R$ over $R_0$. Start with $s=0$; this means that $R_n=0$ for all $n>0$, so that $R=R_0$ and $M$ is a finitelygenerated $R_0$ module, hence $M_n=0$ for all large $n$. Thus $P_M^\lambda(t)$ is a polynomial in this case.

        Now suppose $s>0$ and the theorem true for $s-1$. Multiplication by $x_s$ is an $R$-module homomorphism of $M_n$ into $M_{n+k_s}$, hence it gives an exact sequence, say

        \begin{equation*}
            0 \rightarrow K_n \rightarrow M_n \xrightarrow{x_s} M_{n+k_s} \rightarrow L_{n+k_s} \rightarrow 0 .
        \end{equation*}


        Let $K=\oplus_n K_n, L=\oplus_n L_n$; these are both finitely-generated $A$-modules (because $K$ is a submodule and $L$ a quotient module of $M$ ), and both are annihilated by $x_s$, hence they are $A_0\left[x_1, \ldots, x_{s-1}\right]$-modules. Applying $\lambda$ to (1) we have, by (2.11)

        \begin{equation*}
            \lambda\left(K_n\right)-\lambda\left(M_n\right)+\lambda\left(M_{n+k_s}\right)-\lambda\left(L_{n+k_s}\right)=0 ;
        \end{equation*}

        multiplying by $t^{n+k_s}$ and summing with respect to $n$ we get

        \begin{equation*}
            \left(1-t^{k_s}\right) P(M, t)=P(L, t)-t^{k_s} P(K, t)+g(t)
        \end{equation*}

        where $g(t)$ is a polynomial. Applying the inductive hypothesis the result now follows.
    \end{proof}
\end{theorem}

\begin{corollary}[Existence of the Hilbert polynomial]
    Then there exists a polynomial
    $H(t)\in\mathbb{Q}[t]$ of degree $\dim M$ such that $H(n)=h_M^\lambda(n)$ for all sufficiently large integers $n$.

    This polynomial is called the \textbf{Hilbert polynomial} of $M$ with respect to $\lambda$.
\end{corollary}

\begin{corollary}
    There exists a $\chi(t,A,M)\in \mathbb{Q}[t]$ of degree $\dim M$ such that
    \begin{equation*}
        \sum_{i=0}^n \lambda\left(M_{i}\right)
        =
        \chi(n)
    \end{equation*}
    for all sufficiently large integers $n$.
\end{corollary}

\begin{definition}
    The \textbf{multiplicity} of $M$ with respect to $\lambda$ is defined to be
    \begin{equation*}
        e(M, \lambda)
        :=
        d! \cdot a_{d(M)}
    \end{equation*}
    where $a_{d(M)}$ is the leading coefficient of the Hilbert polynomial $\chi(t)$.
\end{definition}

\section{Krull Dimension}
In this section, all rings $A$ are commutative Noetherian ring with identity and modules $M$ are finitely generated $A$-module. Although some definitions and results hold in greater generality, the most of useful properties stay in this case.
\begin{definition}
    Let $A$-module $M$.
    \begin{enumerate}
        \item
              The \textbf{Krull dimension} of $A$ is defined to be the supremum of the lengths $n$ of all chains of prime ideals
              \begin{equation*}
                  \mathfrak{p}_0 \subsetneq \mathfrak{p}_1 \subsetneq \cdots \subsetneq \mathfrak{p}_n
              \end{equation*}
              say $\dim(A)=n$ if the supremum is attained, and $\dim(A)=\infty$ otherwise.
        \item
              The \textbf{Krull dimension} of $A$-module $M$ is defined to be the dimension of the quotient ring $A/\operatorname{Ann}(M)$, that is,
              \begin{equation*}
                  \dim(M)
                  :=
                  \dim\left(A/\operatorname{Ann}(M)\right)
              \end{equation*}
    \end{enumerate}
\end{definition}

\begin{proposition}
    .
    \begin{enumerate}
        \item $\dim A = \dim \Spec A$
        \item If $A$ is a local ring with maximal ideal $\mathfrak{m}$, then $\dim A=\sup\left\{n:\mathfrak{p}_0 \subsetneq \mathfrak{p}_1 \subsetneq \cdots \subsetneq \mathfrak{p}_n=\mathfrak{m}\right\}$
        \item if $S$ is a multiplicatively closed subset of $A$, then $\dim S^{-1} A \leq \dim A$
    \end{enumerate}
\end{proposition}

\begin{definition}
    Let $A$ be a commutative ring and $\mathfrak{p}$ a prime ideal of $A$.
    The \textbf{height} of $\mathfrak{p}$ is defined to be
    \begin{equation*}
        \operatorname{ht}(\mathfrak{p})
        :=
        \sup\left\{n: \mathfrak{p}_0 \subsetneq \mathfrak{p}_1 \subsetneq \cdots \subsetneq \mathfrak{p}_n=\mathfrak{p}\right\}
    \end{equation*}
    Thus $\dim A = \sup_{\mathfrak{p} \in \Spec A} \operatorname{ht}(\mathfrak{p})$.
\end{definition}


\begin{theorem}[Krull Principal Ideal Theorem]
    Let $A$ be a Noetherian commutative ring and a ideal $\mathfrak{a}=\left(f_1,\ldots,f_r\right)$.
    If $\mathfrak{p}$ is a minimal element of $\left\{\mathfrak{q}\in \Spec A :\mathfrak{a} \subset \mathfrak{q}\right\}$.
    Then $\operatorname{ht} \left(\mathfrak{p}\right) \leq r$.
\end{theorem}

\section{Weyl Algebraa}
























































































