\documentclass[12pt, oneside]{book}

\usepackage{../mypackages}





\begin{document}
\frontmatter
\title{{\Huge{\textbf{Homological Algebra}}}}
\maketitle

\dominitoc % 初始化minitoc
\pagenumbering{Roman}
\tableofcontents % 主目录


\mainmatter
\pagenumbering{arabic} % 正文编页码字体 


\chapter{Abelian Categories}
\minitoc

\section{Some notion}
\begin{definition}
    Let $\mathcal{C}$ be a category.
    \begin{enumerate}
        \item
              A \textbf{zero object} $0$ in $\mathcal{C}$ is an object that is both initial and terminal.
        \item
              the \textbf{zero morphism} from object $A$ to object $B$ is the unique morphism $0_{A,B}: A \rightarrow 0 \rightarrow B$ that factors through the zero object.
    \end{enumerate}
\end{definition}








\begin{definition}
    Let $\mathcal{C}$ be a category with zero object and morphism $f: A \rightarrow B$, then
    \begin{enumerate}
        \item
              The \textbf{kernel} of $f$ is an equalizer of $f$ and $0_{A,B}$.

        \item
              its \textbf{image} of $f$ is $\Ker\left( \Coker\left( f \right) \right)$, that is,
              $\Im\left( f \right)=   $
        \item
              its \textbf{coimage} of $f$ is $\Coker\left( \Ker\left( f \right) \right)$
    \end{enumerate}

\end{definition}

\section{Additive Categories}

\begin{definition}
    A category $\mathcal{C}$ is \textbf{additive} if
    \begin{enumerate}[label=(\roman*)]
        \item
              $\operatorname{Hom}(A, B)$ is an (additive) abelian group for every $A, B \in \operatorname{obj}(\mathcal{C})$,
        \item
              the distributive laws hold: given morphisms
              \begin{equation*}
                  X \xrightarrow{a} A \underset{g}{\stackrel{f}{\rightrightarrows}} B \xrightarrow{b} Y,
              \end{equation*}
              where $X$ and $Y \in \operatorname{obj}(\mathcal{C})$, then
              \begin{equation*}
                  b(f+g)=b f+b g \quad \text { and } \quad(f+g) a=f a+g a
              \end{equation*}
        \item $\mathcal{C}$ has a zero object $0$,
        \item
              $\mathcal{C}$ has finite products and finite coproducts: for all objects $A, B$ in $\mathcal{C}$, both $A \prod B$ and $A \coprod B$ exist in $\operatorname{obj}(\mathcal{C})$.
    \end{enumerate}
    \begin{remark}
        In an additive category, the product and coproduct of two objects $A$ and $B$ coincide up to isomorphism, called the \textbf{direct sum} and denoted by $A \oplus B$.
    \end{remark}
\end{definition}

\begin{definition}
    If $\mathcal{C}$ and $\mathcal{D}$ are additive categories, a functor $T: \mathcal{C} \rightarrow \mathcal{D}$ (of either variance) is called \textbf{additive functor} if, for all $A, B$ and all $f, g \in \operatorname{Hom}(A, B)$, we have
    \begin{equation*}
        T(f+g)=T f+T g ;
    \end{equation*}
    \begin{remark}
        That is, the function $\operatorname{Hom}_{\mathcal{C}}(A, B) \rightarrow \operatorname{Hom}_{\mathcal{D}}(T A, T B)$, given by $f \mapsto T f$, is a homomorphism of abelian groups.

        Of course, if $T$ is an additive functor, then $T(0)=0$, where 0 is either a zero object or a zero morphism.
    \end{remark}
\end{definition}


\begin{proposition}
    Let $\mathcal{C}$ be an additive category, and let $A, A_1, A_2 \in \operatorname{obj}(\mathcal{C})$. Then $A \cong A_1 \prod A_2$ if and only if there are morphisms $\iota_1: A_1 \rightarrow A, \iota_2: A_2 \rightarrow A$, $p_1: A \rightarrow A_1$, and $p_2: A \rightarrow A_2$ such that
    \begin{equation*}
        p_1 \iota_1=1_{A_1}, \quad p_2 \iota_2=1_{A_2}, \quad p_1 \iota_2=0, \quad p_2 \iota_1=0, \quad \text { and } \quad \iota_1 p_1+\iota_2 p_2=1_A .
    \end{equation*}
    Moreover, $A \sqcap B$ is also a coproduct with injections $i$ and $j$, and so
    \begin{equation*}
        A \sqcap B \cong A \sqcup B .
    \end{equation*}
    \begin{proof}
        The proof of the first statement, left to the reader, is a variation of the proof of Proposition 2.20. The proof of the second statement is a variation of the proof of Proposition 5.8, and it, too, is left to the reader. The last statement holds because two coproducts, here $A \sqcup B$ and $A \sqcap B$, must be isomorphic.
    \end{proof}
\end{proposition}

\begin{corollary}
    If $\mathcal{C}$ and $\mathcal{D}$ are additive categories and $T: \mathcal{C} \rightarrow \mathcal{D}$ is an additive functor of either variance, then $T(A \oplus B) \cong T(A) \oplus T(B)$ for all $A, B \in \operatorname{obj}(\mathcal{C})$.
\end{corollary}




\section{Abelian Categories}
\begin{definition}
    A category $\mathcal{C}$ is an \textbf{abelian category} if
    \begin{enumerate}[label=(\roman*)]
        \item
              it is an additive category
        \item
              every morphism has a kernel and a cokernel,
        \item
              every monomorphism is a kernel and every epimorphism is a cokernel.
    \end{enumerate}
\end{definition}

\begin{definition}
    A category $\mathcal{P}$ is an \textbf{exact category} if $\mathcal{P}$ is a full subcategory of some abelian category $\mathcal{A}$ and if $\mathcal{P}$ is closed under extensions; that is, if $0 \rightarrow P^{\prime} \rightarrow A \rightarrow P^{\prime \prime} \rightarrow 0$ is an exact sequence in $\mathcal{A}$, and if $P^{\prime}, P^{\prime \prime} \in \operatorname{obj}(\mathcal{P})$, then $A \in \operatorname{obj}(\mathcal{P})$.

    Every abelian category is an exact category. The full subcategory of Ab consisting of all torsion-free abelian groups is an exact category, but it is not an abelian category.
\end{definition}

\section{Functors between Abelian Categories}

\begin{definition}
    Let $F:\mathcal{A} \rightarrow \mathcal{B}$ be a covariant functor between abelian categories.
    \begin{enumerate}
        \item
              $F$ is \textbf{left exact} if for every exact sequence $0 \rightarrow A^{\prime} \rightarrow A \rightarrow A^{\prime \prime}\rightarrow 0$ in $\mathcal{A}$, the sequence $0 \rightarrow F\left(A^{\prime}\right) \rightarrow F(A) \rightarrow F\left(A^{\prime \prime}\right)$ is exact in $\mathcal{B}$.
        \item
              $F$ is \textbf{right exact} if for every exact sequence $A^{\prime} \rightarrow A \rightarrow A^{\prime \prime} \rightarrow 0$ in $\mathcal{A}$, the sequence $F\left(A^{\prime}\right) \rightarrow F(A) \rightarrow F\left(A^{\prime \prime}\right) \rightarrow 0$ is exact in $\mathcal{B}$.
        \item
              $F$ is \textbf{exact} if it is both left exact and right exact; that is, for every exact sequence $0 \rightarrow A^{\prime} \rightarrow A \rightarrow A^{\prime \prime} \rightarrow 0$ in $\mathcal{A}$, the sequence $0 \rightarrow F\left(A^{\prime}\right) \rightarrow F(A) \rightarrow F\left(A^{\prime \prime}\right) \rightarrow 0$ is exact in $\mathcal{B}$.
    \end{enumerate}
    \begin{remark}
        Note that a functor $F$ is left exact if and only if it preserves kernels, $F\left(\Ker\left( f \right)\right)=\Ker\left( F(f) \right)$ and $F$ is right exact if and only if it preserves cokernels, $F\left(\Coker\left( f \right)\right)=\Coker\left( F(f) \right)$.
    \end{remark}
\end{definition}
\begin{definition}
    Let $F:\mathcal{A} \rightarrow \mathcal{B}$ be a contravariant functor between abelian categories.
    \begin{enumerate}
        \item
              $F$ is \textbf{left exact} if for every exact sequence $0 \rightarrow A^{\prime} \rightarrow A \rightarrow A^{\prime \prime} \rightarrow 0$ in $\mathcal{A}$, the sequence $0 \rightarrow F\left(A^{\prime \prime}\right) \rightarrow F(A) \rightarrow F\left(A^{\prime}\right) $ is exact in $\mathcal{B}$.
        \item
              $F$ is \textbf{right exact} if for every exact sequence $0\rightarrow A^{\prime} \rightarrow A \rightarrow A^{\prime \prime} \rightarrow 0$ in $\mathcal{A}$, the sequence $ F\left(A^{\prime \prime}\right) \rightarrow F(A) \rightarrow F\left(A^{\prime}\right)\rightarrow 0$ is exact in $\mathcal{B}$.
        \item
              $F$ is \textbf{exact} if it is both left exact and right exact; that is, for every exact sequence $0 \rightarrow A^{\prime} \rightarrow A \rightarrow A^{\prime \prime} \rightarrow 0$ in $\mathcal{A}$, the sequence $0 \rightarrow F\left(A^{\prime \prime}\right) \rightarrow F(A) \rightarrow F\left(A^{\prime}\right) \rightarrow 0$ is exact in $\mathcal{B}$.
    \end{enumerate}
\end{definition}


\section{Diagram Chase}


\begin{lemma}
    [Snake's lemma]
    \label{lem: snake's lemma}
    Let $\mathcal{A}$ be an abelian category. Given a commutative diagram with exact rows
    \begin{equation*}
        \begin{tikzcd}
            0 \arrow[r] & A' \arrow[r,"u"] \arrow[d,"f^\prime"] & A \arrow[r,"v"]\arrow[d,"f"]  & A^{\prime\prime} \arrow[d,"f^{\prime\prime}"] \arrow[r] & 0 \\
            0 \arrow[r] & B^\prime  \arrow[r,"u^{\prime}"]  & B \arrow[r,"v^{\prime}"]  & B^{\prime\prime} \arrow[r] & 0
        \end{tikzcd}
    \end{equation*}
    there is an exact sequence
    \begin{equation*}
        0 \rightarrow \Ker\left(f^\prime\right) \rightarrow \Ker\left(f\right) \rightarrow \Ker\left(f^{\prime\prime}\right) \xrightarrow{\delta} \Coker\left(f^\prime\right) \rightarrow \Coker\left(f\right) \rightarrow \Coker\left(f^{\prime\prime}\right) \rightarrow 0
    \end{equation*}
    \begin{proof}
        Let $i: \Ker\left( f^{\prime\prime} \right) \rightarrow A^{\prime\prime}$ denote the kernel of $f^{\prime\prime}$ and
        pullback $P=A\times_{A^{\prime\prime}} \Ker\left( f^{\prime\prime} \right)$:
        \begin{equation*}
            \begin{tikzcd}
                &&P\arrow[d,"p_A"]\arrow[r,"p_{ker}"]& \Ker\left( f^{\prime\prime} \right) \arrow[d,"i"]&\\
                0 \arrow[r] & A' \arrow[r,"u"] \arrow[d,"f^\prime"] & A \arrow[r,"v"]\arrow[d,"f"]  & A^{\prime\prime} \arrow[d,"f^{\prime\prime}"] \arrow[r] & 0 \\
                0 \arrow[r] & B^\prime \arrow[d,"\pi"]  \arrow[r,"u^{\prime}"]  & B \arrow[r,"v^{\prime}"]  & B^{\prime\prime} \arrow[r] & 0\\
                &\Coker\left( f^\prime \right)&&&
            \end{tikzcd}
        \end{equation*}
        Then we have
        \[
            v' \circ \bigl( f \circ p_A \bigr)
            = f'' \circ  i \circ p_{\ker}
            = 0
        \]
        Thus $f \circ p_A$ factors through $\ker(v') = \left(B^\prime \xrightarrow{u^\prime}B\right)$.
        Since $u'$ is the kernel of $v'$, there exists a unique morphism
        \[
            \phi : P \longrightarrow B'
        \]
        Let $\pi : B' \to \operatorname{coker}(f')$ denote the cokernel of $f'$.
        Consider the composite
        \[
            \pi \circ \phi : P \longrightarrow \operatorname{coker}(f').
        \]

        In an abelian category, pullbacks of epimorphisms are epimorphisms;
        thus $p_{\ker} : P \to \ker(f'')$ is an epimorphism.
        Therefore there exists a unique morphism
        \[
            \delta : \ker(f'') \longrightarrow \operatorname{coker}(f')
        \]
        such that
        \[
            \delta \circ p_{\ker} = \pi \circ \phi .
        \]
        This $\delta$ is the connecting morphism.

        We now verify exactness of the long sequence.

        \smallskip
        \noindent\textbf{Exactness at $\ker f'$ and $\ker f$.}
        The sequence
        \[
            0 \to A' \xrightarrow{u} A \xrightarrow{v} A''
        \]
        is exact, and kernels commute with monomorphisms in abelian categories.
        This implies that the induced map
        \[
            \ker(f') \longrightarrow \ker(f)
        \]
        is a monomorphism, and its image equals the kernel of
        \[
            \ker(f) \longrightarrow \ker(f'').
        \]
        Thus exactness holds.

        \smallskip
        \noindent\textbf{Exactness at $\ker f''$.}
        We must show:
        \[
            \operatorname{im}(\ker f \to \ker f'') = \ker(\delta).
        \]

        First, the composite $\delta \circ (\ker f \to \ker f'')$ is zero.
        Indeed, if $x$ lies in $\ker(f)$ then $f(x)=0$; tracing the universal property
        of the pullback, one obtains a morphism $\widetilde{x}: \ker(f) \to P$ such that
        \[
            \phi \circ \widetilde{x}
        \]
        factors through $\operatorname{im}(f')$.
        Since $q$ kills $\operatorname{im}(f')$, we have
        \[
            \delta \circ p_{\ker} \circ \widetilde{x}
            = q \circ \phi \circ \widetilde{x}
            = 0,
        \]
        and because $p_{\ker}$ is an epimorphism, $\delta$ is zero on the image of $\ker f$.

        Conversely, suppose $\delta(x'')=0$.
        Then $q \circ \phi$ vanishes on $p_{\ker}^{-1}(x'')$, so the morphism
        $\phi$ factors through $\operatorname{im}(f')$.
        Using the universal property of the pullback again, this yields a lift of $x''$
        to an element of $\ker(f)$, proving that $x''$ lies in the image of
        $\ker f \to \ker f''$.

        Thus exactness holds at $\ker f''$.

        \smallskip
        \noindent\textbf{Exactness at $\operatorname{coker}(f')$ and $\operatorname{coker}(f)$.}
        The connecting morphism $\delta$ shows that
        \[
            \operatorname{im}(\delta) \subseteq \ker(\operatorname{coker}(f') \to \operatorname{coker}(f)).
        \]
        Conversely, if an element of $\operatorname{coker}(f')$ maps to zero in
        $\operatorname{coker}(f)$, one checks (using cokernel universal properties and
        commutativity of the diagram) that it arises from a unique element of $\ker(f'')$ via $\delta$.
        The argument is dual to the previous kernel argument and uses only
        the universal property of cokernels.

        \smallskip
        \noindent\textbf{Exactness at $\operatorname{coker}(f'')$.}
        This follows directly from the right exactness of the bottom row
        \[
            B' \xrightarrow{u'} B \xrightarrow{v'} B'' \to 0.
        \]

        \smallskip

        All maps are natural in the entire diagram since every step was constructed only
        by limits, colimits, kernels, and cokernels, which are functorial.
        Thus the long sequence is exact and natural.
    \end{proof}
\end{lemma}




\section{Exact Sequences} % Exact Sequences
\begin{definition}
    Let $R$ be a ring.
    A sequence of left module homomorphisms
    \begin{equation*}
        \cdots \xrightarrow{f_{i-1}} M_{i-1} \xrightarrow{f_i} M_i \xrightarrow{f_{i+1}} M_{i+1} \xrightarrow{f_{i+2}} \cdots
    \end{equation*}
    is \textbf{exact} provided $\Im {f}_i=\Ker {f}_{{i}+1}$ for all $i$.
\end{definition}

\begin{definition}
    A exact sequence
    \begin{equation*}
        \begin{tikzcd}
            0 \arrow[r,""]&A \arrow[r,"f"]&B \arrow[r,"g"]&C\arrow[r,""]&0
        \end{tikzcd}
    \end{equation*}
    is called to be \textbf{short exact sequence}, and we often say that $B$ is an \textbf{extension} of $A$ by $C$.
    \begin{remark}
        \begin{equation*}
            0 \rightarrow \Ker f \rightarrow A \xrightarrow{f} B \rightarrow \Coker f \rightarrow 0
        \end{equation*}
    \end{remark}
\end{definition}



\begin{proposition}.
    \begin{enumerate}
        \item Given two short exact sequence
    \end{enumerate}
\end{proposition}



\begin{theorem}
    Let $R$ be a ring and
    \begin{equation*}
        \begin{tikzcd}
            0 \arrow[r] & A \arrow[r, "f"] \arrow[d,"\alpha"] & B \arrow[r, "g"] \arrow[d,"\beta"] & C \arrow[r] \arrow[d, "\gamma"] & 0 \\
            0 \arrow[r] & A' \arrow[r, "f'"] & B' \arrow[r, "g'"]   & C' \arrow[r]       & 0
        \end{tikzcd}
    \end{equation*}
    a commutative diagram of R-modules and R-module homomorphisms such that each row is a short exact sequence. Then
    \begin{enumerate}
        \item
              $\alpha, \gamma$ monomorphisms $\Rightarrow \beta$ is a monomorphism;

        \item
              $\alpha, \gamma$ epimorphisms $\Rightarrow \beta$ is an epimorphism;

        \item
              $\alpha, \gamma$ isomorphisms $\Rightarrow \beta$ is an isomorphism.
    \end{enumerate}
    \begin{remark}
        Two short exact sequences are said to be \textbf{isomorphic} if $\alpha, \beta$, and $\gamma$ are isomorphisms.
        In fact, isomorphism of short exact sequences is an equivalence relation.
    \end{remark}
\end{theorem}








\begin{lemma}
    Let $f:A\rightarrow B:G$ and if $gf=1_A$, then
    \begin{equation*}
        M=\Im\left( f \right)\oplus\Ker\left( g \right)
    \end{equation*}
\end{lemma}



\begin{theorem}
    \label{thm: Split exact sequence}
    Let $R$ be a ring and $0 \rightarrow A_1 \xrightarrow{{f}} B \xrightarrow{{g}} A_2 \rightarrow 0$ a short exact sequence of $R$-module homomorphisms. Then the following conditions are equivalent.
    \begin{enumerate}
        \item
              $g$ is split epimorphism

        \item
              $f$ is split monomorphism

        \item
              $\Im\left( f \right)=\Ker\left( g \right)$ is a direct summand of $B$.


              In this case,
              we have that $B\cong A_1\oplus A_2$ and $\Im\left( f \right)=\Ker\left( g \right)\cong A_1$.
              $0 \rightarrow A_1 \xrightarrow{\iota_1} A_1 \oplus A_2 \xrightarrow{\pi_2} A_2 \rightarrow 0$.
    \end{enumerate}
    \begin{remark}
        A short exact sequence that satisfies the equivalent conditions is said to be \textbf{split} or a \textbf{split exact sequence}.
    \end{remark}
    \begin{proof}
        1. $\Rightarrow$ 3.
        Verify that that diagram
        \begin{equation*}
            \begin{tikzcd}
                0 \arrow[r] & A_1  \arrow[r,"\iota_1"] \arrow[d,"1_{A_1}"] & A_1\oplus A_2 \arrow[r,"\pi_2"] \arrow[d,"f\oplus h"] & A_2 \arrow[d,"1_{A_2}"] \arrow[r] & 0 \\
                0 \arrow[r] & A_1 \arrow[r,"f"]           & B \arrow[r,"g"]                      & A_2 \arrow[r]           & 0
            \end{tikzcd}
        \end{equation*}
        is commutative, thus $f\oplus h$ is isomorphism.


        3. $\Rightarrow$ 1.
        Given a commutative diagram with exact rows and $\varphi$ an isomorphism:
        \begin{equation*}
            \begin{tikzcd}
                0 \arrow[r] & A_1 \arrow[r,shift left=1,"\iota_1"] \arrow[d,"1_{A_1}"] & A_1\oplus A_2\arrow[l,dashed,shift left=1,"\pi_1"] \arrow[r,shift left=1,"\pi_2" ] \arrow[d,"\varphi"] & \arrow[l,shift left=1,dashed,"\iota_2"]  A_2 \arrow[r] \arrow[d, "1_{A_2}"] & 0 \\
                0 \arrow[r] & A_1 \arrow[r, "f"] & B \arrow[r, "g"]   & A_2 \arrow[r]       & 0
            \end{tikzcd}
        \end{equation*}
        define $h: A_2 \rightarrow B$ to be $\varphi \iota_2$ and $k: B \rightarrow A_1$ to be $\pi_1 \varphi^{-1}$. Use the commutativity of the diagram and the facts $\pi_i l_i=1_{A_i}, \varphi^{-1} \varphi=1_{A_1 \oplus A_2}$ to show that $k f=1_{A_1}$ and $g h=1_{A_2}$.
    \end{proof}
\end{theorem}

\begin{proposition}
    Let $R$ be a ring.
    \begin{enumerate}
        \item
              $\Hom_R(M,-)$ and $\Hom_R(-,N)$ are left exact.
        \item
              if $A$ is right $R$-module and $B$ is left $R$-module,  $A \oplus -$ and $- \oplus B$ are right exact functor.
    \end{enumerate}
\end{proposition}





\section{Projective Object} % Projective modules
\begin{definition}
    Let $\mathcal{C}$ be a category.
    An object $P$ in $\mathcal{C}$ is called a \textbf{projective object} if for every epimorphism $g:A\rightarrow B$ and every morphism $f:P\rightarrow B$, there exists a morphism $\tilde{f}:P\rightarrow A$ such that $g\tilde{f}=f$.
    \begin{equation*}
        \begin{tikzcd}
            & P \arrow[ldd,dashed," \exists \tilde{f}"'] \arrow[dd,"f"] & \\
            & & \\
            A \arrow[r,"g"]& B &
        \end{tikzcd}
    \end{equation*}
\end{definition}

\begin{definition}
    The category $\mathcal{C}$ is said to have \textbf{enough projectives} if for every object $X$ in $\mathcal{C}$, there exists an epimorphism $P \rightarrow X$ with $P$ a projective object.
\end{definition}

\subsection{In Abelian category} % In Abelian category
Let $\mathcal{A}$ be an Abelian category.
\begin{theorem}
    The following conditions on a object $P$ are equivalent.
    \begin{enumerate}
        \item
              $P$ is projective;

        \item
              every short exact sequence $0 \rightarrow A \rightarrow B \rightarrow P \rightarrow 0$ is split;
        \item
              functor $\Hom(P,-)$ is exact.
        \item
              for all object $A$, $\Ext^n\left(P,A\right) = 0$ when $n\geq 1$.
        \item
              for all object $A$, $\Ext^1\left(P,A\right) = 0$
    \end{enumerate}
\end{theorem}



\begin{corollary}
    The direct sum $\bigoplus_{i \in I} {P}_i$ is projective if and only if each ${P}_i$ is projective.
\end{corollary}

\subsection{In $R$-$\Mod$}
In the category $R$-$\Mod$ of left $R$-modules, projective objects are called projective modules.


\begin{theorem}
    \label{thm: Equivalent conditions of projective}
    The $R$-module $P$ is projective if and only if
    \begin{enumerate}
        \item
              $P$ is a direct summand of a free module $F=R^{\oplus(I)}$.
        \item
              there exists a free module $R^{\oplus(I)}$ and an idempotent endomorphism $e:F\rightarrow F$ such that $\Im\left( e \right)\cong P$.
    \end{enumerate}
    \begin{remark}
        If $P$ is finitely generated, then we can choose $F$ to have finite rank $n$ and idempotent matrix $e\in M_n\left( R \right)$, that, $P\cong e \left(R^n \right)$
    \end{remark}
\end{theorem}


\begin{theorem}
    \label{thm: Free module is projective}
    Every free module $F$ over a ring $R$ is projective.
\end{theorem}

\begin{corollary}
    \label{cor: Every module $M$ over a ring $R$ is the homomorphic image of a projective $R$-module.}
    Every module $M$ over a ring $R$ is the homomorphic image (or quotient equivalently ) of a projective $R$-module.
    Thus $R$-$\Mod$ has enough projectives.
\end{corollary}










\begin{theorem}
    Let $R$ be a P.I.D.
    Then a projective $R$-module must be free.
\end{theorem}
\begin{corollary}
    Every submodule of module $M$ over P.I.D $R$ is free.
\end{corollary}

\begin{proposition}
    If $R$ is commutative then the tensor product of two projective $R$-modules is projective.
\end{proposition}













\section{Injective object} % Injective modules
\begin{definition}
    Let $\mathcal{C}$ be a category.
    An object $E$ in $\mathcal{C}$ is called an \textbf{injective object} if for every monomorphism $f:A\rightarrow B$ and every morphism $h:A\rightarrow E$, there exists a morphism $\tilde{h}:B\rightarrow E$ such that $\tilde{h}f=h$.
    \begin{equation*}
        \begin{tikzcd}
            A \arrow[dd,"h"']\arrow[r,"f"] & B \arrow[ldd,dashed, "\exists \tilde{h}"]  \\
            &  \\
            E &
        \end{tikzcd}
    \end{equation*}
\end{definition}
\begin{definition}
    Let $\mathcal{C}$ be a category. We say that $\mathcal{C}$ has \textbf{enough injectives} if for every object $X$ in $\mathcal{C}$, there exists a monomorphism $X \rightarrow E$ with $E$ an injective object.
\end{definition}
\begin{definition}
    Let $\mathcal{C}$ be a category and object $X$.
    If there exists a essential monomorphism $X \rightarrow E\left(X\right)$ with $E$ an injective object, then $E\left(X\right)$ is called an \textbf{injective envelope} of $X$.
\end{definition}




\subsection{In Abelian category} % % In Abelian category
\begin{theorem}
    Let $\mathcal{A}$ be a Abelian category.
    The following conditions on a object $E$ are equivalent.
    \begin{enumerate}
        \item
              $E$ is injective;

        \item
              every short exact sequence $0 \rightarrow E \xrightarrow{f} B \xrightarrow{g} C \rightarrow 0$ is split;
        \item
              functor $\Hom(-,E)$ is exact.
        \item
              for all object $A$, $\Ext^n\left(A,E\right) = 0$ when $n\geq 1$.
        \item
              for all object $A$, $\Ext^1\left(A,E\right) = 0$.
    \end{enumerate}
\end{theorem}

\begin{corollary}
    A direct product of $R$-modules $\prod_{i \in I} E_{i}$ is injective if and only if $E_{i}$ is injective for every $i \in I$.
\end{corollary}


\subsection{In $R$-$\Mod$}

\begin{definition}
    A module $E$ over a ring $R$ is said to be \textbf{injective} if given any diagram of $R$-module homomorphisms with top row exact, there exists an $R$-module homomorphism ${h}: B \rightarrow {J}$ such that the diagram
    \begin{equation*}
        \begin{tikzcd}
            0 \arrow[r] & A \arrow[dd,"h"']\arrow[r,"f"] & B \arrow[ldd,dashed, "\exists \tilde{h}"]  \\
            & &  \\
            & E &
        \end{tikzcd}
    \end{equation*}
    is commutative.
\end{definition}

\begin{theorem}
    The $E$ is injective iff $E$ is a direct summand of any module $B$ of which it is a submodule.
\end{theorem}


\begin{theorem}[Baer's Criterion]
    \label{thm: Baer's Criterion}
    The left $R$-module $E$ is injective iff
    for every left ideal $\mathfrak{a}$ of $R$, any $R$-module homomorphism $\mathfrak{a} \rightarrow E$ may be extended to an $R$-module homomorphism $R \rightarrow E$.
    \begin{proof}
        Suppose conversely that every homomorphism $g: \mathfrak{a} \rightarrow E$ can be lifted to a homomorphism $G: R \rightarrow E$.
        To show that $E$ is injective we must show that if $0 \rightarrow A \rightarrow B$ is exact and $f: A \rightarrow E$ is an $R$ module homomorphism then there is a lift $F: B \rightarrow E$. Without loss of generality we may assume that $A$ is a submodule of $B$ and that the inclusion map is the given monomorphism.

        If $\mathcal{S}$ is the collection $\left(B_i,f_i:B_i \rightarrow E\right)$ such that
        $A\subset B_i\subset B$ and $\left.f_i\right|_{A}=f$, then the ordering $\left(B_i,f_i\right) \leq\left(B_j,f_j\right)$ if $B_i \subseteq B_j$ and $f_j= f_i$ on $B_i$ partially orders $\mathcal{S}$.
        Since $\mathcal{S} \neq \emptyset$, by Zorn's Lemma there is a maximal element $(\tilde{B},\tilde{f})$ in $\mathcal{S}.$

        We claim that $\tilde{B}=B$.
        Suppose that $\tilde{B}\neq B$ and there is some element $b \in B$ not contained in $\tilde{B}$ and let $I=\left\{r \in R \mid r b \in \tilde{B}\right\}$. It is easy to check that $I$ is a left ideal in $R$, and the map $g: I \rightarrow E$ defined by $g(x)=\tilde{f}(x b)$ is an $R$-module homomorphism from $I$ to $E$. By hypothesis, there is a lift $G: R \rightarrow E$ of $g$.

        Consider the submodule $\tilde{B}+R b$ of $B$, and define the map $F^{\prime}: \tilde{B}+R b \rightarrow E$ by $F^{\prime}\left(b^{\prime}+r b\right)=\tilde{f}\left(b^{\prime}\right)+G(r)$. If $b_1+r_1 b=b_2+r_2 b$ then $\left(r_1-r_2\right) b=b_2-b_1 \in \tilde{B}$, which
        shows that $r_1-r_2 \in I$, so that
        \begin{equation*}
            G\left(r_1-r_2\right)=g\left(r_1-r_2\right)=F\left(\left(r_1-r_2\right) m\right)=F\left(m_2-m_1\right) .
        \end{equation*}
        and so $F\left(m_1\right)+G\left(r_1\right)=F\left(m_2\right)+G\left(r_2\right)$. Hence $F^{\prime}$ is well defined and it is then immediate that $F^{\prime}$ is an $R$-module homomorphism extending $f$ to $B^{\prime}+R b$.
        This contradicts the maximality of $B^{\prime}$, so that $B^{\prime}=M$, which completes the proof.
    \end{proof}
\end{theorem}

\begin{proposition}
    Let $R$ be a P.I.D and a $R$-module $E$. then $E$ is injective if and only if $r E=E$ for every nonzero $r \in R$ ($E$ is divisible).
    \begin{proof}
        Suppose $R$ is a P.I.D and $E$ is injective.
        Let $r$ be a nonzero element of $R$ and $x \in E$.
        Consider the $R$-module homomorphism $f: \left(r\right) \rightarrow E$ defined by $f(r)=x$.
        Then ther is a extension $F:R\rightarrow E$ by \cref{thm: Baer's Criterion} and $x=f(r)=F(r)=rF(1)=ry$.
    \end{proof}
\end{proposition}



\begin{proposition}
    $R$-$\Mod$ has enough injectives objects.
\end{proposition}


\section{Flat Modules} % Flat modules

\begin{definition}
    Let $R$ be a ring, then a right $R$-module $A$ is \textbf{flat} if $A \otimes_R \square$ is an exact functor; that is, whenever
    \begin{equation*}
        0 \rightarrow B^{\prime} \xrightarrow{i} B \xrightarrow{p} B^{\prime \prime} \rightarrow 0
    \end{equation*}
    is an exact sequence of left $R$-modules, then
    \begin{equation*}
        0 \rightarrow A \otimes_R B^{\prime} \xrightarrow{1_A \otimes i} A \otimes_R B \xrightarrow{1_A \otimes p} A \otimes_R B^{\prime \prime} \rightarrow 0
    \end{equation*}
    is an exact sequence of abelian groups.
    \begin{remark}
        If $D$ is an $\left(S, R\right)$-bimodule then is an exact sequence of left $S$-modules.
        In particular, if $S=R$ is a commutative ring, then is an exact sequence of $R$-modules with respect to the standard $R$-module structures.
    \end{remark}
\end{definition}

\begin{theorem}
    [A Flatness Criterion]
    Let $A$ be right $R$-module.
    Then $A$ is flat if and only if for every finitely generated ideal $I$ of $R$, the map from $A \otimes_R I \rightarrow A \otimes_R R \cong A$ induced by the inclusion $I \subseteq R$ is again injective (or, equivalently, $A \otimes_R I \cong A I \subseteq A$ ).


\end{theorem}
\begin{corollary}[A Flatness Criterion for quotients]
    Let $A$ be a left $R$-module.
    Suppose $A=F / K$ where $F$ is flat and $K$ is an $R$-submodule of $F$.
    Then $A$ is flat if and only if $F I \cap K=K I$ for every finitely generated ideal $I$ of $R$.
\end{corollary}



\begin{proposition}
    Let $R$ be an arbitrary ring.
    \begin{enumerate}
        \item
              The right $R$-module $R$ is a flat right $R$-module.
        \item  A direct sum $\bigoplus_j M_j$ of right $R$-modules is flat if and only each $M_j$ is flat.
        \item  Every projective right $R$-module $P$ is flat.
    \end{enumerate}
\end{proposition}


\begin{example}
    $\mathbb{Q}\oplus \mathbb{Z}$ is a flat $\mathbb{Z}$-module but neither projective nor injective.
\end{example}



















\chapter{Homological Algebra}
\minitoc


\section{Complex and Homology}
\begin{definition}
    Let $\mathcal{A}$ be an abelian category.
    \begin{enumerate}
        \item
              A \textbf{complex} (abbreviating \textbf{chain complex}) in an abelian category $\mathcal{A}$ is a sequence of objects and morphisms in $\mathcal{A}$ (called \textbf{differentials}),
              \begin{equation*}
                  \cdots \rightarrow  C_{n+1}\xrightarrow{d_{n+1}} C_n\xrightarrow{d_n}C_{n-1}\xrightarrow{d_{n-1}} \rightarrow \cdots
              \end{equation*}
              such that :
              \begin{equation*}
                  d_n d_{n+1}=0 \quad \text { for all } n \in \mathbb{Z} .
              \end{equation*}
              We usually simplify the notation, writing $\mathbf{C}_{\bullet}$ instead of $\left(\mathbf{C}_{\bullet}, d_{\bullet}\right)$.
        \item
              If $\left(\mathbf{C}_{\bullet}, d_{\bullet}\right)$ and $\left(\mathbf{C}_{\bullet}^{\prime}, d_{\bullet}^{\prime}\right)$ are complexes, then a \textbf{chain map}
              \begin{equation*}
                  f=f_{\bullet}:\left(\mathbf{C}_{\bullet}, d_{\bullet}\right) \rightarrow\left(\mathbf{C}_{\bullet}^{\prime}, d_{\bullet}^{\prime}\right)
              \end{equation*}
              is a sequence of morphisms $f_n: C_n \rightarrow C_n^{\prime}$ for all $n \in \mathbb{Z}$ making the following diagram commute:
              \begin{equation*}
                  \begin{tikzcd}
                      \cdots \arrow[r] & C_{n+1} \arrow[d,"f_{n+1}"] \arrow[r,"d_{n+1}"] & C_n \arrow[d,"f_n"] \arrow[r,"d_n"] & C_{n-1} \arrow[d,"f_{n-1}"] \arrow[r] & \cdots \\
                      \cdots \arrow[r] & C_{n+1}^{\prime} \arrow[r,"d_{n+1}^{\prime}"] & C_n^{\prime} \arrow[r,"d_n^{\prime}"] & C_{n-1}^{\prime} \arrow[r] & \cdots
                  \end{tikzcd}
              \end{equation*}


    \end{enumerate}
    \begin{remark}
        There is a Abelian category $\mathbf{Comp}(\mathcal{A})$ of complexes in $\mathcal{A}$.
        The objects are, of course, complexes. The morphisms are chain maps. Exactly,
        \begin{enumerate}
            \item
                  whose kernels and cokernels are defined
                  \begin{equation*}
                      \begin{tikzcd}[row sep=2.2em, column sep=2.2em]
                          \vdots \ar[d] & \vdots \ar[d] & \vdots \ar[d] & \vdots \ar[d] \\
                          \Ker(f_{n+1}) \ar[r, hook, "i_{n+1}"] \ar[d] &
                          C_{n+1} \ar[r, "f_{n+1}"] \ar[d, "d_{n+1}"'] &
                          C_{n+1}' \ar[r, two heads, "\pi_{n+1}"] \ar[d, "d_{n+1}'"'] &
                          \Coker(f_{n+1}) \ar[d] \\
                          \Ker(f_{n}) \ar[r, hook, "i_{n}"] \ar[d] &
                          C_{n} \ar[r, "f_{n}"] \ar[d, "d_{n}"'] &
                          C_{n}' \ar[r, two heads, "\pi_{n}"] \ar[d, "d_{n}'"'] &
                          \Coker(f_{n}) \ar[d] \\
                          \Ker(f_{n-1}) \ar[r, hook, "i_{n-1}"] \ar[d] &
                          C_{n-1} \ar[r, "f_{n-1}"] \ar[d, "d_{n-1}"'] &
                          C_{n-1}' \ar[r, two heads, "\pi_{n-1}"] \ar[d, "d_{n-1}'"'] &
                          \Coker(f_{n-1}) \ar[d] \\
                          \vdots & \vdots & \vdots & \vdots
                      \end{tikzcd}
                  \end{equation*}
            \item
                  products and coproducts
        \end{enumerate}
    \end{remark}
\end{definition}





\section{Homology Functor} % Homology Functor
\begin{definition}
    If $(\mathbf{C}, d)$ is a complex in $\mathbf{C o m p}(\mathcal{A})$, where $\mathcal{A}$ is an abelian category, define
    \begin{equation*}
        \begin{aligned}
            n \text {-chains }      & =C_n,                                              \\
            n \text {-cycles }      & =Z_n(\mathbf{C})=\operatorname{ker} d_n,           \\
            n \text {-boundaries }  & =B_n(\mathbf{C})=\operatorname{im} d_{n+1}         \\
            n \text {-th homology } & =H_n(\mathbf{C})=Z_n(\mathbf{C}) / B_n(\mathbf{C})
        \end{aligned}
    \end{equation*}
    However, if we recognize $\mathcal{A}$ as a full subcategory of $\mathbf{A b}$, then an element of $H_n(\mathbf{C})$ is a coset $z+B_n(\mathbf{C})$; we call this element a \textbf{homology class}, and often denote it by $\operatorname{cls}\left( z \right)$ .

\end{definition}

\begin{theorem}
    The homology $H_n$ is a additive functor from $\mathbf{C o m p}(\mathcal{A})$ to $\mathcal{A}$ for all $n \in \mathbb{Z}$.
    \begin{itemize}
        \item
              $\mathbf{C}$ maps to $H_n(\mathbf{C})$.
        \item
              If $f: \mathbf{C} \rightarrow \mathbf{C}^{\prime}$ is a chain map, then $f_n$ maps $n$-cycles to $n$-cycles and $n$-boundaries to $n$-boundaries, so it induces a morphism
              \begin{equation*}
                  H_n(f): H_n(\mathbf{C}) \rightarrow H_n\left(\mathbf{C}^{\prime}\right)
              \end{equation*}
              defined by $H_n(f): \operatorname{cls}(z) \mapsto \operatorname{cls}\left(f_n z\right) $.
    \end{itemize}
\end{theorem}




\begin{definition}
    A complex $\mathbf{C}$ is a \textbf{positive complex} if $C_n=0$ for all $n<0$.
    All positive complexes form the full subcategory $\mathbf{C o m p}_{\geq 0}(\mathcal{A})$ of $\mathbf{C o m p}(\mathcal{A})$.

    A complex $\mathbf{C}$ is a \textbf{negative complex} (or cochain complex) if $C_n=0$ for all $n>0$.
    All negative complexes form the full subcategory $\mathbf{C o m p}^{\leq 0}(\mathcal{A})$ of $\mathbf{C o m p}(\mathcal{A})$.

\end{definition}


\subsection{Homotopy} % Homotopy
\begin{definition}
    Let $\mathbf{C}$ and $\mathbf{D}$ be complexes, and let $p \in \mathbb{Z}$.
    A \textbf{map of degree} $p$, denoted by $s: \mathbf{C} \rightarrow \mathbf{D}$, is a sequence $s=\left(s_n\right)$ with $s_n: C_n \rightarrow D_{n+p}$ for all $n$.
\end{definition}

\begin{definition}
    Chain maps $f, g:(\mathbf{C}, d) \rightarrow\left(\mathbf{C}^{\prime}, d^{\prime}\right)$ are \textbf{homotopic}, denoted by $f \simeq g$, if there is a map $s=\left(s_n\right): \mathbf{C} \rightarrow \mathbf{C}^{\prime}$ of degree $1$ with
    \begin{equation*}
        f_n-g_n=d_{n+1}^{\prime} s_n+s_{n-1} d_n
    \end{equation*}
    \begin{equation*}
        \begin{tikzcd}
            {} \arrow[r]
            & C_{n+1} \arrow[r, "d_{n+1}"] \arrow[d, ""]
            & C_n \arrow[r, "d_n"] \arrow[d, ""] \arrow[dl, "s_n"']
            & C_{n-1} \arrow[r] \arrow[d, ""] \arrow[dl, "s_{n-1}"']
            & {} \\
            {} \arrow[r]
            & C'_{n+1} \arrow[r, "d'_{n+1}"]
            & C'_n \arrow[r, "d'_n"]
            & C'_{n-1} \arrow[r]
            & {}
        \end{tikzcd}
    \end{equation*}
    A chain map $f:(\mathbf{C}, d) \rightarrow\left(\mathbf{C}^{\prime}, d^{\prime}\right)$ is \textbf{null-homotopic} if $f \simeq 0$, where $0$ is the zero chain map.

    If there is a chain map $\alpha: \mathbf{C} \rightarrow \mathbf{C}^{\prime} $ and $\beta: \mathbf{C}^{\prime}  \rightarrow \mathbf{C}$ such that $\alpha \beta \simeq 1_{\mathbf{C}^{\prime}}$ and $\beta \alpha \simeq 1_{\mathbf{C}}$, then $\mathbf{C}$ and $\mathbf{C}^{\prime}$ are \textbf{homotopy equivalent}.
\end{definition}


\begin{proposition}
    \label{pro: homotopy invariance of homology}
    Homotopic chain maps induce the same homomorphism in homology; that is, if $f, g:(\mathbf{C}, d) \rightarrow\left(\mathbf{C}^{\prime}, d^{\prime}\right)$ are homotopic chain maps, then $H_n(f)=H_n(g): H_n(\mathbf{C}) \rightarrow H_n(\mathbf{C}')$ for all $n \in \mathbb{Z}$.
    \begin{proof}
        If $f \simeq g$ via $s: \mathbf{C} \rightarrow \mathbf{C}^{\prime}$, and if $z_n \in Z_n(\mathbf{C})$, then
        \begin{equation*}
            f_n z_n-g_n z_n
            =
            d_{n+1}^{\prime} s_n z_n+s_{n-1} d_n z_n=d_{n+1}^{\prime} s_n z_n+0
            \in
            B_n\left(\mathbf{C}^{\prime}\right) .
        \end{equation*}
        Thus, $H_n(f): \operatorname{cls}\left(z_n\right) \mapsto \operatorname{cls}\left(f_n z_n\right)$ and $H_n(g): \operatorname{cls}\left(z_n\right) \mapsto \operatorname{cls}\left(g_n z_n\right)$ agree.
    \end{proof}
\end{proposition}

\subsection{Long Exact Sequence in Homology} % Long Exact Sequence in Homology
\begin{theorem}
    \label{thm: long exact sequence in homology}
    Let $\mathcal{A}$ be a abelian category,
    given a short exact sequence of complexes
    \begin{equation*}
        0 \rightarrow \mathbf{C}^{\prime} \xrightarrow{i} \mathbf{C} \xrightarrow{\pi} \mathbf{C}^{\prime \prime} \rightarrow 0
    \end{equation*}
    in $\operatorname{Comp}(\mathcal{A})$, then, for each $n \in \mathbb{Z}$, there is a \textbf{connecting homomorphisms}
    $\delta_n: H_n\left(\mathbf{C}^{\prime \prime}\right) \rightarrow H_{n-1}\left(\mathbf{C}^{\prime}\right)$
    defined by
    $\delta_n: \operatorname{cls}\left(z_n^{\prime \prime}\right) \mapsto \operatorname{cls}\left(i_{n-1}^{-1} d_n \pi_n^{-1} z_n^{\prime \prime}\right)$
    and thus a long exact sequence in $\mathcal{A}$
    \begin{equation*}
        \cdots \rightarrow H_{n+1}\left(\mathbf{C}^{\prime\prime}\right) \xrightarrow{\delta_{n+1}}
        H_n(\mathbf{C}^\prime) \xrightarrow{H_n(i)}
        H_n\left(\mathbf{C}\right) \xrightarrow{H_n(p)}
        H_{n-1}\left(\mathbf{C}^{\prime\prime}\right) \xrightarrow{\delta   _{n}}H_{n-1}\left(\mathbf{C}^{\prime}\right)\rightarrow \cdots
    \end{equation*}
    \begin{proof}
        If follows from \cref{lem: snake's lemma}.
    \end{proof}
\end{theorem}


\begin{theorem}[Naturality of $\delta$]
    Let $\mathcal{A}$ be an abelian category.
    Given a commutative diagram in $\operatorname{Comp}(\mathcal{A})$ with exact rows,
    \begin{equation*}
        \begin{tikzcd}
            0 \arrow{r} & \mathbf{C}^{\prime} \arrow{r}{i} \arrow{d}{f^{\prime}} & \mathbf{C} \arrow{r}{p} \arrow{d}{f} & \mathbf{C}^{\prime \prime} \arrow{r} \arrow{d}{f^{\prime \prime}} & 0 \\
            0 \arrow{r} & \mathbf{D}^{\prime} \arrow{r}{j}                          & \mathbf{D} \arrow{r}{q}                          & \mathbf{D}^{\prime \prime} \arrow{r}                 & 0
        \end{tikzcd}
    \end{equation*}
    there is a commutative diagram in $\mathcal{A}$ with exact rows,
    \begin{equation*}
        \begin{tikzcd}
            \cdots \arrow{r} & H_{n+1}\left(\mathbf{C}^{\prime \prime}\right) \arrow{r}{\delta_{n+1}} \arrow{d}{H_{n+1}\left(f^{\prime \prime}\right)} & H_n\left(\mathbf{C}^{\prime}\right) \arrow{r}{H_n(i)} \arrow{d}{H_n\left(f^{\prime}\right)} & H_n(\mathbf{C}) \arrow{r}{H_n(p)} \arrow{d}{H_n(f)} & H_{n-1}\left(\mathbf{C}^{\prime \prime}\right) \arrow{r}{\delta_n} \arrow{d}{H_{n-1}\left(f^{\prime \prime}\right)} & H_{n-1}\left(\mathbf{C}^{\prime}\right) \arrow{r} \arrow{d}{H_{n-1}\left(f^{\prime}\right)} & \cdots \\
            \cdots \arrow{r} & H_{n+1}\left(\mathbf{D}^{\prime \prime}\right) \arrow{r}{\delta_{n+1}}                          & H_n\left(\mathbf{D}^{\prime}\right) \arrow{r}{H_n(j)}                          & H_n(\mathbf{D}) \arrow{r}{H_n(q)}                          & H_{n-1}\left(\mathbf{D}^{\prime \prime}\right) \arrow{r}{\partial_n}                          & H_{n-1}\left(\mathbf{D}^{\prime}\right) \arrow{r}                          & \cdots
        \end{tikzcd}
    \end{equation*}
    \begin{remark}
        Let $\mathbf{SES}\left(\mathbf{Comp}\left(\mathcal{A}\right)\right)$ be the category of short exact sequences in $\operatorname{Comp}(\mathcal{A})$.
        The functors $F,G:\mathbf{SES}\left(\mathbf{Comp}\left(\mathcal{A}\right)\right) \rightarrow \mathcal{A}$ defined by
        \begin{equation*}
            F_n:
            \quad
            0 \rightarrow \mathbf{C}^{\prime} \xrightarrow{i} \mathbf{C} \xrightarrow{p} \mathbf{C}^{\prime \prime} \rightarrow 0
            \mapsto
            H_n\left(\mathbf{C}^{\prime \prime}\right)
        \end{equation*}
        and
        \begin{equation*}
            G_n:
            \quad
            0 \rightarrow \mathbf{C}^{\prime} \xrightarrow{i} \mathbf{C} \xrightarrow{p} \mathbf{C}^{\prime \prime} \rightarrow 0
            \mapsto
            H_{n-1}\left(\mathbf{C}^{\prime}\right)
        \end{equation*}
        Then $\delta_n$ is a natural transformation from $F_n$ to $G_n$.
    \end{remark}
\end{theorem}



\section{Derived Functor} % Derived Functor
\subsection{$\delta$-Functors}
\begin{definition}
    Let $\mathcal{A}$ and $\mathcal{B}$ be abelian categories.
    A (covariant) \textbf{homological $\delta$-functor} between $\mathcal{A}$ and $\mathcal{B}$ is
    \begin{itemize}
        \item
              a collection of additive functors $T_n: \mathcal{A} \rightarrow \mathcal{B}$ for $n \geq 0$
              (Here we make the convention that $T_n=0$ for $n<0$.)
        \item
              together with a sequence natural transformations $\delta_n$ from the functor sending $(*)$ to $T_i(A^{\prime\prime})$ to the functor sending $(*)$ to $T_{i-1}(A^\prime)$ in $\mathcal{B}^{\mathbf{SES}\left(\mathcal{A}\right)}$, where $*$ is the short exact sequence $0 \rightarrow A^\prime \rightarrow A \rightarrow A^{\prime\prime} \rightarrow 0$ in $\mathcal{A}$

              \begin{remark}
                  That is, for each short exact sequence $0 \rightarrow A^\prime \rightarrow A \rightarrow A^{\prime\prime} \rightarrow 0$ in $\mathcal{A}$, there exists a sequence of \textbf{connecting morphisms} in $\mathcal{B}$
                  \begin{equation*}
                      \delta_n: T_n(A^{\prime\prime}) \rightarrow T_{n-1}(A^\prime)\tag{*}
                  \end{equation*}
                  for all $n \geq 1$.

                  And Naturality:
                  For each morphism in $\mathbf{SES}\left(\mathcal{A}\right)$
                  \begin{equation*}
                      \begin{tikzcd}
                          0 \arrow{r} & A^\prime \arrow{r} \arrow[d ,"f^{\prime}"] & A \arrow{r} \arrow[d ,"f"] & A^{\prime\prime} \arrow{r} \arrow[d ,"f^{\prime\prime}"] & 0 \\
                          0 \arrow{r} & B^{\prime} \arrow{r} & B \arrow{r} & B^{\prime\prime} \arrow{r} & 0
                      \end{tikzcd}
                  \end{equation*}
                  the $\delta$ give a commutative diagram
                  \begin{equation*}
                      \begin{tikzcd}
                          T_n\left(A^{\prime\prime}\right) \arrow{r}{\delta_n} \arrow[d,"T_n(f^{\prime\prime})"]  & T_{n-1}\left(A^{\prime}\right) \arrow[d,"T_{n-1}(f^{\prime})"] \\
                          T_n(B^{\prime\prime}) \arrow{r}{\delta_n} & T_{n-1}(B^{\prime})
                      \end{tikzcd}
                  \end{equation*}
              \end{remark}
    \end{itemize}
    And a condition is imposed:
    \begin{itemize}
        \item
              Long exact sequence:
              For each short exact sequence as above, there is a long exact sequence
              \begin{equation*}
                  \cdots T_{n+1}(A^{\prime\prime}) \xrightarrow{\delta_{n+1}} T_n(A^\prime) \rightarrow T_n(A) \rightarrow T_n(A^{\prime\prime}) \xrightarrow{\delta_n} T_{n-1}(A^\prime) \cdots
              \end{equation*}
    \end{itemize}
\end{definition}

\begin{definition}
    Simliarly, a covariant \textbf{cohomological $\delta$-functor} between $\mathcal{A}$ and $\mathcal{B}$ is

    \begin{equation*}
        \cdots T^{n-1}(A^{\prime\prime}) \xrightarrow{\delta^{n+1}} T^n(A^{\prime}) \rightarrow T^n(A) \rightarrow T^n(A^{\prime\prime}) \xrightarrow{\delta^n} T^{n+1}(A^{\prime}) \cdots
    \end{equation*}
\end{definition}




\begin{corollary}
    Let $\mathcal{A}$ be an abelian category, then homology $H_n:\mathbf{Comp}(\mathcal{A})_{\geq 0} \rightarrow \mathcal{A}$ forms a homological $\delta$-functor.
\end{corollary}


\subsection{Resolutions}
\begin{definition}
    Let $\mathcal{A}$ be an abelian category and $A\in \Obj\left(\mathcal{A}\right)$.
    \begin{enumerate}
        \item
              The \textbf{projective resolution} of $A$ is an exact sequence
              \begin{equation*}
                  \mathbf{P}
                  =
                  \quad
                  \cdots\rightarrow P_2 \xrightarrow{d_2} P_1 \xrightarrow{d_1} P_0 \xrightarrow{\varepsilon} A \rightarrow 0
              \end{equation*}
              in which each $P_n$ is projective.
        \item
              If $\mathbf{P}$ is a projective resolution of $A$, then its \textbf{deleted projective resolution} is the complex
              \begin{equation*}
                  \mathbf{P}_A
                  =
                  \ldots\rightarrow P_2 \xrightarrow{d_2} P_1 \xrightarrow{d_1} P_0 \rightarrow 0 .
              \end{equation*}
    \end{enumerate}
    \begin{remark}
        Deleting $A$ loses no information: $A \cong \operatorname{Coker} d_1= H_0\left(\mathbf{P}_A\right)$; the inverse operation, restoring $A$ to $\mathbf{P}_A$, is called \textbf{augmenting}.

        If $\mathcal{A}$ is ${ }_R \mathbf{M o d}$ or $\mathbf{M o d}_R$, then a free resolution of a module $A$ is a projective resolution in which each $P_n$ is free; a flat resolution is an exact sequence in which each $P_n$ is flat.
    \end{remark}
\end{definition}

\begin{definition}
    Let $\mathcal{A}$ be an abelian category and $A\in \Obj\left(\mathcal{A}\right)$.
    An \textbf{injective resolution} of $A $ is an exact sequence
    \begin{equation*}
        \mathbf{E}
        =
        0 \rightarrow A \xrightarrow{\eta} E^0 \xrightarrow{d^0} E^1 \xrightarrow{d^1} E^2 \rightarrow \cdots
    \end{equation*}
    in which each $E^n$ is injective.
    If $\mathbf{E}$ is an injective resolution of $A$, then its \textbf{deleted injective resolution} is the complex
    \begin{equation*}
        \mathbf{E}^A=0 \rightarrow E^0 \xrightarrow{d^0} E^1 \xrightarrow{d^1} E^2 \rightarrow \cdots
    \end{equation*}
    \begin{remark}
        Deleting $A$ loses no information, for $A \cong \operatorname{ker} d^0$.
    \end{remark}
\end{definition}
\begin{proposition}
    Every (left or right) $R$-module $A$ has projective and injective resolutions respectively.
\end{proposition}


\begin{theorem}[Comparison Theorem]
    \label{thm: comparison theorem}
    Let $\mathcal{A}$ be an abelian category.
    \begin{enumerate}
        \item
              Given a morphism $f: A \rightarrow A^{\prime}$ in $\mathcal{A}$, consider the diagram
              \begin{equation*}
                  \begin{tikzcd}
                      \cdots \arrow[r] & P_2 \arrow[dashed]{d}{f_2} \arrow[r,"d_2"] & P_1 \arrow[dashed]{d}{f_1} \arrow[r,"d_1"] & P_0 \arrow[dashed]{d}{f_0} \arrow[r,"\varepsilon"] & A \arrow{d}{f} \arrow[r] & 0 \\
                      \cdots \arrow[r] & P_2^{\prime} \arrow[r,"d_2^{\prime}"] & P_1^{\prime} \arrow[r,"d_1^{\prime}"] & P_0^{\prime} \arrow[r,"\varepsilon^{\prime}"] & A^{\prime} \arrow[r] & 0
                  \end{tikzcd}
              \end{equation*}
              where the rows are complexes. If each $P_n$ in the top row is projective, and if the bottom row is exact, then there exists a unique chain ( in the sense of homotopy ) map $f: \mathbf{P}_A \rightarrow \mathbf{P}_{A^{\prime}}$ making the completed diagram commute.
        \item
              Given a morphism $g: A^{\prime} \rightarrow A$, consider the diagram of negative complexes
              \begin{equation*}
                  \begin{tikzcd}
                      0 \arrow[r] & A \arrow[r,"\eta"] & E^{0}  \arrow[r,"d^{0}"] & E^{1}  \arrow[r,"d^{1}"] & E^{2}  \arrow[r] & \cdots \\
                      0 \arrow[r] & A^\prime \arrow[u,"g"]\arrow[r,"\eta^\prime"] & E^{0\prime}\arrow[u,dashed,"g^0"] \arrow[r,"d^0"] & E^1 \arrow[r,"d^1"] & E^2 \arrow[r] & \cdots
                  \end{tikzcd}
              \end{equation*}
              If the bottom row is exact and each $E^n$ in the top row is injective, then there exists a unique chain map $\mathbf{E}^{A^{\prime}} \rightarrow \mathbf{E}^A$ making the completed diagram commute.
    \end{enumerate}
\end{theorem}

\begin{corollary}
    The projective (injective) resolution of an object $A$ in an abelian category $\mathcal{A}$ with enough projectives (injectives) is unique up to homotopy equivalence.
    \begin{proof}
        It follows that every object $A$ can be a image of a projective object $P_0$ by an epimorphism $\varepsilon: P_0 \rightarrow A$. (every object can be imbedded into an injective object by a monomorphism $\eta: A \rightarrow E^0$.)
        Then we can construct a projective resolution of $A$ step by step.
    \end{proof}
\end{corollary}

\section{Left Derived Functors} % Derived Functors
\subsection{Basic definition}
\begin{definition}
    Let $F: \mathcal{A} \rightarrow \mathcal{B}$ be a covariant functor between abelian categories and assume $\mathcal{A}$ has enough projectives.
    We now construct its \textbf{left derived functors} $L_n F: \mathcal{A} \rightarrow \mathcal{B}$, for all $n \in \mathbb{Z}$.
    \begin{itemize}
        \item
              For $A\in \mathcal{A}$, choose one deleted projective resolution $\mathbf{P}_A$ of $A$ and define
              \begin{equation*}
                  L_n F \left(A\right)
                  :=
                  H_n\left(F \left(\mathbf{P}_A\right)\right) .
              \end{equation*}
        \item
              Let $f: A \rightarrow A^{\prime}$ be a morphism. By the comparison theorem, there is a chain map $\check{f}: \mathbf{P}_A \rightarrow \mathbf{P}_{A^{\prime}}$ over $f$. Then $F \check{f}: F \mathbf{P}_A \rightarrow F \mathbf{P}_{A^{\prime}}$ is also a chain map, and we define $\left(L_n F\right) f:\left(L_n F\right) A \rightarrow\left(L_n F\right) A^{\prime}$ by
              \begin{equation*}
                  L_n F \left(f\right):=H_n(F \check{f})
              \end{equation*}
    \end{itemize}
    \begin{remark}
        That is,
        \begin{equation*}
            \begin{tikzcd}
                A   \arrow[d,"f"]     & \mathbf{P}_A \arrow[d,"\check{f}"]         & F(\mathbf{P}_A) \arrow[d,"F(\check{f})"]          & H_n (F(\mathbf{P}_A)) \arrow[d,"H_n(F(\check{f}))"]         \\
                A^\prime & \mathbf{P}_{A^\prime} & F(\mathbf{P}_{A^\prime}) & H_n (F(\mathbf{P}_{A^\prime}))
            \end{tikzcd}
        \end{equation*}
    \end{remark}
\end{definition}

\begin{lemma}
    The definition of $L_n F$ is independent of the choice of the $\mathbf{P}_A$ and chain map $\check{f}$ over $f$.
    \begin{proof}
        If $\mathbf{P}_A$ and $\mathbf{P}_A^{\prime}$ are two projective resolutions of $A$, then by \cref{thm: comparison theorem}, they are homotopy equivalent.
        Thus, $H_n\left(F\left(\mathbf{P}_A\right)\right) \cong H_n\left(F\left(\mathbf{P}_A^{\prime}\right)\right)$ by \cref{pro: homotopy invariance of homology}.

        \smallskip

        If $\check{f}, \check{f}^{\prime}: \mathbf{P}_A \rightarrow \mathbf{P}_{A^{\prime}}$ are two chain maps over $f$, then by \cref{thm: comparison theorem}, they are homotopic; that is, $\check{f} \simeq \check{f}^{\prime}$.
        Thus, $H_n(F(\check{f}))=H_n\left(F\left(\check{f}^{\prime}\right)\right)$ by \cref{pro: homotopy invariance of homology}.
    \end{proof}
\end{lemma}




\begin{lemma}[Horseshoe Lemma]
    \label{lem: horseshoe lemma}
    Let $\mathcal{A}$ be an abelian category with enough projectives.
    Suppose given a commutative diagram
    \begin{equation*}
        \begin{tikzcd}
            &&& 0 \arrow[d,""] &\\
            \cdots\arrow[r,""]&P_1^\prime\arrow[r,""] \arrow[d,dashed,""]&P_0^\prime \arrow[d,dashed,""]\arrow[r,""] &A^\prime \arrow[r,""]\arrow[d,""]&0\\
            &P_0^\prime \oplus P_0^{\prime\prime} \arrow[r,dashed,""] \arrow[d,dashed,""]&P_0^\prime \oplus P_0^{\prime\prime}\arrow[d,dashed,""] \arrow[r,dashed,""]& A \arrow[d,""]&\\
            \cdots\arrow[r,""]&P_1^{\prime\prime}\arrow[r,""]&P_0^{\prime\prime}\arrow[r,""]&A^{\prime\prime}\arrow[r,""]\arrow[d,""]&0\\
            &&&0&\\
        \end{tikzcd}
    \end{equation*}
    where the column is exact and the rows are projective resolutions. Set $P_n= P_n^{\prime} \oplus P_n^{\prime \prime}$. Then the $P_n$ assemble to form a projective resolution $\mathbf{P}_A$ of $A$, and the right-hand column lifts to an split exact sequence of complexes
    \begin{equation*}
        0 \rightarrow \mathbf{P}_{A^{\prime}} \xrightarrow{i} \mathbf{P}_A \xrightarrow{\pi} \mathbf{P}_{A^{\prime \prime}} \rightarrow 0,
    \end{equation*}
\end{lemma}

\begin{theorem}
    If $F:\mathcal{A} \rightarrow \mathcal{B}$ is a additive right exact covariant functor between abelian categories, then its left derived functors $L_n F$ form a homological $\delta$-functor.
    \begin{proof}
        Let
        \begin{equation*}
            0 \rightarrow A^{\prime} \rightarrow A \rightarrow A^{\prime \prime} \rightarrow 0
        \end{equation*}
        be a short exact sequence in $\mathcal{A}$.
        By the \cref{lem: horseshoe lemma}, there is an split exact sequence of complexes
        \begin{equation*}
            0 \rightarrow \mathbf{P}_{A^{\prime}} \xrightarrow{i} \mathbf{P}_A \xrightarrow{\pi} \mathbf{P}_{A^{\prime \prime}} \rightarrow 0
        \end{equation*}
        Applying the right exact functor $F$ gives an exact sequence of complexes
        \begin{equation*}
            0 \rightarrow F\left(\mathbf{P}_{A^{\prime}}\right) \xrightarrow{F(i)} F\left(\mathbf{P}_A\right) \xrightarrow{F(\pi)} F\left(\mathbf{P}_{A^{\prime \prime}}\right) \rightarrow 0
        \end{equation*}
        and hence a long exact sequence in homology by \cref{thm: long exact sequence in homology}
        \begin{equation*}
            \cdots \rightarrow H_{n+1}\left(F\left(\mathbf{P}_{A^{\prime \prime}}\right)\right) \xrightarrow{\delta_{n+1}} H_n\left(F\left(\mathbf{P}_{A^{\prime}}\right)\right) \xrightarrow{(L_n F)(i)} H_n\left(F\left(\mathbf{P}_A\right)\right) \xrightarrow{(L_n F)(\pi)} H_n\left(F\left(\mathbf{P}_{A^{\prime \prime}}\right)\right) \xrightarrow{\delta_n} H_{n-1}\left(F\left(\mathbf{P}_{A^{\prime}}\right)\right) \rightarrow \cdots
        \end{equation*}
        which is precisely
        \begin{equation*}
            \cdots \rightarrow L_{n+1} F\left(A^{\prime \prime}\right) \xrightarrow{\delta_{n+1}} L_n F\left(A^{\prime}\right) \xrightarrow{L_n F(i)} L_n F(A) \xrightarrow{L_n F(\pi)} L_n F\left(A^{\prime \prime}\right) \xrightarrow{\delta_n} L_{n-1} F\left(A^{\prime}\right) \rightarrow \cdots
        \end{equation*}
    \end{proof}
\end{theorem}

\begin{proposition}
    \label{pro: axiom}
    If $F:\mathcal{A} \rightarrow \mathcal{B}$ is an right exact    covariant functor between abelian categories and left derived functors $L_* F$, then
    \begin{enumerate}
        \item
              $L_0F \cong F$
        \item
              if $P$ is projective, then $L_n F\left(P\right)=0$ for all $n>0$.
        \item Long exact sequence:
        \item
              $\delta_n$ is natural
    \end{enumerate}
\end{proposition}


\subsection{Tor} % Tor
\begin{definition}
    If $B$ is a left $R$-module and $T=\square \otimes_R B$, define
    \begin{equation*}
        \operatorname{Tor}_n^R(\square, B)
        :=
        L_n T : \mathbf{Mod}_R \rightarrow \mathbf{A b} .
    \end{equation*}
    Thus, if $\mathbf{P}=\rightarrow P_2 \xrightarrow{d_2} P_1 \xrightarrow{d_1} P_0 \xrightarrow{\varepsilon} A \rightarrow 0$ is the chosen projective resolution of a right $R$-module $A$, then
    \begin{equation*}
        \operatorname{Tor}_n^R(A, B)
        =
        H_n\left(\mathbf{P}_A \otimes_R B\right)
    \end{equation*}
    \begin{remark}
        if $B$ is an $\left(R, S\right)$-bimodule, then the target is $\mathbf{M o d}_S$.
        In particular, if $R$ is commutative, then $A \otimes_R B$ is an $R$-module, and so the values of $\operatorname{Tor}_n^R(\square, B)$ lie in ${ }_R \mathbf{Mod}$.
    \end{remark}
\end{definition}

\begin{definition}
    If $A$ is a right $R$-module and $T=A \otimes_R \square$, define
    \begin{equation*}
        \operatorname{tor}_n^R(A, \square):=L_n T
    \end{equation*}
    \begin{remark}
        for all right $R$-modules $A$ and left $R$-modules $B$, and all $n \geq 0$,
        \begin{equation*}
            \operatorname{Tor}_n^R(A, B) \cong \operatorname{tor}_n^R(A, B)
        \end{equation*}
        Thus, the notation $\operatorname{tor}_n^R(A, B)$ is only temporary.
    \end{remark}
\end{definition}


\subsection{Properties of Tor}

\begin{theorem}
    Let $R$ be a ring, $A$ a right $R$-module, and $B$ a left $R$-module.
    \begin{enumerate}
        \item
              then
              \begin{equation*}
                  \operatorname{Tor}_n^R(A, B) \cong \operatorname{Tor}_n^{R^{\mathrm{op}}}(B, A)
              \end{equation*}
              for all $n \geq 0$, where $R^{\mathrm{op}}$ is the opposite ring of $R$.
        \item
              If $R$ is a commutative ring, then for all $n \geq 0$,
              \begin{equation*}
                  \operatorname{Tor}_n^R(A, B) \cong \operatorname{Tor}_n^R(B, A)
              \end{equation*}
    \end{enumerate}
\end{theorem}


\begin{theorem}
    A right $R$-module $F$ is flat iff $\operatorname{Tor}_n^R(F, M)=0$ for all $n \geq 1$ and every left $R$-module $M$.
\end{theorem}


\begin{theorem}
    The functors $\operatorname{Tor}_n^R(A, \square)$ and $\operatorname{Tor}_n^R(\square, B)$ can be computed using flat resolutions of either variable; more precisely, for all flat resolutions $\mathbf{F}$ and $\mathbf{G}$ of $A$ and $B$, respectively, and for all $n \geq 0$,
    \begin{equation*}
        H_n\left(\mathbf{F}_A \otimes_R B\right) \cong \operatorname{Tor}_n^R(A, B) \cong H_n\left(A \otimes_R \mathbf{G}_B\right) .
    \end{equation*}
\end{theorem}


\section{Right Derived Functor}
\begin{definition}
    Let $F: \mathcal{A} \rightarrow \mathcal{B}$ be a covariant functor between abelian categories and assume $\mathcal{A}$ has enough injectives.
    We now construct its \textbf{right derived functors} $R^n F: \mathcal{A} \rightarrow \mathcal{B}$, for all $n \in \mathbb{Z}$.
    \begin{itemize}
        \item
              For $A\in \mathcal{A}$, choose one deleted injective resolution $\mathbf{E}^A$ of $A$ and define
              \begin{equation*}
                  R^n F \left(A\right)
                  :=
                  H^n\left(F \left(\mathbf{E}^A\right)\right) .
              \end{equation*}
        \item
              Let $f: A \rightarrow A^{\prime}$ be a morphism. By the comparison theorem, there is a chain map $\check{f}: \mathbf{E}^A \rightarrow \mathbf{E}^{A^{\prime}}$ over $f$. Then $F \check{f}: F \mathbf{E}^A \rightarrow F \mathbf{E}^{A^{\prime}}$ is also a chain map, and we define $R^n F\left(f\right):\left(R^n F\right) A \rightarrow\left(R^n F\right) A^{\prime}$ by
              \begin{equation*}
                  R^n F \left(f\right)
                  :=
                  H^n(F \check{f})
              \end{equation*}
    \end{itemize}
\end{definition}


\subsection{Ext}
\begin{definition}
    Let $R$ be a ring. If $T=\operatorname{Hom}_R(A, \square)$, define $\operatorname{Ext}_R^n(A, \square):=R^n T$.
    \begin{remark}
        If the chosen injective resolution of $B$ is $\mathbf{E}^B=0 \rightarrow B \xrightarrow{\eta} E^0 \xrightarrow{d^0} E^1 \xrightarrow{d^1} E^2 \rightarrow$, then
        \begin{equation*}
            \operatorname{Ext}_R^n(A, B)
            =
            H^n\left(\operatorname{Hom}_R\left(A, \mathbf{E}^B\right)\right)
        \end{equation*}
        If $R$ is commutative, then $\operatorname{Hom}_R(A, B)$ is an $R$-module, and so the values of $\operatorname{Ext}_R^n(A, \square)$ lie in .
    \end{remark}
\end{definition}

\section{Adjoint Functors and Left/Right Exactness}
\begin{theorem}
    Let $L: \mathcal{A} \rightarrow \mathcal{B}$ be left adjoint to a functor $R: \mathcal{B} \rightarrow \mathcal{A}$, where $\mathcal{A}$ and $\mathcal{B}$ are arbitrary categories. Th
    en
    \begin{enumerate}
        \item
              $L$ preserves all colimits (coproducts, direct limits, cokernels, etc.). That is, if $A: I \rightarrow \mathcal{A}$ has a colimit, then so does $L A: I \rightarrow \mathcal{B}$, and

              \begin{equation*}
                  L\left(\operatorname{colim} A_i\right)=\operatorname{colim} L\left(A_i\right) .
              \end{equation*}

        \item
              $R$ preserves all limits (products, inverse limits, kernels, etc.). That is, if $B: I \rightarrow \mathcal{B}$ has a limit, then so does $R B: I \rightarrow \mathcal{A}$, and

              \begin{equation*}
                  R\left(\lim _{i \in I} B_i\right)=\lim _{i \in I} R\left(B_i\right) .
              \end{equation*}
    \end{enumerate}
\end{theorem}


\chapter{Tor and Ext}
\section{Tor for Abelian Groups}

\begin{example}
    Let $B$ be an abelian group and $p$ be a prime number.
    Then
    \begin{equation*}
        \operatorname{Tor}_n^{\mathbb{Z}}(\mathbb{Z}_p, B)
        =
        \begin{cases}
            B / pB  & n=0      \\
            { }_p B & n=1      \\
            0       & n \geq 2
        \end{cases}
    \end{equation*}
    where $pB=\{pb : b \in B\}$ and ${ }_p B=\{b \in B : p b=0\}$.
    To see this, use the deleted resolution
    \begin{equation*}
        \mathbf{P}_B: \cdots \rightarrow0 \rightarrow \mathbb{Z} \xrightarrow{p} \mathbb{Z}  \rightarrow 0
    \end{equation*}
\end{example}


\begin{proposition}
    For all abelian groups $A$ and $B$ :
    \begin{enumerate}
        \item
              $\operatorname{Tor}_1^{\mathbb{Z}}(A, B)$ is a torsion abelian group.
        \item
              $\operatorname{Tor}_n^{\mathbb{Z}}(A, B)=0$ for $n \geq 2$.
    \end{enumerate}
\end{proposition}






\chapter{Homological Dimension}
\section{Dimension of Rings and Modules}


\begin{definition}
    Let $R$ be a ring and $A$ be a left $R$-module, then $\operatorname{pd}_R(A) \leq n$ (pd abbreviates \textbf{projective dimension}) if there is a finite projective resolution
    \begin{equation*}
        0 \rightarrow P_n \rightarrow \cdots \rightarrow P_1 \rightarrow P_0 \rightarrow A \rightarrow 0
    \end{equation*}
    If no such finite resolution exists, then $\operatorname{pd}(A)=\infty ;$ otherwise $\operatorname{pd}(A)=n$ if $n$ is the length of a shortest projective resolution of $A$.
\end{definition}






\chapter{Sheaf}
\section{}
\begin{definition}
    Let $X$ be a topological space and a category $\mathcal{C}$.
    A \textbf{presheaf} $\mathscr{F}$ of $\mathcal{C}$-objects on $X$ is a contravariant functor $\mathbf{Open}(X)^{\op} \rightarrow \mathcal{C}$.

    \begin{enumerate}
        \item
              Each $\mathscr{F}(U)$ is called the \textbf{sections} of $\mathscr{F}$ over $U$,
        \item
              The morphism $\rho_{U V}: \mathscr{F}(U) \rightarrow \mathscr{F}(V)$ is called the \textbf{restriction} map, and we often denote $\rho_{U V}(s)$ by $s|_{V}$, if $s\in \mathscr{F}(U)$.
    \end{enumerate}
\end{definition}



\begin{definition}
    If $\mathscr{F}$ is a presheaf on $X$, and if $P$ is a point of $X$, we define the \textbf{stalk} $\mathscr{F}_P$ of $\mathscr{F}$ at $P$ to be the direct limit of the groups $\mathscr{F}(U)$ for all open sets $U$ containing $P$, via the restriction maps $\rho$.
\end{definition}

\begin{definition}
    If $\mathscr{F}$ and $\mathscr{G}$ are presheaves on $X$, a \textbf{morphism} $\varphi: \mathscr{F} \rightarrow \mathscr{G}$ consists of a morphism of abelian groups $\varphi(U): \mathscr{F}(U) \rightarrow \mathscr{G}(U)$ for each open set $U$, such that whenever $V \subseteq U$ is an inclusion, the diagram
    \begin{equation*}
        \begin{tikzcd}
            \mathscr{F}(U) \arrow{r}{\varphi(U)} \arrow{d}{\rho_{U V}} & \mathscr{G}(U) \arrow{d}{\rho^\prime_{U V}} \\
            \mathscr{F}(V) \arrow{r}{\varphi(V)}                         & \mathscr{G}(V)
        \end{tikzcd}
    \end{equation*}
    commutes.
\end{definition}









\begin{definition}
    Suppose $C$ has finite limit.
    A \textbf{sheaf} of $\mathcal{C}$-objects on $X$ is a presheaf satisfying the sheaf axiom:
    If open set $U\subset X$ has an open cover $\left\{U_i\right\}_{i \in I}$, then the
    \begin{equation*}
        \begin{tikzcd}
            \mathscr{F}(U) \arrow[r,"\alpha"] & \prod_{i \in I} \mathscr{F}\left(U_i\right) \arrow[r,shift left=0.5ex] \arrow[r,shift right=0.5ex] & \prod_{i,j \in I} \mathscr{F}\left(U_i \cap U_j\right)
        \end{tikzcd}
    \end{equation*}
    is an equalizer diagram in $\mathcal{C}$ where $\alpha$ is the morphism induced by $\rho_{U, U_i}$ and two morphisms from $\prod_{i \in I} \mathscr{F}\left(U_i\right)$ to $\prod_{i,j \in I} \mathscr{F}\left(U_i \cap U_j\right)$ are induced by $\rho_{U_i, U_i \cap U_j}$ and $\rho_{U_j, U_i \cap U_j}$ respectively.

    \begin{remark}
        The sheaf axiom is equivalent to the following two conditions:
        \begin{enumerate}[label=(\roman*)]
            \item (Locality axiom)
                  For every open cover $\left\{U_i\right\}_{i \in I}$ of an open set $U$ in $X$, and for every section $s,t \in \mathscr{F}(U)$ such that for all $i,j \in I$,
                  \begin{equation*}
                      s|_{U_i \cap U_j}=t|_{U_i \cap U_j} ,
                  \end{equation*}
                  then $s=t$.
            \item (Gluing axiom)
                  For every open cover $\left\{U_i\right\}_{i \in I}$ of an open set $U$ in $X$, and for every family of sections $s_i \in \mathscr{F}\left(U_i\right)$ such that for all $i,j \in I$,
                  \begin{equation*}
                      s_i|_{U_i \cap U_j}=s_j|_{U_i \cap U_j} ,
                  \end{equation*}
                  there exists a unique section $s \in \mathscr{F}(U)$ such that for all $i \in I$,
                  \begin{equation*}
                      s|_{U_i}=s_i .
                  \end{equation*}
        \end{enumerate}
    \end{remark}
\end{definition}



\begin{proposition}
    Note that a morphism $\varphi: \mathscr{F} \rightarrow \mathscr{G}$ of presheaves on $X$ induces a morphism $\varphi_P: \mathscr{F}_P \rightarrow \mathscr{G}_P$ on the stalks, for any point $P \in X$.

    Let $\varphi: \mathscr{F} \rightarrow \mathscr{G}$ be a morphism of sheaves on a topological space $X$. Then $\varphi$ is an isomorphism if and only if the induced map on the stalk $\varphi_P: \mathscr{F}_P \rightarrow \mathscr{G}_P$ is an isomorphism for every $P \in X$.
\end{proposition}




\begin{definition}
    Let $f: X \rightarrow Y$ be a continuous map of topological spaces.

    \begin{enumerate}
        \item
              For any sheaf $\mathscr{F}$ on $X$, we define the \textbf{direct image sheaf} $f_* \mathscr{F}$ on $Y$ by
              \begin{equation*}
                  \left(f_* \mathscr{F}\right)(V):=\mathscr{F}\left(f^{-1}(V)\right)
              \end{equation*}
              for any open set $V \subseteq Y$.

        \item
              For any sheaf $\mathscr{G}$ on $Y$, we define the \textbf{inverse image sheaf} $f^{-1} \mathscr{G}$ on $X$ to be the sheaf associated to the presheaf
              \begin{equation*}
                  f^{-1}(U)
                  :=
                  \lim _{V \supseteq f(U)} \mathscr{G}(V)
              \end{equation*}
              where $U$ is any open set in $X$, and the directlimit is taken over all open sets $V$ of $Y$ containing $f(U)$.
    \end{enumerate}
\end{definition}


\begin{definition}
    If $Z$ is a subset of $X$, regarded as a topological subspace with the induced topology, if $i: Z \rightarrow X$ is the inclusion map, and if $\mathscr{F}$ is a sheaf on $X$, then we call $i^{-1} \mathscr{F}$ the restriction of $\mathscr{F}$ to $Z$, and we often denote it by $\left.\mathscr{F}\right|_Z$. Note that the stalk of $\left.\mathscr{F}\right|_Z$ at any point $P \in Z$ is just $\mathscr{F}_P$.
\end{definition}










\end{document}
