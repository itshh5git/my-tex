\documentclass[12pt, oneside]{book}

\usepackage{../mypackages}





\begin{document}
\frontmatter
\title{{\Huge{\textbf{Representation of Groups}}}}
\maketitle

\dominitoc % 初始化minitoc
\pagenumbering{Roman}
\tableofcontents % 主目录



\chapter{Introduction}
\section{}

\begin{definition}
    Suppose now $G$ is a finite group and $k$ is a field.
    \begin{enumerate}
        \item
              A \textbf{$k$-linear representation} of $G$
              is a group homomorphism
              \begin{equation*}
                  \rho : G \rightarrow\GL(V)
              \end{equation*}
              where $V$ is a $k$ vector space of dimension $n$.

        \item
              Let $\rho$ and $\rho^{\prime}$ be two representations of the same group $G$ in vector spaces $\mathbf{V}$ and $\mathbf{V}^{\prime}$.
              These representations are said to be similar (or isomorphic) if there exists a linear isomorphism $\tau: \mathbf{V} \rightarrow \mathbf{V}^{\prime}$ which "transforms" $\rho$ into $\rho^{\prime}$, that is, which satisfies the identity

              \begin{equation*}
                  \tau \circ \rho(s)=\rho^{\prime}(s) \circ \tau \quad \text { for all } s \in \mathbf{G} .
              \end{equation*}


              When $\rho$ and $\rho^{\prime}$ are given in matrix form by $\mathrm{R}_s$ and $\mathrm{R}_s^{\prime}$ respectively, this means that there exists an invertible matrix $T$ such that

              \begin{equation*}
                  \mathrm{T} \cdot \mathrm{R}_s=\mathrm{R}_s^{\prime} \cdot \mathrm{T}, \quad \text { for all } s \in \mathrm{G},
              \end{equation*}

              which is also written $\mathrm{R}_s^{\prime}=\mathrm{T} \cdot \mathrm{R}_s \cdot \mathrm{~T}^{-1}$. We can identify two such representations (by having each $x \in \mathrm{~V}$ correspond to the element $\tau(x) \in \mathrm{V}^{\prime}$ ); in particular, $\rho$ and $\rho^{\prime}$ have the same degree.
    \end{enumerate}
    \begin{remark}
        A $k$-linear representation of $G$ is equivalent to a finite dimension $k[G]$-module
    \end{remark}
\end{definition}



\mainmatter
\pagenumbering{arabic} % 正文编页码字体 





















\end{document}