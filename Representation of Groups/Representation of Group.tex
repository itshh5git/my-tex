\documentclass[12pt, oneside]{book}

\usepackage{../mypackages}





\begin{document}
\frontmatter
\title{{\Huge{\textbf{Representation of Groups}}}}
\maketitle

\dominitoc % 初始化minitoc
\pagenumbering{Roman}
\tableofcontents % 主目录




\mainmatter
\pagenumbering{arabic} % 正文编页码字体 








\chapter{Introduction}
\section{Basic Definitions}

\begin{definition}
    Suppose now $G$ is a finite group and $k$ is a field.
    \begin{enumerate}
        \item
              A \textbf{$k$-linear representation} of $G$
              is a group homomorphism
              \begin{equation*}
                  \rho : G \rightarrow\GL(V)
              \end{equation*}
              where $V$ is a $k$-vector space.
              The space $V$ is called the \textbf{representation space} of $\rho$ and $\dim_k V$ (if finite) is called the \textbf{degree} of the representation.

        \item
              Let $\rho$ and $\rho^{\prime}$ be two representations of the same group $G$ in vector spaces $V$ and $V^{\prime}$.
              The \textbf{transforms} of $V$ and $V^\prime$ is a $k$-linear map from $V$ to $V^{\prime}$ such that
              \begin{equation*}
                  \tau \circ \rho(s)=\rho^{\prime}(s) \circ \tau \quad \text { for all } s \in G .
              \end{equation*}
              These representations are said to be \textbf{similar} (or isomorphic) if $\tau$ is an isomorphism.
    \end{enumerate}
    \begin{remark}
        A $k$-linear representation $\rho : G \rightarrow\GL(V)$ is equivalent to a left $k[G]$-module $V$.
        In this case, the transforms between two representations are exactly the $k[G]$-module homomorphisms.
    \end{remark}
\end{definition}

\begin{definition}
    Let $\rho : G \rightarrow\GL(V)$ be a $k$-linear representation of a finite group $G$ in the vector space $V$.
    A subspace $W$ of $V$ is called a \textbf{subrepresentation} if it is a $k[G]$-submodule of $V$.

    A representation is called \textbf{irreducible} if it has no proper non-zero subrepresentation.
\end{definition}


\begin{definition}
    Let $\rho_i : G \rightarrow\GL(V_i)$ be $k$-linear representation of a finite group $G$ in the vector space $V_i$ or $\char k=0$.
    \begin{enumerate}
        \item
              The \textbf{sum representation} $\rho_1 \oplus \rho_2$ is defined on the vector space $V_1 \oplus V_2$ by
              \begin{equation*}
                  \left(\rho_1 \oplus \rho_2\right)(s)\left(v_1, v_2\right)=\left(\rho_1(s) v_1, \rho_2(s) v_2\right)
              \end{equation*}
              for all $s \in G, v_1 \in V_1, v_2 \in V_2$.
              (the direct sum of $k[G]$-modules $V_1 \oplus V_2$).
        \item
              The \textbf{tensor product representation} $\rho_1 \otimes \rho_2$ is defined on the vector space $V_1 \otimes V_2$ by
              \begin{equation*}
                  \left(\rho_1 \otimes \rho_2\right)(s)\left(v_1 \otimes v_2\right)=\left(\rho_1(s) v_1\right) \otimes\left(\rho_2(s) v_2\right)
              \end{equation*}
              for all $s \in G, v_1 \in V_1, v_2 \in V_2$.
    \end{enumerate}
\end{definition}










\section{Maschke's Theorem}
In this section, we always assume that $k$ be a field whose characteristic does not divide the order of the finite group $G$
\begin{lemma}
    Let $V_i$ be $k[G]$-module for $i=1,2$ and $f:V_1\rightarrow V_2$ be a $k$-linear map, then
    \begin{equation*}
        F(x):=
        \frac{1}{\left|G\right|} \sum_{g \in G} g \cdot f\left( g^{-1}x\right)
    \end{equation*}
    is a $k[G]$-module homomorphism from $V_1$ to $V_2$.
    \begin{remark}
        Thus every $k$-linear map $h:V_1 \to V_2$ can be ``averaged'' to a transform.
        \begin{equation*}
            h^0=\frac{1}{\left|G\right|} \sum_{g \in G}\left(\rho_g^2\right)^{-1} h \rho_g^1 .
        \end{equation*}
    \end{remark}
\end{lemma}

\begin{theorem}[Maschke]
    Then $k[G]$ is semisimple.
    \begin{proof}
        Let $V$ be a $k[G]$-module, $W$ be a submodule of $V$. And let $W^{\prime}$ be a $k$-complement of $W$ in $V$, and let $p$ be the corresponding projection of $V$ onto $V$.
        Define a map $P: V \rightarrow W$ by
        \begin{equation*}
            P(v)
            :=
            \frac{1}{\left|G\right|} \sum_{g \in G} g \cdot p\left( g^{-1}v\right)
        \end{equation*}
        where $1/\left|G\right|$ is the inverse of the order of $G$ in the field $k$ (or say $\mathbb{F}_p$).
        Then $P$ is a $k[G]$-projection from $V$ to $W$ (i.e. $P$ is a $k[G]$-module homomorphism with $P^2 = P$ and $W=\operatorname{Im}(P)$), so $\ker(P)$ is a $k[G]$-complement of $W$ in $V$.
    \end{proof}
\end{theorem}

\begin{corollary}
    Every representation of $G$ over $k$ is a direct sum of irreducible representations and has finite number of irreducible components up to isomorphism.
    \begin{remark}
        Noted that every irreducible representation is finite degree. (Since it is a simple module over the finite dimensional algebra $k[G]$)
    \end{remark}
\end{corollary}

\subsection{}

\begin{corollary}
    For a finite dimension $k[G]$, we have the following decomposition of the group algebra $k[G]$:
    \begin{equation*}
        k[G]\cong \bigoplus_{i=1}^r M_{n_i}(k)
    \end{equation*}
    and each $M_{n_i}(k)$ has only one simple module
    $V_i=k^{n_i}$ up to isomorphism.
\end{corollary}


\begin{corollary}
    Then $V$ decomposes as a direct sum :
    \begin{equation*}
        V\cong \bigoplus_{i=1}^r V_i^{\oplus n_i}
    \end{equation*}
    where $V_i$ are all non-isomorphic simple $k[G]$-modules and $n_i=\dim_k \Hom_{k[G]}(V_i, V)$.
\end{corollary}



\section{}
\begin{proposition}
    Let $z=\sum a_g g \in k[G]$ with coefficients $a_g$ in $k$.
    Then $z \in Z\left(k[G]\right)$ if and only if $a_{h g h^{-1}}=a_g$ for all $g, t \in G$.
\end{proposition}



\begin{corollary}
    For each conjugacy class $C$ of $G$,
    The elements
    \begin{equation*}
        z_C:=\sum_{g \in C} g
    \end{equation*}
    where $C$ runs through the conjugacy classes of $G$, form a $k$-basis of $Z\left(k[G]\right)$.
    Thus $\dim_k Z\left(k[G]\right)$ is equal to the number of conjugacy classes of $G$.
\end{corollary}

























\chapter{Character Theory}


\section{Characters of Representations}
\begin{definition}
    Let $V$ be a $k[G]$-module of finite dimension.
    \begin{enumerate}
        \item
              The \textbf{character} $\chi$ of the representation is defined by
              \begin{equation*}
                  \chi(s)
                  :=
                  \operatorname{Tr}\left(\rho_s\right) .
              \end{equation*}
        \item
              If $W$ be a submodule of $V$, the \textbf{subcharacter} $\chi_{W}$ of the is defined by
              \begin{equation*}
                  \chi_{W}(s)
                  :=
                  \operatorname{Tr}\left(\rho_s|_{W}\right) .
              \end{equation*}
        \item
              A character $\chi$ is called \textbf{irreducible} if it is the character of an irreducible representation.
    \end{enumerate}
\end{definition}



\begin{proposition}
    If $\chi$ is the character of a representation $\rho$ of degree $n$, we have:
    \begin{enumerate}
        \item
              $\chi(1)=n$,
        \item
              $\chi\left(s^{-1}\right)=\overline{\chi(s)}$ for $s \in G$,
        \item
              $\chi\left(t s t^{-1}\right)=\chi(s)$ for $s, t \in G$. $\chi(a b)=\chi(b a)$
    \end{enumerate}
\end{proposition}
\begin{proposition}
    Let $\rho^1: G \rightarrow \GL\left(V_1\right)$ and $\rho^2: G \rightarrow \GL\left(V_2\right)$ be two linear representations of $G$.
    Then:
    \begin{enumerate}
        \item
              $\chi_{\rho^1 \oplus \rho^2} = \chi_{\rho^1} + \chi_{\rho^2}$.
        \item
              $\chi_{\rho^1 \otimes \rho^2} = \chi_{\rho^1} \cdot \chi_{\rho^2}$.
    \end{enumerate}
\end{proposition}

\section{Schur's lemma}% Schur's lemma
We first give the general version of Schur's lemma in module theory.
\begin{lemma}
    Let $A$ be a ring and $M_1,M_2$ be simple left $A$-modules.
    Then
    \begin{equation*}
        \Hom_{A}\left( M_1, M_2\right)
        =
        \begin{cases}
            0                    & ,M_1 \not\cong M_2 \\
            \text{division ring} & ,M_1 \cong M_2
        \end{cases}
    \end{equation*}

\end{lemma}
\begin{lemma}
    Let $k$ be an algebraically closed field and $A$ be $k$-algebra.
    If $V_i$ are simple left $A$-module of finite dimension and $V_1$ is $A$-isomorphic to $V_2$
    Then
    \begin{equation*}
        \dim_k\Hom_{A}\left( V_1,V_2 \right)=1
    \end{equation*}
    Indeed, $\Hom_{A}\left( V_1,V_2 \right)=k\cdot \phi$ where $\phi$ is the $A$-isomorphism of $V_1$ and $V_2$.
    \begin{remark}
        Especially, for a simple left $A$-module $V$ of finite dimension, we have
        \begin{equation*}
            \operatorname{End}_{ A}(V) = k\cdot \id
        \end{equation*}
    \end{remark}
\end{lemma}


\begin{proposition}[Schur's lemma]
    Let $k$ be a algebraically closed field and $G$ be a finite group.
    Let $\rho^1: G \rightarrow \GL\left(V_1\right)$ and $\rho^2: G \rightarrow \GL\left(V_2\right)$ be two irreducible representations of $G$, and let $f:V_1 \rightarrow V_2$ be a transform.
    Then:
    \begin{enumerate}
        \item
              If $\rho^1$ and $\rho^2$ are not isomorphic, we have $f=0$.

        \item
              If $\rho^1 \cong \rho^2$ ($\rho^1= \rho^2$ and $V_1 =   V_2$ ) then
              $f$ is a homothety (i.e., a scalar multiple of the identity).
    \end{enumerate}
\end{proposition}

\begin{corollary}
    Let $h$ be a linear mapping of $V_1$ into $V_2$, and put:
    \begin{equation*}
        h^0=\frac{1}{\left|G\right|} \sum_{g \in G}\left(\rho_g^2\right)^{-1} h \rho_g^1 .
    \end{equation*}
    Then:
    \begin{enumerate}
        \item
              If $\rho^1$ and $\rho^2$ are not isomorphic, we have $h^0=0$.
        \item
              If $V_1=V_2$ and $\rho^1=\rho^2, h^0$ is a homothety of ratio $(1 / n) \operatorname{Tr}(h)$, with $n=\operatorname{dim}\left(V_1\right)$.
    \end{enumerate}
\end{corollary}

\begin{corollary}

\end{corollary}


\section{Main }
In this section, we will derive the character theory of finite degree representations over a algebraically closed field.

\begin{definition}
    Let $G$ be a finite group and $\varphi,\phi$ be complex valued functions on $G$.
    The \textbf{inner product} of $\varphi$ and $\phi$ is defined by
    \begin{equation*}
        \left(\varphi ,\phi\right)
        :=
        \frac{1}{\left|G\right|} \sum_{g \in G} \varphi(g) \overline{\phi(g)}.
    \end{equation*}
\end{definition}




\begin{definition}
    For any $z\in Z\left(k[G]\right)$ acting on an irreducible representation $V$, by Schur's lemma, we know that $z$ acts as a homothety on $V$.
    Thus one can define the $k$-algebra homomorphism
    \begin{equation*}
        \omega_V : Z\left(k[G]\right) \to k
    \end{equation*}
    called \textbf{central character} of irreducible representation $V$ over $k$
\end{definition}

\begin{proposition}
    Let $V$ be an irreducible representation of $G$ over $k$ and $C$ be a conjugacy class of $G$. And let $z_C:=\sum_{g \in C} g\in Z\left(k[G]\right)$
    \begin{equation*}
        \omega_V\left( z_C \right)
        =
        \frac{\# C}{\dim V} \chi_V\left( c \right)
    \end{equation*}

\end{proposition}













\end{document}