\documentclass[12pt, a4paper, oneside]{book}

\usepackage{../mypackages}


\begin{document}
\frontmatter
\title{{\Huge{\textbf{Real Analysis}}}}
\maketitle

\dominitoc % 初始化minitoc
\pagenumbering{Roman}
\tableofcontents % 主目录


\mainmatter

\chapter{Measure} % Measure
\section{Measurable space} % Measurable space
\subsection{Definition}

\begin{definition}
    Let $X$ be a nonempty set.
    An \textbf{algebra} of sets on $X$ is a nonempty collection $\mathcal{A}$ of subsets of $X$ that is closed under finite unions and complements.
\end{definition}

\begin{definition}
    A family of sets $\mathcal{R} \subset \mathcal{P}(X)$ is called a \textbf{ring} if it is closed under finite unions and differences.

    A ring that is closed under countable unions is called a \textbf{$\sigma$-ring}.
\end{definition}

\begin{proposition}.
    \begin{enumerate}
        \item
              Rings (resp. $\sigma$-rings) are closed under finite (resp. countable) intersections.
        \item
              If $\mathcal{R}$ is a ring (resp. $\sigma$-ring), then $\mathcal{R}$ is an algebra (resp. $\sigma$-algebra) iff $X \in \mathcal{R}$.
        \item
              If $\mathcal{R}$ is a $\sigma$-ring, then $\left\{E \subset X: E \in \mathcal{R}\right.$ or $\left.E^c \in \mathcal{R}\right\}$ is a $\sigma$-algebra.
        \item
              If $\mathcal{R}$ is a $\sigma$-ring, then $\{E \subset X: E \cap F \in \mathcal{R}$ for all $F \in \mathcal{R}\}$ is a $\sigma$-algebra.
    \end{enumerate}
\end{proposition}


\begin{definition}
    A collection $\mathcal{M}$ of subsets of a set $X$ is said to be a \textbf{$\sigma$-algebra} or \textbf{$\sigma$-field} in $X$
    if $\mathcal{M}$ has the following properties:
    \begin{enumerate}[label=(\roman*)]
        \item
              If $E_n\in \mathcal{M}$, then $E=\bigcup\limits_{n=1}^{\infty}E_n\in \mathcal{M}$
        \item
              If $E\in\mathcal{M}$, then $E^c\in\mathcal{M}$
    \end{enumerate}
    If $\mathcal{M}$ is a $\sigma$-algebra in $X$, then $\left(X,\mathcal{M}\right)$ is called a \textbf{measurable space},
    and the members of $\mathcal{M}$ are called the \textbf{measurable sets} in $X$.
\end{definition}
\begin{proposition}
    Let $\left(X,\mathcal{M}\right)$ be a Measurable space.
    \begin{enumerate}
        \item
              $\varnothing\in\mathcal{M}$, $X\in \mathcal{M}$

        \item
              $A_1\cup A_2\cup\ldots\cup A_n\in\mathcal{M}$ if $A_i\in\mathcal{M}$  for $i=1,\ldots,n$.

        \item
              $\bigcap_{n=1}^\infty A_n=\left(\bigcup_{n=1}^\infty A_n^c\right)^c$ is measurable if $A_i\in\mathcal{M}$  for $i=1,\ldots,n$

        \item
              $A-B\in \mathcal{M}$ if $A\in \mathcal{M}$ and $B\in \mathcal{M}$.
    \end{enumerate}
\end{proposition}


\begin{definition}
    If $\mathcal{E}$ is any collection of subsets of $X$, there exists a smallest
    $\sigma$-algebra $\mathcal{M}\left(\mathcal{E}\right)$ in $X$ such that $\mathcal{E}\subset\mathcal{M}\left(\mathcal{E}\right)$ called
    \textbf{the $\sigma$-algebra generated by $\mathcal{E}$}.

    Let $X$ be a topological space. There exists a smallest $\sigma$-algebra $\mathcal{B}$ in $X$
    such that every open set in $X$ belongs $\mathcal{B}$. The  members of $\mathcal{B}$ are called the \textbf{Borel set of $X$}.
    All countable unions of closed set are called $F_{\sigma}$ set, all countable
    intersection of open  set are called $G_{\delta}$ set.
\end{definition}

\begin{proposition}
    Then $\mathcal{M}\left(\mathcal{E}\right)$ is the union of the $\sigma$-algebras generated by $\mathcal{F}$ as $\mathcal{F}$ ranges over all countable subsets of $\mathcal{E}$ i.e.
    \begin{equation*}
        \mathcal{M}\left(\mathcal{E}\right)
        =
        \bigcup_{\substack{\mathcal{F}\subset \mathcal{E} \\ \mathcal{F} \text{ countable }}}
        \mathcal{M}\left(\mathcal{F}\right)
    \end{equation*}
\end{proposition}


\subsection{Product of measurable space} % Product of measurable space
\begin{definition}
    Let $\left\{\left(X_\alpha,\mathcal{M}_\alpha\right)\right\}_{\alpha \in A}$ be an indexed collection of measurable space, $X=\prod_{\alpha \in A} X_\alpha$, and $\pi_\alpha: X \rightarrow X_\alpha$ the coordinate maps.
    The \textbf{product $\sigma$-algebra} on $X$ is the $\sigma$-algebra generated by
    \begin{equation*}
        \left\{\pi_\alpha^{-1}\left(E_\alpha\right): E_\alpha \in \mathcal{M}_\alpha, \alpha \in A\right\}
    \end{equation*}
    We denote this $\sigma$-algebra by $\bigotimes_{\alpha \in A} \mathcal{M}_\alpha$.
\end{definition}


\begin{proposition}
    \label{pro: Product of measurable space}
    Let $\left\{\left(X_\alpha,\mathcal{M}\right)\right\}_{\alpha \in A}$ be an indexed collection of measurable space, and product $\bigotimes_{\alpha \in A} \mathcal{M}_\alpha$
    \begin{enumerate}
        \item
              If $A$ is countable, then $\bigotimes_{\alpha \in A} \mathcal{M}_\alpha$ is generated by $\left\{\prod_{\alpha \in A} E_\alpha: E_\alpha \in \mathcal{M}_\alpha\right\}$.
        \item
              Suppose that $\mathcal{M}_\alpha$ is generated by $\mathcal{E}_\alpha$. Then $\bigotimes_{\alpha \in A} \mathcal{M}_\alpha$ is generated by $\mathcal{F}_1=\left\{\pi_\alpha^{-1}\left(E_\alpha\right): E_\alpha \in \mathcal{E}_\alpha, \alpha \in A\right\}$.
        \item
              If $A$ is countable and $X_\alpha \in \mathcal{E}_\alpha$ for all $\alpha, \bigotimes_{\alpha \in A} \mathcal{M}_\alpha$ is generated by $\mathcal{F}_2=\left\{\prod_{\alpha \in A} E_\alpha: E_\alpha \in \mathcal{E}_\alpha\right\}$.
    \end{enumerate}
\end{proposition}


\begin{proposition}
    Let $X_1, \ldots, X_n$ be topological spaces and let $X=\prod_1^n X_j$, equipped with the product topology.
    Then $\bigotimes_1^n \mathcal{B}_{X_j} \subset \mathcal{B}_X$. If the $X_j$ are $C_2$, then $\bigotimes_1^n \mathcal{B}_{X_j}=\mathcal{B}_X$.
    \begin{proof}
        By \ref{pro: Product of measurable space}, $\bigotimes_1^n \mathcal{B}_{X_j}$ is generated by $\left\{\pi_j^{-1}\left(U_j\right): 1 \leq j \leq n,U_j \text{ is open in } X_j\right\}$, thus $\bigotimes_1^n \mathcal{B}_{X_j} \subset \mathcal{B}_X$.

        Let $\mathcal{E}_j$ be a countable basis of $X_j$, then $\mathcal{E}_j=\left\{\prod U_i :U_i\in \mathcal{E}_i\right\}$ is a countable basis of $X$,
        It follows that $\mathcal{B}_{X_j}$ is generated by $\mathcal{E}_j$ and $\mathcal{B}_X$ is generated by $\left\{\prod_1^n E_j: E_j \in \mathcal{E}_j\right\}$. Therefore $\mathcal{B}_X=\bigotimes_1^n \mathcal{B}_{X_j}$ by \ref{pro: Product of measurable space}.
    \end{proof}
\end{proposition}

\begin{corollary}
    $\mathcal{B}_{\mathbb{R}^n}={\mathcal{B}_{\mathbb{R}}}^{\otimes n}$.
\end{corollary}

\subsection{Elementary family}
\begin{definition}
    We define an \textbf{elementary family} to be a collection $\mathcal{E}$ of subsets of $X$ such that
    \begin{enumerate}[label=(\roman*)]
        \item
              $\varnothing \in \mathcal{E}$,
        \item
              if $E, F \in \mathcal{E}$ then $E \cap F \in \mathcal{E}$,
        \item
              if $E \in \mathcal{E}$ then $E^c$ is a finite disjoint union of members of $\mathcal{E}$.
    \end{enumerate}
\end{definition}

\begin{proposition}
    If $\mathcal{E}$ is an elementary family, the collection $\mathcal{A}$ of finite disjoint unions of members of $\mathcal{E}$ is an algebra.
\end{proposition}







\section{Measure} % Measure 
\begin{definition}
    Let $(X,\mathcal{M})$ be a measurable space. \begin{enumerate}
        \item
              A \textbf{(positive) measure} on $\left(X,\mathcal{M}\right)$ is a function $\mu$, s.t.
              \begin{enumerate}[label=(\roman*)]
                  \item
                        $\mu :\mathcal{M}\rightarrow \left[0,+\infty\right]$ and $\mu\left(\varnothing\right)=0$
                  \item
                        If $E_i$ is a sequence of disjoint sets in $\mathcal{M}$, then $\mu\biggl(\bigcup\limits_{i=1}^{\infty}E_{i}\biggr)=\sum\limits_{i=1}^{\infty}\mu(E_{i})$
              \end{enumerate}
              A \textbf{measure space} is a measurable space which has a positive measure defined on the $\sigma$-algebra, denoted by $\left(X,\mathcal{M},\mu\right)$

        \item
              If $\mu(X)<\infty$, $\mu$ is called \textbf{finite}.



        \item
              If $E=\bigcup_1^{\infty} E_j$ where $E_j \in \mathcal{M}$ and $\mu\left(E_j\right)<\infty$ for all $j$, the set $E$ is said to be \textbf{$\sigma$-finite} for $\mu$.
              If $X$ is $\sigma$-finite, $\mu$ is called \textbf{$\sigma$-finite}.

        \item
              If for each $E$ with $\mu(E)=\infty$ there exists a measurable subset $F$ of $E$ s.t. $0<\mu(F)<\infty$, then $\mu$ is called \textbf{semifinite}.
              \begin{remark}
                  In this case, $\mu(F)$ can indeed be made arbitrarily large (though still finite).
              \end{remark}
        \item
              A set $E \in \mathcal{M}$ such that $\mu(E)=0$ is called a \textbf{null set}.

        \item
              If a statement about points $x \in X$ is true except for $x$ in some null set, we say that it is true $\mu$-\textbf{almost everywhere} (abbreviated a.e.), or for almost every $x$.
    \end{enumerate}
\end{definition}
\begin{theorem}
    Let $\mu$ be a measure on $(X, \mathcal{M})$, define $\mu_0$ on $\mathcal{M}$ by $\mu_0(E)=\sup \{\mu(F): F \subset E$ and $\mu(F)<\infty\}$.
    \begin{enumerate}
        \item
              $\mu_0$ is a semifinite measure. It is called the \textbf{semifinite part} of $\mu$.
        \item
              If $\mu$ is semifinite, then $\mu=\mu_0$.

        \item
              There is a measure $\nu$ on $\mathcal{M}$ (in general, not unique) which assumes only the values 0 and $\infty$ such that $\mu=\mu_0+\nu$.
    \end{enumerate}
\end{theorem}

\begin{proposition}
    Let $\mu$ be a measure on a $\sigma$-algebra $\mathcal{M}$.
    \begin{enumerate}
        \item
              (finitely additivity)
              $\mu ( A_{1}\cup \ldots \cup A_{n}) = \mu ( A_{1}) + \ldots+ \mu ( A_{n})$ if $A_1,\ldots,A_{n}$ are pairwise disjoin members of $\mathcal{M}$


        \item
              (Monotonicity)
              $A\subset B$ implies $\mu(A)\leq\mu(B)$ if $A\in\mathcal{M},B\in\mathcal{M}$

        \item (Subadditivity) If $\left\{E_j\right\}_1^{\infty} \subset \mathcal{M}$, then $\mu\left(\bigcup_1^{\infty} E_j\right) \leq \sum_1^{\infty} \mu\left(E_j\right)$.

        \item (Continuity from below) If $\left\{E_j\right\}_1^{\infty} \subset \mathcal{M}$ and $E_1 \subset E_2 \subset \cdots$, then $\mu\left(\bigcup_1^{\infty} E_j\right)=\lim _{j \rightarrow \infty} \mu\left(E_j\right)$.

        \item(Continuity from above) If $\left\{E_j\right\}_1^{\infty} \subset \mathcal{M}, E_1 \supset E_2 \supset \cdots$, and $\mu\left(E_1\right)<\infty$, then $\mu\left(\bigcap_1^{\infty} E_j\right)=\lim _{j \rightarrow \infty} \mu\left(E_j\right)$.
    \end{enumerate}
\end{proposition}



\begin{definition}[Completion]
    Let $(X,\mathcal{M},\mu)$ be a measure space, let $\overline{\mathcal{M}}$ be the collection of all $E\subset X$ for which there exist sets $A$ and $B\in \mathcal{M}$ such that $A\subset E\subset B$ and $\mu(B-A)=0$, and define $\overline{\mu}(E)=\mu(A)$ in this situation.
    \begin{remark}
        It follows that $\overline{\mathcal{M}}=\left\{E\cup N: E\in \mathcal{M}, N \text{ a subset of a null set}\right\}$
    \end{remark}
\end{definition}



\begin{proposition}
    Let $(X, \mathcal{M}, \mu)$ be a measure space and $E \in \mathcal{M}$, define $\mu_E(A)=\mu(A \cap E)$ for $A \in \mathcal{M}$ and $\mathcal{M}_E=\left\{A\cap E :A \in \mathcal{M}\right\}$.
    Then
    \begin{enumerate}
        \item
              $\left(X,\mathcal{M},\mu_E\right)$ is a measure space.
        \item
              $\left(E,\mathcal{M}_E,\mu_E\right)$ is a measure space.
    \end{enumerate}
\end{proposition}


\section{Outer Measures}

\begin{definition}
    An \textbf{outer measure} on a nonempty set $X$ is a function  that satisfies
    \begin{enumerate}[label=(\roman*)]
        \item
              $\mu^*: \mathcal{P}(X) \rightarrow[0, \infty]$ and $\mu^*(\varnothing)=0$,
        \item
              $\mu^*(A) \leq \mu^*(B)$ if $A \subset B$,
        \item
              $\mu^*\left(\bigcup_1^{\infty} A_j\right) \leq \sum_1^{\infty} \mu^*\left(A_j\right)$.
    \end{enumerate}
\end{definition}

\begin{proposition}
    \label{pro: Outer measure}
    Let $\mathcal{E} \subset \mathcal{P}(X)$ and $\rho: \mathcal{E} \rightarrow[0, \infty]$ be such that $\varnothing \in \mathcal{E}, X \in \mathcal{E}$, and $\rho(\varnothing)=0$. For any $A \subset X$, define
    \begin{equation*}
        \mu^*(A)=\inf \left\{\sum_1^{\infty} \mu\left(E_j\right): E_j \in \mathcal{E} \text { and } A \subset \bigcup_1^{\infty} E_j\right\} .
    \end{equation*}
    Then $\mu^*$ is an outer measure.
    \begin{proof}
        For any $A \subset X$ there exists $\left\{E_j\right\}_1^{\infty} \subset \mathcal{E}$ such that $A \subset \bigcup_1^{\infty} E_j$ (take $E_j=X$ for all $j$ ) so the definition of $\mu^*$ makes sense. Obviously $\mu^*(\varnothing)=0$ (take $E_j=\varnothing$ for all $j$ ), and $\mu^*(A) \leq \mu^*(B)$ for $A \subset B$ because the set over which the infimum is taken in the definition of $\mu^*(A)$ includes the corresponding set in the definition of $\mu^*(B)$. To prove the countable subadditivity, suppose $\left\{A_j\right\}_1^{\infty} \subset \mathcal{P}(X)$ and $\epsilon>0$. For each $j$ there exists $\left\{E_j^k\right\}_{k=1}^{\infty} \subset \mathcal{E}$ such that $A_j \subset \bigcup_{k=1}^{\infty} E_j^k$ and $\sum_{k=1}^{\infty} \rho\left(E_j^k\right) \leq \mu^*\left(A_j\right)+\epsilon 2^{-j}$. But then if $A=\bigcup_1^{\infty} A_j$, we have $A \subset \bigcup_{j, k=1}^{\infty} E_j^k$ and $\sum_{j, k} \rho\left(E_j^k\right) \leq \sum_j \mu^*\left(A_j\right)+\epsilon$, whence $\mu^*(A) \leq \sum_j \mu^*\left(A_j\right)+\epsilon$. Since $\epsilon$ is arbitrary, we are done.
    \end{proof}
\end{proposition}


\begin{definition}
    Let $\mu^*$ be an outer measure on $X$, a set $A \subset X$ is called \textbf{$\boldsymbol{\mu}^{\star}$-measurable} if
    \begin{equation*}
        \mu^*(E)=\mu^*(E \cap A)+\mu^*\left(E \cap A^c\right) \text { for all } E \subset X .
    \end{equation*}
    So we see that $A$ is $\mu^*$-measurable iff
    \begin{equation*}
        \mu^*(E) \geq \mu^*(E \cap A)+\mu^*\left(E \cap A^c\right) \text { for all } E \subset X \text { such that } \mu^*(E)<\infty .
    \end{equation*}

    Some motivation for the notion of $\mu^*$-measurability can be obtained by referring to the discussion at the beginning of this section. If $E$ is a "well-behaved" set such that $E \supset A$, the equation $\mu^*(E)=\mu^*(E \cap A)+\mu^*\left(E \cap A^c\right)$ says that the outer measure of $A, \mu^*(A)$, is equal to the "inner measure" of $A, \mu^*(E)-\mu^*\left(E \cap A^c\right)$. The leap from "well-behaved" sets containing $A$ to arbitrary subsets of $X$ a large one, but it is justified by the following theorem.
\end{definition}


\begin{theorem}[Carathéodory's Theorem]
    Let $\mu^*$ be an outer measure on $X$, then the collection $\mathcal{M}$ of $\mu^*$-measurable sets is a $\sigma$-algebra, and $\left(X,\mathcal{M},\mu=\left.\mu^*\right|_{\mathcal{M}}\right)$ is a complete measure space.
\end{theorem}



\subsubsection{Premeasure}

\begin{definition}
    Let $\mathcal{A} \subset \mathcal{P}(X)$ be an algebra on $X$, a function $\mu$ will be called a \textbf{premeasure} if
    \begin{enumerate}
        \item
              $\mu_0: \mathcal{A} \rightarrow[0, \infty]$ and $\mu_0(\varnothing)=0$,
        \item
              if $\left\{A_j\right\}_1^{\infty}$ is a sequence of disjoint sets in $\mathcal{A}$ such that $\bigcup_1^{\infty} A_j \in \mathcal{A}$, then $\mu_0\left(\bigcup_1^{\infty} A_j\right)=\sum_1^{\infty} \mu_0\left(A_j\right)$.
    \end{enumerate}
    If $\mu_0$ is a premeasure on $\mathcal{A} \subset \mathcal{P}(X)$, it induces an outer measure on $X$ in accordance with \ref{pro: Outer measure}, namely,
    \begin{equation*}
        \mu^*(E)=\inf \left\{\sum_1^{\infty} \mu_0\left(A_j\right): A_j \in \mathcal{A}, E \subset \bigcup_1^{\infty} A_j\right\} .
    \end{equation*}
\end{definition}





\section{Product Measure} % Product Measure













\chapter{Integration} % Integration
\minitoc

\section{Measurable Function} % Measurable Function
\subsection{Basic Definition}
\begin{definition}
    Let $\left(X, \mathcal{M}\right)$ and $\left(Y, \mathcal{N}\right)$ be measurable spaces, a mapping $f: X \rightarrow Y$ is called $(\mathcal{M}, \mathcal{N})$-\textbf{measurable} if $f^{-1}(E) \in \mathcal{M}$ for all $E \in \mathcal{N}$.

    If $X$ and $Y$ are topological spaces, $f: X \rightarrow Y$ is called \textbf{Borel measurable} if it is $\left(\mathcal{B}_X, \mathcal{B}_Y\right)$-measurable.
\end{definition}


\begin{lemma}
    Let $f: X \rightarrow Y$ be a map inducing a mapping $f^{-1}: \mathcal{P}(Y) \rightarrow \mathcal{P}(X)$, defined by $f^{-1}(E)=\{x \in X: f(x) \in E\}$, which preserves countable unions, intersections, and complements.
    Thus, if $\mathcal{N}$ is a $\sigma$-algebra on $Y$, $\left\{f^{-1}(E): E \in \mathcal{N}\right\}$ is a $\sigma$-algebra on $X$.
\end{lemma}

\begin{proposition}
    Let $\left(X, \mathcal{M}\right)$, $\left(Y, \mathcal{N}\right)$ and $\left(Z,\mathcal{O}\right)$ be measurable spaces.
    \begin{enumerate}
        \item
              If $f: X \rightarrow Y$ is $(\mathcal{M}, \mathcal{N})$-measurable and $g: Y \rightarrow Z$ is $(\mathcal{N}, \mathcal{O})$-measurable, then $g \circ f$ is $(\mathcal{M}, \mathcal{O})$-measurable.
        \item
              If $\mathcal{N}$ is generated by $\mathcal{E}$, then $f: X \rightarrow Y$ is $(\mathcal{M}, \mathcal{N})$-measurable iff $f^{-1}(E) \in \mathcal{M}$ for all $E \in \mathcal{E}$.
        \item
              If $X$ and $Y$ are topological spaces, every continuous $f: X \rightarrow Y$ is $\left(\mathcal{B}_X, \mathcal{B}_Y\right)$-measurable.
    \end{enumerate}
\end{proposition}


\begin{definition}
    Let $(X, \mathcal{M})$ be a measurable space, $f$ is a function on $X$, and $E \in \mathcal{M}$, we say that $f$ is measurable on $E$ if $f^{-1}(B) \cap E \in \mathcal{M}$ for all Borel sets $B$. (Equivalently, $\left.f\right|_E$ is $\mathcal{M}_E$-measurable, where $\mathcal{M}_E=\{F \cap E: F \in \mathcal{M}\}$.)
\end{definition}




\begin{proposition}
    Let $(X, \mathcal{M})$ and $\left(Y_\alpha, \mathcal{N}_\alpha\right)(\alpha \in A)$ be measurable spaces, $Y= \prod_{\alpha \in A} Y_\alpha, \mathcal{N}=\otimes_{\alpha \in A} \mathcal{N}_\alpha$, and $\pi_\alpha: Y \rightarrow Y_\alpha$ the coordinate maps. Then $f: X \rightarrow Y$ is $(\mathcal{M}, \mathcal{N})$-measurable iff $f_\alpha=\pi_\alpha \circ f$ is $\left(\mathcal{M}, \mathcal{N}_\alpha\right)$-measurable for all $\alpha$.
\end{proposition}


\subsection{Complex-valued measurable function}

\begin{definition}
    Let $\left(X, \mathcal{M}\right)$ be a measurable space, a  complex-, real- or extended real-valued function $f$ on $X$ will be called $\mathcal{M}$-measurable, if it is $\left(\mathcal{M}, \mathcal{B}_{\mathbb{C}}\right)$, $\left(\mathcal{M}, \mathcal{B}_{\mathbb{R}}\right)$ or $\left(\mathcal{M}, \mathcal{B}_{\overline{\mathbb{R}}}\right)$ measurable.

    In particular, $f: \mathbb{R}^n \rightarrow \mathbb{C}, \mathbb{R}$ or $\overline{\mathbb{R}}$ is \textbf{Lebesgue  measurable} if it is $\left(\mathcal{L}, \mathcal{B}_{\mathbb{C}}\right)$, $\left(\mathcal{L}, \mathcal{B}_{\mathbb{R}}\right)$ or $\left(\mathcal{L}, \mathcal{B}_{\overline{\mathbb{R}}}\right)$ measurable respectively.
\end{definition}

\begin{proposition}
    Let $\left(X,\mathcal{M}\right)$ be a measurable space
    \begin{enumerate}
        \item
              A function $f: X \rightarrow \overline{\mathbb{R}}$ is measurable iff $f^{-1}(\{-\infty\}) \in \mathcal{M}$, $f^{-1}(\{\infty\}) \in \mathcal{M}$, and $f\chi_{\left\{\left|f\right|<\infty\right\}}:X\to \mathbb{R}$ is measurable.
        \item
              A function $f:X\to \mathbb{C}$ is measurable iff $\operatorname{Re} f$ and $\operatorname{Im} f: X\to \mathbb{R}$ are measurable.
    \end{enumerate}
\end{proposition}




\begin{proposition}

    2. Suppose $f, g: X \rightarrow \overline{\mathbb{R}}$ are measurable.
    a. $f g$ is measurable (where $0 \cdot( \pm \infty)=0$ ).
    b. Fix $a \in \overline{\mathbb{R}}$ and define $h(x)=a$ if $f(x)=-g(x)= \pm \infty$ and $h(x)= f(x)+g(x)$ otherwise. Then $h$ is measurable.
\end{proposition}



\begin{proposition}
    Let $f_n :\left(X,\mathcal{M}\right)\rightarrow \overline{\mathbb{R}}$ be measurable. Then
    \begin{enumerate}
        \item
              $g=\sup f_n, h=\inf f_n,\lim\inf f_n$ and $\lim\sup f_n$ are measurable.

        \item
              The limit of every pointwise convergent sequence of complex measurable functions is measurable.
    \end{enumerate}
\end{proposition}


\begin{definition}
    Let $\left(X,\mathcal{M}\right)$ be a measurable space.

    \begin{enumerate}
        \item
              if $f: X \rightarrow \overline{\mathbb{R}}$, we define the \textbf{positive and negative parts} of $f$ to be
              \begin{equation*}
                  f^{+}(x)=\max (f(x), 0), \quad f^{-}(x)=\max (-f(x), 0)
              \end{equation*}
              Then $f=f^{+}-f^{-}$. If $f$ is measurable, so are $f^{+}$and $f^{-}$, by Corollary 2.8.

        \item
              if $f: X \rightarrow \mathbb{C}$, we have its \textbf{polar decomposition}:
              \begin{equation*}
                  f=(\operatorname{sgn} f)|f|, \quad \text { where } \quad \operatorname{sgn} z= \begin{cases}z /|z| & \text { if } z \neq 0 \\ 0 & \text { if } z=0\end{cases}
              \end{equation*}
              if $f$ is measurable, so are $|f|$ and $\operatorname{sgn} f$.
    \end{enumerate}
\end{definition}



\subsection{Simple Functions} % Simple Functions
\begin{definition}
    Let $s:X\longrightarrow \mathbb{C}$ a complex function  on a measurable space $X$.
    If the range of $s$ consists of only finitely many points, $s$ is called a \textbf{simple function}.
    If $\alpha_1, \ldots, \alpha_n$ are the distinct values of a simple function $s$,
    then clearly
    \begin{equation*}
        s=\sum_{i=1}^{n}\alpha_{i}\chi_{f^{-1}\left(\alpha_i\right)}
    \end{equation*}
    It is also clear that $s$ is measurable if and only if each of the sets $A_{\iota}$ is measurable.

    Among these are the nonnegative simple functions, whose range is a finite subset of $[0,\infty).$
\end{definition}



\begin{theorem}
    \label{thm: convergence pointwise by simple function}
    Let $f:\left(X,\mathcal{M}\right)\rightarrow[0,\infty]$ be measurable. There exist simple measurable functions $s_n$ on $X$ such thet
    \begin{enumerate}[label=(\roman*)]
        \item
              $0\leq s_1\leq s_2\leq \ldots \leq f.$
        \item
              $s_{n}( x) \to f( x)$ $as$ $n\to \infty$  for every $x\in X.$  pointwise, and $\phi_n \rightarrow f$ uniformly on any set on which $f$ is bounded.
    \end{enumerate}
    \begin{proof}
        To each positive integer $n$ and $t\in R$, define
        \begin{center}
            $\varphi_n(t)=
                \begin{cases}
                    2^{-n}[2^nt] & \quad 0\leq t < n         \\
                    \quad n      & \quad n\leq t \leq \infty
                \end{cases}$
        \end{center}
        Each $\varphi_n$ is then a Borel function on $[0,\infty]$, $0\leq \varphi_1\leq\varphi_2\leq\ldots\varphi_n\leq t$ and $\varphi_n\to t$ as $n\to \infty$ for every $t\in [0,\infty]$. It
        follows that the function
        \begin{center}
            $s_n=\varphi_n\circ f$
        \end{center}
        are measurable simple function satisfied (i) and (ii).
    \end{proof}
\end{theorem}

\begin{corollary}
    Let $(X, \mathcal{M})$ be a measurable space.
    If $f: X \rightarrow \mathbb{C}$ or $\overline{\mathbb{R}}$ is measurable, there is a sequence $\left\{\phi_n\right\}$ of simple functions s.t. $0 \leq\left|\phi_1\right| \leq\left|\phi_2\right| \leq \cdots \leq|f|, \phi_n \rightarrow f$ pointwise, and $\phi_n \rightarrow f$ uniformly on any set on which $f$ is bounded.
\end{corollary}


\subsection{Completion}
\begin{proposition}
    Let $\left(X,\mathcal{M},\mu\right)$ be a measure space. Then the following proposition are equivalent
    \begin{enumerate}
        \item
              $(X, \mathcal{M}, \mu)$ is complete.
        \item
              If $f$ is measurable and $f=g$ a.e., then $g$ is measurable.
        \item
              If $f_n$ is measurable and $f_n \rightarrow f$ a.e., then $f$ is measurable.
    \end{enumerate}
\end{proposition}
\begin{proposition}
    Let $(X, \mathcal{M}, \mu)$ be a measure space and $(X, \overline{\mathcal{M}}, \bar{\mu})$ be its completion. If $f$ is an $\overline{\mathcal{M}}$-measurable function on $X$, there is an $\mathcal{M}$-measurable function $g$ such that $f(x)=g(x)$ a.e. $\bar{\mu}$.
\end{proposition}


\section{Lebesgue Integration} % Lebesgue Integration
\begin{definition}
    Let $(X,\mathcal{M},\mu)$ be a measure space.
    \begin{enumerate}
        \item
              If $s:X\longrightarrow[0,\infty)$ is a measurable simple function of the form
              $$s=\sum_{i=1}^{n}\alpha_{i}\chi_{A_{i}}$$
              where $\alpha_1,\ldots,\alpha_n$ are the distinct values of $s$, and if $E\in\mathcal{M}$, we define
              \begin{equation*}
                  \int_{E}s\d \mu=\sum_{i=1}^{n}\alpha_{i}\:\mu(A_{i}\cap E)
              \end{equation*}

        \item
              If $f:X\to[0,\infty]$ is a nonnegative measurable function, and $E\in\mathcal{M}$, we defne
              $$\int_{E}f\d \mu=\sup\int_{E}s\d \mu$$
              the supremum being taken over all simple measurable functions such that $0\leq s\leq f$.

        \item
              If $f:X\to[-\infty,\infty]$ is measurable, and $E\in\mathcal{M}$, we define
              \begin{equation*}
                  \int_Ef\d \mu=\int_Ef^+\d \mu-\int_Ef^-\d \mu
              \end{equation*}
              provided that at least one of the integrals on the right is finite. The left side is then a number in $[-\infty,\infty]$ called the \textbf{Lebesgue integral} of $f$ over $E$,  with respect to the measure $\mu$.
              We say $f$ is integrablr if both $\int_Ef^+\d \mu$ and $\int_Ef^-\d \mu$ are finite.
    \end{enumerate}
\end{definition}

\begin{proposition}
    Let $(X, \mathcal{M}, \mu)$ be a measure space and let $(X, \overline{\mathcal{M}}, \bar{\mu})$ be its completion. If $f$ is an $\overline{\mathcal{M}}$-measurable function on $X$, there is an $\mathcal{M}$-measurable function $g$ such that $f=g$ $\bar{\mu}$-almost everywhere.
\end{proposition}


\subsection{Convergence Theorem}

\begin{theorem}[Lebesgue's Monotone Convergence Theorem]
    Let $\left\{f_n\right\}$ be a sequence of measurable functions on $X$, and suppose that
    \begin{enumerate}[label=(\roman*)]
        \item $0\leq f_{1}( x) \leq f_{2}( x) \leq \cdots \leq \infty$ for every $x\in X$
        \item $\lim f_{n}(x)\to f(x)$ for every $x\in X$
    \end{enumerate}
    Then $f$ is measurable, and
    \begin{equation*}
        \lim_{n\to \infty}\int_X f_n\d \mu
        =
        \int_X f\d \mu
    \end{equation*}
    \begin{proof}
        Since $\int f_{n}\leq\int f_{n+1}\leq \int f$, there exists an $\alpha\in[0,\infty]$ such that
        \begin{equation*}
            \int_Xf_n\d \mu\to\alpha\leq \int_X fd\mu
        \end{equation*}
        Let $s$ be any simple measurable function such that $0\leq s\leq f$, let $c\in (0,1)$ be a constant, and define
        \begin{equation*}
            E_{n}=\left\{x:f_{n}(x)\geq cs(x)\right\} \quad(n=1,2,\ldots)
        \end{equation*}
        Each $E_n$ is measurable, $E_1\subset E_2\subset E_3\subset\ldots$, and $X=\bigcup E_n$. Also,
        $$\alpha\geq\int_{X}f_{n}\d \mu\geq\int_{E_{n}}f_{n}\d \mu\geq c\int_{E_{n}}s\d \mu\quad(n=1,2,\ldots)$$
        Let $n\to\infty$ and then $c\to 1^-$, the result is
        \begin{equation*}
            \alpha\geq\int_X s\d \mu
        \end{equation*}
        for every simple measurable s satisfying $0\leq s\leq f$, so that
        \begin{equation*}
            \alpha\geq\int_X f\d \mu
        \end{equation*}
    \end{proof}
\end{theorem}
\begin{corollary}
    Let $\left\{f_n\right\}$ be a sequence of measurable functions on $X$, and suppose that
    \begin{enumerate}
        \item
              $\infty \geq f_{1}( x) \geq f_{2}( x) \geq \cdots \geq 0$ for every $x\in X$
        \item
              $\lim f_{n}(x)= f(x)$ for every $x\in X$
        \item
              $f_1 \in L^1(\mu)$
    \end{enumerate}
    Then $f$ is measurable, and
    $$\int_X f_n\d \mu\to\int_X f\d \mu\quad\quad as\:n\to\infty$$
\end{corollary}

\begin{corollary}
    If $f_n:X\longrightarrow [0,\infty]$ is measurable,  for $n=1,2,\ldots,$ and
    $$f(x)=\sum_{n=1}^{\infty}f_{n}(x)\quad(x\in X),$$
    is measurable, then
    $$\int_{X}f\d \mu=\sum_{n=1}^{\infty}\int_{X}f_{n}\d \mu$$
\end{corollary}

\begin{theorem}[Fatou's Lemma]
    If $f_n:X\to[0,\infty]$ is measurable for each positive integer $n$, then
    \begin{equation*}
        \int_X \left( \underset{n\to\infty}{\lim\inf}      f_n\right) \d \mu
        \leq
        \underset{n\to\infty}{\lim\inf} \int_X f_n \d \mu
    \end{equation*}
    \begin{proof}
        Put
        $$g_{k}(x)=\inf_{i\geq k}f_i(x)\quad(k=1,2,\ldots;x\in X)$$
        Then $g_k\leq f_k$. Also, $0\leq g_{1}\leq g_{2}\leq\cdots$, each $g_k$ is measurable, so that
        $$\int_X \lim\inf f_n\xleftarrow[\text{Monotone}]{\lim\inf}\int_Xg_k \leq\int_X f_k\xrightarrow{\lim\inf}\lim\inf\int_X f_n$$
    \end{proof}
\end{theorem}

\begin{theorem}[Lebesgue's Dominated Convergence Theorem]
    Suppose $\left\{f_n\right\}$ is a sequence of complex measurable functions on $X$ such that
    \begin{enumerate}[label=(\roman*)]
        \item
              $\lim\limits_{n\to\infty}f_{n}(x)=f(x)$ exists for every x $\in X$. \\
        \item
              there is a function $g\in L^{1}(\mu)$ such that
              $$\left|f_{n}(x)\right| \leq \left|g(x)\right| \quad(n=1,2,\ldots;x\in X)$$
    \end{enumerate}
    then $f\in L^1(\mu)$ and
    \begin{equation*}
        \lim_{n\to\infty}\int_X\left|f_n-f\right| d\mu=0
    \end{equation*}
    \begin{proof}
        Since $|f|\leq g$ and $f$ is measurable, $f\in L^1(\mu).$ Since $|f_n-f|\leq2g$, Fatou's lemma applies to the functions $2g-|f_n-f|$ and yields
        $$\begin{aligned}
                \int_{X}2g\d \mu & \leq\lim\inf\int_{X}(2g-|f_{n}-f|)\d \mu                        \\
                                 & =\int_{X}2g\d \mu+\lim\inf\left(-\int_{X}|f_{n}-f|\d \mu\right) \\
                                 & =\int_{X}2g\d \mu-\lim\sup\int_{X}|f_{n}-f|\d \mu
            \end{aligned}$$
        Since $\int 2g\d \mu$ is finite, we may subtract it and obtain
        \begin{equation*}
            \lim\sup\int_X\left|f_{n}-f\right|\d \mu\leq 0
        \end{equation*}
        Thus $\lim\int_X\left|f_{n}-f\right|\d \mu\leq 0$.
    \end{proof}
\end{theorem}

\begin{theorem}[A generalized Dominated Convergence Theorem]
    Suppose $g_n,g \in L^1$. If
    \begin{enumerate}[label=(\roman*)]
        \item
              $ f_n \rightarrow f$ for all $x\in X$
        \item
              $g_n \rightarrow g$ for all $x\in X$,  $\left|f_n\right| \leq g_n$ and $\int g_n \rightarrow \int g$
    \end{enumerate}
    then $f \in L^1$ and $f \xrightarrow{L^1}f$
    \begin{proof}
        Apply Fatou's lemma to
        \begin{equation*}
            g_n+g -\left|f_n-f\right|
        \end{equation*}
    \end{proof}
\end{theorem}


\begin{theorem}
    Suppose ${f_n}$ is a sequence of complex measurable functions on $\left(X,\mathcal{M},\mu\right)$ s.t,
    \begin{equation*}
        \sum_{n=1}^\infty\int_X\lvert f_n\rvert \d \mu < \infty
    \end{equation*}
    Then the series
    \begin{equation*}
        f(x)=\sum_{n=1}^{\infty}f_n(x)
    \end{equation*}
    converges absolutely for a.e. $x\in X$ (thus $f$ is well-defined a.e.), and
    \begin{equation*}
        \int_X f\d \mu=\sum_{n=1}^\infty\int_X f_n \d \mu
    \end{equation*}
    \begin{proof}
        Let $S_n$ be the set on which $f_n$ is defined, so that $\mu(S_n^c)=0.$ Put
        $$\varphi(x)=\sum_{n=1}^\infty \lvert f_n(x)\rvert$$
        for $x\in S=\bigcap S_{n}$. Then $\mu(S^c)=0$ and $\varphi$ is defined a.e. $[\mu]$. By monotone convergence theorem,
        \begin{equation*}
            \int_S\varphi\d \mu =
            \sum_{n=1}^\infty\int_S \lvert f_n\rvert \d \mu < \infty
        \end{equation*}

        If $E=\left\{x\in S:\varphi(x)<\infty\right\} \subset S$, it follows that $\mu(S-E)=0$.
        The series $f(x)=\sum f_n(x)$ converges absolutely for every $x\in E$, and if $f(x)$ is defined for $x\in E$ then $\left|f(x)\right|\leq \varphi(x)$ on $E$, so that $f\in L^1(\mu)$ on $E$. If $g_n=\sum\limits_{k=1}^n f_k$ then $\left|g_n\right|\leq\varphi,\:g_n(x)\to f(x)$ for all $x\in E$, then
        \begin{equation*}
            \sum_{n=1}^\infty\int_X f_n \d \mu
            =\sum_{n=1}^\infty\int_E f_n \d \mu
            =\int_E f\d \mu
            =\int_X f\d \mu
        \end{equation*}
    \end{proof}
\end{theorem}
\begin{corollary}
    Let $\left\{E_k\right\}$ be a sequence of measurable sets n $X$, such that
    $$\sum_{k=1}^{\infty}\mu(E_{k})<\infty$$
    Then almost all $ x\in X$ lie in at mostf finitely many of the sets $E_k$.
\end{corollary}
\subsection{\texorpdfstring{$L^1(\mu)$}{} Space}
\begin{definition}
    Let $\left(X,\mu\right)$ be a measure.
    We define $L^1(\mu)$ to be the collection of all complex measurable functions $f$ on $X$ for which
    $$\int_X\lvert f\rvert\d \mu<\infty$$
    The members of $L^1(\mu)$ are called \textbf{Lebesgue integrable functions (with respect to $\mu$)} or summable functions.

    If $f = u + \mathrm{i} v $, where $u$ and $\upsilon$ are real measurable
    functions on $X$, and if $f\in L^1(\mu)$, we define
    $$\int_Ef \d \mu = \int_E u^+ \d \mu - \int_E u^- \d \mu + \mathrm{i} \int_E v^+ \d \mu - \mathrm{i} \int_Ev^-\d \mu $$for every measurable set $E$.
\end{definition}


\begin{theorem}
    The operator $\int \cdot \d \mu$ is a bounded $\mathbb{C}$-linear functional in $L^1$ with norm $1$.
    \begin{proof}
        Put $z=\int_X f\d \mu$. Since $z$ is a complex number, there is a complex number $\alpha$, with $\left|\alpha\right|=1$,such that $\alpha z=\left|z\right|$.
        Let $u$ be the real part of $\alpha f$. Then $u\leq\left|\alpha f\right|=\left|f\right|$. Hence
        \begin{equation*}
            \left|\int_{X}f\d \mu\right|=\alpha\int_{X}f\d \mu=\int_{X}\alpha f\d \mu=\int_{X}u\d \mu\leq\int_{X}|f|\d \mu
        \end{equation*}
    \end{proof}
\end{theorem}



\begin{theorem}\

    (1) Suppose $f:X\longrightarrow[0,\infty]$ is measurable, $E\in\mathcal{M}$, and $\int_E f \d \mu=0$. Then $f=0$ a.e. on $E$.

    (2) Suppose $f\in L^1(\mu)$ and $\int_E f\d \mu=0$ for every $E\in\mathcal{M}$. Then $f=0$ a.e. on $X$.

    (3) Suppose $f\in L^{1}(\mu)$ and
    $$\int_X f\d \mu=\int_X\left|f\right| \d \mu$$
    Then there is a constant $\alpha$ such that $f = \alpha \left|f\right|$ a.e. on $X$.
\end{theorem}

\begin{theorem}[Absolutely continuous]
    Suppoes $f\in L^1(\mu)$. Then to each $\varepsilon > 0$ there exists a $\delta > 0$ such that
    \begin{equation*}
        \int_E \left|f\right|
        d\mu
        <
        \varepsilon
    \end{equation*}
    whenever $\mu(E)< \delta$.
\end{theorem}





\section{Modes of Convergence}
\subsection{Convergence in measure}
\begin{definition}
    Let $\left(X,\mathcal{M},\mu\right)$ be a measure space.
    We say that a sequence $\left\{f_n\right\}$ of measurable complex-valued functions on $\left(X, \mathcal{M}, \mu\right)$ is \textbf{Cauchy in measure} if for every $\varepsilon>0$,
    \begin{equation*}
        \mu\left(\left\{x:\left|f_n(x)-f_m(x)\right| \geq \varepsilon\right\}\right) \rightarrow 0 \text { as } m, n \rightarrow \infty,
    \end{equation*}
    and that $\left\{f_n\right\}$ converges in measure to $f$ if for every $\varepsilon>0$,
    \begin{equation*}
        \mu\left(\left\{x:\left|f_n(x)-f(x)\right| \geq \varepsilon\right\}\right) \rightarrow 0 \text { as } n \rightarrow \infty .
    \end{equation*}
    \begin{remark}
        If we relax the condition so that the function $f_n$ and $f$ can take $\infty$, then this definition is not well-defined. Therefore, we only consider $\mathbb{R}$-valued functions here, and furthermore, $\mathbb{C}$-valued functions.
    \end{remark}
\end{definition}

\begin{theorem}
    Let $\left(X,\mathcal{M},\mu\right)$ be a measure space.
    Then $f_n \rightarrow f$ in measure iff for every $\epsilon>0$ there exists $N \in \mathbb{N}$ such that $\mu \left(\left\{ x: \left|f_n(x)-f(x)\right| \geq \epsilon\right\}\right)<\epsilon$ for $n \geq N$.
\end{theorem}





\begin{theorem}
    Suppose that $\left\{f_n\right\}$ is Cauchy in measure. Then there is a unique (in the sense of a.e.) measurable function $f$ such that $f_n \rightarrow f$ in measure, and there is a subsequence $\left\{f_{n_j}\right\}$ that converges to $f$ a.e.
    \begin{proof}
        We can choose a subsequence $\left\{g_j\right\}$ of $\left\{f_n\right\}$ such that $E_j= \left\{x:\left|g_j(x)-g_{j+1}(x)\right| \geq 2^{-j}\right\}$ of measure $\mu\left(E_j\right) \leq 2^{-j}$. If $F_k=\bigcup_{j=k}^{\infty} E_j$, then $\mu\left(F_k\right) \leq \sum_k^{\infty} 2^{-j}=2^{1-k}$, and if $x \notin F_k$, for $i \geq j \geq k$ we have
        \begin{equation*}
            \left|g_j(x)-g_i(x)\right| \leq \sum_{l=j}^{i-1}\left|g_{l+1}(x)-g_l(x)\right| \leq \sum_{l=j}^{i-1} 2^{-l} \leq 2^{1-j}
        \end{equation*}
        Thus $\left\{g_j\right\}$ is pointwise Cauchy on $F_k^c$. Let $F=\bigcap_1^{\infty} F_k=\limsup E_j$. Then $\mu(F)=0$, and if we set $f(x)=\lim g_j(x)$ for $x \notin F$ and $f(x)=0$ for $x \in F$, then $f$ is measurable and $g_j \rightarrow f$ a.e.

        Also, $\left|g_j(x)-f(x)\right| \leq 2^{1-j}$ for $x \notin F_k$ and $j \geq k$. Since $\mu\left(F_k\right) \rightarrow 0$ as $k \rightarrow \infty$, it follows that $g_j \rightarrow f$ in measure. But then $f_n \rightarrow f$ in measure, because
        \begin{equation*}
            \left\{x:\left|f_n(x)-f(x)\right| \geq \varepsilon\right\} \subset\left\{x:\left|f_n(x)-g_j(x)\right| \geq \frac{1}{2} \varepsilon\right\} \cup\left\{x:\left|g_j(x)-f(x)\right| \geq \frac{1}{2} \varepsilon\right\},
        \end{equation*}
        and the sets on the right both have small measure when $n$ and $j$ are large. Likewise, if $f_n \rightarrow g$ in measure,
        \begin{equation*}
            \{x:|f(x)-g(x)| \geq \varepsilon\} \subset\left\{x:\left|f(x)-f_n(x)\right| \geq \frac{1}{2} \varepsilon\right\} \cup\left\{x:\left|f_n(x)-g(x)\right| \geq \frac{1}{2} \varepsilon\right\}
        \end{equation*}
        for all $n$, hence $\mu(\{x:|f(x)-g(x)| \geq \varepsilon\})=0$ for all $\varepsilon$. Letting $\varepsilon$ tend to zero through some sequence of values, we conclude that $f=g$ a.e.
    \end{proof}
\end{theorem}



\begin{proposition}
    $f_n \rightarrow f$ in $L^1 \Rightarrow f_n \rightarrow f$ in measure $\Rightarrow f_{n_j}\rightarrow f$ a.e.
\end{proposition}








\begin{proposition}
    Let $\left(X,\mathcal{M},\mu\right)$ be a finite measure space.
    If $f$ and $g$ are complex-valued measurable functions on $X$, define
    \begin{equation*}
        \rho(f, g)=\int \frac{|f-g|}{1+|f-g|} \d \mu .
    \end{equation*}
    Then $\rho$ is a metric on $L^0\left(\mu\right)$, and $f_n \rightarrow f$ with respect to this metric iff $f_n \rightarrow f$ in measure.
\end{proposition}

\begin{proposition}
    Suppose $f_n \rightarrow f$ in measure and $g_n \rightarrow g$ in measure. Then
    \begin{enumerate}
        \item
              $f_n+g_n \rightarrow f+g$ in measure.

        \item
              $f_n g_n \rightarrow f g$ in measure if $\mu(X)<\infty$.
    \end{enumerate}

\end{proposition}


\begin{theorem}
    Let $\left(X,\mathcal{M},\mu\right)$ be a finite measure space, then $L^0\left(\mu\right)$ is a linear topological space with complete metric.
\end{theorem}



















\subsection{Almost uniform convergence}
\begin{theorem}[Egoroff's Theorem]
    Let $\left(X,\mathcal{M},\mu\right)$ be a finite measure space, $f_1, f_2, \ldots$ and $f$ be measurable complex-valued functions on $X$ such that $f_n \rightarrow f$ a.e. Then for every $\varepsilon>0$ there exists $E \subset X$ such that $\mu(E)<\varepsilon$ and $f_n \rightarrow f$ uniformly on $E^c$.
    \begin{remark}
        The type of convergence involved in the conclusion of Egoroff's theorem is sometimes called \textbf{almost uniform convergence}.

        The hypothesis " $\mu(X)<\infty$ " can be replaced by " $\left|f_n\right| \leq g$ for all $n$, where $g \in L^1(\mu)$."
    \end{remark}
    \begin{proof}
        Without loss of generality we may assume that $f_n \rightarrow f$ everywhere on $X$. For $k, n \in \mathbb{N}$ let
        \begin{equation*}
            E_n(k)=\bigcup_{m=n}^{\infty}\left\{x:\left|f_m(x)-f(x)\right| \geq k^{-1}\right\}
        \end{equation*}
        Then, for fixed $k, E_n(k)$ decreases as $n$ increases, and $\bigcap_{n=1}^{\infty} E_n(k)=\varnothing$, so since $\mu(X)<\infty$ we conclude that $\mu\left(E_n(k)\right) \rightarrow 0$ as $n \rightarrow \infty$. Given $\varepsilon>0$ and $k \in \mathbb{N}$, choose $n_k$ so large that $\mu\left(E_{n_k}(k)\right)<\varepsilon 2^{-k}$ and let $E=\bigcup_{k=1}^{\infty} E_{n_k}(k)$. Then $\mu(E)<\varepsilon$, and we have $\left|f_n(x)-f(x)\right|<k^{-1}$ for $n>n_k$ and $x \notin E$. Thus $f_n \rightarrow f$ uniformly on $E^c$.
    \end{proof}


\end{theorem}



\subsection{Relation between different convergence modes}
\subsubsection{}
\begin{proposition}[The Vitali Convergence Theorem]
    Suppose $1 \leq p<\infty$ and $\left\{f_n\right\}_{1}^{\infty} \subset L^p$. Then $\left\{f_n\right\}$ is Cauchy in the $L^p$ iff
    \begin{enumerate}[label=(\roman*)]
        \item
              $\left\{f_n\right\}$ is Cauchy in measure;
        \item
              the sequence $\left\{\left|f_n\right|^p\right\}$ is uniformly integrable.
        \item
              for every $\epsilon>0$ there exists $E \subset X$ such that $\mu(E)<\infty$ and $\int_{E^c}\left|f_n\right|^p<\epsilon$ for all $n$.
    \end{enumerate}
    \begin{remark}
        It follows that
        \begin{equation*}
            f_n \xrightarrow{L^p} f \Rightarrow f_n \xrightarrow{\mu} f
        \end{equation*}
        On the other hand,
        \begin{equation*}
            f_n\xrightarrow{\mu} f \text{ and } \left|f_n\right|\leq g \in L^p \Rightarrow f_n\xrightarrow{L^p}f
        \end{equation*}
    \end{remark}
\end{proposition}






\begin{proposition}
    Suppose $1 \leq p<\infty$. If $f_n, f \in L^p$ and $f_n \rightarrow f$ a.e., then $\left\|f_n-f\right\|_p \rightarrow 0$ iff $\left\|f_n\right\|_p \rightarrow\|f\|_p$.
    \begin{proof}
        Apply Fatou's lemma to
        \begin{equation*}
            h_n=2^{n-1}\left(\left|f_n\right|^p+\left|f\right|^p\right)-\left|f_n-f\right|^{p}\geq 0
        \end{equation*}
    \end{proof}
\end{proposition}







\chapter{Radon measure} % Radon measure

\section{Some topology Preliminaries}
\begin{definition}
    Let $X$ be a topological space.

    $X$ is \textbf{locally compact} if every point of $X$ has a neighborbood whose closure is compact.

    If $X$ is itself compact, then $X$ is called a compact space.
\end{definition}
\begin{theorem}
    Suppose $K$ is compact and $F$ is closed, in a topological space $X$. If $F \subset K$, then $F$ is compact.
\end{theorem}
\begin{corollary}
    If $A \subset B$ and if $B$ has compact closure, so does $A$.
\end{corollary}
\begin{theorem}
    Suppose $X$ is a Hausdorff space, a compact $K\subset X$ and $p \notin K$. Then there are open sets $p \in U$ and $K\subset V$ such that $U \cap V=\varnothing$.
\end{theorem}
\begin{corollary}
    Let $X$ be a Hausdorff space.

    (1) Compact subsets are closed.

    (2) If $F$ is closed and $K$ is compact, then $F \cap K$ is compact.
\end{corollary}
\begin{theorem}\
    If $\left\{K_\alpha\right\}$ is a collection of compact subsets of a Hausdorff space and if $\bigcap_a K_\alpha=\varnothing$, then some finite subcollection of $\left\{K_{p_k}\right\}_{k=1}^n$ also has empty intersection.

    Proof:
    Fix a member $K_1$ of $\left\{K_a\right\}$. Since no point of $K_1$ belongs to every $K_a$, then $\left\{K_\alpha^c\right\}$ is an open cover of $K_1$. Hence $K_1 \subset K_{\alpha_1}^c \cup \cdots \cup K_{\alpha_{n}}$ for some finite collection $\left\{K_{\alpha_i}\right\}$. This implies that
    $$
        K_1 \cap K_{\alpha_1} \cap \cdots \cap K_{\alpha_n}=\varnothing
    $$
\end{theorem}
\begin{theorem}\
    Suppose $X$ is a locally compact Hausdorff space $X$, $K \subset U$, $U$ is open, and $K$ is compact. Then there is an open set $V$ with compact closure such that
    $$K \subset V \subset \overline{V} \subset U$$

    Proof:
    Since every point of $K$ has a neighborhood with compact closure, and since $K$ is covered by the union of finitely many of these neighborhoods, $K$ lies in an open set $G$ with compact closure. If $U=X$, take $V=G$.

    Otherwise, theorem 2.1.4 shows that to each $p \in U^c\subset K^c$ there corresponds an open set $W_p$ such that $K \subset W_p$ and $p \notin \overline{W}_p$ (). Hence $\left\{U^c \cap \overline{G} \cap \overline{W}_p\right\}_{p\in U^c}$ is a collection of compact sets with empty intersection. By Theorem 2.6 there are points $p_1, \ldots, p_n \in C$ such that
    $$U^c \cap \overline{G} \cap \overline{W}_{p_1} \cap \ldots \cap \overline{W}_{p_n}=\varnothing$$
    then
    $$\overline{G} \cap \overline{W}_{p_1} \cap \ldots \cap \overline{W}_{p_n}\subset U $$
    The set $V=G \cap W_{p_1} \cap \ldots \cap W_{p_n}$
\end{theorem}
\section{Convex Function and Inequalities}
\begin{definition}
    A real function $\varphi$ defined on a segment $( a, b )$, where $-\infty \leq a<b \leq \infty$, is called convex if the inequality
    $$
        \varphi((1-\lambda) x+\lambda y) \leq(1-\lambda) \varphi(x)+\lambda \varphi(y)
    $$
    holds whenever $a<x<b, a<y<b$, and $0 \leq \lambda \leq 1$.

    Also, it is equivalent to the requirement that
    $$
        \frac{\varphi(t)-\varphi(s)}{t-s}
        \leq
        \frac{\varphi(u)-\varphi(t)}{u-t}
    $$
    whenever $a<s<t<u<b$.
\end{definition}
\begin{theorem}
    If $\varphi$ is convex on $(a, b)$ then $\varphi$ is continuous on $(a, b)$
\end{theorem}
\begin{theorem}
    Let $\left\{ \varphi_\alpha\right\}$ be a collection of convex function on $(a,b)$, then

    (1) $ f(x) = \sup\limits_\alpha \varphi_\alpha (x)$
    , assume that it is finite, is convex (and lower semicontinuous).

    (2) $g(x) = \overline{\lim} \psi_n (x)$ is convex (and lower semicontinuous)

    Proof: (i)Suppose that $x,y \in (a,b)$ and $0 \leq \lambda \leq 1$. Since $f_\alpha$ is convex
    $$
        \varphi_\alpha ((1-\lambda) x+\lambda y)
        \leq (1-\lambda) \varphi_\alpha (x)+\lambda \varphi_\alpha (y)
        \leq (1-\lambda) f (x)+\lambda f(y)
    $$
    for all $\alpha$. We have
    $$
        f((1-\lambda) x+\lambda y) \leq (1-\lambda) f (x)+\lambda f(y)
    $$

    As for (ii)
    $$
        \psi_n ((1-\lambda) x+\lambda y)
        \leq (1-\lambda) \psi_n (x)+\lambda \psi_n (y)
    $$
    and let $b$
\end{theorem}
\begin{corollary}

\end{corollary}

\subsection{Inequality}

\begin{theorem}[Jensen's Inequality]
    Let $(\Omega,\mu,\mathcal{M})$ be a positive measure space with $\mu(\Omega)=1$. If $f(x)$ is a real function in $L^1(\mu)$, if $a<f(x)<b$ for all $x\in \Omega$, and if $\varphi$ is convex on $(a,b)$, then
    $$\varphi\left(\int_{\Omega}f\d \mu\right)\leq
        \int_{\Omega} \varphi \circ f \d \mu$$

    Note: The cases $a=-\infty$ and $b=\infty$ are not excluded. It may happen that $\varphi \circ f$ is not in $L^{1}(\mu)$; in that case, the integral of $\varphi \circ f$ exists in the extended sense, and its value is $+\infty$.

    Proof:
    Put $t=\int_{\Omega} f \d  \mu$, then $a<t<b$. Since $\varphi$ is convex on $(a,b)$, there is a $\beta \in \mathbb{R}$ that

    \begin{equation}
        \varphi(s) \geq \varphi(t)+\beta(s-t) \quad(a<s<b)  \tag{2}
    \end{equation}
    Hence
    \begin{equation}
        \varphi(f(x))-\varphi(t)-\beta[f(x)-t] \geq 0 \tag{3}
    \end{equation}
    for every $x \in \Omega$. Since $\varphi$ is continuous, $\varphi \circ f$ is measurable. If we integrate both sides with respect to $\mu$, then we get the inequality.
\end{theorem}
\begin{definition}
    If $p$ and $q$ are positive real number such that
    $$\frac{1}{p} + \frac{1}{q} = 1$$
    then we call $p$ and $q$ a pair of conjugate exponents. $1$ and $\infty$ are also regarded as a pair of conjugate exponents.
\end{definition}
\begin{theorem}[Hölder's inequality]
    Let $p$ and $q$ be conjugate exponents, $1<p<\infty$. Let $X$ be a measure space, with measure $\mu$. Let $f$ and $g$ be measurable functions on $X$, with range in $[0, \infty]$. Then

    \begin{equation}
        \int_{X} f g \d  \mu \leq\left\{\int_{X} f^{p} \d  \mu\right\}^{1 / p}\left\{\int_{X} g^{q} \d  \mu\right\}^{1 / q} \tag{1}
    \end{equation}

    Proof:
    Let $A$ and $B$ be the two factors on the right of (1). If $A=0$, then $f=0$ a.e. ; hence $f g=0$ a.e. , so (1) holds. If $A>0$ and $B=\infty$, (1) is again trivial. So we need consider only the case $0<A<\infty, 0<B<\infty$. Put
    \begin{equation}
        F=\frac{f}{A}, \quad G=\frac{g}{B} \tag{3}
    \end{equation}
    This gives
    \begin{equation}
        \int_{X} F^{p} \d  \mu=\int_{X} G^{q} \d  \mu=1 \tag{4}
    \end{equation}

    Since $1 / p+1 / q=1$, the convexity of the exponential function implies that
    \begin{equation}
        F(x) G(x) \leq \frac{F(x)^{p}}{p}+ \frac{G(x)^{q}}{q}  \tag{6}
    \end{equation}
    for every $x \in X$. Integration of (6) yields

    \begin{equation}
        \int_{X} F G \d  \mu \leq p^{-1}+q^{-1}=1 \tag{7}
    \end{equation}
    by (4); inserting (3) into (7), we obtain (1).
\end{theorem}
\begin{theorem}\
    \begin{equation}
        \left\{\int_{X}(f+g)^{p} \d  \mu\right\}^{1 / p} \leq\left\{\int_{X} f^{p} \d  \mu\right\}^{1 / p}+\left\{\int_{X} g^{p} \d  \mu\right\}^{1 / p} \tag{2}
    \end{equation}

    Proof:
    We write
    \begin{equation}
        (f+g)^{p}=f \cdot(f+g)^{p-1}+g \cdot(f+g)^{p-1} \tag{8}
    \end{equation}
    Hölder's inequality gives
    \begin{equation}
        \int f \cdot(f+g)^{p-1} \leq\left\{\int f^{p}\right\}^{1 / p}\left\{\int(f+g)^{(p-1) q}\right\}^{1 / q} \tag{9}
    \end{equation}
    and
    \begin{equation}
        \int g \cdot(f+g)^{p-1} \leq \left\{\int g^{p}\right\}^{1 / p}\left\{\int(f+g)^{(p-1) q}\right\}^{1 / q} \tag{9'}
    \end{equation}
    Since $(p-1) q=p$, addition of (9) and (9') gives
    \begin{equation}
        \int(f+g)^{p} \leq\left\{\int(f+g)^{p}\right\}^{1 / q}\left[\left\{\int f^{p}\right\}^{1 / p}+\left\{\int g^{p}\right\}^{1 / p}\right] \tag{10}
    \end{equation}


    Clearly, it is enough to prove (2) in the case that the left side is greater than 0 and the right side is less than $\infty$. The convexity of the function $t^{p}$ for $0<t<\infty$ shows that
    $$
        \left(\frac{f+g}{2}\right)^{p} \leq \frac{1}{2}\left(f^{p}+g^{p}\right)
    $$
    Hence the left side of (2) is less than $\infty$, and (2) follows from (10) if we divide by the first factor on the right of ( 10 ), bearing in mind that $1-1 / q=1 / p$. This completes the proof.
\end{theorem}




\section{Semicontinwous}
\begin{definition}
    Let $f$ be a real (or extended-real) function on a topological space. If
    $$\left\{x: f(x)>\alpha\right\}$$
    is open for every real $\alpha, f$ is said to be \textbf{lower semicontinuous}. If
    $$\left\{x: f(x)<\alpha\right\}$$
    is open for every real $\alpha, f$ is said to be upper semicontinuous
\end{definition}

\begin{theorem}
    Let $X$ be topology space.

    (1) $\chi_V (x)$ is lower semicontinuous iff $V$ is open.

    (2) $\chi_F (x)$ is uper semicontinuous iff $F$ is closed.

    (3) The supremum of any collection of lower semicontinuous functions is lower semicontinuous.

    (4) The infimum of any collection of upper semicontinuous
    functions is upper semicontinwous.
\end{theorem}

\begin{theorem}
    Let $X$ be a topological space and $f: X \rightarrow \mathbb{R}$

    (1) $f$ is lower semicontinuous if and only if given $x \in X$, for every $\left\{x_n\right\} \subseteq X \backslash\left\{x\right\}$ converging to $x$
    $$\underset{n\to \infty}{\lim\inf} f(x_n)\geq f(x)$$


    (2) $f$ is upper semicontinuous if and only if given $x \in X$, for every $\left\{x_n\right\} \subseteq X \backslash\left\{x\right\}$ converging to $x$
    $$\underset{n\to \infty}{\lim\sup} f(x_n)\leq f(x)$$
\end{theorem}

\begin{theorem}\
    Let $\left\{f_{n}\right\}$ be a sequence of real functions on $\mathbb{R}$, and consider the following four statements:

    (1) If $f_{1}$ and $f_{2}$ are upper semicontinuous, then $f_{1}+f_{2}$ is upper semicontinuous.

    (2) If $f_{1}$ and $f_{2}$ are lower semicontinuous, then $f_{1}+f_{2}$ is lower semicontinuous.

    (3) If each $f_{n}$ is lower semicontinuous and nonnegative, then $\sum\limits_{1}^{\infty} f_{n}$ is lower semicontinuous.

    Proof:
    Noticed that
    $$\left\{f_1+f_2 > \alpha\right\}= \bigcup_{\beta_1 +\beta_2 > \alpha} \left(\left\{ f_1 > \beta_1\right\}\cap \left\{ f_2 > \beta_2\right\}\right)$$
    is open if $f_1$ and $f_2$ are lower semicontinuous. Then $g_n=\sum\limits_{1}^n f_k$ is lower semicontinuous and
    $$f=\sum\limits_{1}^{\infty} f_{n} = \sup\limits_n g_n$$
    is lower semicontinuous.
\end{theorem}
\begin{theorem}[Extreme Value Theorem for semicontinuous function]
    If $X$ is compact and $f: X \longrightarrow R$ is upper (lower) semicontinuous, then $f$ attains its maximum (minimum) at some point of $X$.
\end{theorem}

\begin{theorem}
    Suppose that $X$ is a metric space with metric $d$, and let $f: X \longrightarrow[0, \infty]$ that $f(p)<\infty$ for at least one $p \in X$
    (or $f : X\longrightarrow \mathbb{R}$). Then $f(x)$ is lower semicontinuous if and only if there is real nonnegative continuous function sequence
    $\left\{f_n\right\}$ that
    $$\lim_{n \to \infty} f_n (x) \rightarrow f(x) \quad \text{ for all } x \in X$$

    Proof:
    For $n=1,2,3, \ldots, x \in X$, define
    $$f_{n}(x)=\inf \left\{f(p)+n d(x, p): p \in X\right\}$$
    and prove that
    $$\left|f_{n}(x)-f_{n}(y)\right| \leq n d(x, y)$$
    and
    $$0 \leq f_{1} \leq f_{2} \leq \cdots \leq f$$

    (iii) $f_{n}(x) \rightarrow f(x)$ as $n \rightarrow \infty$, for all $x \in X$.
\end{theorem}

\section{The completion of \texorpdfstring{$C_0(X)$}{}}
\begin{definition}
    A complex function $f$ on a locally compact Hausdorff space $X$ is said to \textbf{vanish at infinity} if to every $\varepsilon>0$ there exists a compact set $K \subset X$ such that $|f(x)| < \varepsilon$ for all $x \in X \backslash K $. The class of all continuous $f$ on $X$ which vanish at infinity is called $C_{0}(X)$.

    It is clear that $C_{c}(X) \subset C_{0}(X)$, and that the two classes coincide if $X$ is compact. In that case we write $C(X)$ for either of them.
\end{definition}
\begin{theorem}
    Let $X$ be locally compact Hausdorff space, then $C_0(X)$ is Banach space with norm $\cdot_\infty$
\end{theorem}
\begin{theorem}
    If $X$ is a locally compact Hausdorff space, then $C_{0}(X)$ is the completion of $C_{c}(X)$, relative to the metric defined by the supremum norm
    $$
        \left\| f \right\|=\sup _{x \in X} |f(x)|
    $$

    Proof:
    An elementary verification shows that $C_{0}(X)$ satisfies the axioms of a metric space if the distance between $f$ and $g$ is taken to be $\|f-g\|$. We have to show that (a) $C_{c}(X)$ is dense in $C_{0}(X)$ and (b) $ C_{0}(X)$ is a complete metric space.

    Given $f \in C_{0}(X)$ and $\varepsilon > 0$, there is a compact set $K$ so that $|f(x)| < \varepsilon$ outside $K$. Urysohn's lemma gives us a function $g \in C_{c}(X)$ such that $0 \leq g \leq 1$ and $g(x)=1$ on $K$. Put $h=f g$. Then $h \in C_{c}(X)$ and $\|f-h\|<\varepsilon$. This proves (a).

    To prove $(b)$, let $\left\{f_{n}\right\}$ be a Cauchy sequence in $C_{0}(X)$, i.e, assume that $\left\{f_{n}\right\}$ converges uniformly. Then its pointwise limit function $f$ is continuous. Given $\varepsilon>0$, there exists an $n$ so that $\left\|f_{n}-f\right\|<\varepsilon / 2$ and there is a compact set $K$ so that $\left|f_{n}(x)\right|<\varepsilon / 2$ outside $K$. Hence $|f(x)|<\varepsilon$ outside $K$, and we have proved that $f$ vanishes at infinity. Thus $C_{0}(X)$ is complete.
\end{theorem}

\section{Urysohn's Lemma (LCH's version)}
\begin{definition}
    Let $X$ be a topological space.

    (1) The notation
    $$ K \prec f$$
    will mean that $K$ is a compact subset of $X$, that $f \in C_c(X)$, that $0 \leq f(x) \leq 1$ for all $x \in X$, and that $f(x)=1$ for all $x \in K$.

    (2) The notation
    $$f \prec V$$
    will mean that $V$ is open, that $f \in C_c(X), 0 \leq f \leq 1$, and that the support of $f$ lies in $V$.

    (3) The notation
    $$K \prec f \prec V$$
    will be used to indicate that the both hold.
\end{definition}

\begin{theorem}[Urysohn's Lemma]
    \label{thm: Urysohn's Lemma}
    Suppose $X$ is a locally compact Hausdorff space, $F$ is closed, $K$ is compact in $X$, and $K \cap V\neq \varnothing$. Then there exists an $f \in C_c(X)$ maps $X$ into $[0,1]$, such that
    \begin{equation*}
        f(K)=1 ,\quad f(F) =0
    \end{equation*}
\end{theorem}

\begin{corollary}
    Suppose $X$ is a locally compact Hausdorff space, $V$ is open, $K$ is compact in $X$, and $K \subset V$. Then there exists an $f \in C_c(X)$, such that
    \begin{equation*}
        K \prec f\prec V
    \end{equation*}
\end{corollary}
\begin{theorem}
    Suppose $V_1, \ldots, V_n$ are open subsets of a locally compact Hausdorff space $X$, $K$ is compact, and
    $$
        K \subset V_1 \cup \cdots \cup V_n
    $$
    Then there exist functions $h_i \prec V_i \quad (i=1, \ldots, n)$ such that
    $$
        h_1(x)+\cdots+h_n(x)=1 \quad(x \in K)
    $$
    The collection $\left\{h_1, \ldots, h_n\right\}$ is called a \textbf{partition of unity on $K$}, subordinate to the cover $\left\{V_1, \ldots, V_n\right\}$.


    Proof:
    Each $x \in K$ has a neighborhood $W_x$ with compact closure $\overline{W}_x \subset V_i$ for some $i_x$. There are points $x_1, \ldots, x_m$ such that $W_{x_1} \cup \cdots \cup W_{x_m} \supset K$. If $1 \leq i \leq n$, let $H_1$ be the union of those $\overline{W}_x$ which lie in $V_i$. By Urysohn's lemma, there are functions $g_i$ such that $H_i \prec g_i \prec V_i$. Define
    \begin{equation*}
        \begin{aligned}
             & h_1=g_1                                                                     \\
             & h_2=\left(1-g_1\right) g_2                                                  \\
             & \cdots \cdots \cdots                                                        \\
             & h_n=\left(1-g_1\right)\left(1-g_2\right) \cdots\left(1-g_{n-1}\right) g_n .
        \end{aligned}
    \end{equation*}
    Then $h_i \prec V_i$. It is easily verified, by induction, that
    \begin{equation*}
        h_1+h_2+\cdots+h_n=1-\left(1-g_1\right)\left(1-g_2\right) \cdots\left(1-g_n\right) .
    \end{equation*}
    Since $K \subset H_1 \cup \cdots \cup H_n$, at least one $g_(x)=1$ at each point $x \in K$; hence (3) shows that (1) holds.
\end{theorem}


Throughout this chapter, $X$ will denote an $LCH$ space and $\mu$ a positive Borel measure on $X$.

\section{Basic Definition} % Basic Definition
\begin{definition}
    Let $\mu$ be a Borel measure on locally compact
    Hausdorff space $\left(X,\tau,\mathcal{B}\right)$ and $E$ a Borel set of $X$.
    \begin{enumerate}
        \item
              The measure $\mu$ is called \textbf{outter regular} on $E$ if
              \begin{equation*}
                  \mu(E)=\inf\left\{\mu(V): E \subset V,\: V \text{ open}\right\}
              \end{equation*}
        \item
              is called \textbf{inner regular} on $E$ if
              \begin{equation*}
                  \mu(E)=\sup\left\{\mu(K):K\subset E,\: K \text{ compact}\right\}
              \end{equation*}
        \item
              If $\mu$ is outer and inner regular on all Borel sets, $\mu$ is called \textbf{regular}.
    \end{enumerate}
\end{definition}

\begin{definition}
    A \textbf{Radon measure} $\mu$ on $X$ is a Borel measure that
    \begin{enumerate}
        \item
              is finite on all compact sets,
        \item
              is outer regular on all Borel sets,
        \item
              is inner regular on all open sets.
    \end{enumerate}
    \begin{remark}
        It follows the definition that $\mu$ is inner regular on $\sigma$-finite sets.


    \end{remark}
\end{definition}


\begin{corollary}
    Suppose that $\mu$ is a $\sigma$-finite Radon measure on $X$ and $E$ is a Borel set in $X$.
    \begin{enumerate}
        \item
              For every $\varepsilon>0$ there exist an open $U$ and a closed $F$ with $F \subset E \subset U$ and $\mu(U \backslash F)<\varepsilon$.
        \item
              There exist an $F_\sigma$ set $A$ and a $G_\delta$ set $B$ such that $A \subset E \subset B$ and $\mu(B \backslash A)=0$.
    \end{enumerate}
\end{corollary}

\begin{corollary}
    $\sigma$-compact $\Rightarrow \sigma$-finite $\Rightarrow $ regular
\end{corollary}





\section{Riesz Representation Theorem}
\begin{theorem}[Riesz Representation Theorem]
    Let $\left(X,\tau\right)$ be a locally compact Hausdorff space, and let $\Lambda$ be a positive linear functional on $C_c(X)$.
    Then there is a unique Radon measure $\mu$ on $X$ such that $\Lambda(f)=\int_X f \d \mu$ for all $f \in C_c(X)$.
    \begin{proof}
        Step 1. Construction of $\mu$ and $\mathcal{M}$
        For every open set $V$ in $X$, define
        \begin{equation*}
            \mu(V)=\sup \left\{\Lambda f : f\prec V\right\}
            \tag{1}
        \end{equation*}
        If $V_1 \subset V_2$, it is clear that (1) implies $\mu\left(V_1\right) \leq \mu\left(V_2\right)$. Hence
        \begin{equation*}
            \mu(E)=\inf \left\{\mu(V): E \subset V, V \text { open }\right\}
            \tag{2}
        \end{equation*}
        if $E$ is an open set, and it is consistent with (1) to define $\mu(E)$ by (2), for every $E \subset X$.

        Let $\mathcal{M}_F$ be the class of all $E \subset X$ which satisfy two conditions: $\mu(E)<\infty$, and
        \begin{equation*}
            \mu(E)=\sup \left\{\mu(K): K \subset E, K \text { compact }\right\}
            \tag{3}
        \end{equation*}
        Finally, let $\mathcal{M}$ be the class of all $E \subset X$ such that $E \cap K \in \mathcal{M}_F$ for every compact $K$.


        Step 2. Subadditivity. If $E_1, E_2, E_3, \ldots$ are arbitrary subsets of $X$, then
        \begin{equation*}
            \mu\left(\bigcup_{i=1}^{\infty} E_i\right) \leq \sum_{i=1}^{\infty} \mu\left(E_j\right)
            \tag{4}
        \end{equation*}
        We first show that
        \begin{equation*}
            \mu\left(V_1 \cup V_2\right) \leq \mu\left(V_1\right)+\mu\left(V_2\right)
        \end{equation*}
        if $V_1$ and $V_2$ are open. For any $g\prec V_1 \cup V_2$,
        there are functions $h_1$ and $h_2$ such that $h_i\prec V_i$ and $h_1(x)+h_2(x)=1$ for all $x$ in the support of $g$. Hence $h_i g\prec V_i, g=h_1 g+h_2 g$, and so
        \begin{equation*}
            \Lambda g=\Lambda\left(h_1 g\right)+\Lambda\left(h_2 g\right) \leq \mu\left(V_1\right)+\mu\left(V_2\right)
        \end{equation*}
        It follows that $\mu\left(V_1 \cup V_2\right) \leq \mu\left(V_1\right)+\mu\left(V_2\right)$.

        If $\mu\left(E_i\right)=\infty$ for some $i$, then (4) is trivially true. Suppose therefore that $\mu\left(E_i\right)<\infty$ for every $i$.
        Choose $\varepsilon>0$. By (2) there are open sets $V_i \supset E_i$ such that
        \begin{equation*}
            \mu\left(V_i\right)<\mu\left(E_i\right)+2^{-i} \varepsilon \quad(i=1,2,3, \ldots)
        \end{equation*}
        Put $V=\bigcup_i^{\infty} V_i$, and choose $f \prec V$. Since $f$ has compact support, we see that $f \prec V_1 \cup \cdots \cup V_n$ for some $n$. we therefore obtain
        \begin{equation*}
            \Lambda f
            \leq
            \mu\left(V_1 \cup \cdots \cup V_n\right)
            \leq
            \mu\left(V_1\right)+\cdots+\mu\left(V_n\right)
            \leq \sum_{i=1}^{\infty} \mu\left(E_i\right)+\varepsilon
        \end{equation*}
        Since this holds for every $f \prec V$, and since $\bigcup E_i \subset V$, it follows that
        \begin{equation*}
            \mu\left(\bigcup_{i=1}^{\infty} E_i\right) \leq \mu(V) \leq \sum_{i=1}^{\infty} \mu\left(E_i\right)+\varepsilon,
        \end{equation*}
        which proves (4), since $\varepsilon$ was arbitrary.

        Step 3. If $K$ is compact, then $K \in \mathcal{M}_F$ and
        \begin{equation*}
            \mu(K)=\inf \left\{\Lambda f: K \prec f\right\}
            \tag{5}
        \end{equation*}
        If $K \prec f$ and $0<\alpha<1$, let $V_\alpha=\left\{x: f(x)>\alpha\right\}$. Then $K \subset V_a$, and $\alpha g \leq f$ whenever $g \prec V_a$. Hence
        \begin{equation*}
            \mu(K) \leq \mu\left(V_a\right)
            =
            \sup \left\{\Lambda g: g \prec V_a\right\} \leq \alpha^{-1} \Lambda f
        \end{equation*}
        Let $\alpha \rightarrow 1^-$, to conclude that
        \begin{equation*}
            \mu(K) \leq \Lambda f
        \end{equation*}
        Thus $\mu(K)<\infty$. Since $K$ evidently satisfies (3), $K \in \mathcal{M}_F$.

        If $\varepsilon>0$, there exists $V \supset K$ with $\mu(V)<\mu(K)+\varepsilon$. By Urysohn's lemma, $K \prec f \prec V$ for some $f$. Thus
        \begin{equation*}
            \Lambda f \leq \mu(V)<\mu(K)+\varepsilon
        \end{equation*}
        which gives (5).

        Step 4. Every open set satisfies (3) is inner regular. Hence $\mathcal{M}_F$ contains every open set $V$ with $\mu(V)<\infty$.

        Let $\alpha$ be a real number such that $\alpha<\mu(V)$. There exists an $f \prec V$ with $\alpha<\Lambda f$. If $W$ is any open set which contains the support $K$ of $f$, then $f \prec W$, hence $\Lambda f \leq \mu(W)$. Thus
        \begin{equation*}
            \Lambda f
            \leq
            \inf \left\{\mu(W) : K \subset W , W \text{ is open}\right\}
            =
            \mu(K)
        \end{equation*}
        This exhibits a compact $K \subset V$ with $\alpha<\mu(K)$, so that (3) holds for $V$.

        Step 5. Suppose $E=\bigcup_{i=1}^{\infty} E_i$, where $E_1, E_2, E_3, \ldots$ are pairwise disjoint members of $\mathcal{M}_F$. Then
        \begin{equation*}
            \mu(E)=\sum_{i=1}^{\infty} \mu\left(E_i\right)
        \end{equation*}
        If, in addition, $\mu(E)<\infty$, then also $E \in \mathcal{M}_F$.

        We first show that
        \begin{equation*}
            \mu\left(K_1 \cup K_2\right)=\mu\left(K_1\right)+\mu\left(K_2\right)
        \end{equation*}
        if $K_1$ and $K_2$ are disjoint compact sets. Choose $\varepsilon>0$. By Urysohn's lemma, there exists $f \in C_c(X)$ such that $f(x)=1$ on $K_1, f(x)=0$ on $K_2$, and $0 \leq f \leq 1$. By 3. there exists $g$ such that
        \begin{equation*}
            K_1 \cup K_2<g \text { and } \Lambda g<\mu\left(K_1 \cup K_2\right)+\varepsilon
        \end{equation*}
        Note that $K_1 \prec f g$ and $K_2 \prec (1-f) g$. Since $\Lambda$ is linear, it follows from  that
        \begin{equation*}
            \mu\left(K_1\right)+\mu\left(K_2\right) \leq \Lambda(f g)+\Lambda(g-f g)=\Lambda g<\mu\left(K_1 \cup K_2\right)+\varepsilon
        \end{equation*}
        Since $\varepsilon$ was arbitrary, (10) follows now from Step 1.

        If $\mu(E)=\infty$, (9) follows from Step I. Assume therefore that $\mu(E)<\infty$, and choose $\varepsilon>0$. Since $E_i \in \mathcal{M}_F$, there are compact sets $H_i \subset E_i$ with
        \begin{equation*}
            \mu\left(H_i\right)>\mu\left(E_i\right)-2^{-i} E \quad(i=1,2,3, \ldots)
        \end{equation*}
        Putting $K_n=H_1 \cup \cdots \cup H_n$ and using induction on (10), we obtain
        \begin{equation*}
            \mu(E) \geq \mu\left(K_n\right)=\sum_{i=1}^n \mu\left(H_i\right)>\sum_{i=1}^n \mu\left(E_i\right)-\varepsilon
        \end{equation*}
        Since (12) holds for every $n$ and every $\varepsilon>0$, the left side of (9) is not smaller than the right side, and so (9) follows from Step I.
        But if $\mu(E)<\infty$ and $\varepsilon>0$, (9) shows that
        \begin{equation*}
            \mu(E) \leq \sum_{i=1}^N \mu\left(E_i\right)+\varepsilon
        \end{equation*}
        for some $N$. By (12), it follows that $\mu(E) \leq \mu\left(K_N\right)+2 \varepsilon_{\text {, }}$ and this shows that $E$ satisfies (3); hence $E \in \mathcal{M}_F$.


        Step 6. If $E \in \mathcal{M}_{\mathrm{F}}$ and $\in>0$, there is a compact $K$ and an open $V$ such that $K \subset E \subset V$ and $\mu(V-K)<\varepsilon$.Our definitions show that there exist $K \subset E$ and $V \supset E$ so that
        \begin{equation*}
            \mu(V)-\frac{\varepsilon}{2}<\mu(E)<\mu(K)+\frac{\varepsilon}{2}
        \end{equation*}
        Since $V-K$ is open, $V-K \in \mathcal{M}_F$, by Step III. Hence Step IV implies that
        \begin{equation*}
            \mu(K)+\mu(V-K)=\mu(V)<\mu(K)+\varepsilon
        \end{equation*}

        Step 7.
        If $A \in \mathcal{M}_F$ and $B \in \mathcal{M}_F$, then $A-  B, A \cup B$, and $A \cap B$ belong to $\mathcal{M}_F$.

        If $\varepsilon>0$, Step 6 shows that there are sets $K_i$ and $V_i$ such that $K_1 \subset A \subset V_1, K_2 \subset B \subset V_2$, and $\mu\left(V_i-K_i\right)<\varepsilon$, for $i=1$, 2. Since
        \begin{equation*}
            A-B \subset V_1-K_2 \subset\left(V_1-K_1\right) \cup\left(K_1-V_2\right) \cup\left(V_2-K_2\right),
        \end{equation*}
        Step I shows that
        \begin{equation*}
            \mu(A-B) \leq \varepsilon+\mu\left(K_1-V_2\right)+\varepsilon .
        \end{equation*}
        Since $K_1-V_2$ is a compact subset of $A-B$, (14) shows that $A-B$ satisfies (3), so that $A-B \in \mathbb{M}_F$.
        Since $A \cup B=(A-B) \cup B$, an application of Step IV shows that $A \cup B \in \mathcal{M}_f$. Since $A \cap B=A-(A-B)$, we also have $A \cap B \in \mathcal{M}_F$.
    \end{proof}
\end{theorem}

\begin{theorem}
    Let $X$ be a locally compact Hausdorff space in which every open set is $\sigma$-compact (implies $X$ is $\sigma$-compact). Let $\lambda$ be any positive Borel measure on ${X}$ such that $\lambda(K)<\infty$ for every compact set $K$. Then $\lambda$ is regular.

    Proof:
    Put $\Lambda f=\int_X f \d  \lambda$, for $f \in C_c(X)$. Since $\lambda(K)<\infty$ for every compact $K$, $\Lambda$ is a positive linear functional on $C_c(X)$, and there is a regular measure $\mu$, satisfying the conclusions of previous Theorem , such that
    $$
        \int_X f \d  \lambda=\int_X f \d  \mu \quad f \in C_c(X).
    $$

    We will show that $\lambda=\mu$. Let $V$ be open in $X$. Then $V=\bigcup K_i$, where $K_i$ is compact, $i=1,2,3, \ldots$. By Urysohn's lemma we can choose $f_i$ so that $K_i \prec f_i \prec V$. Let $g_n=\max \left(f_1, \ldots, f_n\right)$. Then $g_n \in C_c(X)$ and $g_n(x)$ increases to $\chi_V(x)$ at every point $x \in X$. Hence (1) and the monotone convergence theorem imply
    $$
        \lambda(V)=\lim _{n \rightarrow \infty} \int_X g_n\d  \lambda=\lim _{n \rightarrow \infty} \int_X g_n \d  \mu=\mu(V)
    $$

    Now let $E$ be a Borel set in $X$, there is a closed set $F$ and an open set $V$ such that $F \subset E \subset V$ and $\mu(V-F)<\varepsilon$. Hence $\mu(V) \leq \mu(F)+\varepsilon \leq \mu(E)+\varepsilon$.

    Since $V-F$ is open, shows that $\lambda(V-F)<\varepsilon$, hence $\lambda(V) \leq \lambda(E)+\varepsilon$. Consequently
    and
    $$\begin{aligned}
             & \lambda(E) \leq \lambda(V)=\mu(V) \leq \mu(E)+\varepsilon \\
             & \mu(E) \leq \mu(V)=\lambda(V) \leq \lambda(E)+\varepsilon
        \end{aligned}$$
    so that $|\lambda(E)-\mu(E)|<\varepsilon$ for every $\varepsilon>0$. Hence $\lambda(E)=\mu(E)$.
\end{theorem}



\section{Approximation} % Approximation
\begin{proposition}
    \label{pro: C_c(X) is dense in L^p(mu)}
    If $\mu$ is a Radon measure on $X$, $C_c(X)$ is dense in $L^p(\mu)$ for $1 \leq p<\infty$.
    \begin{proof}
        Since the $L^p$ simple functions are dense in $L^p$, it suffices to show that for any Borel set $E$ with $\mu(E)<\infty, \chi_E$ can be approximated in
        the $L^p$ norm by elements of $C_c(X)$. Given $\varepsilon>0$, by Proposition 7.5 we can choose a compact $K \subset E$ and an open $U \supset E$ such that $\mu(U \backslash K)<\varepsilon$, and by Urysohn's lemma we can choose $f \in C_c(X)$ such that $\chi_K \leq f \leq \chi_U$. Then $\left\|\chi_E-f\right\|_p \leq \mu(U \backslash K)^{1 / p}<\varepsilon^{1 / p}$, so we are done.
    \end{proof}
\end{proposition}

\begin{theorem}[Lusin's Theorem]
    \label{pro: Lusin's Theorem}
    Suppose that $\mu$ is a Radon measure on $X$ and $f: X \rightarrow \mathbb{C}$ is a measurable function that vanishes outside a set of finite measure. Then for any $\varepsilon>0$ there exists $\phi \in C_c(X)$ such that $\phi=f$ except on a set of measure $<\varepsilon$. If $f$ is bounded, $\phi$ can be taken to satisfy $\|\phi\|_u \leq\left\| f \right\|_u$.
    \begin{proof}
        Let $E=\left\{x: f(x) \neq 0\right\}$ of finite measure, and suppose to begin with that $f$ is bounded. Then $f \in L^1(\mu)$, so by \ref{pro: C_c(X) is dense in L^p(mu)} there is a sequence $\left\{g_n\right\}$ in $C_c(X)$ that converges to $f$ in $L^1$, and hence by Corollary 2.32 a subsequence (still denoted by $\left.\left\{g_n\right\}\right)$ that converges to $f$ a.e. By Egoroff's theorem there is a set $A \subset E$ such that $\mu(E \backslash A)<\varepsilon / 3$ and $g_n \rightarrow f$ uniformly on $A$, and there exist a compact $K \subset A$ and
        an open $U \supset E$ such that $\mu(A \backslash K)<\varepsilon / 3$ and $\mu(U \backslash E)<\varepsilon / 3$.
        Since $g_n \rightarrow f$ uniformly on $B, \left.f\right|_B$ is continuous, so by \ref{thm: Urysohn's Lemma} there exists $h \in C_c(X)$ such that $K\prec h \prec U$. But then $\{x: f(x) \neq h(x)\}$ is contained in $U \backslash B$, which has measure $<\varepsilon$.

        To complete the proof for $f$ bounded, define $\beta: \mathbb{C} \rightarrow \mathbb{C}$ by $\beta(z)=z$ if $|z| \leq\left\| f \right\|_u$ and $\beta(z)=\left\| f \right\|_u \operatorname{sgn} z$ if $|z|>\left\| f \right\|_u$, and set $\phi=\beta \circ h$. Then $\phi \in C_c(X)$ since $\beta$ is continuous and $\beta(0)=0$. Moreover, $\|\phi\|_u \leq\left\| f \right\|_u$, and $\phi=f$ on the set where $h=f$, so we are done.

        If $f$ is unbounded, let $A_n=\{x: 0<|f(x)| \leq n\}$. Then $A_n$ increases to $E$ as $n \rightarrow \infty$, so $\mu\left(E \backslash A_n\right)<\varepsilon / 2$ for sufficiently large $n$. By the preceding argument there exists $\phi \in C_c(X)$ such that $\phi=f \chi_{A_n}$ except on a set of measure $<\varepsilon / 2$, and hence $\phi=f$ except on a set of measure $<\varepsilon$.
    \end{proof}
\end{theorem}




\begin{theorem}[Borel measurable version]
    Let $f$ be a real-valued Lebesgue measurable function on $\mathbb{R}^k$. Prove that there exist Borel functions $g$ and $h$ such that $g(x)=h(x)$ a.e. $[m]$, and $g(x) \leq f(x) \leq h(x)$ for every $x \in \mathbb{R}^k$.
\end{theorem}

\begin{theorem}[The Vitali-Caratheodory Theorem]
    Suppose $f \in L^{1}(\mu)$, $f$ is real-valued, and $\varepsilon>0$. Then there exist functions $u$ and $v$ on $X$ such that $u \leq f \leq v, u$ is upper semicontinuous and bounded above, $v$ is lower semicontinuous and bounded below, and
    \begin{equation}
        \int_{X}(v-u) d \mu<\varepsilon \tag{1}
    \end{equation}

    Proof:
    Assume first that $f \geq 0$ and that $f$ is not identically 0 . Since $f$ is the pointwise limit of an increasing sequence of simple functions $s_{n}$, $f$ is the sum of the simple functions $t_{n}=s_{n}-s_{n-1}$ (taking $s_{0}=0$ ), and since $t_{n}$ is a linear combination of characteristic functions, we see that there are measurable sets $E_{i}$ (not necessarily disjoint) and constants $c_{i}>0$ such that
    \begin{equation}
        f(x)=\sum_{i=1}^{\infty} c_{i} \chi_{E_{i}}(x) \quad(x \in X) \tag{2}
    \end{equation}
    Since
    \begin{equation}
        \int_{X} f d u=\sum_{i=1}^{\infty} c_{i} \mu\left(E_{i}\right) \tag{3}
    \end{equation}
    the series in (3) converges. There are compact sets $K_{i}$ and open sets $V_{i}$ such that $K_{i} \subset E_{i} \subset V_{i}$ and

    \begin{equation}
        c_{i} \mu\left(V_{i}-K_{i}\right)<2^{-i-1} \varepsilon \quad(i=1,2,3, \ldots) \tag{4}
    \end{equation}
    Put
    \begin{equation}
        v=\sum_{i=1}^{\infty} c_{i} \chi_{V_{i}} \quad u=\sum_{i=1}^{N} c_{i} \chi_{K_{i}}, \tag{5}
    \end{equation}
    where $N$ is chosen so that

    \begin{equation}
        \sum_{N+1}^{\infty} c_{i} \mu\left(E_{j}\right)<\frac{\varepsilon}{2} . \tag{6}
    \end{equation}
    Then $v$ is lower semicontinuous, $u$ is upper semicontinuous, $u \leq f \leq 0$, and
    $$
        \begin{aligned}
            v-u & =\sum_{i=1}^{N} c_{i}\left(\chi_{V_{i}}-\chi_{K_{i}}\right)+\sum_{N+1}^{\infty} c_{i} \chi_{V_{i}}          \\
                & \leq \sum_{i=1}^{\infty} c_{i}\left(\chi_{V_{i}}-\chi_{K_{i}}\right)+\sum_{N+1}^{\infty} c_{i} \chi_{E_{i}}
        \end{aligned}
    $$
    so that (4) and (6) imply (1).

    In the general case, write $f=f^{+}-f^{-}$, attach $u_{1}$ and $v_{1}$ to $f^{+}$, attach $u_{2}$ and $v_{2}$ to $f^{-}$, as above, and put $u=u_{1}-v_{2}, v=v_{1}-u_{2}$. Since $-v_{2}$ is upper semicontinuous and since the sum of two upper semicontinuous functions is upper semicontinuous (similarly for lower semicontinuous; we leave the proof of this as an exercise), $u$ and $v$ have the desired properties.
\end{theorem}

\section{Lebesgue Measure}
\begin{theorem}[Lebesgue Measure on $\mathbb{R}^n$]
    There exists a positive complete measure $m$ defined on a $\sigma$-algebra $\mathcal{M}$ in $\mathbb{R}^{k}$, with the following properties:

    (a) $m(W)=\mathrm{vol}(W)$ for every $k$-cell $W$.

    (b) $\mathcal{M}$ contains all Borel sets in $\mathbb{R}^{k}$; more precisely, $E \in \mathfrak{P}$ if and only if there are sets $A$ and $B \subset R^{k}$ such that $A \subset E \subset B, A$ is an $F_{\sigma}, B$ is $a G_{d}$, and $m(B-A)=0$. Also, $m$ is regular.

    (c) $m$ is translation-invariant, i.e.,
    $$
        m(E+x)=m(E)
    $$
    for every $E \in \mathcal{M}$ and every $x \in \mathbb{R}^{k}$.



    (e) To every linear transformation $T$ of $R^{k}$ into $R^{k}$ corresponds a real number $\Delta(T)$ such that
    $$m(T(E))=\Delta(T) m(E)$$
    for every $E \in \mathbb{M}$. In particular, $m(T(E))=m(E)$ when $T$ is a rotation.

    The members of $\mathcal{M}$ are the Lebesgue measurable sets in $\mathbb{R}^{k}$; $m$ is the Lebesgue measure on $\mathbb{R}^{k}$.

    \textbf{Proof}:
    If $f$ is any complex function on $\mathbb{R}^{k}$, with compact support, define
    \begin{equation}
        \Lambda_{n} f=2^{-n k} \sum_{2^n x \in \mathbb{Z}} f(x) \quad (n=1,2,3, \ldots) \tag{1}
    \end{equation}
    Now suppose $f \in C_{c}\left(R^{k}\right)$, $f$ is real, $W$ is an open $k$-cell which contains the support of $f$, and $\varepsilon>0$. The uniform continuity of $f$ shows that there is an integer $N$ and that there are functions $g$ and $h$ with support in $W$, such that (i) $g$ and $h$ are constant on each box belonging to $\Omega_{N}$, (ii) $g \leq f \leq h$, and (iii) $h-g<\varepsilon$. If $n>N$, then
    \begin{equation}
        \Lambda_{N} g=\Lambda_{\mathrm{n}} g \leq \Lambda_{\mathrm{n}} f \leq \Lambda_{n} h=\Lambda_{N} h \tag{2}
    \end{equation}
    Thus the upper and lower limits of $\left\{\Lambda_{n} f\right\}$ differ by at most $ \mathrm{vol}(W)\varepsilon$, and since $\varepsilon$ was arbitrary, we have proved the existence of
    \begin{equation}
        \Lambda f=\lim _{n \rightarrow \infty} \Lambda_{n} f \quad f \in C_{c}\left(R^{k}\right) \tag{3}
    \end{equation}
    It is immediate that $\Lambda$ is a positive linear functional on $C_{c}\left(R^{k}\right)$. (In fact, $\Lambda f$ is precisely the Riemann integral of $f$ over $R^{k}$.)

    2. We define $m$ and $\mathcal{M}$ to be the measure and $\sigma$-algebra associated with this $\Lambda$ as in Riesz Representation Theorem. Since Theorem 2.14 gives us a complete measure and since $R^{k}$ is $\sigma$ compact, Theorem 2.17 011implies assertion (b).

    3. To prove (a), let $W$ be the open cell , let $E_{r}$ be the union of those boxes belonging to $\Omega_{r}$ whose closures lie in $W$, choose $f_{r}$ so that
    $\overline{E}_{r}\prec$ $f_{r} \prec W$
    , and put $g_{r}=\max \left\{f_{1}, \ldots, f_{r}\right\}$. Our construction of $\Lambda$ shows that
    \begin{equation}
        \mathrm{vol}\left(E_{r}\right) \leq \Lambda f_{r} \leq \Lambda g_{r} \leq \mathrm{vol} (W) \tag{4}
    \end{equation}
    As $r \rightarrow \infty, \mathrm{vol}\left(E_{r}\right) \rightarrow \mathrm{vol}(W)$, and
    \begin{equation}
        \Lambda g_r=\int g_{r} \d  m \rightarrow m(W) \tag{5}
    \end{equation}
    by the monotone convergence theorem, since $g_{r}(x) \rightarrow \chi_{W}(x)$ for all $x \in R^{k}$. Thus $m(W)=\mathrm{vol}(W)$ for every open cell $W$, and since every $k$-cell is the intersection of a decreasing sequence of open $k$-cells, we obtain (a).

    The proofs of $(c),(d)$, and $(e)$ will use the following observation: If $\lambda$ is a positive Borel measure on $R^{k}$ and $\lambda(E)=m(E)$ for all boxes $E$, then the same equality holds for all open sets $E$, by property $2.19(d)$, and therefore for all Borel sets $E$, since $\lambda$ and $m$ are regular (Theorem 2.18).

    To prove (c), fix $x \in R^{k}$ and define $\lambda(E)=m(E+x)$. It is clear that $\lambda$ is then a measure; by $(a), \lambda(E)=m(E)$ for all boxes, hence $m(E+x)=m(E)$ for all Borel sets $E$. The same equality holds for every $E \in \mathbb{P}$, because of ( $b$ ).

    Suppose next that $\mu$ satisfies the hypotheses of (d). Let $Q_{0}$ be a 1 -box, put $c=\mu\left(Q_{0}\right)$. Since $Q_{0}$ is the union of $2^{n k}$ disjoint $2^{-n}$ boxes that are translates of each other, we have
    $$
        2^{n k} \mu(Q)=\mu\left(Q_{0}\right)=c m\left(Q_{0}\right)=c \cdot 2^{n k} m(Q)
    $$
    for every $2^{-n}$-box $Q$. Property 2.19(d) implies now that $\mu(E)=c m(E)$ for all open sets $E \subset R^{k}$. This proves ( $d$ ).

    To prove (e), let $T: R^{k} \rightarrow R^{k}$ be linear. If the range of $T$ is a subspace $Y$ of lower dimension, then $m(Y)=0$ and the desired conclusion holds with $\Delta(T)=0$. In the other case, elementary linear algebra tells us that $T$ is a one-to-one map of $R^{k}$ onto $R^{k}$ whose inverse is also linear. Thus $T$ is a homeomorphism of $R^{k}$ onto $R^{k}$, so that $T(E)$ is a Borel set for every Borel set $E$, and we can therefore define a positive Borel measure $\mu$ on $R^{k}$ by
    $$
        \mu(E)=m(T(E))
    $$

    The linearity of $T$, combined with the translation-invariance of $m$, gives
    $$
        \mu(E+x)=m(T(E+x))=m(T(E)+T x)=m(T(E))=\mu(E) .
    $$

    Thus $\mu$ is translation-invariant, and the first assertion of (e) follows from (d), first for Borel sets $E$, then for all $E \in \mathcal{M}$ by (b).

    To find $\Delta(T)$, we merely need to know $m(T(E) / m(E)$ for one set $E$ with $0<m(E)<\infty$. If $T$ is a rotation, let $E$ be the unit ball of $R^{k}$; then $T(E)=E$, and $\Delta(T)=1$.
\end{theorem}

\begin{theorem}
    If $\mu$ is any positive translation-invariant Borel measure on $R^{k}$ such that $\mu(K)<\infty$ for every compact set $K$, then there is a constant $c$ such that $\mu(E)=cm(E)$ for all Borel sets $E \subset R^{k}$.
\end{theorem}





\chapter{\texorpdfstring{$L^p$}{} Space}

\section{Definition}
\begin{definition}
    Let $\left(X,\mathcal{M},\mu\right)$ be a measure space.
    If $ 0<p<\infty$ and if $f$ is a complex measurable function on $X$, define
    \begin{equation*}
        \left\| f \right\|_{p}=\left\{\int_{X} |f|^{p} \d  \mu\right\}^{1 / p}
    \end{equation*}
    and let $L^{p}(\mu)$ consist of all $f$ for which
    \begin{equation*}
        \left\| f \right\|_{p}<\infty
    \end{equation*}
    We call $\left\| f \right\|_{p}$ the $L^p$-norm of $f$.
    If $\mu$ is Lebesgue measure on $\mathbb{R}^{k}$, we write $L^{p}\left(\mathbb{R}^{k}\right)$ instead of $L^{p}(\mu)$.
\end{definition}
\begin{definition}
    Suppose $g: X \longrightarrow [0, \infty]$ is measurable. Let $S$ be the set of all real $\alpha$ such that
    \begin{equation*}
        \mu\left(g^{-1}((\alpha, \infty])\right)=0
    \end{equation*}
    If $S=\varnothing$, put $\beta=\infty$. If $S \neq \varnothing$, put $\beta=\inf S$. Since
    \begin{equation*}
        g^{-1}((\beta, \infty]) = \bigcup_{n=1}^{\infty} g^{-1} \left((\beta+\frac{1}{n}, \infty]\right)
    \end{equation*}
    and since the union of a countable collection of sets of measure 0 has measure 0 , we see that $\beta \in S$. We call $\beta$ the \textbf{essential supremum} of $g$.

    If $f$ is a complex measurable function on $X$, we define $\left\| f \right\|_{\infty}$ to be the essential supremum of $|f|$, and we let $L^{\infty}(\mu)$ consist of all $f$ for which $\left\| f \right\|_{\infty}<\infty$. The members of $L^{\infty}(\mu)$ are sometimes called essentially bounded measurable functions on $X$.

    \begin{remark}
        It follows from this definition that the inequality $|f(x)| \leq \lambda$ holds for almost all $x$ if and only if $\lambda \geq\left\| f \right\|_{\infty}$.
    \end{remark}
\end{definition}

\begin{proposition}
    If $f$ is a measurable function on $X$, define the \textbf{essential range} $R_f$ of $f$ to be the set of all $z \in \mathbb{C}$ such that $\{x:|f(x)-z|<\epsilon\}$ has positive measure for all $\epsilon>0$. Then
    \begin{enumerate}
        \item
              $R_f$ is closed.
        \item
              If $f \in L^{\infty}$, then $R_f$ is compact and $\|f\|_{\infty}=\max \left\{|z|: z \in R_f\right\}$.
    \end{enumerate}
\end{proposition}


\begin{theorem}[Holder's Inequality]
    Suppoes $p$ and $q$ are conjugate exponents, $1 \leq p \leq \infty$, and if $f \in L^{p}(\mu)$ and $g \in L^{g}(\mu)$, then $f g \in L^{1}(\mu)$, and
    \begin{equation*}
        \left\| f g \right\|_1 \leq \left\| f \right\|_p  \left\| g \right\|_q
    \end{equation*}
    and in this case equality holds iff
    \begin{center}
        $\left|f\right|^p$ and $\left|g\right|^q$ are linear dependent in $L^0$.  $1 < p < \infty$ \\
        $\left|g(x)\right|=\left\| g \right\|_{\infty}$ a.e. on the set $\left\{x:f(x) \neq 0\right\}$.
    \end{center}

\end{theorem}

\begin{corollary}[Generalized]
    For $\sum\frac{1}{p_i}=\frac{1}{p}$ with $1 \leq  p_i$ we have
    \begin{equation*}
        \left\| \prod f_i \right\|_p
        \leq
        \prod\left\| f_i \right\|_{p_i}
    \end{equation*}
\end{corollary}




\begin{corollary}[Interpolation]
    Suppose $\frac{\theta}{p}=\sum \frac{\theta_i}{p_i}$ with $\sum \theta_i=\theta$
    \begin{equation*}
        \left\| f   \right\|_{p}^{\theta}
        \leq
        \prod \left\| f \right\|_{p_i}^{\theta_i}
    \end{equation*}
\end{corollary}

\begin{theorem} [Minkowski's Inequality]
    Suppose $1 \leq p \leq \infty$, and $f \in L^{p}(\mu), g \in L^{p}(\mu)$. Then $f+g \in L^{p}(\mu)$, and
    \begin{equation*}
        \left\| f +g\right\|_{p} \leq\left\| f \right\|_{p}+\left\| g \right\|_{p}
    \end{equation*}
\end{theorem}


\section{\texorpdfstring{$L^p$}{} Space}
\begin{theorem}
    Let $\left(X,\mathcal{M},\mu\right)$ be a measure space. Then $L^{p}(\mu)$ is a banach space for $1 \leq p \leq \infty$.
    \begin{proof}
        Assume first that $1 \leq p<\infty$. Let $\left\{f_{n}\right\}$ be a Cauchy sequence in $L^p(\mu)$. There is a subsequence $\left\{f_{n}\right\}, n_{1}<n_{2}<\cdots$, such that
        \begin{equation*}
            \left\|f_{n_{i+1}}-f_{n_i}\right\|_{p}<2^{-i}
        \end{equation*}
        Put
        \begin{equation*}
            g_{k}=\sum_{i=1}^{k}\left|f_{n_{i-1}}-f_{n_{i}}\right|, \quad g=\sum_{i=1}^{\infty}\left|f_{n_{i}+1}-f_{m_{i}}\right|
        \end{equation*}
        the Minkowski inequality shows that
        $\left\|g_{k}\right\|_{p}<1$
        for $k=1$, $2,3, \ldots$. Hence an application of Fatou's lemma to
        $\left\{g^p\right\}$ gives $\left\| g \right\|_{p} \leq 1$.
        In particular, $g(x)<\infty$ a.e., so that the series
        \begin{equation*}
            f_{n_{1}}(x)+\sum_{i=1}^{\infty}\left(f_{n_{i+1}}(x)-f_{n_i}(x)\right)
        \end{equation*}
        converges absolutely for a.e. $x \in X$. Denote the sum of (3) by $f(x)$, for those $x$ at which (3) converges; put $f(x)=0$ on the remaining set of measure zero. Since
        $$f_{n+1}+\sum_{i=1}^{k-1}\left(f_{n+1}-f_{n i}\right)=f_{n t} $$
        we see that
        $$f(x)=\lim _{i \rightarrow \infty} f_{m}(x) \quad \text { a.e. } $$

        Having found a function $f$ which is the pointwise limit a.e. of $\left\{f_{n}\right\}$, we now have to prove that this $f$ is the $L^{p}$-limit of $\left\{f_{m}\right\}$. Choose $\varepsilon>0$. There exists an $N$ such that $\left\|f_{n}-f_{m}\right\|_{p}<\varepsilon$ if $n>N$ and $m>N$. For every $m>N$, Fatou's lemma shows therefore that
        $$
            \int_{X}\left|f-f_{m}\right|^{p} d \mu \leq \liminf _{i \rightarrow \infty} \int_{X}\left|f_{m_{i}}-f_{m}\right|^{p} d \mu \leq \varepsilon^{p}
        $$

        We conclude from (6) that $f-f_{m} \in L^p(\mu)$, hence that $f \in L(\mu)$ [since $f=$ $\left(f-f_{m}\right)+f_{m}$, and finally that $\left\|f-f_{m}\right\|_{p} \rightarrow 0$ as $m \rightarrow \infty$. This completes the proof for the case $1 \leq p<\infty$.

        In $L^{\infty}(\mu)$ the proof is much easier. Suppose $\left\{f_{n}\right\}$ is a Cauchy sequence in $L^{\infty}(\mu)$, let $A_{k}$ and $B_{m, n}$ be the sets where $\left|f_{k}(x)\right|>\left\|f_{k}\right\|_{\infty}$ and where $\left|f_{m}(x)-f_{m}(x)\right|>\left\|f_{n}-f_{m}\right\|_{\infty}$, and let $E$ be the union of these sets, for $k, m$, $n=1,2,3, \ldots$ Then $\mu(E)=0$, and on the complement of $E$ the sequence $\left\{f_{n}\right\}$ converges uniformly to a bounded function $f$. Define $f(x)=0$ for $x \in E$. Then $f \in L^{\infty}(\mu)$, and $\left\|f_{n}-f\right\|_{\infty} \rightarrow 0$ as $n \rightarrow \infty$.
    \end{proof}
\end{theorem}



\begin{proposition}
    If $1 \leq p<q<r \leq \infty$,
    \begin{enumerate}
        \item
              $L^p \cap L^r$ is a Banach space with norm $\left\| f \right\|=\left\| f \right\|_{p}+\left\| f \right\|_{r}$
        \item
              The inclusion map $L^p \cap L^r \rightarrow L^q$ is continuous.
    \end{enumerate}
    \begin{remark}
        It follows $\left\| f \right\|_{q}\leq \left\| f \right\|_{p}^{\theta_1}\left\| f \right\|_{r}^{\theta_2}\leq \theta_1\left\| f \right\|_{p}+\theta_2\left\| f \right\|_{r} \leq \left\| f \right\|_{}$ that $\left\| \cdot \right\|_{}$ and $\left\| \cdot \right\|_{q}$ are equivalent in $L^p\cap L^r$.
    \end{remark}
\end{proposition}

\begin{proposition}
    If $1 \leq p<q<r \leq \infty$,
    \begin{enumerate}
        \item
              $ L^p+L^r$ is a Banach space with norm
              $\left\| f \right\|_{}=\inf \left\{\left\| g \right\|_{p}+\left\| h  \right\|_{r} :f=g+h\right\}$
        \item
              The inclusion map $L^q \rightarrow L^p+L^r$ is continuous.
    \end{enumerate}
    \begin{proof}
        First, given $f\in L^p$, we have
        \begin{equation*}
            f=\chi_E f +\chi_{E^c} f
        \end{equation*}
        where $E=\left\{x:\left|f(x)\right|>1\right\}$ of finite measure. Thus $g=\chi_E f \in L^p$, $h=\chi_{E^c} f\in L^r$ and $L^q \subset L^p+L^r$.
        Furthermore,
        \begin{equation*}
            \left\| f \right\|_{}\leq \left\| \chi_E f  \right\|_{p}+\left\| \chi_{E^c} f \right\|_{r}
        \end{equation*}
    \end{proof}
\end{proposition}



\section{Approximation in \texorpdfstring{$L^p$}{}}
\begin{theorem}[Simple function]
    For $1\leq p < \infty$, the set
    $S=\left\{f=\sum_1^n a_j \chi_{E_j}: \mu\left(E_j\right)<\infty \right\}$  is dense in $L^{p}(\mu)$.
\end{theorem}

\begin{theorem}[$C_c(X)$ on local compact Hausdorff space $X$]
    Let $X$ be a LCHS and $\mu$ be a measure on a $\sigma$-algebra $\mathcal{M}$ on $X$ with the properties stated in Riesz Representation Theorem.
    For $1 \leq p<\infty$, $C_{c}(X)$ is dense in $L^p(\mu)$.

    Proof:
    Suppose $f \in L^p(\mu)$.
    Let $s$ be a complex, measurable, simple functions with
    $$\|s-f\|_p < \frac{\varepsilon}{2}$$
    Also, there exists a $g \in$ $C_{c}(X)$ such that $g(x)=s(x)$ except on a set of measure $<\varepsilon$, and $|g| \leq\|s\|_{\infty}$ (Lusin's theorem). Hence
    $$
        \|g-s\|_{p} \leq 2 \varepsilon^{1 / p}\|s\|_{\infty} .
    $$
    This completes the proof.

    Thus $L^p (\mu)$ is the completion of the metric space which is obtained by endowing $C_{c} (\mu)$ with the $L^{p}$-metric.
\end{theorem}


\begin{theorem}[Step function on $\mathbb{R}$]
    A step function is, by definition, a finite linear combination of characteristic functions of bounded intervals in $\mathbb{R}$.
\end{theorem}

\section{Distribution Functions and Weak \texorpdfstring{$L^p$}{}} % Distribution Functions


\begin{definition}
    If $f$ is a measurable function on $(X, \mathcal{M}, \mu)$, we define its \textbf{distribution function} $\lambda_f:(0, \infty) \rightarrow[0, \infty]$ by
    \begin{equation*}
        \lambda_f(\alpha)=\mu(\{x:|f(x)|>\alpha\}) .
    \end{equation*}
\end{definition}

\begin{proposition}.
    \begin{enumerate}
        \item
              $\lambda_f$ is decreasing and right continuous.
        \item
              If $|f| \leq|g|$, then $\lambda_f \leq \lambda_g$.
        \item
              If $\left|f_n\right|$ increases to $|f|$, then $\lambda_{f_n}$ increases to $\lambda_f$.
        \item
              If $f=g+h$, then $\lambda_f(\alpha) \leq \lambda_g\left(\frac{1}{2} \alpha\right)+\lambda_h\left(\frac{1}{2} \alpha\right)$.
    \end{enumerate}
\end{proposition}

\begin{theorem}
    If $\lambda_f(\alpha)<\infty$ for all $\alpha>0$ and $\phi$ is a nonnegative Borel measurable function on $(0, \infty)$, then
    \begin{equation*}
        \int_X \phi \circ|f| d \mu=-\int_0^{\infty} \phi(\alpha) d \lambda_f(\alpha)
    \end{equation*}
\end{theorem}

\begin{definition}
    Let $\left(X,\mathcal{M},\mu\right)$ be measure space.
    If $f$ is a measurable function on $X$ and $0<p<\infty$, we define
    \begin{equation*}
        [f]_p=\left(\sup _{\alpha>0} \alpha^p \lambda_f(\alpha)\right)^{1 / p}
    \end{equation*}
    and we define weak $L^p$ to be the set of all $f$ such that $[f]_p<\infty$。 $\left[\cdot\right]_p$ is a quasinorm ()

    However, weak $L^p$ is a topological vector space; see Exercise 35.

    The relationship between $L^p$ and weak $L^p$ is as follows. On the one hand,
    \begin{equation*}
        L^p \subset \text { weak } L^p, \quad \text { and } \quad[f]_p \leq\|f\|_p
    \end{equation*}
    (This is just a restatement of Chebyshev's inequality.) On the other hand, if we replace $\lambda_f(\alpha)$ by $\left([f]_p / \alpha\right)^p$ in the integral $p \int_0^{\infty} \alpha^{p-1} \lambda_f(\alpha) d \alpha$, which equals $\|f\|_p^p$, we obtain a constant times $\int_0^{\infty} \alpha^{-1} d \alpha$, which is divergent at both 0 and $\infty$ - but
\end{definition}







\section{Mapping from \texorpdfstring{$L^p$}{} to \texorpdfstring{$L^1$}{}}

\begin{lemma}
    For $x \geq 0, y \geq 0$, that
    \begin{equation*}
        \left|x^p-y^p\right| \leq \begin{cases}|x-y|^p & \text { if } 0<p<1 \\ p|x-y|\left(x^{p-1}+y^{p-1}\right) & \text { if } 1 \leq p<\infty\end{cases}
    \end{equation*}

    Note that (a) and (b) establish the continuity of the mapping $f \rightarrow|f|^\rho$ that carries $E(\mu)$ into $L^1(\mu)$.
\end{lemma}


\begin{theorem}
    Suppose $\mu$ is a positive measure, $f,g \in L^p(\mu)$. Let the natural map
    \begin{equation*}
        f\rightarrow \left|f\right|^p
    \end{equation*}
    carries $L^p(\mu)$ into $L^1(\mu)$

    (1) Lipschitz continuous. If $ 0<p<1$, prove that
    \begin{equation*}
        \int\left||f|^p-|g|^p\right| d \mu \leq \int|f-g|^p d \mu
    \end{equation*}
    that $\Delta(f, g)=\int|f-g|^p d \mu$ defines a metric on $L^p(\mu)$, and that the resulting metric space is complete.

    (2) Local lipschitz continuous. If $1 \leq p<\infty$ and $\left\| f \right\|_p \leq R,\left\| g \right\|_p \leq R$, prove that
    \begin{equation*}
        \int\left| \left|f\right|^p-\left|g\right|^p \right| d \mu \leq 2 p R^{p-1}\|f-g\|_p
    \end{equation*}
\end{theorem}


\section{}
\begin{theorem}[Young's]
    Let $p, q\in[1, \infty]$ and $f \in L^p$ and $g\in L^q$. Then $f * g$ $\in L^r$ and
    \begin{equation*}
        \|f * g\|_{L^r}
        \leq
        \left\| f \right\|_{L^p}
        \left\| g \right\|_{L^q}
    \end{equation*}
    where
    \begin{equation*}
        \frac{1}{p}+\frac{1}{q}=\frac{1}{r}+1
    \end{equation*}

    Proof:
    Show that the desired inequality is equivalent to
    \begin{equation*}
        \left|\int f(y) g(x-y) h(x) \d x \d y\right|
        \leq
        \left\| f \right\|_{L^p}\left\| g \right\|_{L^q}\|h\|_{L^s}
    \end{equation*}
    for all $h \in L^s$ with $\frac{1}{r}+\frac{1}{s}=1$ (prove that $f^*g \in \left(L^r\right)^*$)
    \begin{equation*}
        \begin{aligned}
                 & \left|\int f(y) g(x-y) h(x) \d x \d y\right|                                                                                                                                                                                                                               \\
            \leq & \int \left|f(y) g(x-y) h(x)\right| \d x \d y                                                                                                                                                                                                                               \\
                 & \leq \left\|\left(|g(x-y)|^q|h(x)|^s\right)^{\frac{p-1}{p}}\right\|_{\frac{p}{p-1}\left(\mathbb{R}^2\right)}\left\|\left(|f(y)|^p|h(x)|^s\right)^{\frac{q-1}{q}}\right\|_{L^{\frac{q}{q-1}\left(\mathbb{R}^2\right)}}                                                      \\
                 & \quad \cdot\left\|\left(|f(y)|^p|g(x-y)|^q\right)^{\frac{s-1}{s}}\right\|_{L^{\frac{s}{s-1}}\left(\mathbb{R}^2\right)}^{\frac{p-1}{p}}\left(\int_{(x, y) \in \mathbb{R}^2}|f(y)|^p|h(x)|^s d x d y\right)^{\frac{q-1}{q}}                                                  \\
                 & =\left(\int_{(x, y) \in \mathbb{R}^2}|g(x-y)|^q|h(x)|^s d x d y\right)^{\frac{s-1}{s}}                                                                                                                                                                                     \\
                 & \quad \cdot\left(\int_{(x, y) \in \mathbb{R}^2}|f(y)|^p|g(x-y)|^q d x d y\right)^{\frac{q-1}{q}+\frac{s-1}{s}}\left(\int_{y \in \mathbb{R}}|g(y)|^q d y\right)^{\frac{p-1}{p}+\frac{s-1}{s}}\left(\int_{x \in \mathbb{R}}|h(x)|^s d y\right)^{\frac{p-1}{p}+\frac{q-1}{q}} \\
                 & =\left(\int_{y \in \mathbb{R}}|f(y)|^p d y\right)
        \end{aligned}
    \end{equation*}
    (from Fubini's theorem of integrating with respect to $x$ or $y$ first)
    \begin{equation*}
        =\left(\int_{y \in \mathbb{R}}|f(y)|^p d y\right)^{\frac{1}{p}}\left(\int_{y \in \mathbb{R}}|g(y)|^q d y\right)^{\frac{1}{q}}\left(\int_{x \in \mathbb{R}}|h(x)|^s d y\right)^{\frac{1}{s}}
    \end{equation*}
    which is the inequality we would like to prove, where the last step comes from
    \begin{equation*}
        \frac{q-1}{q}+\frac{s-1}{s}=\frac{1}{p}, \quad \frac{s-1}{s}+\frac{p-1}{p}=\frac{1}{q}, \quad \text { and } \quad \frac{p-1}{p}+\frac{q-1}{q}=\frac{1}{s} .
    \end{equation*}



    Another proof:
    Indeed, $\frac{1}{r}+ \frac{1}{q'}=\frac{1}{p}$, $\frac{1}{r}+ \frac{1}{p'}=\frac{1}{q}$ where $\frac{1}{p}+\frac{1}{p'}=1$ and $\frac{1}{q}+\frac{1}{q'}=1$
    \begin{equation*}
        \begin{aligned}
            \left|f*g(x) \right|
             & \leq
            \int_{\mathbb{R}^n} \left|f(y) g(x-y)\right| \d y                 \\
             & =\int_{\mathbb{R}^n}\left|f(y)^p\right|^{\frac{1}{q^{\prime}}}
            \left|f(y)^p g\left(y-x\right)^q\right|^{\frac{1}{r}}
            \left|g\left(y -x\right)^q\right|^{\frac{1}{p^{\prime}}}  \d y    \\
             & \leq \left\| f^{\frac{p}{q'}}\right\|_{L^{q'}}
            \left(\int_{\mathbb{R}^n} f(y)^p g(x-y)^q \d y\right)^{\frac{1}{r}}
            \left\|g^{\frac{q}{p'}}\right\|_{L^{p'}}
        \end{aligned}
    \end{equation*}
    Thus
    \begin{equation*}
        \begin{aligned}
            \left\|f*g\right\|_{L^r} & \leq \left\| f\right\|_{L^p}^{\frac{p}{q'}} \left\| g\right\|_{L^q}^{\frac{q}{p'}}
            \left(\int \int_{\mathbb{R}^n} f(y)^p g(x-y)^q \d y \d x\right)^{\frac{1}{r}}                                 \\
                                     & = \left\| f\right\|_{L^p}
            \left\| g\right\|_{L^q}
        \end{aligned}
    \end{equation*}
\end{theorem}


\chapter{Signed Measures}
\section{}
\begin{definition}
    Let $\left(X, \mathcal{M}\right)$ be a measurable space.
    A \textbf{signed measure} $\nu$ on $\left(X, \mathcal{M}\right)$ is a function  such that
    \begin{enumerate}[label=(\roman*)]
        \item
              $\nu: \mathcal{M} \rightarrow[-\infty, \infty]$ and $\nu$ assumes at most one of the values $\pm \infty$;
        \item
              $\nu(\varnothing)=0$;
        \item
              if $\left\{E_j\right\}$ is a sequence of disjoint sets in $\mathcal{M}$, then $\nu\left(\bigcup_1^{\infty} E_j\right)=\sum_1^{\infty} \nu\left(E_j\right)$, where the latter sum converges absolutely if $\nu\left(\bigcup_1^{\infty} E_j\right)$ is finite.
    \end{enumerate}
    Thus every measure is a signed measure; for emphasis we shall sometimes refer to measures as \textbf{positive measures}.
\end{definition}

\begin{proposition}
    Let $\left(X, \mathcal{M}\right)$ be a measurable space.
    \begin{enumerate}
        \item
              First, if $\mu_1, \mu_2$ are measures on $\mathcal{M}$ and at least one of them is finite, then $\nu=\mu_1-\mu_2$ is a signed measure.
        \item
              Second, if $\mu$ is a measure on $\mathcal{M}$ and $f: X \rightarrow[-\infty, \infty]$ is a measurable function such that at least one of $\int f^{+} d \mu$ and $\int f^{-} d \mu$ is finite (in which case we shall call $f$ an extended $\mu$-integrable function), then the set function $\nu$ defined by $\nu(E)=\int_E f d \mu$ is a signed measure.
    \end{enumerate}

    In fact, we shall see shortly that these are really the only examples: Every signed measure can be represented in either of these two forms.
\end{proposition}




\begin{theorem}[The Hahn Decomposition Theorem]
    Let $\nu$ be a signed measure on $(X, \mathcal{M})$, there exist a positive set $P$ and a negative set $N$ for $\nu$ such that $P \cup N=X$ and $P \cap N=\varnothing$. If $P^{\prime}, N^{\prime}$ is another such pair, then $P \Delta P^{\prime}\left(=N \Delta N^{\prime}\right)$ is null for $\nu$.
\end{theorem}




\begin{definition}[The Jordan Decomposition Theorem]
    Let $\nu$ be a signed measure on $(X, \mathcal{M})$, there exist unique positive measures $\nu^{+}$and $\nu^{-}$such that $\nu=\nu^{+}-\nu^{-}$and $\nu^{+} \perp \nu^{-}$.

    Proof. Let $X=P \cup N$ be a Hahn decomposition for $\nu$, and define $\nu^{+}(E)= \nu(E \cap P)$ and $\nu^{-}(E)=-\nu(E \cap N)$. Then clearly $\nu=\nu^{+}-\nu^{-}$and $\nu^{+} \perp \nu^{-}$. If also $\nu=\mu^{+}-\mu^{-}$and $\mu^{+} \perp \mu^{-}$, let $E, F \in \mathcal{M}$ be such that $E \cap F=\varnothing, E \cup F=X$, and $\mu^{+}(F)=\mu^{-}(E)=0$. Then $X=E \cup F$ is another Hahn decomposition for $\nu$, so $P \Delta E$ is $\nu$-null. Therefore, for any $A \in \mathcal{M}, \mu^{+}(A)=\mu^{+}(A \cap E)=\nu(A \cap E)= \nu(A \cap P)=\nu^{+}(A)$, and likewise $\nu^{-}=\mu^{-}$.

    The measures $\nu^{+}$and $\nu^{-}$ are called the \textbf{positive and negative variations} of $\nu$, and $\nu=\nu^{+}-\nu^{-}$is called the \textbf{Jordan decomposition} of $\nu$. Furthermore, we define the \textbf{total variation} of $\nu$ to be the measure $|\nu|$ defined by
    \begin{equation*}
        |\nu|=\nu^{+}+\nu^{-} .
    \end{equation*}
\end{definition}


\begin{proposition}
    Let $\nu$ be a signed measure on $(X, \mathcal{M})$

    It is easily verified that $E \in \mathcal{M}$ is $\nu$-null iff $|\nu|(E)=0$, and $\nu \perp \mu$ iff $|\nu| \perp \mu$ iff $\nu^{+} \perp \mu$ and $\nu^{-} \perp \mu$ (Exercise 2.)

    We observe that if $\nu$ omits the value $\infty$ then $\nu^{+}(X)=\nu(P)<\infty$, so that $\nu^{+}$is a finite measure and $\nu$ is bounded above by $\nu^{+}(X)$; similarly if $\nu$ omits the value $-\infty$. In particular, if the range of $\nu$ is contained in $\mathbb{R}$, then $\nu$ is bounded.



    Integration with respect to a signed measure $\nu$ is defined in the obvious way: We set
    \begin{equation*}
        \begin{gathered}
            L^1(\nu)=L^1\left(\nu^{+}\right) \cap L^1\left(\nu^{-}\right) \\
            \int f d \nu=\int f d \nu^{+}-\int f d \nu^{-} \quad\left(f \in L^1(\nu)\right)
        \end{gathered}
    \end{equation*}

    One more piece of terminology: a signed measure $\nu$ is called finite (resp. $\sigma$-finite) if $|\nu|$ is finite (resp. $\sigma$-finite).
\end{proposition}























\chapter{Complex Measure}
\section{Total Variation}
\begin{definition}
    Let $\mathcal{M}$ a $\sigma$-algebra in a set $X$. Call a countable collection $\left\{E_{i}\right\}$ of members of $\mathcal{M}$ a partition of $E$ if $E_{i} \cap E_{j}=\varnothing$ whenever $i \neq j$, and if $E=$ $\cup E_{i}$.

    A complex measure $\mu$ on $\mathcal{M}$ is then a complex function on $\mathcal{M}$ such that
    $$
        \mu(E)=\sum_{i=1}^{\infty} \mu\left(E_{i}\right) \quad(E \in \mathcal{M})
    $$
    for every partition $\left\{E_{i}\right\}$ of $E$.

    \textbf{Remark} Since the union of the sets $E_{i}$ is not changed if the subscripts are permuted, every rearrangement of the series must also converge. Hence the series actually converges absolutely.
\end{definition}
\begin{definition}
    The function $|\mu|$ on $\mathcal{M}$ called the total variation of complex measure $\mu$, defined by
    $$
        |\mu|(E)=\sup \sum_{i=1}^{\infty}\left|\mu\left(E_{i}\right)\right| \quad(E \in \mathcal{M})
    $$
    is a positive measure on $\mathcal{M}$.

    Proof:
    Let $\left\{E_{i}\right\}$ be a partition of $E \in \mathcal{M}$.
    Let any $\varepsilon>0$,
    then each $E_{i}$ has a partition $\left\{A_{i j}\right\}_{j=1}^\infty$ such that
    $$
        |\mu|(E_i)-\frac{\varepsilon }{2^i} < \sum_{j}\left|\mu\left(A_{i j}\right)\right|
    $$
    for $i=1,2,3, \ldots$. Since $\left\{A_{i j}\right\}(i, j=1,2,3, \ldots)$ is a partition of $E$, it follows that
    $$
        |\mu|(E) \geq
        \sum_{i, j}\left|\mu\left(A_{i j}\right)\right| \geq
        \sum_i |\mu|(E_i) -\varepsilon
    $$
    We see that
    $$
        \sum_{i}|\mu|\left(E_{i}\right) \leq|\mu|(E)
    $$

    To prove the opposite inequality, let $\left\{A_{j}\right\}$ be any partition of $E$. Then for any fixed $j,\left\{A_{j} \cap E_{i}\right\}$ is a partition of $A_{j}$, and for any fixed $i,\left\{A_{j} \cap E_{j}\right\}$ is a partition of $E_{i}$. Hence
    $$
        \begin{aligned}
            \sum_{j}\left|\mu\left(A_{j}\right)\right| & =\sum_{j}\left|\sum_{i} \mu\left(A_{j} \cap E_{i}\right)\right|                                      \\
                                                       & \leq \sum_{j} \sum_{i}\left|\mu\left(A_{j} \cap E_{j}\right)\right|                                  \\
                                                       & =\sum_{i} \sum_{j}\left|\mu\left(A_{j} \cap E_{i}\right)\right| \leq \sum_{i}|\mu|\left(E_{i}\right)
        \end{aligned}
    $$
    holds for every partition $\left\{A_{j}\right\}$ of $E$, we have
    \begin{equation}
        |\mu|(E) \leq \sum_{i}|\mu|\left(E_{i}\right) . \tag{5}
    \end{equation}
\end{definition}
\begin{lemma}
    If $z_{1}, \ldots, z_{N}$ are complex numbers then there is a subset $S$ of $\left\{1, \ldots, N\right\}$ for which
    $$
        \left|\sum_{k \in S} z_{k}\right| \geq \frac{1}{\pi} \sum_{k=1}^{N}\left|z_{k}\right|
    $$

    Proof:
    Write $z_{k}=\left|z_{k}\right| e^{i a_k}$. For $-\pi \leq \theta \leq \pi$, let $S(\theta)$ be the set of all $k$ for which $\cos \left(\alpha_{k}-\theta\right)>0$. Then
    $$
        \left|\sum_{S(\theta)} z_{k}\right|=\left|\sum_{S(\theta)} e^{-i \theta} z_{k}\right| \geq \operatorname{Re} \sum_{S(\theta)} e^{-i \theta} z_{k}=\sum_{k=1}^{N}\left|z_{k}\right| \cos ^{+}\left(\alpha_{k}-\theta\right) .
    $$
    Choose $\theta_{0}$ so as to maximize the last sum, and put $S=S\left(\theta_{0}\right)$. This maximum is at least as large as the average of the sum over $[-\pi, \pi]$, and this average is $\pi^{-1} \sum\left|z_{k}\right|$, because
    $$
        \frac{1}{2 \pi} \int_{-\pi}^{\pi} \cos ^{+}(\alpha-\theta) d \theta=\frac{1}{\pi}
    $$
    for every $\alpha$.
\end{lemma}
\begin{theorem}
    If $\mu$ is a complex measure on $X$, then
    $$
        \left|\mu\right|(X)<\infty
    $$
\end{theorem}
\begin{definition}
    Let us now consider a real measure (signed measures) $\mu$ on a $\sigma$-algebra $\mathcal{M}$. Define $|\mu|$ as before, and define
    \begin{equation*}
        \mu^{+}=\frac{1}{2}(|\mu|+\mu), \quad \mu^{-}=\frac{1}{2}(|\mu|-\mu)
    \end{equation*}

    Then both $\mu^{+}$and $\mu^{-}$are positive measures on $\mathcal{M}$, and they are bounded, by Theorem 6.4. Also,
    $$
        \mu=\mu^{+}-\mu^{-}, \quad|\mu|=\mu^{+}+\mu^{-}
    $$

    The measures $\mu^{+}$and $\mu^{-}$are called the positive and \textbf{negative variations} of $\mu$, respectively. This representation of $\mu$ as the difference of the positive measures $\mu^{+}$ and $\mu^{-}$is known as the Jordan decomposition of $\mu$.
\end{definition}

\subsection{Absolutely Continuous}
\begin{definition}
    Let $\mu$ be a positive measure on a $\sigma$-algebra $\mathcal{M}$, and let $\lambda$ be an arbitrary measure on $\mathcal{M}$; $\lambda$ may be positive or complex.

    (1) We say that \textbf{$\lambda$ is absolutely continuous with respect to $\mu$}, and write
    $$
        \lambda \ll \mu
    $$
    if $\lambda(E)=0$ for every $E \in \mathcal{M}$ for which $\mu(E)=0$.

    (2) If there is a set $A \in \mathcal{M}$ such that $\lambda(E)=\lambda(A \cap E)$ for every $E \in \mathcal{M}$, we say that \textbf{$\lambda$ is concentrated on $A$}.

    (3) Suppose $\lambda_1$ and $\lambda_2$ are measures on $\mathcal{M}$, and suppose there exists a pair of disjoint sets $A$ and $B$ such that $\lambda_1$ is concentrated on $A$ and $\lambda_2$ is concentrated on $B$. Then we say that \textbf{$\lambda_1$ and $\lambda_2$ are mutually singular}, and write
    $$
        \lambda_1 \perp \lambda_2
    $$
\end{definition}
\begin{theorem}
    Suppose, $\mu, \lambda, \lambda_1$ and $\lambda_2$ are measures on a $\sigma$-algebra $\mathcal{M}$, and $\mu$ is positive.

    (1) If $\lambda$ is concentrated on $A$, so is $|\lambda|$.

    (2) If $\lambda_1 \perp \lambda_2$, then $\left|\lambda_1\right| \perp\left|\lambda_2\right|$.

    (3) If $\lambda_1 \perp \mu$ and $\lambda_2 \perp \mu$, then $\lambda_1+\lambda_2 \perp \mu$.

    (4) If $\lambda_1 \ll \mu$ and $\lambda_2 \ll \mu$, then $\lambda_1+\lambda_2<\mu$.

    (5) If $\lambda \ll \mu$, then $|\lambda| \ll \mu$.

    (6) If $\lambda_1 \ll \mu$ and $\lambda_2 \perp \mu$, then $\lambda_1 \perp \lambda_2$.

    (7) If $\lambda \ll \mu$ and $\lambda \perp \mu$, then $\lambda=0$.
\end{theorem}

\subsubsection{}
\begin{lemma}[Borel-Cantelli Lemma]
    Suppose $E_k$ is a countable family of measurable sets on $(X, \mathcal{M}, \mu)$. Let
    \begin{equation*}
        \begin{aligned}
            E & =\left\{x\in X: x\in E_k, \text{ for infinitely many } k\right\} \\
              & =\bigcap_{n=1}^{\infty} \bigcup_{i=n}^{\infty} E_i
        \end{aligned}
    \end{equation*}
    and
    \begin{equation*}
        \mu(E)= \lim_{n\to \infty} \mu(\bigcup_{i=n}^{\infty} E_i)
    \end{equation*}
\end{lemma}
\begin{theorem}
    Suppose $\mu$ and $\lambda$ are measures on a $\sigma$-algebra $\mathcal{M}$, $\mu$ is positive, and $\lambda$ is complex. Then the following two conditions are equivalent:
    \begin{enumerate}[]
        \item $\lambda \ll \mu$.
        \item To every $\varepsilon>0$ corresponds a $\delta>0$ such that $ |\lambda(E)| <\varepsilon$ for all $E \in \mathcal{M}$ with $\mu(E)<\delta$.
    \end{enumerate}

    Proof:
    Suppose (ii) holds. If $\mu(E)=0$, then $\mu(E)<\delta$ for every $\delta>0$, hence $|\lambda(E)|<\varepsilon$ for every $\varepsilon>0$, so $\lambda(E)=0$. Thus (ii) implies (i).

    Suppose (ii) is false. Then there exists an $\varepsilon>0$ and there exist sets $E_n \in$ $\mathcal{M}(n=1,2,3, \ldots)$ such that $\mu\left(E_n\right)<2^{-n}$ but $\left|\lambda\left(E_n\right)\right| \geq \varepsilon$. Hence $|\lambda|\left(E_n\right) \geq \varepsilon$. Put
    $$
        A=\bigcap_{n=1}^{\infty} \bigcup_{i=n}^{\infty} E_i
    $$
    Then $\mu(A)=0$ and
    $$
        |\lambda|(A) = \lim _{n \rightarrow \infty} |\lambda|\left(\bigcup_{i=n}^{\infty} E_i\right) \geq \varepsilon
        >0,
    $$
    It follows that we do not have $|\lambda| \ll \mu$, hence (ii) is false.
\end{theorem}


\section{Lebesgue-Radon-Nikodym}
\begin{theorem}
    If $\mu$ is a positive $\sigma$-finite measure on a $\sigma$-algebra $\mathcal{M}$ in a set $X$, then there is a function $w \in L^1(\mu)$ such that $0<w(x)<1$ for every $x \in X$.

    Proof:
    To say that $\mu$ is $\sigma$-finite means that $X$ is the union of countably many sets $E_n \in \mathcal{M}(n=1,2,3, \ldots)$ for which $\mu\left(E_n\right)$ is finite. Put
    $$
        w_n(x)=\frac{1}{ 2^{n} [1+\mu\left(E_n\right)]}   \chi_{E_n}(x)
    $$
    Then $w=\sum\limits_1^{\infty} w_n$ has the required properties.

    \textbf{Remark} The point of the theorem is that $\mu$ can be replaced by a finite measure $\tilde{\mu}$ (namely, $d \tilde{\mu}=w d \mu$ ) which, because of the strict positivity of $w$, has precisely the same sets of measure 0 as $\mu$.
\end{theorem}
\begin{theorem}[Lebesgue-Radon-Nikodym]
    Let $\mu$ be a positive $\sigma$-finite measure on a $\sigma$-algebra $\mathcal{M}$ in a set $X$, and let $\lambda$ be a complex measure on $\mathcal{M}$.

    (1) There is then a unique pair of complex measures $\lambda_{a}$ and $\lambda_s$ on $\mathcal{M}$ such that
    \begin{equation*}
        \lambda=\lambda_a+\lambda_s, \quad \lambda_a \ll \mu, \quad \lambda_s \perp \mu
    \end{equation*}
    If $\lambda$ is positive and finite, then so are $\lambda_a$ and $\lambda_b$.

    (2) There is a unique $h \in L^1(\mu)$ such that
    $$
        \lambda_a(E)=\int_E h \d  \mu
    $$
    for every set $E \in \mathcal{M}$. We may express this in the form $d \lambda_{\mathrm{a}}=h d \mu$, or even in the form $h=d \lambda_{\mathrm{a}} / d \mu$.

    The pair $\left(\lambda_a, \lambda_b\right)$ is called the \textbf{Lebesgue decomposition of $\lambda$ relative to $\mu$}.
    The function $h$ which occurs in (2) is called the \textbf{Radon-Nikodym derivative of $\lambda_a$ with respect to $\mu$}.

    Proof:
    The uniqueness of (a) and (b) is easily seen.

    Assume first that $\lambda$ is a positive bounded measure on $M$. Associate $w$ to $\mu$ as in previous theorem. Then
    $$
        d \varphi=d \lambda+w \d  \mu
    $$
    defines a positive bounded measure $\varphi$ on $\mathcal{M}$.

    The definition of the sum of two measures shows that
    $$
        \int_X f d \varphi=\int_X f d \lambda+\int_X f w \d  \mu
    $$
    for $f=\chi_E$, hence for simple $f$, hence for any nonnegative measurable $f$. If $f \in L^2(\varphi)$, the Schwarz inequality gives
    $$
        \left|\int_X f \d  \lambda\right| \leq \int_X|f| \d  \lambda \leq \int_X|f| \d  \varphi \leq\left\{\int_X |f|^2 \d  \varphi\right\}^{1 / 2}\left\{\varphi(X)\right\}^{1 / 2}
    $$
    Since $\varphi(X)<\infty$, we see that
    $$
        f \rightarrow \int_X f \d  \lambda
    $$
    is a bounded linear functional on $L^2(\varphi)$. Hence there exists a $g \in L^2(\varphi)$ such that
    \begin{equation}
        \int_X f \d  \lambda=\int_X f g \d  \varphi \tag{1}
    \end{equation}
    for every $f \in L^2(\varphi)$.
    Observe also that although $g$ is defined uniquely as an element of $L^2(\varphi), g$ is determined only a.e. $[\varphi]$ as a point function on $X$.

    Put $f=\chi_{E}$ in (1), for any $E \in \mathcal{M}$ with $\varphi(E)>0$, and since $0 \leq \lambda \leq \varphi$, we have

    $$
        \varphi (E) \geq \lambda(E) =  \int_E g \d  \varphi
    $$
    $$
        0 \leq \frac{1}{\varphi(E)} \int_E g \d  \varphi=\frac{\lambda(E)}{\varphi(E)} \leq 1
    $$
    Hence $g(x) \in[0,1]$ for almost all $x$ with respect to $\varphi$. We may therefore assume that $0 \leq g(x) \leq 1$ for every $x \in X$, without affecting (1), and we rewrite (1) in the form
    \begin{equation}
        \int_X (1-g) f \d  \lambda=\int_X f g w \d  \mu \tag{2}
    \end{equation}

    Put
    $$
        A=\left\{x: 0 \leq g(x)<1\right\}, \quad B=\left\{x: g(x)=1\right\}
    $$
    and define measures $\lambda_a$ and $\lambda_s$ by
    $$
        \lambda_a (E)=\lambda(A \cap E), \quad \lambda_{s}(E)=\lambda(B \cap E),
    $$
    for all $E \in \mathcal{M}$.
    If $f=\chi_B$ in (2), the left side is 0 , the right side is $\int_B w \d  \mu$. Since $w(x)>0$ for all $x$, we conclude that $\mu(B)=0$. Thus $\lambda_s \perp \mu$.

    Since $g$ is bounded, (2) holds if $f$ is replaced by
    $$
        \left(1+g+\cdots+g^n\right) \chi_E
    $$
    for $n=1,2,3, \ldots, E \in \mathcal{M}$. For such $f$, (2) becomes
    \begin{equation}
        \int_E\left(1-g^{n+1}\right) \d  \lambda=\int_E g\left(1+g+\cdots+g^n\right) w \d  \mu \tag{3}
    \end{equation}
    Note that $1- g^{n+1}(x) < \chi_{A}(x)$ and $\lambda(A) < \lambda (X) < \infty$; $\chi_A(x) \in L^1 (\lambda)$. The Dominated Convergence Theorem shows that the left side of (3) converges therefore to $\lambda(A \cap E)=\lambda_a(E)$ as $n \rightarrow \infty$. The monotone convergence theorem shows that the right side of (2) tends to $\int_E h \d  \mu$ as $n \rightarrow \infty$ ,where $h=\frac{g}{1-g}$.
    \begin{equation}
        \lambda_a(E)=\int_E h \d  \mu \tag{4}
    \end{equation}
    Taking $E=X$, we see that $h \in L^1(\mu)$, since $\lambda_a (X)<\infty$. Finally, (4) shows that $\lambda_a \ll \mu$, and the proof is complete for positive $\lambda$.

    If $\lambda$ is a complex measure on $\mathcal{M}$, then $\lambda=\lambda_1+i \lambda_2$, with $\lambda_1$ and $\lambda_2$ real, and we can apply the preceding case to the positive and negative variations of $\lambda_1$ and $\lambda_2$.

    \textbf{Remark} If $\lambda$ (and $\mu$) are positive and $\sigma$-finite, we can now write
    $$
        X=\bigcup_{n=1}^\infty X_n
    $$
    where $X_1 \subset X_2 \subset \cdots$ and $\lambda\left(X_n\right)<\infty$, for $n=1,2,3$, ....
    Considering the Lebesgue decompositions of the measures
    $$\lambda_n(E) = \lambda\left(E \cap X_n\right)$$
    on $\mathcal{M}$ that is positive and finite, we still get a function $h_n\in L^1 (\mu)$ which satisfies
    $$\lambda_{n,a}(E) = \int_{E} h_n \d \mu$$
    Let $n\to \infty$, we have
    $$\lambda_a (E) =  \int_{E} h \d \mu$$
    by the monotone convergence theorem, where $h=\lim h_n$

    We observe that $\lambda_n(E)$
    however, it is no longer true that $h \in L^1(\mu)$, although $h$ is "locally in L," i.e., $\int_{x_n} h d \mu<\infty$ for each $n$.

    If we go beyond $\sigma$-finiteness of $\lambda$, we meet situations where the two theorems under consideration actually fail. For example, let $\mu$ be Lebesgue measure on $(0,1)$, and let $\lambda$ be the counting measure on the $\sigma$-algebra of all Lebesgue
\end{theorem}


\section{Consequence Radon-Nikodym Theorem}

\begin{theorem}[Polar Representation of Complex Measure $\mu$]
    Let $\mu$ be a complex measure on a $\sigma$-algebra $\mathcal{M}$ in $X$. Then there is a measurable function $h\in L^1 (\left|\mu\right|)$ such that $\left|h(x)\right|=1$ for all $x \in X$ and such that
    \begin{equation*}
        d \mu
        =
        h \d \left|\mu\right|
    \end{equation*}

    Proof:
    It is trivial that $\mu \ll|\mu|$ and $\left|\mu\right|$ is finite, therefore the Radon-Nikodym theorem guarantees the existence of some $h \in L^1(|\mu|)$ which satisfies $d \mu=h \d |\mu|$ .

    Let $A_r=\left\{x:|h(x)|<r\right\}$, where $r$ is some positive number, and let $\left\{E_j\right\}$ be a partition of $A_r$.
    Then
    $$
        \sum_j \left|\mu\left(E_j\right)\right|
        =\sum_j \left|\int_{E_j} h \d | \mu| \right|
        \leq \sum_j r|\mu|\left(E_j\right)=r|\mu|\left(A_r\right)
    $$
    for all partition of $A_r$, so that $|\mu|\left(A_r\right) \leq r|\mu|\left(A_r\right)$. If $r<1$, this forces $|\mu|\left(A_r\right)=0$. Thus $|h| \geq 1$ a.e. $\left|\mu\right|$

    On the other hand, if $|\mu|(E)>0$
    $$
        \left| \frac{1}{\left|\mu\right| (E)}  \int_E h \d  \left| \mu\right| \right| =\frac{|\mu(E)|}{|\mu|(E)} \leq 1
    $$

    We now apply Theorem 1.40 (with the closed unit disc in place of $S$ ) and conclude that $|h| \leq 1$ a.e. $\left|\mu\right|$

    Let $B=\left\{x \in X:|h(x)| \neq 1\right\}$. We have shown that $|\mu|(B)=0$, and if we redefine $h$ on $B$ so that $h(x)=1$ on $B$, we obtain a function with the desired properties.
\end{theorem}
\begin{theorem}[The Hahn Decomposition Theorem]
    Let $\mu$ be a real measure on a $\sigma$ algebra $\mathcal{M}$ in a set $X$. Then there exist sets $A$ and $B \in \mathcal{M}$ such that
    $A \cup B=X, A \cap B=\varnothing$, and such that the positive and negative variations $\mu^{+}$and $\mu^{-}$of $\mu$ satisfy
    $$
        \mu^{+}(E)=\mu(A \cap E), \quad \mu^{-}(E)=-\mu(B \cap E) \quad(E \in \mathcal{M})
    $$
    The pair $(A, B)$ is called a \textbf{Hahn decomposition} of $X$, induced by $\mu$.

    Proof:
    By previous theorem, $d \mu=h \d |\mu|$, where $|h|=1$. Since $\mu$ is real, it follows that $h$ is real (a.e. $\left|\mu\right|$), hence $h= \pm 1$. Put
    $$
        A=\left\{x: h(x)=1\right\}, \quad B=\left\{x: h(x)=-1\right\} .
    $$
    Since $\mu^{+}=\frac{1}{2}(|\mu|+\mu)$,
    we have, for any $E \in \mathcal{M}$,
    $$
        \mu^{+}(E)=\frac{1}{2} \int_E(1+h) \d |\mu|=\int_{E \cap A} h \d |\mu|=\mu(E \cap A)
    $$
    Since $\mu(E)=\mu(E \cap A)+\mu(E \cap B)$ and since $\mu=\mu^{+}-\mu^{-}$, the second half follows from the first.
\end{theorem}
\begin{corollary}
    Let $\mu$ be a real measure on $\mathcal{M}$. If $\mu=\lambda_1-\lambda_2$, where $\lambda_1$ and $\lambda_2$ are positive measures, then $\lambda_1 \geq \mu^{+}$ and $\lambda_2 \geq \mu^{-}$.

    Proof:
    Since $\mu \leq \lambda_1$, we have
    $$
        \mu^{+}(E)=\mu(E \cap A) \leq \lambda_1(E \cap A) \leq \lambda_1(E) .
    $$

    \noindent\textbf{Remark} This is the minimum property of the Jordan decomposition.
\end{corollary}
\begin{corollary}
    Suppose $\mu$ is a positive measure on $\mathcal{M}, g \in L^1(\mu)$, and
    $$
        \lambda(E)=\int_{E} g \d  \mu \quad(E \in \mathcal{M})
    $$
    Then
    $$
        |\lambda|(E)=\int_{E} |g| \:  d \mu \quad(E \in \mathcal{M})
    $$

    Proof:
    By previous Theorem , there is a function $h$, of absolute value $1$ , such that $d \lambda=h \d |\lambda|$. By hypothesis, $d \lambda=g \d  \mu$. Hence
    $$
        h \d |\lambda|=g \d  \mu
    $$
    This gives $d|\lambda|=\overline{h} g \d  \mu$.
    Since $|\lambda| \geq 0$ and $\mu \geq 0$, it follows that $\overline{h} g \geq 0$ a.e. $[\mu]$, so that $\overline{h} g=|g|$ a.e. $[\mu]$.
\end{corollary}

\section{Riesz Theorem}
\begin{theorem}
    Let $X$ be a locally compact Hausdorff space, then every bounded linear functional $\Phi$ on $C_0(X)$ is represented by a unique regular complex Borel measure $\mu_\Phi$, in the sense that
    \begin{equation*}
        \Phi f
        =
        \int_X f \d  \mu
    \end{equation*}
    for every $f \in C_0(X)$. Moreover, $
        \|\Phi\|    = \|\mu\| = |\mu|(X)$.


    Proof:
    We first settle the uniqueness question. Suppose $\mu$ is a regular complex Borel measure on $X$ and $\int f \d  \mu=0$ for all $f \in C_0(X)$. By Theorem 6.12 there is a Borel function $h$, with $|h|=1$, such that $d \mu=h \d |\mu|$. For any sequence $\left\{f_n\right\}$ in $C_0(X)$ we then have
    $$
        |\mu|(X)=\int_X\left(\bar{h}-f_k\right) h \d |\mu| \leq \int_X \left|\bar{h}-f_n\right| d|\mu|
    $$
    and since $C_c(X)$ is dense in $L^1(|\mu|)$, thus $|\mu|(X)=0$, and $\mu=0$.

    Now consider a given bounded linear functional $\Phi$ on $C_0(X)$. Assume $\|\Phi\|=1$, without loss of generality. We shall construct a positive linear functional $\Lambda$ on $C_c(X)$, such that
    $$
        |\Phi(f)| \leq \Lambda(|f|) \leq\left\| f \right\| \quad\left(f \in C_c(X)\right)
    $$
    where $\left\| f \right\|$ denotes the supremum norm.
    Once we have this $\Lambda$, we associate with it a positive Borel measure $\lambda$, as in Theorem 2.14. The conclusion of Theorem 2.14 shows that $\lambda$ is regular if $\lambda(X)<\infty$. Since
    $$
        \lambda(X)=\sup \left\{\Lambda f: 0 \leq f \leq 1, f \in C_c(X)\right\}
    $$
    and since $|\Lambda f| \leq 1$ if $\left\| f \right\| \leq 1$, we see that actually $\lambda(X) \leq 1$.
    We also deduce from (4) that
    $$
        |\Phi(f)| \leq \Lambda(|f|)=\int_x|f| d \lambda=\left\| f \right\|_1 \quad\left(f \in C_c(X)\right)
    $$

    The last norm refers to the space $L^1(\lambda)$. Thus $\Phi$ is a linear functional on $C_c(X)$ of norm at most 1 , with respect to the $L^1(\lambda)$-norm on $C_c(X)$. There is a normpreserving extension of $\Phi$ to a linear functional on $L^1(\lambda)$, and therefore Theorem 6.16 (the case $p=1$ ) gives a Borel function $g$, with $|g| \leq 1$, such that
    $$
        \Phi(f)=\int_X f g d \lambda \quad\left(f \in C_c(X)\right)
    $$

    Each side of (6) is a continuous functional on $C_0(X)$, and $C_c(X)$ is dense in $C_0(X)$. Hence (6) holds for all $f \in C_0(X)$, and we obtain the representation (1) with $d \mu=g d \lambda$.

    Since $\|\boldsymbol{\Phi}\|=1,(6)$ shows that
    $$
        \int_X|g| d \lambda \geq \sup \left\{|\Phi(f)|: f \in C_0(X),\left\| f \right\| \leq 1\right\}=1
    $$

    We also know that $\lambda(X) \leq 1$ and $|g| \leq 1$. These facts are compatible only if $\lambda(X)=1$ and $|g|=1$ a.e. [ $\lambda]$. Thus $d|\mu|=|g| d \lambda=d \lambda$, by Theorem 6.13, and
    $$
        |\mu|(X)=\lambda(X)=1=\|\Phi\|
    $$
    which proves (2).
    So all depends on finding a positive linear functional $\Lambda$ that satisfies (4). If $f \in C_c^{+}(X)$ [the class of all nonnegative real members of $C_c(X)$ ], define
    $$
        \Lambda f=\sup \left\{|\Phi(h)|: h \in C_c(X),|h| \leq f\right\}
    $$

    Then $\Lambda f \geq 0, \Lambda$ satisfies (4), $0 \leq f_1 \leq f_2$ implies $\Delta f_1 \leq \Delta f_2$, and $\Lambda(c f)=c \Delta f$ if $c$ is a positive constant. We have to show that
    $$
        \Lambda(f+g)=\Lambda f+\Lambda g \quad\left(f \text { and } g \in C_c^{+}(X)\right)
    $$
    and we then have to extend $\Lambda$ to a linear functional on $C_c(X)$.
    Fix $f$ and $g \in C_c^{+}(X)$. If $\varepsilon>0$, there exist $h_1$ and $h_2 \in C_c(X)$ such that $\left|h_1\right| \leq f,\left|h_2\right| \leq g$, and
    $$
        \Lambda f \leq\left|\Phi\left(h_1\right)\right|+\varepsilon, \quad \Lambda g \leq\left|\Phi\left(h_2\right)\right|+\varepsilon
    $$
    There are complex numbers $\alpha_i,\left|\alpha_i\right|=1$, so that $\alpha_i \Phi\left(h_i\right)=\left|\Phi\left(h_i\right)\right|, i=1,2$. Then
    $$
        \begin{aligned}
            \Lambda f+\Lambda g & \leq\left|\Phi\left(h_1\right)\right|+\left|\Phi\left(h_2\right)\right|+2 \varepsilon \\
                                & =\Phi\left(\alpha_1 h_1+\alpha_2 h_2\right)+2 \varepsilon                             \\
                                & \leq \Lambda\left(\left|h_1\right|+\left|h_2\right|\right)+2 \varepsilon              \\
                                & \leq \Lambda(f+g)+2 \varepsilon
        \end{aligned}
    $$
    so that the inequality $\geq$ holds in (10).
    Next, choose $h \in C_c(X)$, subject only to the condition $|h| \leq f+g$, let $V=\left\{x: f(x)+g(x)>0\right\}$, and define
    $$
        \begin{aligned}
             & h_1(x)=\frac{f(x) h(x)}{f(x)+g(x)}, \quad h_2(x)=\frac{g(x) h(x)}{f(x)+g(x)} \quad(x \in V), \\
             & h_1(x)=h_2(x)=0 \quad(x \notin V) .
        \end{aligned}
    $$

    It is clear that $h_1$ is continuous at every point of $V$. If $x_0 \notin V$, then $h\left(x_0\right)=0$; since $h$ is continuous and since $\left|h_1(x)\right| \leq|h(x)|$ for all $x \in X$, it follows that $x_0$ is a point of continuity of $h_1$. Thus $h_1 \in C_c(X)$, and the same holds for $h_2$.

    Since $h_1+h_2=h$ and $\left|h_1\right| \leq f,\left|h_2\right| \leq g$, we have
    $$
        |\Phi(h)|=\left|\Phi\left(h_1\right)+\Phi\left(h_2\right)\right| \leq\left|\Phi\left(h_1\right)\right|+\left|\Phi\left(h_2\right)\right| \leq \Lambda f+\Lambda g .
    $$

    Hence $\Lambda(f+g) \leq \Lambda f+\Lambda g$, and we have proved (10).
    If $f$ is now a real function, $f \in C_c(X)$, then $2 f^{+}=|f|+f$, so that $f^{+} \in$ $C_c^{+}(X)$; likewise, $f^{-} \in C_c^{+}(X)$; and since $f=f^{+}-f^{-}$, it is natural to define
    $$
        \Lambda f=\Lambda f^{+}-\Lambda f^{-} \quad\left(f \in C_{\mathrm{r}}(X), f \text { real }\right)
    $$
    and
    $$
        \Lambda(u+i v)=\Lambda u+i \Lambda v .
    $$

    Simple algebraic manipulations, just like those which occur in the proof of Theorem 1.32, show now that our extended functional $\Lambda$ is linear on $C_c(X)$.

    This completes the proof.
\end{theorem}

\chapter{Integration On Product Spaces}





\section{Application}
\begin{theorem}[Minkowski]
    Suppose that $(X, \mathscr{L}, \mu)$ and $(Y, \mathscr{T}, \nu)$ are $\sigma$-finite measure spaces. Let $f$ be an $(\mathscr{L} \times \mathscr{T})$ measurable function on $X \times Y$. If $0 \leq f \leq \infty$ and $1 \leq p<\infty$, then we have the analogy of Minkowski's inequality
    \begin{equation*}\left\{\int \left[\int f(x,y) \d \lambda(y)\right]^p \d \mu(x)\right\}^{\frac{1}{p}}
        \leq
        \int \left[\int f(x,y)^p \d \lambda(y)\right]^{\frac{1}{p}} \d \mu(x)
    \end{equation*}

    Proof: If $p=1$, then it is exactly The Fubini Theorem. Suppose that $1<p<\infty$ and $q$ is the conjugate exponent of $p$. Since the inequality holds trivially when
    $$
        \int\left[\int f^p(x, y) \d \mu(x)\right]^{\frac{1}{p}} \d  \lambda(y)=\infty
    $$
    so without loss of generality, we assume further that this integral is finite. Let $g \in L^q(\mu)$, we define $\Phi$ by
    $$
        \Phi(g)=\int\left[\int f(x, y) \d \lambda(y)\right] \cdot g(x) \d \mu(x)
    $$
    we have
    $$
        \begin{aligned}
            \Phi (g) & =\iint f(x, y) \cdot g(x) \d \mu(x) \d \lambda(y)                                                                          \\
                     & \leq \int \left[\int f^p(x, y) \d \mu(x)\right]^{\frac{1}{p}}\left[\int|g|^q \d \mu(x)\right]^{\frac{1}{q}} \d  \lambda(y) \\
                     & =\left\| g \right\|_q \int\left[\int f^p(x, y) \d \mu(x)\right]^{\frac{1}{p}} \d \lambda(y)
        \end{aligned}
    $$
    then $\Phi$ is a bounded linear functional on $L^q(\mu)$ and there exists a unique $h \in L^p(\mu)$ such that
    $$
        \Phi(g)=\int h g \d \mu(x)
    $$
    the uniqueness of $h$ and the comparison of the representations and show that
    $$
        h(x)=\int f(x, y) \d \lambda(y)
    $$
    Furthermore, we deduce that
    $$
        \begin{aligned}
            \left\{\int\left[\int f(x, y) \d \lambda(y)\right]^p \d \mu(x)\right\}^{\frac{1}{p}} & =\|h\|_p                                                                   \\
                                                                                                 & =\|\Phi\|                                                                  \\
                                                                                                 & \leq \int\left[\int f^p(x, y) \d \mu(x)\right]^{\frac{1}{p}} \d \lambda(y)
        \end{aligned}
    $$
    which completes the proof of the problem.
\end{theorem}
\begin{theorem}[Hardy's Inequality]
    Suppose $f\geq 0$ on $(0, \infty)$, $f\in L^p$, $1\leq p \leq \infty$, and
    \begin{equation*}F(x) = \frac{1}{x} \int_0^x f(t) \d t\end{equation*}


    Proof. Since $f(t) t^\alpha$ and $t^{-\alpha}$ are nonnegative, we have estimate by Holder's
    $$
        \begin{aligned}
            x F(x) & =\int_0^x f(t) t^\alpha t^{-\alpha} \d  t                                                                                       \\
                   & \leq\left(\int_0^x f^p t^{\alpha p} \d t\right)^{\frac{1}{p}}\left(\int_0^x t^{\alpha q} \d t\right)^{\frac{1}{q}}              \\
                   & =\left(\int_0^x f^p t^{\alpha p} \d t\right)^{\frac{1}{p}} \times\left(\frac{x^{-\alpha q+1}}{-\alpha q+1}\right)^{\frac{1}{q}}
        \end{aligned}
    $$
    which gives
    $$
        F^p(x) \leq(1-\alpha q)^{1-p} x^{-1-\alpha p} \int_0^x f^p t^{\alpha p} \d t
    $$
    and then
    $$
        \begin{aligned}
            \int_0^{\infty} F^p(x) \d  x & \leq(1-\alpha q)^{1-p} \int_0^{\infty} x^{-1-\alpha p} \int_0^x f^p t^{\alpha p} \d  t \d x                        \\
                                         & = (1-\alpha q)^{1-p} \int_0^{\infty} f^p t^{\alpha p} \int_0^{\infty} \chi_{(0, x)}(t) x^{-1-\alpha p} \d  x \d  t \\
                                         & =(1-\alpha q)^{1-p} \int_0^{\infty} f^p t^{\alpha p} \times \frac{t^{-\alpha p}}{\alpha p} \d  t                   \\
                                         & =(1-\alpha q)^{1-p}(\alpha p)^{-1} \int_0^{\infty} f^p \d  t                                                       \\
                                         & = \left(\frac{p}{p-1}\right)^p \int_0^{\infty} f^p \d  t \quad \quad (\text{if we choose } \alpha=\frac{1}{pq})
        \end{aligned}
    $$
    as the space $(0, \infty)$ is $\sigma$-finite, $\chi_{(0, x)} x^{-1-\alpha p} f^p t^{\alpha p} \geq 0$ and measurable on $(0, \infty) \times(0, \infty)$.
\end{theorem}

\begin{lemma}
    Suppose that $f$ and $K$ are measurable functions on $\mathbb{R}$, then $F(x,y)=K(x-y)f(y)$ is measurable on $\mathbb{R}^2$

    Proof:
    There exist Borel functions $f_0$ and $K_0$ such that $f_0=f$ a.e. and $K_0=K$ a.e. We claim that $F_0(x,y)=K_0(x-y)f_0(y)=F(x,y)$ a.e. $(x,y)\in \mathbb{R}^2$.

    As we can see, the set $\left\{(x,y) : F_0(x,y)\neq F_(x,y)\right\}$ is contained in
    \begin{equation*}\left\{(x,y) :f_0(y)\neq f(y)\right\} \cup
        \left\{(x,y) :K_0(x-y)\neq K(x-y)\right\} \end{equation*}
    where $\left\{(x,y) :f_0\neq f\right\}=\mathbb{R} \times \left\{y :f_0\neq f\right\}$
    is $0$ measure and
    \begin{equation*}
        \begin{aligned}
            m\left(\left\{(x,y) :K_0(x-y)\neq K(x-y)\right\}\right) &
            =\int_{\mathbb{R}^2} \chi_E (x-y) \d x\d y                                                                                     \\
                                                                    & =\int_\mathbb{R} \left[\int_\mathbb{R} \chi_{y+E}(x)\d x\right] \d y \\
                                                                    & =0
        \end{aligned}
    \end{equation*}
    Consequently, we conclude that $m\left(\left\{(x,y) : F_0(x,y)\neq F(x,y)\right\}\right)=0$ which implies that $F_0(x,y)=F(x,y)$ a.e. $(x,y)\in \mathbb{R}^2$.

    Next, we define $\varphi: \mathbb{R}^2 \rightarrow \mathbb{R}^1$ and $\psi: \mathbb{R}^2 \rightarrow \mathbb{R}^1$ by
    \begin{equation*}
        \varphi(x, y)=x-y, \quad \psi(x, y)=y .
    \end{equation*}
    Then $f_0(x-y)=(f_0 \circ \varphi)(x, y)$ and $g_0(y)=(g_0 \circ \psi)(x, y)$ are Borel functions on $\mathbb{R}^2$ since $\varphi$ and $\psi$ are Borel(continuous) functions. Hence so is their product $F_0$ is a Borel functions on $\mathbb{R}^2$. Finally we conclude that $F$ is measurable on $\mathbb{R}^2$
\end{lemma}
\begin{theorem}[Young's convolution inequality]
    Suppose $1\leq p \leq \infty $, $K\in L^1$, and $f \in L^p$. Then the integral defining
    \begin{equation*}(K*f)(x)= \int_\mathbb{R} K(x-y)f(y) \d y\end{equation*}
    exists for almost every $x\in \mathbb{R}$, $f*K\in L^1$, and
    \begin{equation*}
        \|K*f\|_p \leq \|K\|_1 \left\| f \right\|_p
    \end{equation*}

    Proof:
    Suppoes that $p=1$ and write $F(x,y) = K(x-y) f(y)$. We observe that
    \begin{equation*}
        \begin{aligned}
            \int_{-\infty}^{\infty} d y \int_{-\infty}^{\infty}|F(x, y)| d x=\int_{-\infty}^{\infty}|g(y)| d y \int_{-\infty}^{\infty}|f(x-y)| d x=\left\| f \right\|_1\left\| g \right\|_1
        \end{aligned}
    \end{equation*}
    Thus $F \in L^1\left(\mathbb{R}^2\right)$, and Fubini's theorem implies that the integral
    \begin{equation*}\left(K*f\right)(x)=\int_\mathbb{R} F(x,y) \d y\end{equation*}
    exists for almost all $x \in \mathbb{R}^1$ and that $K*f \in L^1\left(\mathbb{R}^1\right)$. Finally,
    \begin{equation*}
        \begin{aligned}
            \|K*f\|_1 & =\int_{-\infty}^{\infty} \left|\int_\mathbb{R} F(x,y) \d y \right|\d x    \\
                      & \leq \int_{-\infty}^{\infty} \d  x \int_{-\infty}^{\infty}|F(x, y)| \d  y \\
                      & =\left\| f \right\|_1\left\| g \right\|_1
        \end{aligned}
    \end{equation*}

    Now suppose that $1<p<\infty$. By Hölder's Inequality, we derive that
    \begin{equation*}
        \begin{aligned}
            \int_\mathbb{R} |K(x-y) f(y)| \d y & =\int_{-\infty}^{\infty} |K(x-y)|^{\frac{1}{q}} \cdot |K(x-y)|^{\frac{1}{p}}| f(y)| \d y                \\
                                               & \leq\left(\int_\mathbb{R}|K(x-y)| \d y\right)^{\frac{1}{q}}
            \left(\int_{-\infty}^{\infty} |K(x-y)| |f(y)|^p  \d y\right)^{\frac{1}{p}}                                                                   \\
                                               & =\|K\|_1^{\frac{1}{q}} \cdot \left(\int_{-\infty}^{\infty} |K(x-y)| |f(y)|^p  \d y\right)^{\frac{1}{p}}
        \end{aligned}
    \end{equation*}
    which gives
    \begin{equation*}
        \left(\int_{-\infty}^{\infty}|K(x-y) f(y)| \d y\right)^p \leq\|K\|_1^{\frac{p}{q}} \int_{-\infty}^{\infty}K|(x-y)| \cdot |f(y)|^p \d y
    \end{equation*}
    According to the case $p=1$, we have $\int_{-\infty}^{\infty}|f(x-y)| \cdot|g(y)|^p \d y$ exist a.e. $x \in \mathbb{R}$ since $f^p$ and $K$ are belong to $L^1$. Then $K*f(x)$ exist a.e. by this estimate.
    Also,
    \begin{equation*}\begin{aligned}
            \|K*f\|_p^p & =  \int_{\mathbb{R}} \left|\int_\mathbb{R} K(x-y) f(y) \d y\right|^p \d x      \\
                        & \leq \int_{\mathbb{R}}\left( \int_\mathbb{R} |K(x-y) f(y)| \d y\right)^p \d x  \\
                        & \leq
            \|K\|_1^{\frac{p}{q}} \int  \left(\int_{-\infty}^{\infty} |K(x-y)| |f(y)|^p  \d y\right)\d x \\
                        & =\|K\|_1^p \left\| f \right\|_p^p
        \end{aligned}
    \end{equation*}
    which implies that $\|K* f\|_p \leq\|K\|_1 \cdot\left\| f \right\|_p<\infty$.

    \textbf{Remark}
\end{theorem}


\chapter{Fourier Transforms}


\section{Formal Properties}

\subsection{Definition}
\begin{definition}
    If $f, g\in L^1(\mathbb{R}^n)$, we define Fourier transform
    \begin{equation*}
        \hat{f} (\xi )
        = \mathcal{F}f (\xi)
        = \frac{1}{(2\pi)^{\frac{n}{2}}} \int_{\mathbb{R}^n} f(x) e^{-i x \cdot \xi} \d  x \quad \left(\xi \in \mathbb{R}^n \right)
    \end{equation*}
    and
    $$
        \check{g} (\xi)
        =\mathcal{F}^{-1} g (\xi)
        = \frac{1}{(2\pi)^{\frac{n}{2}}}
        \int_{\mathbb{R}^n}
        g(x) e^{i x \cdot \xi} \d  x \quad \left(\xi \in \mathbb{R}^n \right)$$
\end{definition}
\begin{definition}
    If $f\in L^2(\mathbb{R}^n)$, we choose a sequence $\left\{f_k\right\}_{k=1}^{\infty} \subset L^1\left(\mathbb{R}^n\right) \cap L^2\left(\mathbb{R}^n\right)$ with
    $$
        f_k \xrightarrow{L^2(\mathbb{R}^n)} f
    $$
    Then
    $$\left\|\hat{f}_k-\hat{f}_j\right\|_{L^2\left(\mathbb{R}^n\right)}
        =\left\|\widehat{f_k-f_j}\right\|_{L^2\left(\mathbb{R}^n\right)}
        =\left\|f_k-f_j\right\|_{L^2\left(\mathbb{R}^n\right)}$$
    and thus $\left\{\hat{f}_k\right\}_{k=1}^{\infty}$ is a Cauchy sequence in $L^2\left(\mathbb{R}^n\right)$. This sequence consequently converges to a limit $u$ in $L^2\left(\mathbb{R}^n\right)$, which we define to be $\mathcal{F} f=u$.

    The definition of $u$ does not depend upon the choice of approximating sequence $\left\{f_k\right\}$. We can choose $f_k(x)= \chi_{[-k,k]^n} (x) \cdot f(x)$
    then
    $$\mathcal{F}f (\xi)
        =\lim_{k \to +\infty} \frac{1}{(2\pi)^{\frac{n}{2}}} \int_{[-k,k]^n} f(x) e^{-i x \cdot \xi} \d x \quad \text{ in } L^2\left(\mathbb{R}^n\right)
    $$

    We similarly define $\mathcal{F}^{-1}$
\end{definition}
\begin{theorem}[Plancherel's Theorem in $L^1\left(\mathbb{R}^n\right) \cap L^2\left(\mathbb{R}^n\right)$]
    Assume $u \in L^1\left(\mathbb{R}^n\right) \cap L^2\left(\mathbb{R}^n\right)$. Then $\hat{u}, \check{u} \in L^2\left(\mathbb{R}^n\right)$ and
    \begin{equation*}
        \|\hat{u}\|_{L^2\left(\mathbb{R}^n\right)}=\|\check{u}\|_{L^2\left(\mathbb{R}^n\right)}=\|u\|_{L^2\left(\mathbb{R}^n\right)}
    \end{equation*}

    Proof:
    1. First we note that if $v, w \in L^1\left(\mathbb{R}^n\right)$, then $\hat{v}, \hat{w} \in L^{\infty}\left(\mathbb{R}^n\right)$. Also
    \begin{equation*}
        \int_{\mathbb{R}^n} v(x) \hat{w}(x) d x
        =\frac{1}{(2 \pi)^{n / 2}} \int_{\mathbb{R}^n} \int_{\mathbb{R}^n} e^{-i x \cdot y} v(x) w(y) d x d y
        =\int_{\mathbb{R}^n} \hat{v}(y) w(y) dy
    \end{equation*}
    Consequently if $\varepsilon>0$ and $v_{\varepsilon}(x):=e^{-\varepsilon|x|^2}$, we have $\hat{v}_{\varepsilon}(y)=\frac{e^{-\frac{|y|^2}{\varepsilon}}}{(2 \varepsilon)^{n / 2}}$. Thus implies for each $\varepsilon>0$ that
    \begin{equation*}
        \int_{\mathbb{R}^n} \hat{w}(y) e^{-\varepsilon|y|^2} d y
        =
        \frac{1}{(2 \varepsilon)^{n / 2}} \int_{\mathbb{R}^n} w(x) e^{-\frac{|x|^2}{4 \varepsilon}} d x
    \end{equation*}

    2. Now take $u \in L^1\left(\mathbb{R}^n\right) \cap L^2\left(\mathbb{R}^n\right)$ and set $v(x):=\bar{u}(-x)$. Let
    \begin{equation*}
        w:=u * v \in L^1\left(\mathbb{R}^n\right) \cap C\left(\mathbb{R}^n\right)
    \end{equation*}
    and check that
    \begin{equation*}
        \hat{w}=(2 \pi)^{n / 2} \hat{u} \hat{v}
        =
        (2 \pi)^{n / 2} \left|\hat{u}\right|^2
        \in L^1 \left(\mathbb{R}^n\right)
    \end{equation*}
    Since $w \in C(\mathbb{R}^n)\cap L^1(\mathbb{R}^n)$, we have
    \begin{equation*}
        \lim _{\varepsilon \rightarrow 0} \frac{1}{(2 \varepsilon)^{n / 2}} \int_{\mathbb{R}^n} w(x) e^{-\frac{|x|^2}{4 \varepsilon}} d x=(2 \pi)^{n / 2} w(0)
        =
        (2 \pi)^{n / 2} \|u\|_{L^2}
    \end{equation*}
    Also,
    \begin{equation*}
        \lim_{\varepsilon\to 0^+}
        \int_{\mathbb{R}^n} \hat{w}(y) e^{-\varepsilon|y|^2} d y
        =
        \int_{\mathbb{R}^n} \hat{w}(y) dy
        =
        (2 \pi)^{n / 2} \|\hat{u}\|_{L^2}
    \end{equation*}

    The proof for $\check{u}$ is similar.
\end{theorem}

\subsection{Properties}
\begin{theorem}
    Assume $\mathcal{F}f$ exist, and $h$ and $\lambda$ belong to $\mathbb{R}^n$. Then

    (i)
    $$
        \mathcal{F}(f(x)e^{i h \cdot x}) (\xi) = \hat{f}(\xi- h)
    $$
    (ii)
    $$\mathcal{F}(f(x+h))(\xi) = \hat{f} (\xi) e^{i h \cdot \xi}
    $$
    (iii)
    $$\mathcal{F}(f(-x)) (\xi)
        =\mathcal{F}^{-1}f (\xi)$$
    (vi)
    \begin{equation*}
        \mathcal{F}\bar{f}
        =
        \overline{\mathcal{F}^{-1}f}
    \end{equation*}
    (v)
    $$
        \mathcal{F}\left( \frac{1}{\left|\lambda\right|^n} f\left(\frac{x}{\lambda} \right)\right)(\xi)
        = \mathcal{F}f (\lambda \xi)
    $$
\end{theorem}
\begin{theorem}
    (i) If $f,g \in L^1$, then $f*g \in L^1 $ and
    $$\mathcal{F}(f*g)
        = \frac{1}{(2\pi)^\frac{n}{2}} \mathcal{F}(f) \cdot \mathcal{F}(g)$$


    (ii) If $g(x)=(-i x)^\alpha f(x)$ and $g \in L^{1}$, then is differentiable and
    $$ \mathcal{F} ((-ix)^\alpha f(x))
        =D^\alpha (\mathcal{F}f)$$

    (iii) If $D^\alpha u \in L^2(\mathbb{R}^n)$ for multiindex $\alpha$, then
    \begin{equation*}
        \mathcal{F}D^\alpha u
        =
        (iy)^\alpha \mathcal{F}u
    \end{equation*}
\end{theorem}


\section{p=1}
\begin{theorem}
    Suppose $1 \leq p < \infty$ and any function $f$ on $L^p (\mathbb{R}^n)$, the mapping
    $$
        y \rightarrow f_y (x)=f(x-y)
    $$
    is a uniformly continuous mapping of $\mathbb{R}^n$ into $L^p (\mathbb{R}^n)$.

    Proof:
    Fix $\varepsilon>0$. Since $f \in L^p$, there exists a continuous function $g$ whose support lies in a bounded interval $[-A, A]^n$, such that
    $$
        \|f-g\|_p<\varepsilon
    $$
    The uniform continuity of $g$ shows that there exists a $\delta >0$ such that $|s-t|<\delta$ implies
    $$
        |g(s)-g(t)|<\frac{\varepsilon}{(3 A)^{\frac{1}{p}}}
    $$
    If $|s-t|<\delta$, it follows that
    $$
        \int_{\mathbb{R}^n} |g(x-s)-g(x-t)|^p \d  x
        <
        \frac{\varepsilon^p}{3A} (2 A+\delta)^n
        <
        \varepsilon^p
    $$
    Thus
    $$
        \begin{aligned}
            \left\|f_s-f_t\right\|_p & \leq\left\|f_s-g_s\right\|_p+\left\|g_s-g_t\right\|_p+\left\|g_t-f_t\right\|_p \\
                                     & =\left\|(f-g)_s\right\|_p+\left\|g_s-g_t\right\|_p+\left\|(g-f)_t\right\|_p    \\
                                     & < 3 \varepsilon
        \end{aligned}
    $$
    whenever $|s-t|<\delta$. This completes the proof.
\end{theorem}
\begin{theorem}
    If $f\in L^1$, then $\hat{f} \in C_0$ and
    $$
        \|\hat{f}\|_{\infty} \leq\left\| f \right\|_1
    $$

    Proof:
    1. If $t_n \rightarrow t$, then
    $$
        |\hat{f}\left(t_n\right)-\hat{f}(t)|
        \leq
        \frac{1}{(2\pi)^{\frac{n}{2}}} \int_{\mathbb{R}^n} |f(x)||e^{-i t_n x}-e^{-i t x}| \d  x
    $$
    The integrand is bounded by $2|f(x)|$ and tends to 0 for every $x$, as $n \rightarrow \infty$. Hence $\hat{f}\left(t_n\right) \rightarrow \hat{f}(t)$, by the dominated convergence theorem. Thus $\hat{f}$ is continuous.

    2. Since $e^{i\pi}=-1$
    $$
        f(t)= - \frac{1}{(2\pi)^{\frac{n}{2}}} \int_{\mathbb{R}^n} f(x) e^{-i t(x+\frac{\pi}{t})} \d  x
        =-\frac{1}{(2\pi)^{\frac{n}{2}}} \int_{\mathbb{R}^n}
        f(x-\frac{\pi}{t}) e^{-i t x} \d  x
    $$
    Hence
    $$
        2 f(t)= \frac{1}{(2\pi)^{\frac{n}{2}}} \int_{\mathbb{R}^n}\left\{f(x)-f\left(x-\frac{\pi}{t}\right)\right\} e^{-i t x} \d  x
    $$
    so that
    $$
        2|\hat{f}(t)| \leq\left\|f-f_{\frac{\pi}{t}}\right\|_1
    $$
    which tends to 0 as $t \rightarrow \pm \infty$, by previous theorem.
\end{theorem}
\begin{theorem}[The Inversion Theorem for $L^1(\mathbb{R}^n)$]
    If $f \in L^1$ and $\hat{f} \in L^1$, then $\mathcal{F}^{-1}(f)(x)$ exist for all $x\in \mathbb{R}^n$ and
    $$
        \mathcal{F}^{-1}\hat{f} = f
    $$
    for almost every $x \in \mathbb{R}^n$.

    Proof:
    Consider
    \begin{equation*}
        \begin{aligned}
            f*\frac{1}{\left(2\varepsilon\right)^{\frac{n}{2}}} e^{-\frac{\left|x\right|^2}{4\varepsilon}}
             & =
            f*\mathcal{F}^{-1}(e^{\varepsilon\left|y\right|^2})                                                                                                             \\
             & =\int_{\mathbb{R}^n} f(x-y) \left(\frac{1}{\left(2 \pi \right)^\frac{n}{2}}\int_{\mathbb{R}^n} e^{-\varepsilon \left|t\right|^2} e^{it\cdot y} dt \right)dy  \\
             & =\int_{\mathbb{R}^n} e^{-\varepsilon \left|t\right|^2}\left( \frac{1}{\left(2 \pi \right)^\frac{n}{2}}\int_{\mathbb{R}^n} f(x-y) e^{it\cdot y} dy \right) dt \\
             & = \int_{\mathbb{R}^n} e^{-\varepsilon \left|t\right|^2} \hat{f}(t) e^{i t\cdot x} dt
        \end{aligned}
    \end{equation*}
    Let $\varepsilon \to 0^+$, we have
    \begin{equation*}
        f(x)
        =
        \mathcal{F}^{-1}(\hat{f})(x)
    \end{equation*}
    a.e. $x\in \mathbb{R}^n$ since $f\in L^1(\mathbb{R}^n)$ and $\hat{f}\in C^0(\mathbb{R}^n)$.

    \textbf{Remark} We can see $f$ has a version $\mathcal{F}^{-1}f \in C^0(\mathbb{R}^n)$ in this case.
\end{theorem}
\begin{theorem}[The Uniqueness Theorem]
    If $f \in L^1$ and $\hat{f}(t)=0$ for all $t \in R$, then $f(x)=0$ a.e. $[m]$
\end{theorem}


\section{p=2}
\begin{theorem}[The Plancherel Theorem on $L^2(\mathbb{R}^n)$]
    $\mathcal{F}$ is a unitary operator on $L^2(\mathbb{R}^n)$.
\end{theorem}
\begin{theorem}
    $\mathcal{F}$ is a self-adjoint operator on $L^2(\mathbb{R}^n)$.

    Proof:
    We observe that if $u, v \in L^1(\mathbb{R}^n) \cap L^2\left(\mathbb{R}^n\right)$, then
    \begin{equation*}
        \int_{\mathbb{R}^n} \hat{u} v d x=\int_{\mathbb{R}^n} u \hat{v} d x
    \end{equation*}
    since both sides equal
    \begin{equation*}
        \frac{1}{(2 \pi)^{n / 2}} \int_{\mathbb{R}^n}
        \int_{\mathbb{R}^n}
        u(y) v(x) e^{-i x \cdot y}
        dx dy
    \end{equation*}
\end{theorem}
\begin{theorem}[The Inversion Theorem for $L^2(\mathbb{R}^n)$]
    In $L^2(\mathbb{R}^n)$
    \begin{equation*}
        \mathcal{F}^{-1}\mathcal{F}
        =\mathcal{F}\mathcal{F}^{-1}
        =I
    \end{equation*}


    Proof:
    Ues the Plancherel Theorem
    \begin{equation*}
        \int_{\mathbb{R}^n}\mathcal{F}^{-1} \left(\hat{u}\right) v d x
        =
        \int_{\mathbb{R}^n} \hat{u} \check{v} d x=\int_{\mathbb{R}^n} \hat{u} \overline{\mathcal{F}(\bar{v})} d x
        =
        \int_{\mathbb{R}^n} u v d x
    \end{equation*}
    This holds for all $v \in L^2\left(\mathbb{R}^n\right)$, and so statement follows.
\end{theorem}


\section{The Schwarz Space}
\begin{definition}
    The Schwarz space $\mathcal{S}(\mathbb{R}^n)$ consists of all infinitely differentiable functions $f$ on $\mathbb{R}^n$ such that
    $$\sup_{x\in \mathbb{R}^n} \left|x^\alpha \left(\frac{\partial}{\partial x}\right)^\beta f(x)\right| < \infty$$
    for every multi-index $\alpha$ and $\beta$. In other words, $f$ and all its derivatives are required to be rapidly decreasing.

    Noted that $\mathcal{S}\subset L^1\cap L^\infty \cap C^\infty $
\end{definition}
\begin{theorem}[Poisson Summation Formula]
    Let $f \in \mathcal{S}(\mathbb{R})$, then
    $$
        \sum_{k=-\infty}^{\infty} f(2 k \pi)
        =
        \frac{1}{\sqrt{2\pi}}\sum_{n=-\infty}^{\infty}  \mathcal{F}f(n)
    $$

    Proof: Let
    $$
        F(x)=\sum_{k=-\infty}^{\infty} f(x+2 k \pi)
    $$
    then $F$ is periodic, with period $2 \pi$, the $n$-th Fourier coefficient of $F$ is
    $$\begin{aligned}
            \hat{F}(n) & =\frac{1}{2\pi} \int_{-\pi}^\pi F(x) e^{-inx} \d x \\
                       & =\frac{1}{2\pi} \int_\mathbb{R} f(x) e^{-inx} \d x \\
                       & =\frac{1}{\sqrt{2\pi}} \mathcal{F}f(n)
        \end{aligned}
    $$
    hence $F(x)=\sum \frac{1}{\sqrt{2\pi}} \mathcal{F}f(n) e^{i n x}$. In particular, let $x=0$, we have


    More generally, if $\alpha>0, \beta>0, \alpha \beta=2 \pi$,    we have
    $$
        \sum_{k=-\infty}^{\infty} f(k \beta)=\frac{\alpha}{\sqrt{2 \pi}} \sum_{n=-\infty}^{\infty} \varphi(n \alpha)
    $$
\end{theorem}













\chapter{Differentiation}
\section{Derivates}
\subsection{Derivates of Measures}
\begin{definition}
    Let us fix a dimension $k$, denote the open ball with center $x \in R^k$ and radius $r>0$ by
    \begin{equation*}
        B(x, r)
        =
        \left\{y \in R^k:|y-x|<r\right\}
    \end{equation*}
    associate to any complex Borel measure $\mu$ on $\mathbb{R}^k$ the quotients
    $$
        \left(Q_r \mu\right)(x)=\frac{\mu(B(x, r))}{m(B(x, r))}
    $$
    where $m$ is Lebesgue measure on $R^k$, and define the \textbf{symmetric derivative} of $\mu$ at $x$ to be
    \begin{equation*}
        (D \mu)(x)
        =
        \lim _{r \rightarrow 0^+}\left(Q_r \mu\right)(x)
    \end{equation*}
    at those points $x \in R^k$ at which this limit exists.

    We shall study $D \mu$ by means of the \textbf{maximal function} $M \mu$ that  is defined by
    $$
        (M \mu)(x)=\sup _{0<r<\infty}\left(Q_r \left|\mu\right|\right)(x)
    $$
\end{definition}

\begin{theorem}
    The functions $M \mu: R^k \rightarrow[0, \infty]$ are lower semicontinuous, hence measurable.

    Proof:
    To see this, assume $\mu \geq 0$, pick $\lambda>0$, let $E=\left\{M \mu>\lambda\right\}$, and fix $x \in E$. Then there is an $r>0$ such that
    $$
        \mu(B(x, r))=tm(B(x, r))
    $$
    for some $t>\lambda$. Then there are small $\delta$ and $r$ that
    $$
        \begin{aligned}
            \mu(B(y, r+\delta)) & \geq \mu (B(x,r))                  \\
                                & = tm(B(x, r))                      \\
                                & =(t-\varepsilon) m(B(y, r+\delta)) \\
                                & >\lambda m(B(y, r+\delta))
        \end{aligned}
    $$
    Thus $B(x, \delta) \subset E$.
\end{theorem}



\begin{lemma}
    If $W$ is the union of a finite collection of balls $B\left(x_i, r_i\right) \quad  i =1,2 \ldots , N$, then there is a set $S \subset\left\{1, \ldots, N\right\}$ so that
    \begin{enumerate}[label=(\roman*)]
        \item the balls $B\left(x_i, r_i\right)$ with $i \in S$ are disjoint,
        \item $W \subset \bigcup\limits_{i \in S} B\left(x_i, 3 r_i\right)$
    \end{enumerate}

    Proof:
    Order the balls $B_i=B\left(x_i, r_i\right)$ so that $r_1 \geq r_2 \geq \ldots \geq r_N$. Put $i_1=1$. Discard all $B_j$ that intersect $B_{i_1}$. Let $B_{i_2}$ be the first of the remaining $B_j$, if there are any. Discard all $B_j$ with $j>i_2$ that intersect $B_{i_2}$, let $B_{i_3}$ be the first of the remaining ones, and so on, as long as possible. This process stops after a finite number of steps and gives $S=\left\{i_1, i_2, \ldots\right\}$.

    It is clear that (a) holds. Every discarded $B_j$ is a subset of $B\left(x_i, 3 r_i\right)$ for some $i \in S$, for if $r^{\prime} \leq r$ and $B\left(x^{\prime}, r^{\prime}\right)$ intersects $B(x, r)$, then $B\left(x^{\prime}, r^{\prime}\right) \subset B(x, 3 r)$. This proves (b)
\end{lemma}

\newpage

\begin{theorem}
    If $\mu$ is a complex Borel measure on $R^k$ and $\lambda$ is a positive number, then
    \begin{equation*}
        m  \left\{M \mu> \lambda\right\}
        \leq
        3^k \lambda^{-1} \|\mu\|
    \end{equation*}

    Proof:
    Fix $\mu$ and $\lambda$. Let $K$ be a compact subset of the open set
    $\left\{M \mu>\lambda\right\}$. Each $x \in K$ is the center of an open ball $B$ for which
    \begin{equation*}
        \left|\mu\right|(B)
        >
        \lambda m(B)
    \end{equation*}
    Some finite collection of these Balls covers $K$, and previous Lemma gives us a disjoint subcollection, say $\left\{B_1, \ldots, B_n\right\}$, that satisfies
    \begin{equation*}
        \begin{aligned}
            m(K) & \leq \sum_1^n m\left(B_i(x_i,3 r_i)\right)                                       \\
                 & = 3^k \sum_1^n m\left(B_i\right)                                                 \\
                 & \leq 3^k \lambda^{-1} \sum_1^n|\mu|\left(B_i\right) \leq 3^k \lambda^{-1}\|\mu\|
        \end{aligned}
    \end{equation*}
    Now taking the supremum over all compact $K \subset\left\{M \mu>\lambda\right\}$.
\end{theorem}

\begin{definition}
    Any measurable function $f$ for which
    $$
        \lambda \cdot m\left\{|f|>\lambda\right\}
    $$
    is a bounded function of $\lambda$ on $(0, \infty)$ is said to belong to\textbf{ weak $L^1$}.
    Thus weak $L^1$ contains $L^1$.
\end{definition}

\subsection{Derivates of functions}
\begin{definition}
    We associate to each $f \in L^1\left(R^k\right)$ its \textbf{maximal function} $M f: R^k \rightarrow[0, \infty]$, by setting
    \begin{equation*}
        (M f)(x)
        =
        \sup _{0<r<\infty} \frac{1}{m\left(B_r\right)} \int_{B(x, r)}|f| \d  m
    \end{equation*}
    If we identify $f$ with the measure $\mu$ given by $d \mu=f d m$, we see that $M f$ agrees with the previously defined $M \mu$. Theorem 7.4 states therefore that the "maximal operator " $M$ sends $L^1$ to weak $L^1$, with a bound that depends only on the space $R^k$ :
    For every $f \in L^1\left(R^k\right)$ and every $\lambda>0$,
    \begin{equation*}
        m\left\{M f>\lambda\right\}
        \leq
        3^k \lambda^{-1}\left\| f \right\|_1
    \end{equation*}

\end{definition}

\begin{theorem}[Lebesgue points]
    If $f \in L^1\left(R^k\right)$, any $x \in R^k$ for which it is true that
    \begin{equation*}
        \lim _{r \rightarrow 0} \frac{1}{m\left(B_r\right)} \int_{B(x, r)}
        \left|f(y)-f(x) \right|\d  m(y)
        =
        0
    \end{equation*}
    is called a \textbf{Lebesgue point} of $f$.
    If $f \in L^1\left(R^k\right)$, then almost every $x \in R^k$ is a Lebesgue point of $f$.

    Proof: Define
    \begin{equation*}
        (T_r f)(x)=\frac{1}{m(B_r)} \int_{B(x, r)}|f-f(x)| \d  m
    \end{equation*}
    for $x \in R^k, r>0$, and put
    \begin{equation*}
        (T f)(x)=\underset{r \rightarrow 0}{\lim \sup} \left(T_r f\right)(x)
    \end{equation*}
    We have to prove that $Tf=0$ a.e. $[\mathrm{m}]$.

    Pick $y>0$. Let $n$ be a positive integer. By Theorem 3.14, there exists $g \in C\left(R^k\right)$ so that $\|f-g\|_1<1 / n$. Put $h=f-g$, then
    \begin{equation*}
        \left(T_r h\right)(x)
        \leq
        \frac{1}{m\left(B_r\right)} \int_{B(x, r)}|h| d m+|h(x)|
    \end{equation*}
    Let $ r \to 0 $, we have
    \begin{equation*}
        T h \leq M h+|h|
    \end{equation*}

    Since $g$ is continuous, $T g=0$. Since $T_{r} f \leq T_{r} g+T_{r} h$, it follows that
    \begin{equation*}
        T f \leq T h\leq M h+|h|
    \end{equation*}
    Therefore
    \begin{equation*}
        \left\{T f>2 y\right\} \subset\left\{M h>y\right\} \cup\left\{|h|>y\right\}
    \end{equation*}
    Denote the union on the right $E(y, n)$. Since $\|h\|_1<1 / n$, Theorem 7.4 and the inequality $7.5(1)$ show that
    \begin{equation*}
        \begin{aligned}
            m(E(y, n)) & \leq m(\left\{M h>y\right\}) + m(\left\{|h|>y\right\}) \\
                       & \leq 3^k y^{-1} \|h\|_1 + y^{-1} \|h\|_1               \\
                       & \leq \frac{ 3^k+1 }{ yn }
        \end{aligned}
    \end{equation*}
    The left side is independent of $n$. Hence
    \begin{equation*}
        \left\{T f>2 y\right\} \subset \bigcap_{n=1}^{\infty} E(y, n)
    \end{equation*}
    This intersection has measure 0, so that $\left\{T f>2 y\right\}$  has measure 0 . This holds for every positive $y$. Hence $T f=0$ a.e. $[m]$.
\end{theorem}



\begin{definition}
    A sequence $\left\{E_i\right\}$ of Borel sets in $R^k$ is said to \textbf{shrink to $x$ nicely} if there is a number $\alpha>0$ with the following property: There is a sequence of balls $B\left(x, r_i\right)$, with $\lim r_i=0$, such that $E_i \subset B\left(x, r_i\right)$ and
    \begin{equation*}
        m\left(E_i\right) \geq \alpha \cdot m\left(B\left(x, r_j\right)\right)
    \end{equation*}
    for $i=1,2,3, \ldots$.
\end{definition}



\begin{theorem}
    Associate to each $x \in R^k $ a sequence $\left\{E_i(x)\right\}$ that shrinks to $x$ nicely, and let $f \in L^1\left(R^k\right)$. Then
    \begin{equation*}
        f(x)
        =
        \lim_{i \to\infty} \frac{1}{m\left(E_i(x)\right)} \int_{E_i(x)} f \d m
    \end{equation*}
    at every Lebesgue point of $f$, hence a.e. $[m]$.

    Proof:
    Let $x$ be a Lebesgue point of $f$ and let $\alpha(x)$ and $B\left(x, r_j\right)$ be the positive number and the balls that are associated to the sequence $\left\{E_i (x)\right\}$. Then, because $E_i(x) \subset B\left(x, r_i\right)$,
    \begin{equation*}
        \frac{\alpha(x)}{m\left(E_i(x)\right)}
        \int_{E_i(x)} \left|f-f(x)\right| \d m
        \leq
        \frac{1}{m\left(B\left(x, r_i\right)\right)} \int_{B(x, r_i)}\left|f-f(x)\right| \d m
    \end{equation*}
    The right side converges to $0$ as $i \to \infty$, because $r_i \rightarrow 0$ and $x$ is a Lebesgue point of $f$. Hence the left side converges to $0$.
\end{theorem}


\subsection{Consequence}

\begin{theorem}
    If $f \in L^1\left(R^1\right)$ and
    \begin{equation*}
        F(x)
        =
        \int_{-\infty}^x f \d m
    \end{equation*}
    then $F^{\prime}(x)=f(x)$ at every Lebesgue point of $f$, hence a.e. $[m]$.

    Proof:
    Let $\left\{\delta_i\right\}$ be a sequence of positive numbers that converges to $0$ .
    Theorem 7.10 , with $E(x)=\left[x, x+\delta_i\right]$, shows then that
    \begin{equation*}
        \frac{F(x+\delta_i)-F(x)}{\delta_i}
        =
        \frac{1}{\delta} \int_{E_i(x)} f \d m
        \to f(x)
    \end{equation*}
    at all Lebesgue points of $x$ of $f$.

    If we let $E_i(x)$ be $\left[x-\delta_i, x\right]$ instead, we obtain the same result for the let-hand derivative of $F$ at $x$.
\end{theorem}

\begin{theorem}[Metric density]
    Let $E$ be a Lebesgue measurable subset of $R^k$. The \textbf{metric density} of $E$ at a point $x \in R^k$ is defined to be
    \begin{equation*}
        \lim _{r \rightarrow 0} \frac{m(E \cap B(x, r))}{m(B(x, r))}
    \end{equation*}
    provided, of course, that this limit exists.

    If we let $f$ be the characteristic function of $E$ and apply Theorem 7.8 or Theorem 7.10, we see that the metric density of $E$ is 1 at almost every point of $E$, and that it is 0 at almost every point of the complement of $E$.
\end{theorem}

\begin{proposition}
    If $A \subset \mathbb{R}^1$ and $B \subset \mathbb{R}^1$, define $A+B=\left\{a+b: a \in A, b \in B\right\}$. Suppose $m(A)>0, m(B)>0$. Then $A+B$ contains a segment.

    Proof:
    There are points $a_0$ and $b_0$ where $A$ and $B$ have metric density $1$. Put $c_0=a_0+b_0$. We can assume that $a_0=b_0=c_0$, and prove that $A+B$ contains segment $\left(-\delta, \delta\right)$ for some small $\delta>0$.
\end{proposition}

\begin{question}
    Suppose $G$ is a proper subgroup of $\left(\mathbb{R}^1,+\right)$ and $G$ is Lebesgue measurable. Prove that then $m(G)=0$.
\end{question}


\section{The Fundamental Theorem for Calcus}




\subsection{Bounded Variation}
\begin{definition}
    If $f$ is any (complex) function on $I=[a,b]$, define
    \begin{equation*}
        V_a^b \left(f\right)
        =
        \sup \sum_{i=1}^N\left|f\left(t_i\right)-f\left(t_{i-1}\right)\right|
    \end{equation*}
    where the supremum is taken over all $N$ and over all choices of $\left\{t_i\right\}_{i=1}^N$ such that
    \begin{equation*}
        a
        =
        t_0<t_1<\cdots<t_N
        =
        x
    \end{equation*}
    If $V_a^b \left(f\right) < \infty$ called \textbf{total variation function of $f$}, then $f$ is said to have \textbf{bounded variation} on $I$ (briefly, f is BV on $I$; $f \in BV[a,b]$).
\end{definition}

\begin{proposition}
    Fix $I=[a,b]$,

    (1) Bounded variation function is bounded on $I$

    (2) Monotonic function on $I$ has total variation $\left|f(b) - f(a)\right|$

    (3) $BV[a,b]$ is a linear space on $\mathbb{R}$
    (or $\mathbb{C}$)

    (4) $f \in BV[a,b]$ implies $\left|f\right| \in BV[a,b]$

    (5) for any $c \in \left(a,b\right)$
    \begin{equation*}
        V_a^c \left(f\right)
        +
        V_c^b \left(f\right)
        =
        V_a^b \left(f\right)
    \end{equation*}
\end{proposition}

\begin{theorem}
    Suppose $f: I \rightarrow R$ is BV on $I=[a, b]$. Let
    \begin{equation*}
        F(x)
        =
        V_a^x \left(f\right)\quad(a \leq x \leq b)
    \end{equation*}
    The functions $F, F+f, F-f$ are then nondecreasing and BV on $I$.

    Proof:
    Step 1. If $x<y \leq b$, then
    \begin{equation*}
        F(y) \geq|f(y)-f(x)|+\sum_{i=1}^N\left|f\left(t_i\right)-f\left(t_{i-1}\right)\right| .
    \end{equation*}
    Hence $F(y) \geq|f(y)-f(x)|+F(x)$. In particular
    \begin{equation*}
        F(y) \geq f(y)-f(x)+F(x) \text { and } F(y) \geq f(x)-f(y)+F(x)
    \end{equation*}
    This proves that $F, F+f, F-f$ are nondecreasing.

    Step 2. Since sums of two BV functions are obviously BV, it only remains to be proved that $F$ is BV on $I$.
\end{theorem}

\begin{theorem}[Jordan Decomposition]
    $f \in BV[a,b]$ if and only if $f =g-h $ where $g$ and $h$ are nondecreasing function on $[a,b]$
\end{theorem}



\subsubsection{Differentiable of nondecreasing function}

\begin{definition}
    A collection $\mathcal{B}$ of balls $\left\{B\right\}$ is said to be a \textbf{Vitali covering} of a set $E$ if for every $x \in R$ and any $\varepsilon >0$ there is a ball $B \in \mathcal{B}$, such that $x\in B$ and $m\left(B\right) < \varepsilon$.
\end{definition}

\begin{theorem}[Vitali covering theorem]

\end{theorem}


\begin{theorem}[Lebesgue Theorem]
    Nondecreasing function $f$ on $I=[a,b]$ is differentiable a.e. on $I$. And
    \begin{equation*}
        \int_{[a,b]} f' \d x \leq f(b) -f(a)
    \end{equation*}

    Proof:
    Let
    \( E =  \left\{x \in [a, b] : D^+ f(x) > D_- f(x) \right\} \), we prove that \( \mu(E) = 0 \).
    For every pair $\left(p,q\right)$, define
    \begin{equation*}
        E_{p,q}
        =
        \left\{x\in E : D_{-}f(x)<p<q< D^+ f(x) \right\}
    \end{equation*}
    we have
    \begin{equation*}
        E
        =
        \bigcup_{p,q \in \mathbb{Q}} E_{p,q}
    \end{equation*}

    For any $E_{p,q}$ and $x\in E_{p,q}$, there exists arbitrarily small $h >0$ such that
    \begin{equation*}
        \frac{f(x)-f(x-h)}{h} <p
    \end{equation*}
    Thus $\left\{[x-h,x]\right\}$ is a Vitali covering of $E$, then there exists finite disjoint sets $\left\{[x_i-h_i,x_i]\right\}$ such that
    \begin{equation*}
        m\left(E_{p,q} - \bigcup_i [x_i-h_i,x_i]\right)
        <
        \varepsilon
    \end{equation*}
    We have the estimate for left intervals
    \begin{equation*}
        \sum_i h_i
        \leq
        m\left(\bigcup_i [x_i-h_i,x_i]\right)
        \leq
        m(E_{p,q}) +\varepsilon
    \end{equation*}
    it follows that the estimate
    \begin{equation*}
        \sum_i f(x_i) -f(x_i - h_i)
        <
        p \sum_i h_i
        \leq
        p(m(E_{p,q}) +\varepsilon)
    \end{equation*}

    Similarly,
    \begin{equation*}
        q(m(E_{p,q}-2\varepsilon))
        \geq
        q \sum_j k_j
        <
        \sum_j f(x_i+k_i) -f(x_i )
    \end{equation*}













\end{theorem}
\begin{corollary}
    Bounded variation functions are differentiable a.e. on $I$
\end{corollary}

\subsection{Absolutely Continuous }

\begin{definition}
    A complex function $f$, defined on an interval $I=[a, b]$, is said to be \textbf{absolutely continuous} on $I$ (briefly, $f$ is AC on $I$ ) if there corresponds to every $\varepsilon>0$ a $\delta>0$ so that
    \begin{equation*}
        \sum_{i=1}^n
        \left|f\left(\beta_i\right)-f\left(\alpha_i\right)\right|<\varepsilon
    \end{equation*}
    for any $n$ and any disjoint collection of segments $\left(\alpha_1, \beta_1\right), \ldots,\left(\alpha_n, \beta_n\right)$ in $I$ whose lengths satisfy
    \begin{equation*}
        \sum_{i=1}^n\left(\beta_i-\alpha_i\right)<\delta .
    \end{equation*}
\end{definition}

\begin{proposition}
    Fix $I=[a,b]$,

    (1) $AC[a,b] \subset BV[a,b]$

    (2) $AC[a,b]\subset C(I) $

    (3) $AC[a,b]$ is a linear space over $\mathbb{R}$

    (4) If $f \in AC[a,b]$, then $F(x)= V_a^x\left(f\right)$ is nondecreasing and AC .

    Proof:
    If $(a, \beta) \subset I$ then
    \begin{equation*}
        F(\beta)-F(\alpha)=\sup \sum_i^n\left|f\left(t_i\right)-f\left(t_{i-1}\right)\right|,
    \end{equation*}
    the supremum being taken over all $\left\{t_i\right\}$ that satisfy $\alpha=t_0<\cdots<t_n=\beta$.
    Note that $\sum\left(t_i-t_{i-1}\right)=\beta-\alpha$.
    Now pick $\varepsilon>0$, associate $\delta>0$ to $f$ and $\varepsilon$ , choose disjoint segments $\left(\alpha_j, \beta_j\right) \subset I$ with $\sum\left(\beta_j-\alpha_j\right)<\delta$, and apply (5) to each $\left(\alpha_j, \beta_j\right)$.
    It follows that
    \begin{equation*}
        \sum_j\left(F\left(\beta_j\right)-F\left(\alpha_j\right)\right) \leq \varepsilon
    \end{equation*}
    by our choice of $\delta$. Thus $F$ is AC on $I$.
\end{proposition}

\begin{theorem}
    Let $I=[a, b]$, let $f: I \rightarrow R^1$ be continuous and nondecreasing. Each of the following three statements about fimplies the other two:

    (1) $f$ is AC on $I$.

    (2) Lusin's property. $f$ maps sets of measure $0$ to sets of measure $0$ .

    (3) $f$ is differentiable a.e. on $I$, $f^{\prime} \in L^1$, and
    \begin{equation*}
        f(x)-f(a)=\int_a^x f^{\prime}(t) \d t \quad(\alpha \leq x \leq b) .
    \end{equation*}


    Proof:
    $(1) \rightarrow (2)$.
    Assume $f$ is AC on $I$, pick $E \subset I$ so that $E \in \mathcal{M}$ and $m(E)=0$.
    We have to show that $f(E) \in \mathcal{M}$ and $m(f(E))=0$. Without loss of generality, assume that neither $a$ nor $b$ lie in $E$.

    Choose $\varepsilon>0$. Associate $\delta>0$ to $f$ and $\varepsilon$.
    There is then an open set $V$ with $m(V)<\delta$, so that $E \subset V \subset I$. Let $\left(\alpha_i, \beta_i\right)$ be the disjoint segments whose union is $V$. Then $\sum\left(\beta_1-\alpha_i\right)<\delta$, and our choice of $\delta$ shows that therefore
    \begin{equation*}
        \sum_i\left(f\left(\beta_i\right)-f\left(\alpha_i\right)\right) \leq \varepsilon
    \end{equation*}
    Since $E \subset V, f(E) \subset \bigcup\left[f\left(\alpha_i\right), f\left(\beta_i\right)\right]$, then $m(f(E))=0$.

    $(2) \rightarrow (3)$.
    Assume next that (2) holds. Define
    \begin{equation*}
        g(x)=x+f(x) \quad(a \leq x \leq b) .
    \end{equation*}
    If the $f$-image of some segment of length $\eta$ has length $\eta^{\prime}$, then the $g$-image of that same segment has length $\eta+\eta^{\prime}$. From this it follows easily that $g$ satisfies (2), since $f$ does.

    Now suppose $E \subset I, E \in \mathcal{M}$. Then $E=E_1 \cup E_0$ where $m\left(E_0\right)=0$ and $E_1$ is an $F_g$.
    Thus $E_1$ is a countable union of compact sets, and so is $g\left(E_1\right)$, because $g$ is continuous.
    Since $g$ satisfies $(b), m\left(g\left(E_0\right)\right)=0$. Since $g(E)=g\left(E_1\right) \cup g\left(E_0\right)$, we conclude: $g(E) \in \mathcal{M}$.

    Therefore we can define
    \begin{equation*}
        \mu(E)=m(g(E)) \quad (E \subset I, E \in \mathcal{M})
    \end{equation*}
    Since $g$ is one-to-one, disjoint sets in $I$ have disjoint $g$-images. The countable additivity of $m$ shows therefore that $\mu$ is a positive, bounded measure on $\mathcal{M}$.
    Also, $\mu \ll m$, because $g$ satisfies (2).
    Thus
    \begin{equation*}
        d \mu=h \d m
    \end{equation*}
    for some $h \in L^1(m)$, by the Radon-Nikodym theorem.
    If $E=[a, x]$, then $g(E)=[g(a), g(x)]$, and
    \begin{equation*}
        g(x)-g(a)
        =
        m(g(E))=\mu(E)
        =
        \int_E h \d m
        =
        \int_a^x h(t) \d t
    \end{equation*}
    If we now use $g(x)=f(x)+x$, we conclude that
    \begin{equation*}
        f(x)-f(a)=\int_a^x[h(t)-1] d t \quad(\alpha \leq x \leq b) .
    \end{equation*}
    Thus $f^{\prime}(x)=h(x)-1$ a.e. $[m]$, by Theorem 7.11.

    The discussion that preceded Definition 7.17 showed that $(3)$ implies $(1)$.
\end{theorem}

\subsection{Main Objects}
\begin{theorem}
    If $f$ is a (complex) function that is AC on $I=[a, b]$, then

    (1) $f$ is differentiable at almost all points of $I$

    (2) $f^{\prime} \in L^1\left[a,b\right]$, and

    (3)
    \begin{equation*}
        f(x)-f(a)=\int_a^x f^{\prime}(t) \d t \quad(a \leq x \leq b) .
    \end{equation*}

    (4) Lusin's property

    Proof:
    It is of course enough to prove this for real $f$. Let $F$ be its total variation function, define
    \begin{equation*}
        f_1=\frac{1}{2}(F+f), \quad f_2=\frac{1}{2}(F-f),
    \end{equation*}
    and then $f_1$ and $f_2$ are nondecreasing and AC. Since
    \begin{equation*}
        f=f_1-f_2
    \end{equation*}
    this yields (1).
\end{theorem}




\begin{theorem}
    If $f:[a, b] \rightarrow \mathbb{R}^1$ is differentiable at every point of $[a, b]$ and $f^{\prime} \in L^1$ on $[a, b]$, then
    \begin{equation*}
        f(x)-f(a)=\int_a^x f^{\prime}(t) \d t \quad(a \leq x \leq b)
    \end{equation*}

    Proof:
    It is clear that it is enough to prove this for $x=b$. Fix $\varepsilon>0$. Theorem 2.25 ensures the existence of a lower semicontinuous function $g$ on $[a, b]$ such that $f^{\prime} < g$ and
    \begin{equation*}
        \int_a^b g(t) d t<\int_a^b f^{\prime}(t) d t+\varepsilon .
    \end{equation*}

    For any $\eta>0$, define
    \begin{equation*}
        F_\eta(x)
        =
        \int_a^x g(t) \d t
        -
        \left[f(x)-f(a)\right]
        +
        \eta(x-a) \quad(a \leq x \leq b) .
    \end{equation*}
    To each $x \in[a, b)$ there corresponds a $\delta_x>0$ such that
    \begin{equation*}
        g(t)>f^{\prime}(x) \text { and } \frac{f(t)-f(x)}{t-x}<f^{\prime}(x)+\eta
    \end{equation*}
    for all $t \in\left(x, x+\delta_x\right)$, since $g$ is lower semicontinuous and $g(x)>f^{\prime}(x)$.
    For any such $t$ we therefore have
    \begin{equation*}
        \begin{aligned}
            F_\eta(t)-F_\eta(x) & =\int_x^t g(s) \d s-[f(t)-f(x)]+\eta(t-x)                           \\
                                & >(t-x) f^{\prime}(x)-(t-x)\left[f^{\prime}(x)+\eta\right]+\eta(t-x) \\
                                & =0
        \end{aligned}
    \end{equation*}
    Thus $F_\eta$ is increasing on $[a,b)$. Since $F_\eta(a)=0$ and $F_\eta$ is continuous
    \begin{equation*}
        F_\eta(b)
        =
        \int_a^b g(t) \d t
        -
        \left[f(b)-f(a)\right]
        +
        \eta(b-a)
        \geq 0
    \end{equation*}
    Since this holds for every $\eta>0$, we have
    \begin{equation*}
        f(b)-f(a) \leq \int_a^b g(t) d t<\int_a^b f^{\prime}(t) d t+\varepsilon,
    \end{equation*}
    and since $\varepsilon$ was arbitrary, we conclude that
    \begin{equation*}
        f(b)-f(a) \leq \int_a^b f^{\prime}(t) \d t
    \end{equation*}

    Similarly, we conclude that
    \begin{equation*}
        -f(b)-\left[-f(a)\right] \leq \int_a^b -f^{\prime}(t) \d t
    \end{equation*}
    Finally,
    \begin{equation*}
        f(b)-f(a) = \int_a^b f^{\prime}(t) \d t
    \end{equation*}
\end{theorem}


\section{Differentiable Transforms}


\begin{theorem}
    Suppose that

    \begin{enumerate}[label=(\roman*)]
        \item  $ V \subset R^k$, $V$ is open, and $T: V \rightarrow R^k$ is continuous;
        \item  $X \subset V$ is Lebesgue measurable, $T$ is one-to-one on $X$, and $T$ is differentiable at every point of $X$;
        \item  $m(T(V-X))=0$.
    \end{enumerate}
    Then, setting $Y=T(X)$,
    \begin{equation*}
        \int_Y f \d m
        =
        \int_X (f \circ T)\left|J_T\right| \d m
    \end{equation*}
    for every measurable $f: R^k \rightarrow[0, \infty]$.
\end{theorem}



\begin{corollary}
    Suppose $\varphi:[a, b] \rightarrow[\alpha, \beta]$ is AC, monotonic, $\varphi(a)=\alpha, \varphi(b)=\beta$, and $f \geq 0$ is Lebesgue measurable. Then
    \begin{equation*}
        \int_a^\beta f(t) \d t
        =
        \int_a^b f(\varphi(x)) \varphi^{\prime}(x) \d x
    \end{equation*}
\end{corollary}

\section{Problem}


\begin{question}
    Suppose that $E$ is a measurable subset of $\mathbb{R}$ with arbitrarily small periods. Prove that then either $E$ or its complement has measure $0$.

    Proof:
    Pick $\alpha \in \mathbb{R}^1$ and period $p_i$, put
    $F(x)=m\left(E \cap[\alpha, x]\right)$
    for $x>\alpha$, show that
    \begin{equation*}
        F\left(x+p_i\right)-F\left(x-p_i\right)=F\left(y+p_i\right)-F\left(y-p_j\right)
    \end{equation*}
    if $\alpha+p_i<x<y$. It follows that $F'(x)=C$ a.e. on $\mathbb{R}$.
\end{question}

\begin{question}
    Suppose $f$ is a real Lebesgue measurable function with periods $s$ and $t$ whose quotient $s / t$ is irrationial. Prove that there is a constant $c$ such that $f(x)=c$ a.e. on $\mathbb{R}$.

    Hint: Apply Exercise 3 to the sets $\left\{f>\lambda\right\}$.
\end{question}










\end{document}