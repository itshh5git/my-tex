\chapter{Galois Theory}
It can be viewed as another (extremely elegant) application of the important idea in mathematics that one object acting on another provides structural information about both.
\section{Basic Definition}
\begin{definition}\
     Let $K$ be a field.

     (1) An isomorphism $\sigma$ of $K$ with itself is called an automorphism of $K$. The collection of automorphisms of $K$ is denoted $\operatorname{Aut}(K)$.

     (2) An automorphism $\sigma \in \operatorname{Aut}(K)$ is said to fix an element $\alpha \in K$ if $\sigma \alpha=\alpha$. If $F$ is a subset of $K$ (for example, a subfield), then an automorphism $\sigma$ is said to fix $F$ if it fixes all the elements of $F$.
     
     Noted that any automorphism of a field $K$ fixes its prime subfield. In particular we see that $\mathbb{Q}$ and $\mathbb{F}_p$ have only the trivial automorphism: $\operatorname{Aut}(\mathbb{Q})=\{1\}$ and $\operatorname{Aut}\left(\mathbb{F}_p\right)=\{1\}$.
\end{definition}
\begin{definition}\
     Let $K/F$ be an extension of fields. Let $Aut( K/F)$ be the collection of automorphisms of $K$ which fix $F$.
\end{definition}
\begin{theorem}\
     $\operatorname{Aut}(K)$ is a group under composition and $\operatorname{Aut}(K / F)$ is a subgroup.
\end{theorem}
\begin{theorem}\
     Let $K / F$ be a field extension and let $\alpha \in K$ be algebraic over $F$. Then for any $\sigma \in \operatorname{Aut}(K / F), \sigma \alpha$ is a root of the minimal polynomial for $\alpha$ over $F$ i.e., $\operatorname{Aut}(K / F)$ permutes the roots of irreducible polynomials. 
     
     Equivalently, any polynomial with coefficients in $F$ having $\alpha$ as a root also has $\sigma \alpha$ as a root.
     
     Proof: Suppose $\alpha$ satisfies the equation
     $$
     \alpha^n+a_{n-1} \alpha^{n-1}+\cdots+a_1 \alpha+a_0=0
     $$
     where $a_0, a_1, \ldots, a_{n-1}$ are elements of $F$. Applying the automorphism $\sigma \in \operatorname{Aut}(K / F)$ we obtain
     $$
     (\sigma \alpha)^n+a_{n-1}(\sigma \alpha)^{n-1}+\cdots+a_1(\sigma \alpha)+a_0=0
     $$
     This says precisely that $\sigma \alpha$ is a root of the same polynomial over $F$ as $\alpha$.
\end{theorem}
\begin{definition}\
     If $H \leq \operatorname{Aut}(K)$ be a subgroup of the group of automorphisms of $K$, the collection $F$ of elements of $K$ fixed by all the elements of $H$ is a subfield of $K$ called the fixed field of $H$.
\end{definition}
\begin{theorem}\
     
     (1) if $F_1 \subseteq F_2 \subseteq K$ are subfields of $K$ then $\operatorname{Aut}\left(K / F_2\right) \leq \operatorname{Aut}\left(K / F_1\right)$

     (2) if $\operatorname{Aut}(K/F_1) \leq \operatorname{Aut}(K/F_2)  \leq \operatorname{Aut}(K)$, then $F_2 \subseteq F_1$.
\end{theorem}  
\begin{theorem}\
     Let $F$ be a field and let $E$ be the splitting field over $F$ of $f(x) \in F[x]$. any isomorphism $\varphi: F \xrightarrow{\sim} F^{\prime}$ of $F$ with $F^{\prime}$ can be extended to an isomorphism $\sigma: E \xrightarrow{\sim} E^{\prime}$ between $E$ and the splitting field $E^{\prime}$ for $f^{\prime}(x)=\varphi(f(x)) \in F^{\prime}[x]$.

     The number of such extensions (isomorphism of the splitting field over $F$ of $f$) is is equal to the
     number of distinct roots of $f$(hence $\leq [E: F]$, with equality if $f(x)$ is separable over $F$)

     Proof:
     We now show by induction on $[E: F]$ that the number of such extensions is at most $[E: F]$, with equality if $f(x)$ is separable over $F$. 
     
     If $[E: F]=1$ then $E=F$, $E^{\prime}=F^{\prime}, \sigma=\varphi$ and the number of extensions is $1$.
     
     If $[E: F]>1$ then $f(x)$ has at least one irreducible factor $p(x)$ of degree $>1$ with corresponding irreducible factor $p^{\prime}(x)$ of $f^{\prime}(x)$. Let $\alpha$ be a fixed root of $p(x)$. If $\sigma$ is any extension of $\varphi$ to $E$, then $\sigma$ restricted to the subfield $F(\alpha)$ of $E$ is an isomorphism $\tau$ of $F(\alpha)$ with some subfield of $E^{\prime}$. The isomorphism $\tau$ is completely determined by its action on $\alpha$, i.e., by $\tau \alpha$, since $\alpha$ generates $F(\alpha)$ over $F$. Just as in Proposition 2, we see that $\tau \alpha$ must be some root $\beta$ of $p^{\prime}(x)$. Then we have a diagram

     Conversely, for any $\beta$ a root of $p^{\prime}(x)$ there are extensions $\tau$ and $\sigma$ giving such a diagram (this is Theorem 13.8 and Theorem 13.27). Hence to count the number of extensions $\sigma$ we need only count the possible number of these diagrams.
     
     The number of extensions of $\varphi$ to an isomorphism $\tau$ is equal to the number of distinct roots $\beta$ of $p^{\prime}(x)$. Since the degree of $p(x)$ and $p^{\prime}(x)$ are both equal to $[F(\alpha): F]$, we see that the number of extensions of $\varphi$ to a $\tau$ is at most $[F(\alpha): F]$, with equality if the roots of $p(x)$ are distinct.
     
     Since $E$ is also the splitting field of $f(x)$ over $F(\alpha), E^{\prime}$ is the splitting field of $f^{\prime}(x)$
     
     Sec. 14.1 Basic Definitions
     561
     over $F^{\prime}(\beta)$, and $[E: F(\alpha)]<[E: F]$, we may apply our induction hypothesis to these field extensions. By induction, the number of extensions of $\tau$ to $\sigma$ is $\leq[E: F(\alpha)]$, with equality if $f(x)$ has distinct roots.
     
     From $[E: F]=[E: F(\alpha)][F(\alpha): F]$ it follows that the number of extensions of $\varphi$ to $\sigma$ is $\leq[E: F]$. We have equality if $p(x)$ and $f(x)$ have distinct roots, which is equivalent to $f(x)$ having distinct roots since $p(x)$ is a factor of $f(x)$, completing the proof by induction.
     
     In the particular case when $F=F^{\prime}$ and $\varphi$ is the identity map we have $f(x)=f^{\prime}(x)$ and $E=E^{\prime}$ so the isomorphisms of $E$ to $E^{\prime}$ restricting to $\varphi$ on $F$ are the automorphisms of $E$ fixing $F$. We state this as follows:
     
    
\end{theorem}
\begin{corollary}\
     Let $E$ be the splitting field over $F$ of the polynomial $f(x) \in F[x]$. Then
     $$
     |\operatorname{Aut}(E / F)| \leq[E: F]
     $$
     with equality if $f(x)$ is separable over $F$.
\end{corollary}
\begin{definition}\
     Let $K / F$ be a finite extension. Then $K$ is said to be Galois over $F$ and $K / F$ is a Galois extension if $|\operatorname{Aut}(K / F)|=[K: F]$. If $K / F$ is Galois the group of automorphisms Aut $(K / F)$ is called the Galois group of $K / F$, denoted $\operatorname{Gal}(K / F)$.

     If $K$ is the splitting field over $F$ of a separable polynomial $f (x)$ then $K/F$ is Galois and $\operatorname{Gal}(K / F)$ is called the Galois group of $f (x)$ over $F$.
\end{definition}

\section{The Fundamental Theorem}
\begin{definition}\
     A linear character $ \chi$ of a group $G$ with values in a field $L$ is a homomorphism from $G$ to the multiplicative group of $L^{\times }$ :
     $$
     \chi: G \rightarrow L^{\times}
     $$
\end{definition}
\begin{definition}\
     The characters $\chi_1, \chi_2, \ldots, \chi_n$ of $G$ are said to be linearly independent over $L$ if they are linearly independent as functions on $G$, i.e., if there is no nontrivial relation
     $$
     a_1 \chi_1+a_2 \chi_2+\cdots+a_n \chi_n=0 \quad\left(a_1, \ldots, a_n \in L \text { not all } 0\right)
     $$
     as a function on $G$.
\end{definition}
\begin{theorem}\
     If $\chi_1, \chi_2, \ldots, \chi_n$ are distinct characters of $G$ with values in $L$ then they are linearly independent over $L$.

     Proof: 
     Suppose the characters were linearly dependent. Among all the linear dependence relations, choose one with the minimal number $m$ of nonzero 
     coefficients $a_i$.We may also suppose (by renumbering, if necessary) that the $m$ nonzero coefficients are $a_1, a_2, \ldots, a_m$ :
     $$
     a_1 \chi_1+a_2 \chi_2+\cdots+a_m \chi_m=0
     $$
     Then for any $g \in G$ we have
     $$
     a_1 \chi_1(g)+a_2 \chi_2(g)+\cdots+a_m \chi_m(g)=0
     $$
     Let $g_0$ be an element with $\chi_1\left(g_0\right) \neq \chi_m\left(g_0\right)$, we have
     $$
     a_1 \chi_1\left(g_0 g\right)+a_2 \chi_2\left(g_0 g\right)+\cdots+a_m \chi_m\left(g_0 g\right)=0
     $$
     i.e.,
     $$
     a_1 \chi_1\left(g_0\right) \chi_1(g)+a_2 \chi_2\left(g_0\right) \chi_2(g)+\cdots+a_m \chi_m\left(g_0\right) \chi_m(g)=0
     $$
     Multiplying equation (3) by $\chi_m\left(g_0\right)$ and subtracting from equation (4) we obtain
     $$
     \begin{aligned}
     {\left[\chi_m\left(g_0\right)-\chi_1\left(g_0\right)\right] a_1 \chi_1(g) } & +\left[\chi_m\left(g_0\right)-\chi_2\left(g_0\right)\right] a_2 \chi_2(g)+\cdots \\
     & +\left[\chi_m\left(g_0\right)-\chi_{m-1}\left(g_0\right)\right] a_{m-1} \chi_{m-1}(g)=0
     \end{aligned}
     $$
     which holds for all $g \in G$. But the first coefficient is nonzero and this is a relation with fewer nonzero coefficients, a contradiction.
\end{theorem}
\begin{corollary}\
     If $\sigma_1, \sigma_2, \ldots, \sigma_n$ are distinct embeddings of a field $K$ into a field $L$ be viewed as  characters of $K^{\times}$ with values in $L$, then they are linearly independent as functions on $K$. In particular distinct automorphisms of a field $K$ are linearly independent as functions on $K$.
\end{corollary}

\begin{theorem}\
     Let $G=\left\{\sigma_1=1, \sigma_2, \ldots, \sigma_n\right\}$ be a subgroup of $\mathrm{Aut}(K)$ and let $F$ be the fixed field. Then
     $$
     [K: F]=n=|G| 
     $$
     
     Proof: 
     Suppose first that $n>[K: F]=m$ and let $\omega_1, \omega_2, \ldots, \omega_m$ be a basis for $K$ over $F$. Then the system
     $$
     \begin{gathered}
     \sigma_1\left(\omega_1\right) x_1+\sigma_2\left(\omega_1\right) x_2+\cdots+\sigma_n\left(\omega_1\right) x_n=0 \\
     \vdots \\
     \sigma_1\left(\omega_m\right) x_1+\sigma_2\left(\omega_m\right) x_2+\cdots+\sigma_n\left(\omega_m\right) x_n=0
     \end{gathered}
     $$
     of $m$ equations has a nontrivial solution $\left(\beta_1, \beta_2, \ldots, \beta_n\right)$ in $K$.
     
     Let $a_1, a_2, \ldots, a_m$ be $m$ arbitrary elements of $F$.  Multiplying the first equation above by $a_1$, the second by $a_2, \ldots$, the last by $a_m$ then gives the system of equations
     $$
     \begin{gathered}
     \sigma_1\left(a_1 \omega_1\right) \beta_1+\sigma_2\left(a_1 \omega_1\right) \beta_2+\cdots+\sigma_n\left(a_1 \omega_1\right) \beta_n=0 \\
     \vdots \\
     \sigma_1\left(a_m \omega_m\right) \beta_1+\sigma_2\left(a_m \omega_m\right) \beta_2+\cdots+\sigma_n\left(a_m \omega_m\right) \beta_n=0
     \end{gathered}
     $$
     Since $\sigma_i(a_j)=a_j$.
     Adding these equations we see that
     $$
     \sigma_1\left(a_1 \omega_1+a_2 \omega_2+\cdots+a_m \omega_m\right) \beta_1+\cdots+\sigma_n\left(a_1 \omega_1+a_2 \omega_2+\cdots+a_m \omega_m\right) \beta_n=0
     $$
     for all choices of $a_1, \ldots, a_m$ in $F$. Since $\omega_1, \ldots, \omega_m$ is an $F$-basis for $K$ so the previous equation means
     $$
     \sigma_1(\alpha) \beta_1+\cdots+\sigma_n(\alpha) \beta_n=0
     $$
     for all $\alpha \in K$. But this means the distinct automorphisms $\sigma_1, \ldots, \sigma_n$ are linearly dependent over $K$, contradicting Corollary 8.
     
     We have proved $n \leq[K: F]$. 
     Suppose now that $n<[K: F]$. Then there are more than n $ F$-linearly independent elements of $K$, say $\alpha_1, \ldots, \alpha_{n+1}$. The system
     $$
     \begin{gathered}
     \sigma_1\left(\alpha_1\right) x_1+\sigma_1\left(\alpha_2\right) x_2+\cdots+\sigma_1\left(\alpha_{n+1}\right) x_{n+1}=0 \\
     \vdots \\
     \sigma_n\left(\alpha_1\right) x_1+\sigma_n\left(\alpha_2\right) x_2+\cdots+\sigma_n\left(\alpha_{n+1}\right) x_{n+1}=0
     \end{gathered}
     $$
     of $n$ equations in $n+1$ unknowns $x_1, \ldots, x_{n+1}$ has a solution $\beta_1, \ldots, \beta_{n+1}$ in $K$ where not all the $\beta_i, i=1,2, \ldots, n+1$ are 0 . If all the elements of the solution $\beta_1, \ldots, \beta_{n+1}$ were elements of $F$ then the first equation (recall $\sigma_1=1$ is the identity automorphism) would contradict the linear independence over $F$ of $\alpha_1, \alpha_2, \ldots, \alpha_{n+1}$. Hence at least one $\beta_i, i=1,2, \ldots, n+1$, is not an element of $F$.

     Among all the nontrivial solutions  $\left(\beta_1, \ldots, \beta_{n+1}\right)$  of the system (5) choose one with the minimal number $r$ of nonzero $\beta_i$. By renumbering if necessary we may assume $\beta_1, \ldots, \beta_r$ are nonzero. Dividing the equations by $\beta_r$ we may also assume $\beta_r=1$. We have already seen that at least one of $\beta_1, \ldots, \beta_{r-1}, 1$ is not an element of $F$ (which shows in particular that $r>1$ ), say $\beta_1 \notin F$. Then our system of equations reads
     $$
     \begin{gathered}
     \sigma_1\left(\alpha_1\right) \beta_1+\cdots+\sigma_1\left(\alpha_{r-1}\right) \beta_{r-1}+\sigma_1\left(\alpha_r\right)=0 \\
     \vdots \\
     \sigma_n\left(\alpha_1\right) \beta_1+\cdots+\sigma_n\left(\alpha_{r-1}\right) \beta_{r-1}+\sigma_n\left(\alpha_r\right)=0
     \end{gathered}
     $$
     
     Since $\beta_1 \notin F$, there is an automorphism $\sigma_{k_0}\left(k_0 \in\{1,2, \ldots, n\}\right)$ with $\sigma_{k_0} \beta_1 \neq \beta_1$. If we apply the automorphism $\sigma_{k_0}$ to the equations in (6), we obtain the system of equations
     $$
     \sigma_{k_0} \sigma_j\left(\alpha_1\right) \sigma_{k_0}\left(\beta_1\right)+\cdots+\sigma_{k_0} \sigma_j\left(\alpha_{r-1}\right) \sigma_{k_0}\left(\beta_{r-1}\right)+\sigma_{k_0} \sigma_j\left(\alpha_r\right)=0
     $$
     for $j=1,2, \ldots, n$. But the elements
     $$
     \sigma_{k_0} \sigma_1, \sigma_{k_0} \sigma_2, \ldots, \sigma_{k_0} \sigma_n
     $$
     are the same as the elements
     $$
     \sigma_1, \sigma_2, \ldots, \sigma_n
     $$
     in some order since these elements form a group. Hence the equations in (8) can be written
     $$
     \sigma_i\left(\alpha_1\right) \sigma_{k_0}\left(\beta_1\right)+\cdots+\sigma_i\left(\alpha_{r-1}\right) \sigma_{k_0}\left(\beta_{r-1}\right)+\sigma_i\left(\alpha_r\right)=0
     $$
     
     If we now subtract the equations in $\left(8^{\prime}\right)$ from those in (7) we obtain the system
     $$
     \sigma_i\left(\alpha_1\right)\left[\beta_1-\sigma_{k_0}\left(\beta_1\right)\right]+\cdots+\sigma_i\left(\alpha_{r-1}\right)\left[\beta_{r-1}-\sigma_{k_0}\left(\beta_{r-1}\right)\right]=0
     $$
     for $i=1,2, \ldots, n$. But this is a solution to the system of equations (5) with
     $$
     x_1=\beta_1-\sigma_{k_0}\left(\beta_1\right) \neq 0
     $$
     (by the choice of $k_0$ ), hence is nontrivial and has fewer than $r$ nonzero $x_i$. This is a contradiction and completes the proof.
\end{theorem}

\begin{corollary}\
     Let $K / F$ be any finite extension. Then
     $$
     |\operatorname{Aut}(K / F)| \leq[K: F]
     $$
     with equality if and only if $F$ is the fixed field of $\operatorname{Aut}(K / F)$. Put another way, $K / F$ is Galois if and only if $F$ is the fixed field of $\operatorname{Aut}(K / F)$.
\end{corollary}

