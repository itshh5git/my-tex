\chapter{Pseudo-Module}
\begin{theorem}
    If $R$ has an identity and $M$ is an $R$-module, then
    \begin{enumerate}
        \item
              there are submodules $M_1$ and $M_2$ of $M$ such that $M_1$ is unitary, $R M_2=0$ and $M=M_1 \oplus M_2$.

        \item
              Let $N$ be another $R$-module, with $N=N_1 \oplus N_1$ ($N_1$ unitary, $R N_2=0$).
              If $f: M \rightarrow N$ is an $R$-module homomorphism then
              $f(M_1) \subset N_1$ and $f(M_2) \subset N_2$.

        \item
              If the map $f$ of part 2. is an epimorphism [resp. isomorphism], then so are $\left.f\right|_{M_1}: M_1 \rightarrow N_1$ and $\left.f\right|_{M_2}: M_2 \rightarrow N_2$.
    \end{enumerate}
    \begin{proof}
        (1) Let
        \begin{equation*}
            M_1
            =
            \left\{1_R m : m \in M\right\}
        \end{equation*}
        and
        \begin{equation*}
            M_2
            =
            \left\{m \in M : 1_R m=0\right\}
        \end{equation*}
        and observe that for all $m \in M$, $m-1_R m \in M$.

        (2) It follows from $f\left(1_R m\right)=1_R f(m)$
    \end{proof}
\end{theorem}




\begin{theorem}
    Let $R$ be a ring, $M$ an $R$-module, $X$ a subset of $  M$, $\left\{ M_\alpha\right\}$ a family of submodules of $M$ and $ a \in  M$.
    \begin{enumerate}
        \item
              The module $Ra$ is a submodule of $M$ and the map $ {R}\rightarrow  {Ra}$ given by $  {r} \mapsto   {ra}$ is an $R$-module epimorphism.

        \item
              The cyclic submodule C generated by a is $Rx+\mathbb{Z}x$.
              If $R$ has an identity and $C$ is unitary, then $  C=  {Rx}$.

        \item The submodule D generated by X is

              \begin{equation*}
                  RX+\mathbb{Z}X
                  =
                  \left\{\sum_{i=1}^s   r_i   x_i+\sum_{j=1}^t   n_j  y_j :   {s},   {t} \in \mathbb{Z}_{\geq 1} ;   x_i,   y_j \in X ;  r_i \in  R;   n_j \in \mathbb{Z}\right\}
              \end{equation*}
              If $R$ has an identity and $M$ is unitary, then
              \begin{equation*}
                  D=R X=\left\{\sum_{i=1}^s r_i a_i : s \in\mathbb{Z}_{\geq 1} ; a_i \in X ; r_i \in R\right\}
              \end{equation*}

        \item
              If $\left\{M_\alpha\right\}$ is a family of submodules of a module $M$, then $\bigcap M_\alpha$ is a submodule of $M$.

        \item
              The submodule generated by the family $\left\{M_\alpha\right\}$ consists of all finite sums
              that is
              \begin{equation*}
                  \sum M_\alpha
                  =
                  \left\{x_{  i_1}+\cdots+  x_{i_n} :n \in \mathbb{Z}_{\geq 1} ; x_{i_k} \in M_{\alpha_k} \right\}
              \end{equation*}
    \end{enumerate}
\end{theorem}



\begin{proposition}
    Let $R$ be a ring. Let $\left\{X_i : i \in I\right\}$ be a collection of mutually disjoint sets and $F_i$ is a free module on $X_i$, with $\iota_i: X_i \rightarrow F_i$ for each $i \in I$.
    Let $X=\bigcup_{i \in I} X_i$ and $F=\sum_{i \in I}F_i$, with $\phi_i: F_i \rightarrow F$ the canonical injection. Define $\iota: X \rightarrow F$ by $\iota(x)=\phi_i \iota_i(x)$ for $x \in X_i$. Then $F$ is a free module on $X$

\end{proposition}

\begin{theorem}[Existence]
    Let $R$ be an arbitrary ring and $X$ be any set, then there exists a free $R$-module on $X$.
    \begin{proof}
        Since $X$ is the disjoint union of the sets $\left\{x\right\}$ with $x \in X$, it suffices
        to assume $X$ has only one element.

        If $R$ has an identity.
        Let the abelian group $\mathbb{Z}$ be given the trivial left $R$-module structure, so that $R \oplus \mathbb{Z}$ is an $R$-module with
        \begin{equation*}
            r\left(r^{\prime}, m\right)
            =
            \left(r r^{\prime}, 0\right)
        \end{equation*}
        for all $r, r^{\prime} \in R, n\in \mathbb{Z}$.
        Then $R \oplus \mathbb{Z}$ is a free module on $X$ with $\iota(x)=\left(1_R, 1\right)$.(For any function $f:X\rightarrow M$, $\bar{f}\left(\left(r,n\right)\right)=r f(x)$)

        If ${R}$ has no identity, embed $R$ in a ring $S$ with identity and characteristic $0$ where $S=R\times \mathbb{Z}$ and
        \begin{equation*}
            \left(r_1,n_1\right) +\left(r_2,n_2\right)
            =
            \left(r_1+r_2,n_1+n_2\right)
        \end{equation*}
        and
        \begin{equation*}
            \left(r_1,n_1\right) \left(r_1,n_1\right)
            =
            \left(r_1r_2+n_1r_2+n_2r_1,n_1n_2\right)
        \end{equation*}
        Then $S$ is a left $R$-module (Identify $R$ with its image in $S$.)
        \begin{equation*}
            r \cdot \left(r_1,n_2\right)
            =
            \left(r,0\right) \left(r_1,n_1\right)
            =
            \left(rr_1+n_1 r,0\right)
        \end{equation*}
        Now we show that $S$ is a free $R$-module on $X$ with $\iota\left(x\right)=\left(0,1\right)$. Given any $R$-module $M$ and function $f:X\rightarrow M$ (with $f(x)=m \in M$), if we have commutative diagram
        \begin{equation*}
            \begin{tikzcd}
                X =\left\{x\right\} \arrow[dd,"\iota"]\arrow[rr,"f"]& & M\\
                & & \\
                S \arrow[rruu,"\bar{f}"'] & &
            \end{tikzcd}
        \end{equation*}
        then
        \begin{equation*}
            \bar{f}\left(\left(0,1\right)\right)
            =
            f\left(t\right)
            =
            m
        \end{equation*}
        \begin{equation*}
            \bar{f}\left(\left(0,n\right)\right)
            =
            \bar{f}\left(n\left(0,1\right)\right)
            =
            n\bar{f}\left(\left(0,1\right)\right)
            =
            nm
        \end{equation*}
        \begin{equation*}
            \bar{f}\left(\left(r,0\right)\right)
            =
            \bar{f}\left(r\left(0,1\right)\right)
            =
            r\bar{f}\left(r\left(0,1\right)\right)
            =rm
        \end{equation*}
        thus there exists a unique $R$-module homomorphism $\bar{f}:S \rightarrow M$ that
        \begin{equation*}
            \bar{f}\left(\left(r,n\right)\right)
            =
            rm+nm
        \end{equation*}
        This implies $S$ is a free $R$-module on $X=\left\{x\right\}$.
    \end{proof}
\end{theorem}


\begin{theorem}
    Let $R$ be a ring and $X$ a nonempty set.
    \begin{enumerate}
        \item
              $\iota$ must be an injective.

        \item
              $\iota(X)$ is a set of generators of $F$. Furthermore, if $R$ has identity or an element that has cancelltion law in $R$, then $\iota(X)$ is a basis of $F$.

        \item
              The free $R$-module on $X$ is unique up to isomorphic.
    \end{enumerate}
    \begin{proof}
        1.
        Let $x_1$ be arbitrary element in $X$, consider $\mathbb{Z}$ is a trivial $R$-module.
        Define $f:X \rightarrow \tensor[_R]{\mathbb{Z}}{}$ be a function that
        \begin{equation*}
            f(x)=
            \begin{cases}
                1 , & x=x_1         \\
                0 , & \text{others}
            \end{cases}
        \end{equation*}
        Then there exists a $\bar{f} : F \rightarrow \mathbb{Z}$ with $\bar{f}\iota=f$, whence we have
        \begin{equation*}
            \bar{f}(\iota\left(x\right))
            =
            \begin{cases}
                1 , & x=x_1     \\
                0 , & x\neq x_1
            \end{cases}
        \end{equation*}
        thus $\iota(x_1)\neq \iota(x)$ for all $x\neq x_1$(if exists).

        2.
        $\iota(X)$ spans $F$.
        Consider the submodule $S$ generated by $\iota\left(X\right)$ and map
        \begin{equation*}
            f_1
            =
            i_1\iota:X\rightarrow \iota\left(X\right)\hookrightarrow S
        \end{equation*}
        and
        \begin{equation*}
            f_2
            =
            i_2\iota:X\rightarrow \iota\left(X\right)\hookrightarrow F
        \end{equation*}
        then there exists a $R$-module homomorphism
        \begin{center}
            $\bar{f}_1:F \rightarrow S$ such that $\bar{f}_1\iota=f_1$
        \end{center}
        and
        \begin{center}
            $\bar{f}_2:F \rightarrow F$ such that $\bar{f}_2\iota=f_2$
        \end{center}
        respectively. Noted that $\bar{f}_1$ can be extended to a homomorphism $\bar{f}_1:F\rightarrow F$ with $\Im \bar{f}_1=S$ and $\left.\bar{f}_1\right|_S = 1_S$, it is follows from $\iota(X) \subset$ spans $S$ that
        \begin{equation*}
            \bar{f_1}\iota =f_2
        \end{equation*}
        And since the $\bar{f}_2$ is the unique homomorphism from $F$ into $F$ that $\bar{f}_2 \iota = f_2$, we have
        \begin{equation*}
            \bar{f}_2=1_F, \quad \bar{f}_1=\bar{f}_2
        \end{equation*}
        Then $\Im \bar{f}_1=\langle\iota \left(X\right)\rangle=S=F$, that is, $\iota\left(X\right)$ spans $F$.

        Assume that $R$ has identity, then $\iota(X)$ is linear independently.
        Assume that there exists finite linear combination
        \begin{equation*}
            \sum_{k=1}^n r_k \iota\left(x_k\right) =
            0
        \end{equation*}
        where $r_k\in R$ and distinct $x_k\in X$.
        For each $1\leq j \leq n$, define function $f_j:X \rightarrow \mathbb{R}$ that
        \begin{equation*}
            f_j(x_k)=\delta_{jk}
        \end{equation*}
        then there exists $\bar{f}_j$ such that
        \begin{equation*}
            \bar{f}_j(\iota(x_k))=\delta_{jk}
        \end{equation*}
        ($f_j$ and $\bar{f_j}$ vanishe on $\left\{x_j\right\}$ and $\left\{\iota(x_j)\right\}$ respectively)
        Therefore for all $1\leq k \leq n$
        \begin{equation*}
            r_k
            =
            \bar{f}_k\left(\sum_{k=1}^n r_k \iota\left(x_k\right)\right)
            =0
        \end{equation*}
        This implies that $\iota\left(X\right)$ is linear independently.

        3.
        If $F_1,F_2$
        are two free $R$-module on $X$, then there exists $R$ module $\bar{\iota}_1:F_2 \rightarrow F_1$ and $\bar{\iota}_2:F_1 \rightarrow F_2$ such that $\bar{\iota}_2 \iota_1 =\iota_2$ and $\bar{\iota}_1 \iota_2 =\iota_1$
        \begin{equation*}
            \begin{tikzcd}
                X \arrow[dd,"\iota_1"]\arrow[rr,"\iota_2"]& &
                F_2 \\
                & & \\
                F_1 \arrow[rruu,"\bar{\iota}_2"'] & &
            \end{tikzcd}
        \end{equation*}
        Then we have $\bar{\iota}_2 \bar{\iota}_1\iota_2=\iota_2$, whence $\bar{\iota}_2 \bar{\iota}_1=1_{F_2} $ ($\iota_2(X)$ spans $F_2$). Similarly, we have $\bar{\iota}_1 \bar{\iota}_2=1_{F_1}$ and $F_1$ is isomorphic to $F_2$.
    \end{proof}
\end{theorem}





\begin{corollary}
    \label{cor: Every left module $M$ over a ring $R$ is the homomorphic image of a free $R$-module $F$.}
    Every left module $M$ over a ring $R$ is the homomorphic image of a free $R$-module $F$.
    If $M$ is finitely generated, then $F$ may be chosen to be finitely generated.
\end{corollary}


\subsection{Injective modules} % Injective modules

\begin{definition}
    A module $J$ over a ring $R$ is said to be \textbf{injective} if given any diagram of $R$-module homomorphisms with top row exact, there exists an $R$-module homomorphism ${h}: B \rightarrow {J}$ such that the diagram
    \begin{equation*}
        \begin{tikzcd}
            0 \arrow[r] & A \arrow[dd,"f"']\arrow[r,"g"] & B \arrow[ldd,dashed, "\exists \tilde{f}"]  \\
            & &  \\
            & J &
        \end{tikzcd}
    \end{equation*}
    is commutative.
\end{definition}

\begin{proposition}
    A direct product of $R$-modules $\prod_{i \in I} J_{i}$ is injective if and only if $J_{i}$ is injective for every $i \in I$.
\end{proposition}

\begin{theorem}
    Let $R$ be a ring with identity. A unitary $R$-module $J$ is injective if and only if for every left ideal $\mathfrak{a}$ of $R$, any $R$-module homomorphism $\mathfrak{a} \rightarrow J$ may be extended to an $R$-module homomorphism $R \rightarrow J$.
\end{theorem}

\begin{definition}
    An abelian group $D$ is said to be \textbf{divisible} if given any $y \in D$ and $0 \neq n \in \mathbb{Z}$, there exists $x \in D$ such that $n x=y$.
\end{definition}

\begin{proposition}
    \label{pro: Proposition of divisible group}
    We have the following propositions
    \begin{enumerate}
        \item
              An abelian group $D$ is divisible if and only if $D$ is injective as an unitary $\mathbb{Z}$-module.
        \item
              A direct sum of abelian groups is divisible if and only if each summand is divisible

        \item
              The homomorphic image of a divisible group is divisible.
    \end{enumerate}
    \begin{proof}
        \begin{equation*}
            \begin{tikzcd}
                0 \arrow[r,] & \left\langle n  \right\rangle \arrow[d,"f"] \arrow[r,"\id"] & \mathbb{Z} \arrow[ld,dashed,"h"] \\
                & D &
            \end{tikzcd}
        \end{equation*}
    \end{proof}
\end{proposition}


\begin{theorem}
    \label{thm: divisible group and injective module}

    \begin{enumerate}
        \item
              Every abelian group $G$ may be embedded in a divisible abelian group.
        \item
              If $G$ is a divisible abelian group and $R$ is a ring with identity, then  $\Hom_{\mathbb{Z}}(R, G)$ is an injective left $R$-module.
    \end{enumerate}
    \begin{proof}
        1.
        By
        there is a free $\mathbb{Z}$-module $F$ and an epimorphism $F \rightarrow G$ with kernel $K$ so that $F / K \cong G$.
        Since $F$ is a direct sum of copies of $\mathbb{Z}$ by \ref{thm: Free unitary R-module} and $\mathbb{Z} \subset \mathbb{Q}, F$ may be embedded in a direct sum $D$ of copies of the rationals $\mathbb{Q}$.

        If $f: F \rightarrow D$ is the embedding monomorphism, then $f$ induces an isomorphism $F / K \cong f(F) / f(K)$ by Corollary I.5.8. Thus the composition $G \cong F / K \cong f(F) / f(K) \subset D / f(K)$ is a monomorphism. But $D / f(K)$ is divisible by \ref{pro: Proposition of divisible group} since it is the image of canonical maps $D \rightarrow D / f(K)$ and $D$ is divisible.

        2.
        By Lemma 3.8 it suffices to show that for each left ideal $L$ of $R$, every $R$-module homomorphism $f: L \rightarrow \operatorname{Hom}_{\mathbb{Z}}(R, J)$ may be extended to an $R$-module homomorphism $h: R \rightarrow \operatorname{Hom}_{\mathbb{Z}}(R, J)$. The map $g: L \rightarrow J$ given by $g(a)=[f(a)]\left(1_R\right)$ is a group homomorphism. Since $J$ is an injective $\mathbb{Z}$-module by \ref{pro: Proposition of divisible group} and we have the diagram
        \begin{equation*}
            \begin{tikzcd}
                0\arrow[r]& L\arrow[dd,"g"] \arrow[r,] & R  \arrow[ldd,dashed,"\bar{g}"]\\
                & & \\
                & J&
            \end{tikzcd}
        \end{equation*}
        there is a group homomorphism $\bar{g}: R \rightarrow J$ such that $\left.\bar{g}\right|_L =g$.

        Define $h: R \rightarrow \operatorname{Hom}_{\mathbb{Z}}(R, J)$ by $r \mapsto h(r)$, where $h(r): R \rightarrow J$ is the map given by $[h(r)](x)=\bar{g}(x r)$
        ( $x \in R$ ).
        Verify that $h$ is a well-defined function (that is, each $h(r)$ is a group homomorphism $R \rightarrow J$ ) and that $h$ is a group homomorphism $R \rightarrow \operatorname{Hom}_{\mathbf{z}}(R, J)$. If $s, r, x \in R$, then
        \begin{equation*}
            h(s r)(x)=\bar{g}(x(s r))=\bar{g}((x s) r)=h(r)(x s) .
        \end{equation*}

        By the definition of the $R$-module structure of $\operatorname{Hom}_{\mathbf{z}}(R, J), h(r)(x s)=[s h(r)](x)$, whence $h(s r)=s h(r)$ and $h$ is an $R$-module homomorphism. Finally suppose $r \in L$ and $x \in R$. Then $x r \in L$ and
        \begin{equation*}
            h(r)(x)=\bar{g}(x r)=g(x r)=[f(x r)]\left(1_R\right) .
        \end{equation*}

        Since $f$ is an $R$-module homomorphism and $\operatorname{Hom}_{\mathbf{z}}(R, J)$ an $R$-module,
        \begin{equation*}
            [f(x r)]\left(1_R\right)=[x f(r)]\left(1_R\right)=f(r)\left(1_R x\right)=f(r)(x) .
        \end{equation*}

        Therefore, $h(r)=f(r)$ for $r \in L$ and $h$ is an extension of $f$.

    \end{proof}
\end{theorem}




\begin{proposition}
    Every unitary module $M$ over a ring $R$ with identity may be embedded in an injective $R$-module.
    \begin{proof}
        Since $M$ is an abelian group, there is a divisible group $G$ and a group monomorphism $f: M \rightarrow J$ by \ref{thm: divisible group and injective module}.
        The map $\bar{f}: \operatorname{Hom}_{\mathbb{Z}}(R, M) \rightarrow \operatorname{Hom}_{\mathbb{Z}}(R, G)$ given on $g \in \operatorname{Hom}_{\mathbb{Z}}(R, M)$ by $\bar{f}(g)=f g \in \operatorname{Hom}_{\mathbb{Z}}(R, G)$ is easily seen to be an $R$-module monomorphism. Since every $R$-module homomorphism is a $\mathbb{Z}$-module homomorphism, we have $\operatorname{Hom}_R(R, M) \subset \operatorname{Hom}_{\mathbb{Z}}(R, M)$. In fact, it is easy to see that $\operatorname{Hom}_R(R, M)$ is an $R$-submodule of $\operatorname{Hom}_{\mathbb{Z}}(R, M)$. Finally, verify that the map $M \rightarrow \operatorname{Hom}_R(R, M)$ given by $a \mapsto f_a$, where $f_a(r)=r a$, is an $R$-module monomorphism (in fact it is an isomorphism). Composing these maps yields an $R$-module monomorphism
        \begin{equation*}
            M \rightarrow \operatorname{Hom}_R(R, M) \xrightarrow{i} \operatorname{Hom}_{\mathbb{Z}}(R, M) \xrightarrow{\bar{f}} \operatorname{Hom}_{\mathbb{Z}}(R, G) .
        \end{equation*}
        Since $\operatorname{Hom}_{\mathbb{Z}}(R, G)$ is an injective $R$-module by \ref{thm: divisible group and injective module}, we have embedded $M$ in an injective.
    \end{proof}
\end{proposition}


\begin{theorem}
    Let $R$ be a ring with identity.
    The following conditions on a unitary $R$-module $J$ are equivalent.
    \begin{enumerate}
        \item
              $J$ is injective;

        \item
              every short exact sequence $0 \rightarrow {~J} \xrightarrow{f} B \xrightarrow{g} C \rightarrow 0$ is split;

        \item
              $J$ is a direct summand of any module $B$ of which it is a submodule.
    \end{enumerate}
\end{theorem}


\begin{theorem}
    Let $K$ be a commutative ring with identity and $A$ a unitary left $K$-module.
    \begin{enumerate}
        \item
              Then $A$ is a $K$-algebra if and only if there exists a $K$-module homomorphism
              \begin{equation*}
                  \pi: A \otimes_{{K}} A \rightarrow A
              \end{equation*}
              such that the diagram
              \begin{equation*}
                  \begin{tikzcd}
                      A \otimes_K A \otimes_K A \arrow[rr,"\pi \otimes 1_A"] \arrow[dd,"1_A\otimes \pi "]& & A\otimes_K A \arrow[dd,"\pi"] \\
                      & &   \\
                      A\otimes_K A   \arrow[rr,"\pi"]& & A
                  \end{tikzcd}
              \end{equation*}
              is commutative.
              The homomorphism $\pi$ is called the \textbf{product map} of the $K$-algebra $A$ and the product in $A$ is defined by $a \cdot b=\pi\left(a\otimes b\right)$

        \item
              In this case the $K$-algebra $A$ has an identity if and only if there is a $K$-module homomorphism $I: K \rightarrow A$ such that the diagram
              \begin{equation*}
                  \begin{tikzcd}
                      K\otimes_K A \arrow[r,] \arrow[dd,] \arrow[dd,"i \otimes 1_A"] & A \arrow[dd,"1_A"] & A\otimes_K K \arrow[l,] \arrow[dd,"1_A\otimes i"]\\
                      & &\\
                      A\otimes_K A \arrow[r,"\pi"]& A &  A\otimes_K A \arrow[l,"\pi"']
                  \end{tikzcd}
              \end{equation*}
              is commutative, where $\zeta, \theta$ are the isomorphisms of Theorem 5.7.
              The homomorphism $i$ is called the \textbf{unit map}.
    \end{enumerate}
    \begin{proof}
        (1) If $A$ is a $K$-algebra, then the map $A \times A \rightarrow A$ given by $(a, b) \mapsto a b$ is $K$-bilinear, whence there is a $K$-module homomorphism
        \begin{equation*}
            \pi: A \otimes_K A \rightarrow A
        \end{equation*}
        Verify that $\pi$ has the required properties.Conversely, given $A$ and the map $\pi: A \otimes_K A \rightarrow A$, define $a b=\pi(a \otimes b)$ and verify that $A$ is a $K$-algebra.

        (2) If $A$ has an identity $1_A$, then the included map $i: K \rightarrow A$ given by $k \mapsto k 1_A$ is easily seen to be a $K$-module homomorphism with the required properties. If $i: K \rightarrow A$ is also given, then $i\left(1_K\right)$ is an identity for $A$.
    \end{proof}
\end{theorem}