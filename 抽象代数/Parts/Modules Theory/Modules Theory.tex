\chapter{Modules}
\minitoc

\section{Basic Definition} % Basic Definition
\begin{definition}
    Let $R$ be a ring.
    A \textbf{left $R$-module} is an additive abelian group $M$ together with a function $R \times M \rightarrow M$  such that for all $ {r},  {s} \in  {R}$ and $ {a},  {b} \in  M$ :
    \begin{enumerate}[label=(\roman*)]
        \item
              $r(a+b)=r a+r b$.
        \item
              $( {r}+ {s})  {a}= {ra}+ {sa}$
        \item
              $ {r}( {sa})=( {rs})  {a}$
        \item
              $1_{{R}}  {a}= {a}$ for all $ {a} \in  M$
    \end{enumerate}
    then $M$ is said to be a \textbf{$R$-module}. If $R$ is a division ring, then a $R$-module is called a (left) \textbf{vector space}.
    \begin{remark}
        If $R$ has no identity or (iv) fails, we call $M$ \textbf{non-unital module} or \textbf{pseudo-module}.
        In the vast majority of cases, the objects we study are modules.
    \end{remark}
\end{definition}

\begin{definition}
    Let $R$ be a ring, $M$ an $R$-module and $N$ a nonempty subset of $M$. Then $N$ is a \textbf{submodule} of $M$ provided that $N<M$ and $  {rb} \in N$ for all $r \in R$, $  {b} \in   N$.
    A submodule of a vector space over a division ring is called a \textbf{subspace}.
\end{definition}

\begin{proposition}
    Let $M$ be an $R$-module.
    \begin{enumerate}
        \item
              If $N$ is a nonempty subset of $M$, then $N$ is a submodule of $M$ if and only if for all $x, y \in N$ and $r \in R$, $x-y \in N$ and $r x \in N$
        \item
              If $\left\{N_\alpha\right\}$ is a chain of submodules, then so $\bigcup N_\alpha$.
        \item
              If $\left\{N_\alpha\right\}$ is a family of submodules, then so $\bigcap N_\alpha$.
    \end{enumerate}
\end{proposition}















\subsection{Submodule generated by sets}
\begin{definition}
    Let $M$ be a $R$-module. \begin{enumerate}
        \item
              If $X$ is a subset of $M$, then the intersection of all submodules of $M$ containing $X$ is called the \textbf{submodule generated by ${X}$} (or spanned by ${X}$ ).

        \item
              If $X$ is finite, and $X$ generates the submodule $N$, then $N$ is said to be \textbf{finitely generated} and $X$ spans $N$.
              If $X=\left\{a\right\}$, then the submodule generated by $X$ is called the \textbf{cyclic module generated by $a$}.

        \item if $\left\{B_i : i \in I\right\}$ is a family of submodules of $M$, then the submodule generated by $X=\bigcup_{i \in I} B_i$ is called the \textbf{sum of the modules $B_i$}, denoted by $\sum_{i \in I} B_i$.
        \item
              If $I$ is a left ideal of $R$ and $S$ is a nonempty subset of $M$. Then
              \begin{equation*}
                  I S=\left\{\sum_{i=1}^n r_i a_i : r_i \in I ; a_i \in S ; n \in \mathbb{N}^*\right\}
              \end{equation*}
              is a submodule of $M$.
              Similarly if $a \in M$, then $I a=\{r a \mid r \in I\}$ is a submodule of $M$.
    \end{enumerate}
\end{definition}

\begin{theorem}
    Let $R$ be a ring, $M$ an $R$-module, $X$ a subset of $  M$, $\left\{ M_\alpha\right\}$ a family of submodules of $M$.
    \begin{enumerate}
        \item
              The submodule generated by $X$ is
              \begin{equation*}
                  R X
                  =
                  \left\{\sum_{i=1}^s r_i a_i : s \in\mathbb{Z}_{\geq 1} ; a_i \in X ; r_i \in R\right\}
              \end{equation*}
        \item
              The submodule generated by the family $\left\{M_\alpha\right\}$ consists of all finite sums
              that is
              \begin{equation*}
                  \sum M_\alpha
                  =
                  \left\{x_{  i_1}+\cdots+  x_{i_n} :n \in \mathbb{Z}_{\geq 1} ; x_{i_k} \in M_{\alpha_k} \right\}
              \end{equation*}
    \end{enumerate}
\end{theorem}



\subsection{Quotient module and homomorphism} % Quotient module and homomorphism


\begin{definition}
    Let $M$ and $N$ be modules over a ring $R$.
    \begin{enumerate}
        \item
              A function $  {f}:   M \rightarrow   N$ is an \textbf{${R}$-module homomorphism} provided that for all $ x,y \in  M$ and $  {r} \in  {R}$ :
              \begin{equation*}
                  f(x+y)=
                  f(x)+  {f}(y)
                  \text { and }
                  {f}(  {rx})=  {rf}(  {x})
              \end{equation*}
              If $R$ is a division ring, then an $R$-module homomorphism is called a \textbf{linear transformation}.
        \item Four submodules
              $\Ker f=f^{-1}\left(0\right)$, $\Im(F)=f\left(M\right),\Coker\left( f \right)=N/\Im(f),\Coim\left( f \right)=M/\Ker\left( f \right)$
        \item
              define $\Hom_R(M, N)$ to be the set of all $R$-module homomorphisms from $M$ into $N$, which forms a $R$-module.
    \end{enumerate}
\end{definition}
\begin{proposition}
    The following conditions of module-homomorphism $f:M\rightarrow N$ are equivalent
    \begin{enumerate}
        \item $f$ is injective
        \item $\Ker\left( f \right)=0$
        \item $f$ is monomorphism ($fg=fh\Rightarrow g=h$)
        \item $fg=0\Rightarrow g=0$
    \end{enumerate}
\end{proposition}



\begin{definition}
    Let $N$ be a submodule of a module $M$ over a ring $R$.
    Then the quotient group $ M / N$ is an $R$-module with the action of $R$ on $ M / N$ given by:
    \begin{equation*}
        {r}(  x+ N )
        =
        rx + N \quad \text { for all }   {r} \in   {R},   {a} \in   M
    \end{equation*}
    called quotient module.
    The map $\pi:   M \rightarrow   M /  N$ given by $ x \mapsto   x+  N$ is an $R$-module epimorphism with kernel $N$.
    The map $\pi$ is called the \textbf{canonical epimorphism (or projection)}.
\end{definition}


\begin{theorem}
    Let $R$ be a ring, $f: M \rightarrow N$ an $R$-module homomorphism and $L$ is a submodule of $\Ker f$, then
    \begin{enumerate}
        \item
              There is a unique $R$-module homomorphism
              \begin{equation*}
                  \overline{{f}}: M / L \rightarrow N \quad  \text{such that }\overline{{f}}(x + L)= f(x) \text{ for all } x \in M
              \end{equation*}
              with $\Im \overline{{f}}=\Im {f}$ and $\Ker \overline{{f}}=\Ker {f} / L$.
        \item  $ \overline{{f}}$ is an $R$-module isomorphism if and only if $f$ is an $R$-module epimorphism and ${C}=\Ker f$.
              In particular, $M / \Ker {f} \cong \Im {f}$.
    \end{enumerate}
\end{theorem}


\begin{corollary}
    If $R$ is a ring and $M^{\prime}$ is a submodule of the $R$-module $M$ and $N^{\prime}$ a submodule of the $R$-module $N$ and ${f}: M \rightarrow {N}$ is an $R$-module homomorphism such that ${f}\left(A^{\prime}\right) \subset B^{\prime}$, then

    (1) $f$ induces an $R$-module homomorphism $\overline{{f}}: A / A^{\prime} \rightarrow B / B^{\prime}$ given by ${a}+A^{\prime} \mapsto {f}({a})+{B}^{\prime} $.

    $ \overline{{f}}$ is an R -module isomorphism if and only if $\Im {f}+{B}^{\prime}={B}$ and ${f}^{-1}\left({~B}^{\prime}\right) \subset A^{\prime}$.

    In particular if f is an epimorphism such that ${f}\left(A^{\prime}\right)={B}^{\prime}$ and $\Ker {f} \subset A^{\prime}$, then $\overline{{f}}$ is an R-module isomorphism.
\end{corollary}











\begin{theorem}
    Let B and C be submodules of a module A over a ring R .

    (1) There is an R -module isomorphism ${B} /({B} \cap C) \cong({B}+{C}) / C$;

    (2) if ${C} \subset B$, then ${B} / C$ is a submodule of $A / C$, and there is an R -module isomorphism $(A / C) /({B} / C) \cong A / B$.
\end{theorem}



\begin{theorem}
    \label{thm: The Fourth or Lattice Isomorphism Theorem of Modules}
    Let $R$ be a ring and $N$ is a submodule of an $R$-module $M$, then
    \begin{enumerate}
        \item
              There is a one-to-one correspondence between the set of all submodules of $M$ containing $N$ and the set of all submodules of $M/N$, given by $L \mapsto L / N$.
        \item
              Hence every submodule of $M / N$ is of the form $ L / N$, where $L$ is a submodule of $M$ which contains $N$.
    \end{enumerate}
\end{theorem}


\subsection{Annihilator} % Annihilator
\begin{definition}
    Let $R$ be a ring and $M$ an left $R$-module. If $X$ be a subset of $M$
    \begin{enumerate}
        \item The left ideal
              \begin{equation*}
                  \Ann\left(X\right) =\left\{ r \in R \mid rX = 0\right\}
              \end{equation*}
              is called the \textbf{annihilator} of $X$ in $R$.
        \item
              If $\Ann\left( u \right)\neq 0$, $u$ is called to be \textbf{torsion element}.

              If $\Ann\left( u \right)= 0$, $u$ is called to be \textbf{free-torsion element}.
        \item
              If $\Ann\left(u\right) = 0$ for all $u\in M$, then $M$ is called to be a \textbf{torsion-free} module.

              If $\Ann\left(u\right) \neq 0$ for all $u\in M$, then  $M$ is called to be a \textbf{torsion module}.
        \item
              If $R$ is a integral domain, the set $T(M)$ consists of all torsion element is called the \textbf{torsion submodule} of $M$
    \end{enumerate}
\end{definition}

\begin{proposition}
    Let $N,N_1,N_2$ be submodule of $M$.
    \begin{enumerate}
        \item
              $\Ann\left( N_1+N_2 \right)=\Ann\left( N_1 \right)+\Ann\left( N_2 \right)$
        \item
              $\Ann\left( M/N \right)=\left(N:M\right)$
        \item
              $\left(N_1:N_2\right)=\left(N_1:\left(N_1+N_2\right)\right)=\Ann\left( \left(N_1+N_2\right)/N_1 \right)$
    \end{enumerate}
\end{proposition}




\begin{definition}
    The module $M$ is said to be a \textbf{faithful} $R$-module if $\Ann(M)=0$.
    \begin{remark}
        It's obvious that $M$ is a faithful $R/\Ann\left( M \right)$-module
    \end{remark}
\end{definition}

















\section{Modules Category} % Rings Category
We only consider left module in this section.

\subsection{Direct Products and Direct Sums}
% Direct Productsand Direct Sums

\begin{definition}
    Let $R$ be a ring and $\left\{  M_i :  i \in   I\right\}$ a nonempty family of left $R$-modules, $\prod_{i \in I}   M_i$ the direct product of the abelian groups $  M_i$, and $\bigoplus_{i \in I}   M_i$ the direct sum of the abelian groups ${M}_i$.
    \begin{enumerate}
        \item
              $\prod_{i \in I}   M_i$ is an left $R$-module with the action of $R$ given by $  {r}\left\{  {a}_i\right\}=\left\{  {ra}_i\right\}$.

        \item
              $\bigoplus_{i \in I}  M_i$ is a submodule of $\prod_{i \in I}   M_i$.

        \item
              For each $ k \in   I$, the canonical projection $\pi_{ k}: \prod   M_i \rightarrow   M_{ k}$ is an left $R$-module epimorphism.

        \item
              For each $ k \in   I$, the canonical injection $\iota_{ k}:   M_{ k} \rightarrow \bigoplus   M_i$  is an left $R$-module monomorphism.
    \end{enumerate}
    The ring $\prod_{i \in I} M_i$ is called the \textbf{(external) direct product} of the family of $R$-modules $\left\{M_i \mid i \in I\right\}$ and $\bigoplus_{i \in I} M_i$ is its \textbf{(external) direct sum}.
    The maps $\pi_k$ are called the \textbf{canonical projections} ang $\iota_k$ are called \textbf{canonical injections}.
\end{definition}


\begin{theorem}
    Let $R$ be a ring.
    \begin{enumerate}
        \item
              If $\left\{M_i \mid i \in I\right\}$ a family of $R$-modules, $M$ an $R$-module, and
              $\left\{f_i: M \rightarrow M_i \mid i \in I\right\}$
              a family of $R$-module homomorphisms, then there is a unique $R$-module homomorphism
              $f: M \rightarrow \prod_{i \in I} M_i$
              such that the diagram
              \begin{equation*}
                  \begin{tikzcd}
                      M \arrow[rr,"\exists!f"]\arrow[dd,"f_i" ]& &\prod\limits_{i \in I} M_i \\
                      & & \\
                      M_i \arrow[rruu,"\pi_i"]& &
                  \end{tikzcd}
              \end{equation*}
              is commutative for all ${i} \in I$.
              $ \prod_{i \in I} M_i$ is uniquely determined up to isomorphism by this property. In other words, $\prod_{i \in I} M_i$ is a product in the category of left $R$-modules.
        \item
              If $\left\{N_i\right\}_{i\in I}$ is a family of $R$-modules, $N$ an $R$-module, and
              $g_i\in \Hom_R\left(N_i,N\right)$ $i\in I$, then there is a unique $R$-module homomorphism $g: \oplus_{i \in I} N_i \rightarrow N$ such that the diagram
              \begin{equation*}
                  \begin{tikzcd}
                      \bigoplus\limits_{i\in I}N_i \arrow[rr,"\exists!g"] \arrow[dd,"\iota_i"]& & N \\
                      & &\\
                      N_i \arrow[rruu,"g_i"]& &
                  \end{tikzcd}
              \end{equation*}
              is commutative for all ${i} \in I$.
              $ \bigoplus_{i \in I} N_i$ is uniquely determined up to isomorphism by this property. In other words, $\bigoplus_{i \in I} N_i$ is a coproduct in the category of $R$-modules.
    \end{enumerate}
    \begin{remark}
        It follows that
        \begin{equation*}
            \prod \Hom_R\left(A,A_i\right) \cong \Hom_{R}\left(A,\prod_{}A_i\right)
        \end{equation*}
        \begin{equation*}
            \prod \Hom_R\left(A_i,A\right) \cong \Hom_{R}\left(\bigoplus_{}A_i,A\right)
        \end{equation*}
    \end{remark}
\end{theorem}

\begin{theorem}
    Let $R$ be a ring and $  M,   M_1,   M_2, \ldots,   M_{  {n}},  {R}$-modules. Then $  M \cong   M_1 \oplus M_2 \oplus \cdots \oplus   M_n$
    if and only if
    for each $  i=1,2, \ldots,   {n}$ there are $R$-module homomorphisms $\pi_i:   M \rightarrow   M_i$ and $\iota_i:   M_i \rightarrow   M$ such that
    \begin{enumerate}[label=(\roman*)]
        \item $\pi_i \iota_i=1_{M_i}$
              for $i=1,2, \ldots, n$
        \item $\pi_j \iota_i=0$ for $ i \neq j$
        \item $\iota_1 \pi_1+\iota_2 \pi_2+\cdots+\iota_{  {n}} \pi_{  {n}}=1_{  M}$.
    \end{enumerate}
\end{theorem}


\begin{theorem}
    Let $R$ be a ring and $\left\{  M_i : i \in   I\right\}$ a family of submodules of an $R$-module $M$ such that
    \begin{enumerate}[label=(\roman*)]
        \item
              $M=\sum_i M_i$

        \item
              for each $ k \in I$, $M_k\cap \sum_{i\neq k} M_i
                  =
                  0$
    \end{enumerate}
    Then there is an isomorphism $  M \cong \bigoplus_{i \in I}  M_i$.
    The module $M$ is said to be the \textbf{internal direct sum} of $\left\{M_i : i \in I\right\}$.
\end{theorem}
\begin{corollary}
    Let $R$ be a ring and $\left\{  R_i : i \in  I\right\}$ a family of subrings of an $R$ such that
    \begin{enumerate}[label=(\roman*)]
        \item
              $R=\sum_i R_i$

        \item
              for each $ k \in I$, $R_k\cap \sum_{i\neq k} R_i
                  =
                  0$
    \end{enumerate}
    Then there is an isomorphism $  M \cong \bigoplus_{i \in I}  R_i$.
\end{corollary}


\subsection{Free Modules} % Free Modules
\begin{definition}
    Let $R$ be a ring and $M$ an left $R$-module
    \begin{enumerate}
        \item
              A subset $X$ of $M$ is said to be \textbf{linearly independent} provided that for some finite distinct $x_1, \ldots, x_n \in X$ and $r_i \in R$.
              \begin{equation*}
                  r_1 x_1+r_2 x_2+\cdots+r_n x_n=0 \Rightarrow r_i=0 \text { for every } i
              \end{equation*}
              Conversely, a set that is not linearly independent is said to be \textbf{R-linearly dependent}.
        \item
              The $\left\{u_i\right\}_{i\in I}$ is called to be a \textbf{maximal linearly independent subset} of $M$ provided that it is not contained in any larger linearly independent subset of $M$.
        \item
              A linearly independent subset that spans $M$ is called a \textbf{(Hamel) basis} of $M$.
    \end{enumerate}
    \begin{remark}
        A basis must be a maximal linearly independent subset.
    \end{remark}
\end{definition}


\begin{definition}
    Let $R$ be a ring and $X$ a nonempty set.
    An left $R$-module $F$ is called a \textbf{free module} on ${X}$ if $F$ is a free object on $X$ in the category of all left $R$-modules, i.e.  there is a function $\iota: X \rightarrow F$ such that the diagram
    \begin{equation*}
        \begin{tikzcd}
            X \arrow[r, "\iota"] \arrow[dr, "f"'] & F(X) \arrow[d, "\exists ! \tilde{f}", dashed] \\
            & M
        \end{tikzcd}
    \end{equation*}
    is commutative for any left $R$-module $M$ and function $f: X \rightarrow M$ there is a unique $R$-module homomorphism $\tilde{f}: F \left(X\right)\rightarrow M$ with $\tilde{f}\iota=f$.
\end{definition}


\begin{theorem}
    \label{thm: Free unitary R-module}
    Let $R$ be a ring.
    The following conditions on a $R$-module $F$ are equivalent:
    \begin{enumerate}
        \item
              $F$ has a nonempty basis;

        \item
              $F$ is the internal direct sum of a family of cyclic $R$-modules, each of which is isomorphic as a left $R$-module to $R$,

        \item
              $F$ is $R$-module isomorphic to a direct sum of copies of the left $R$-module $R$
    \end{enumerate}
\end{theorem}


\begin{corollary}
    \label{cor: Every unitary module $M$ over a ring $R$ with identity is the homomorphic image of a free  unitary $R$-module $F$.}.

    \begin{enumerate}
        \item
              Every module $M$ over a ring $R$ is the homomorphic image of a free $R$-module $F$.
        \item
              $M$ is a finitely generated $\Leftrightarrow M$ is isomorphic to a quotient of free module $R^n$ for some integer $n>0$.
    \end{enumerate}
\end{corollary}


\subsubsection{Dimension and invariant dimension property} % Dimension and invariant dimension property


\begin{definition}
    Let $R$ be a ring.
    If any two bases of free $R$-module $F$ have the same cardinality, then $R$ is said to have the \textbf{invariant dimension property} and the cardinal number of any basis of $F$ is called the \textbf{dimension} of \textbf{rank} of $F$.
\end{definition}



\subsubsection{Proof of invariant dimension property}
\begin{theorem}
    Let $R$ be a ring and $F$ a free $R$-module with an infinite basis $X$.
    Then every basis of $F$ has the same cardinality as $X$.
    \begin{proof}
        Step 1.
        If $Y$ is another basis of $F$, then we claim that $Y$ is infinite. Suppose on the contrary that $Y$ were finite. Since $Y$ generates $F$ and every element of $Y$ is a linear combination of a finite number of elements of $X$, it follows that there is a finite subset
        $\left\{x_1, \ldots, x_m\right\}$ of $X$, which generates $Y$, thus generates $F$.
        Since $X$ is infinite, there exists
        \begin{equation*}
            x \in X-\left\{x_1, \ldots, x_m\right\}
        \end{equation*}
        Then for some $r_i \in R, x=r_1 x_1+\cdots+r_m x_m$, which contradicts the linear independence of $X$. Therefore, $Y$ is infinite.

        Step 2.
        Let $K(Y)$ be the set of all finite subsets of $Y$. Define a map
        \begin{equation*}
            f: X \rightarrow K(Y)
        \end{equation*}
        by
        \begin{equation*}
            x \mapsto\left\{y_1, \ldots, y_n\right\}
        \end{equation*}
        where $x=r_1 y_1+\cdots+r_n y_n$ and $r_i \neq 0$ for all $i$.
        Since $Y$ is a basis, the $y_i$ are uniquely determined and $f$ is a well-defined function.

        Step 3.
        If $\Im f$ were finite, then
        \begin{equation*}
            \bigcup_{S \in \Im f} S
        \end{equation*}
        would be a finite subset of $Y$ that would generate $X$ and hence $F$. This leads to a contradiction of the linear independence of $Y$ as in the preceding paragraph. Hence $\Im f$ is infinite.

        Step 4.
        Next we show that $f^{-1}(T)$ is a finite subset of $X$ for every $T \in \Im f \subset K(Y)$.
        If $x \in f^{-1}(T)$, then $x$ is contained in the submodule $\langle T\rangle$ of $F$ generated by $T$; that is,
        \begin{equation*}
            f^{-1}(T) \subset \langle T\rangle
        \end{equation*}
        Since $T$ is finite and each $y \in T$ is a linear combination of a finite number of elements of $X$, there is a finite subset $S$ of $X$ such that $\langle T\rangle\subset \langle S\rangle$.
        Thus $x \in f^{-1}(T)$ implies $x \in \langle S\rangle$ and $x$ is a linear combination of elements of $S$.
        Since $x \in X$ and $S \subset X$, this contradicts the linear independence of $X$ unless $x \in S$. Therefore, $f^{-1}(T) \subset S$, whence $f^{-1}(T)$ is finite.

        Step 5.
        Verify that the sets $f^{-1}(T)$ form a partition of $X$,
        \begin{equation*}
            X= \bigsqcup_{T \in \Im f} f^{-1}\left(T\right)
        \end{equation*}
        For each $T \in \Im f$, order the elements of $f^{-1}(T)$, say $x_1, \ldots, x_n$, and define an injective map \begin{center}
            $g_T: f^{-1}(T) \rightarrow \Im f \times {N}\quad $ by $x_k \mapsto(T, k)$.
        \end{center}
        It follows that the map $X \rightarrow \Im f \times {N}$ defined by $x \mapsto g_T(x)$, where $x \in f^{-1}(T)$, is a well-defined injective function, whence $|X| \leq|\Im f \times {N}|$.
        Therefore
        \begin{equation*}
            \left|X\right|
            \leq
            \left|\Im f \times {N}\right|
            =
            \left|\Im f\right|
            \leq|K(Y)|
            =
            \left|Y\right|
        \end{equation*}
        since $\left|\Im f\right|$ is infinite.

        Step 6.
        Interchanging $X$ and $Y$ in the preceding argument shows that $|Y| \leq|X|$. Therefore $|Y|=|X|$ by the Schroeder-Bernstein Theorem.
    \end{proof}
\end{theorem}

\begin{lemma}
    Let $R$ be a ring, ${I}(\neq {R})$ an ideal of ${R}, {F}$ a free $R$-module with basis $X$.
    Then ${F} / {IF}$ is a free $R / I$-module with basis $\pi({X})$ and $\left|\pi({X})\right|=\left|X\right|$ where $\pi: {F} \rightarrow {F} / {IF}$ is the canonical epimorphism.
    \begin{proof}
        If $u+I F \in F / I F$, then $u=\sum_{j=1}^n r_j x_j$ with $r_j \in R, x_j \in X$ since $u \in F$ and $X$ is a basis of $F$. Consequently,
        \begin{equation*}
            u+I F
            =\sum_j\left(r_j x_j+I F\right)
            =
            \sum_j\left(r_j+I\right)\left(x_j+I F\right)
            =
            \sum_j\left(r_j+I\right) \pi\left(x_j\right)
        \end{equation*}
        whence $\pi(X)$ generates $F / I F$ as an $R / I$-module.

        On the other hand, if $\sum_{k=1}^m\left(r_k+I\right) \pi\left(x_k\right)=0$ with $r_k \in R$ and $x_1, \ldots, x_m$ distinct elements of $X$, then
        \begin{equation*}
            0
            =
            \sum_k\left(r_k+I\right) \pi\left(x_k\right)
            =
            \sum_k\left(r_k+I\right)\left(x_k+I F\right)
            =
            \sum_k r_k x_k+I F
        \end{equation*}
        whence $\sum_k r_k x_k \in I F$. Thus $\sum_k r_k x_k=\sum_j s_j u_j$ with $s_j \in I, u_j \in F$. Since each $u_j$ is a linear combination of elements of $X$ and $I$ is an ideal, $\sum_j s_j u_j$ is a linear combination of elements of $X$ with coefficients in $I$.
        Consequently,
        \begin{equation*}
            \sum_{k=1}^m r_k x_k
            =
            \sum_j s_j u_j
            =
            \sum_{t=1}^d c_t y_t
        \end{equation*}
        with $c_t \in I, y_t \in X$. The linear independence of $X$ implies that $r_k \in I$ for every $k$.
        Hence $r_k+I=0$ in $R / I$ for every $k$ and $\pi(X)$ is linearly independent over $R / I$. Thus $F / I F$ is a free $R / I$-module with basis $\pi(X)$ .

        Finally if $x, x^{\prime} \in X$ and $\pi(x)=\pi\left(x^{\prime}\right)$ in $F / I F$, then
        \begin{equation*}
            0
            =
            \left(1_R+I\right) \pi(x)-\left(1_R+I\right) \pi\left(x^{\prime}\right)
            =
            \left(1_R+I\right)\left(x+IF\right)-\left(1_R+I\right)\left(x^\prime+IF\right)
        \end{equation*}
        If $x \neq x^{\prime}$, the preceding argument implies that $1_R + I=0$ in $R/I$, which contradicts the fact that $I \neq R$.
        Therefore, $x=x^{\prime}$ and the map $\pi: X \rightarrow \pi(X)$ is a bijection, whence $|X|=|\pi(X)|$.
    \end{proof}
\end{lemma}


\begin{theorem}.
    \begin{enumerate}
        \item
              Division ring has the invariant dimension property.

        \item
              Let $f: R \rightarrow S$ be a nonzero epimorphism of rings.
              If $S$ has the invariant dimension property, then so does $R$ .

        \item
              If $R$ is a ring that has a homomorphic image which is a division ring, then $R$ has the invariant dimension property.

        \item
              Every commutative ring has the invariant dimension property.
    \end{enumerate}


    \begin{proof}
        (2) Let $I=\Ker f$; then $S \cong R / I$. Let $X$ and $Y$ be two bases of the free unitary $R$-module $F$ and $\pi: F \rightarrow F / I F$ the canonical epimorphism.
        By Lemma $F / I F$ is a free $R / I$-module (and hence a free unitary $S$-module) with bases $\pi(X)$ and $\pi(Y)$ such that $|X|=|\pi(X)|,|Y|=|\pi(Y)|$. Since $S$ has the invariant dimension property, $|\pi(X)|=|\pi(Y)|$. Therefore, $|X|=|Y|$ and $R$ has the invariant dimension property.

        (3) It is obvious from (1) and (2)

        (4) Consider the maximal ideal $\mathfrak{m}$ in $R$ and canonical projective
        \begin{equation*}
            R\rightarrow R/ \mathfrak{m}
        \end{equation*}
    \end{proof}
\end{theorem}



\begin{theorem}
    Let $R$ be a ring and $M$ a module over $R$. Let $I$ be a non-empty set, and let $\left\{x_i\right\}_{i \in I}$ be a basis of $M$. Let $N$ be an $R$-module, and let $\left\{y_i\right\}_{i \in I}$ be a family of elements of $N$. Then there exists a unique homomorphism $f: M \rightarrow N$ such that $f\left(x_i\right)=y_i$ for all $i$.
\end{theorem}




\subsection{Vector Space} % Vector Space

\begin{theorem}
    Let $D$ be a division ring.
    \begin{enumerate}
        \item
              A maximal linearly independent subset $X$ of a vector space $V$ over a division ring $D$ is a basis of $V$ .


        \item
              Basis Extension Theorem.
              Let $V$ be a vector space, $T$ spans $V$ and $S$ be a subset of $T$ which is linearly independent. Then there exists a basis $B$ of $V$ such that $S \subset B \subset T$.

        \item
              Every vector space $V$ over a division ring $D$ has a basis and is therefore a free $D$-module.

        \item
              Let $V$ be a vector space over $D$. Then two bases of $V$ over $D$ have the same cardinality.
    \end{enumerate}
\end{theorem}

\begin{definition}
    If $V$ is a finitely generated $D$-module the cardinality of any basis is called the \textbf{dimension of $V$} and is denoted by $\dim_D V$, or just $\dim V$, and $V$ is said to be finite dimensional over $D$.
    If $V$ is not finitely generated, $V$ is said to be \textbf{infinite dimensional} (written $\dim V=\infty$ ).
\end{definition}

\begin{theorem}
    Let $W$ be a subspace of a vector space $V$ over a division ring $D$.
    \begin{enumerate}
        \item
              $\dim_{D} W \leq \dim_{D} V$

        \item
              if $\dim_{D} W=\dim_{D} V < \infty$, then $W=V$

        \item
              $\dim_{D} V=\dim_{D} W+\dim_{D}(V / W)$.

        \item
              If $V_1$ and $V_2$ are finite dimensional subspaces of a vector space over $D$, then
              \begin{equation*}
                  \dim_D V_1 +\dim_{D} V_2
                  =
                  \dim_{D}(V_1 \cap V_2)+\dim_{D}(V_1+V_2)
              \end{equation*}
    \end{enumerate}
\end{theorem}

\begin{corollary}
    If $f: V \rightarrow V^{\prime}$ is a linear transformation of vector spaces over a division ring $D$, then there exists a basis X of V such that $X \cap \Ker f$ is a basis of $\Ker f$ and $X \cap f^{-1}\left(\Im f\right)$ is a basis of $\Im f$.
    In particular,
    \begin{equation*}
        \dim_{D} V
        =
        \dim_{D}(\Ker f)+\dim_{D}(\Im f)
    \end{equation*}
\end{corollary}

\begin{theorem}
    Let ${R}, {S}, {T}$ be division rings such that $R \subset {S} \subset {T}$. Then
    \begin{equation*}
        \dim_{{R}} {T}
        =
        \left(\dim_{{S}} {T}\right)\left(\dim_{{R}} {S}\right)
    \end{equation*}
    Furthermore, $\dim_{{R}} {T}$ is finite if and only if $\dim_{{S}} {T}$ and $\dim_{{R}} {S}$ are finite.
\end{theorem}

\subsection{Pullbacks and Pushout}


\begin{definition}
    Let $R$ be a ring and in $R$-$\Mod$.
    \begin{enumerate}
        \item
              Given a diagram
              \begin{equation*}
                  \begin{tikzcd}
                      A_1 \arrow[r,"f_1"]& B&\arrow[l,"f_2"'] A_2
                  \end{tikzcd}
              \end{equation*}
              The \textbf{pullback} of $f_1$ and $f_2$ is the submodule
              \begin{equation*}
                  P=\left\{(a_1, a_2) \in A_1 \oplus A_2 \mid f_1(a_1)=f_2(a_2)\right\}
              \end{equation*}
              of the direct product $A_1 \oplus A_2$ together with the canonical projections $\pi_1: P \rightarrow A_1$ and $\pi_2: P \rightarrow A_2$.
              That is,
              \begin{equation*}
                  \begin{tikzcd}
                      X \arrow[ddr,bend right=20,"q_1"']\arrow[drr,bend left=20,"q_2"]\arrow[rd,dashed,"\exists! u"]& &\\
                      &P \arrow[d,"\pi_1"]\arrow[r,"\pi_2"]&  A_2\arrow[d,"f_2"]\\
                      &A_1\arrow[r,"f_1"]&  B\\
                  \end{tikzcd}
              \end{equation*}

        \item
              Given a diagram
              \begin{equation*}
                  \begin{tikzcd}
                      B_1 &\arrow[l,"f_1"'] A\arrow[r,"f_2"]& B_2
                  \end{tikzcd}
              \end{equation*}
              The \textbf{pushout} of $f_1$ and $f_2$ is the quotient module
              \begin{equation*}
                  P=(B_1 \oplus B_2) / K
              \end{equation*}
              where $K$ is the submodule of $B_1 \oplus B_2$ generated by all elements of the form $(f_1(a),-f_2(a))$ with $a \in A$, together with the canonical injections $\iota_1: B_1 \rightarrow P$ and $\iota_2: B_2 \rightarrow P$.
              That is,
              \begin{equation*}
                  \begin{tikzcd}
                      A \arrow[r,"f_2"] \arrow[d,"f_1"'] & B_2\arrow[ddr,bend left=20,""] \arrow[d,"\iota_2"]& \\
                      B_1 \arrow[rrd,bend right=20,""]\arrow[r,"\iota_1"'] & P \arrow[rd,dashed,""]&\\
                      && Y
                  \end{tikzcd}
              \end{equation*}
    \end{enumerate}
\end{definition}













\section{Tensor Products} % Tensor Products
\subsection{Basic definition} % Basic definition
\begin{definition}
    Let $R$ be a ring, $A_R$ be a right module and $\prescript{}{R}{B}$ a left module.
    Let $F$ be the free abelian group on the set $A \times B$.
    Let $K$ be the subgroup of $F$ generated by all elements of the following forms :
    \begin{equation*}
        \left(a+a^{\prime}, b\right)-(a, b)-\left(a^{\prime}, b\right), \\
        \left({a}, {b}+{b}^{\prime}\right)-({a}, {b})-\left({a}, {b}^{\prime}\right), \\
        ({ar}, {b})-({a}, {rb})
    \end{equation*}
    The quotient group ${F} / {K}$ is called the \textbf{tensor product} of $A_R$ and $\prescript{}{R}{B} $ ;
    it is denoted $A \otimes_{{R}} B$.
    The coset $({a}, {b})+{K}$ of the element $({a}, {b})$ in F is denoted ${a} \otimes {b}$; the coset of $(0,0)$ is denoted $0$ .
\end{definition}

\begin{theorem}
    Let $R$ and $S$ be rings and $ \tensor[_S]{A}{_R}  ,\tensor[_R]{B}{}, C_{{R}},{ }_{{R}} {D}_{{S}}$ bimodules as indicated.
    \begin{enumerate}
        \item
              ${A} \otimes_{{R}} B$ is a left $S$-module such that
              \begin{equation*}
                  s(a \otimes b):=sa \otimes b
              \end{equation*}
              for all ${s} \in {S}, {a} \in A$, ${b} \in B$.
        \item
              ${C} \otimes_{{R}} {D}$ is a right $S$-module such that $({c} \otimes {d}) {s}={c} \otimes {ds}$ for all ${c} \in C$, ${d} \in {D}, {s} \in {S}$.
    \end{enumerate}
    \begin{remark}
        An important special case occurs when $R$ is a commutative ring and hence every $R$-module $A$ is an $R$-$R$ bimodule.
        In this case $A \otimes_R B$ is also an $R$-$R$ bimodule with
        \begin{equation*}
            r(a \otimes b)=r a \otimes b=a r \otimes b=a \otimes r b=a \otimes b r=(a \otimes b) r
        \end{equation*}
        for all $r \in R, a \in A, b \in B$.
    \end{remark}
\end{theorem}

\begin{definition}
    If $A_R$ and ${ }_R B$ are modules over a ring $R$, and $C$ is an abelian group, then a \textbf{middle linear (or banlanced) map} from $A \times B$ to $C$ is a function $f: A \times B \rightarrow C$ such that for all $a, a_i \in A, b, b_i \in B$, and $r \in R$ :
    \begin{equation*}
        \begin{aligned}
            f\left(a_1+a_2, b\right) & =f\left(a_1, b\right)+f\left(a_2, b\right) \\
            f\left(a, b_1+b_2\right) & =f\left(a, b_1\right)+f\left(a, b_2\right) \\
            f(a r, b)                & =f(a, r b)
        \end{aligned}
    \end{equation*}
    \begin{remark}
        The map $i: A \times B \rightarrow A \otimes_R B$ given by $(a, b) \mapsto a \otimes b$ is a middle linear map, that is,
        for all $a, a_i \in A, b, b_i \in B$, and $r \in R$
        \begin{equation*}
            \begin{aligned}
                \left(a_1+a_2\right) \otimes b & =a_1 \otimes b+a_2 \otimes b \\
                a \otimes\left(b_1+b_2\right)  & =a \otimes b_1+a \otimes b_2 \\
                a r \otimes b                  & =a \otimes r b
            \end{aligned}
        \end{equation*}
        The map $i$ is called the \textbf{canonical middle linear map}.
    \end{remark}
\end{definition}



\begin{theorem}[Universal property of canonical middle linear map]
    \label{thm: Universal property of canonical middle linear map}
    Let $A_{{R}}$ and $\prescript{ }{R}{B}$ be modules over a ring $R$, and let $C$ be an abelian group.
    If ${g}: A \times B \rightarrow C$ is a middle linear map, then there exists a unique group homomorphism $\overline{{g}}: A \otimes_{{R}} B \rightarrow C$ such that
    \begin{equation*}
        \begin{tikzcd}
            A \times B \arrow[rr, "g"] \arrow[dd, "i"]  & & C \\
            &   &      \\
            A \otimes_R B \arrow[rruu,dashed, "\exists! \bar{g}"']  &     &
        \end{tikzcd}
    \end{equation*}
    is commutative.
    This proves that $i: A \times B \rightarrow A \otimes_R B$ is a universal object in the category of all middle linear maps on $A \times B$, whence $A \otimes_R B$ is uniquely determined up to isomorphism.
\end{theorem}

\begin{definition}
    Let $A, B, C$ be modules over a commutative ring $R$. A \textbf{bilinear map} from $A \times B$ to $C$ is a function $f: A \times B \rightarrow C$ such that for all $ a_i \in A, b_i \in B$, and $r \in R$ :
    \begin{equation*}
        \begin{aligned}
            f\left(a_1+a_2, b\right) & =f\left(a_1, b\right)+f\left(a_2, b\right) \\
            f\left(a, b_1+b_2\right) & =f\left(a, b_1\right)+f\left(a, b_2\right) \\
            f(r a, b)                & =r f(a, b)=f(a, r b)
        \end{aligned}
    \end{equation*}
    If $A$ and $B$ are modules over a commutative ring $R$, then so is $A \otimes_R B$ and the canonical middle linear map $i: A \times B \rightarrow A \otimes_R B$ is easily seen to be bilinear. In this context $i$ is called the \textbf{canonical bilinear map}.
\end{definition}

\begin{theorem}[Universal property of canonical
        bilinear map]
    If ${A}, B, C$ are modules over a commutative ring $R$ and ${g}: A \times B \rightarrow C$ is a bilinear map, then there is a unique $R$-module homomorphism $\overline{{g}}: A \otimes_{{R}} B \rightarrow C$ such that
    \begin{equation*}
        \begin{tikzcd}
            A \times B \arrow[rr, "g"] \arrow[dd, "i"]  & & C \\
            &   &      \\
            A \otimes B \arrow[rruu,dashed, "\exists!\bar{g}"']  &     &
        \end{tikzcd}
    \end{equation*}
    where ${i}: A \times B \rightarrow A \otimes_{{R}} B$ is the canonical bilinear map. The module ${A} \otimes_{{R}} B$ is uniquely determined up to isomorphism by this property.
\end{theorem}


\begin{proposition}
    If $A_R, A_R^{\prime}, \prescript{ }{R}{B}$ and $\prescript{ }{R}{B}^{\prime}$ are modules over a [resp. commutative] ring $R$ and ${f}: A \rightarrow A^{\prime}$, ${g}: B \rightarrow B^{\prime}$ are $R$-module homomorphisms, then there is a unique group [resp. $R$-module] homomorphism
    \begin{center}
        $f\otimes g: A \otimes_{{R}} B \rightarrow A^{\prime} \otimes_{{R}} B^{\prime}\quad $ such that ${a} \otimes {b} \mapsto {f}({a}) \otimes {g}({b})$
    \end{center}
    for all ${a} \in A, {b} \in B$.
\end{proposition}



\subsection{Operation of tensor products} % Operation of tensor products

\begin{theorem}
    Let $R$ and $S$ be rings and $ \tensor[_S]{A}{_R}  ,\tensor[_R]{B}{}, C_{{R}},{ }_{{R}} {D}_{{S}}$ (bi)modules as indicated.

    \begin{enumerate}
        \item
              If ${f}: A \rightarrow A^{\prime}$ is a homomorphism of ${S}$-${R}$ bimodules and ${g}: B \rightarrow B^{\prime}$ is an R-module homomorphism, then the induced map ${f} \otimes {g}: A \otimes_{{R}} B \rightarrow A^{\prime} \otimes_{{R}} B^{\prime}$ is a homomorphism of left $S$-modules.
        \item
              If ${h}: C \rightarrow C^{\prime}$ is an R -module homomorphism and ${k}: {D} \rightarrow {D}^{\prime}$ a homomorphism of R-S bimodules, then the induced map ${h} : C \otimes_R {D} \rightarrow C^{\prime} \otimes_{{R}} {D}^{\prime}$ is a homomorphism of right $S$-modules.
    \end{enumerate}
\end{theorem}


\begin{theorem}
    \label{thm: operation laws of tensor products}
    Let $R,S$ be ring. Then
    \begin{enumerate}
        \item
              If  ${A}_{R},{ }_{{R}} B_{S},{ }_{{S}} C$ are (bi)modules, then there is an abelian group isomorphism.
              \begin{equation*}
                  \left({A} \otimes_{{R}} B\right) \otimes_{{S}} C
                  \cong
                  A \otimes_R \left(B \otimes_S C\right)
              \end{equation*}
              If $A$ is an $S$-$R$ module, there is a left $S$-module isomorphism $\left({A} \otimes_R B\right) \otimes_R C
                  \cong
                  A \otimes_R \left(B \otimes_R C\right)$.

        \item
              Let $A$ and $\left\{{A}_i \mid i \in I\right\}$ right $R$-modules, $B$ and $\left\{{B}_{{j}} \mid {j} \in {J}\right\}$ left $R$-modules.
              Then there are abelian group isomorphisms:
              \begin{equation*}
                  \left(\bigoplus_{i \in I} A_i\right) \otimes_{{R}} B
                  \cong
                  \bigoplus_{i \in I}\left({A}_i \otimes_{{R}} B\right)
              \end{equation*}
              and
              \begin{equation*}
                  A \otimes_{{R}}\left(\bigoplus_{j \in J} {~B}_j\right)
                  \cong
                  \bigoplus_{j \in J}\left(A \otimes_R B_j\right)
              \end{equation*}
    \end{enumerate}
    \begin{proof}
        1.
        By definition, we have
        \begin{equation*}
            v
            =
            \sum_i u_i \otimes c_i=\sum_i\left(\sum_j a_{i j} \otimes b_{i j}\right) \otimes c_i
            =
            \sum_i \sum_j\left[\left(a_{i j} \otimes b_{i j}\right) \otimes c_i\right] .
        \end{equation*}
        Therefore, $\left(A \otimes_R B\right) \otimes_S C$ is generated by all elements of the form $(a \otimes b) \otimes c$ $(a \in A, b \in B, c \in C)$. Similarly, $A \otimes_R\left(B \otimes_S C\right)$ is generated by all $a \otimes(b \otimes c)$ with $a \in A, b \in B, c \in C$.

        Verify that the assignment $\left(\sum_{i=1}^n a_i \otimes b_i, c\right) \mapsto \sum_{i=1}^n\left[a_i \otimes\left(b_i \otimes c\right)\right]$ defines an $S$-middle linear map $\left(A \otimes_R B\right) \times C \rightarrow A \otimes_R\left(B \otimes_S C\right)$. Therefore, by \ref{thm: Universal property of canonical middle linear map} there is a homomorphism
        \begin{equation*}
            \alpha:\left(A \otimes_R B\right) \otimes_S C \rightarrow A \otimes_R\left(B \otimes_S C\right)
        \end{equation*}
        with $\alpha[(a \otimes b) \otimes c]=a \otimes(b \otimes c)$ for all $a \in A, b \in B, c \in C$.
        Similarly there is an homomorphism
        \begin{equation*}
            \beta: A \otimes_R\left(B \otimes_S C\right) \rightarrow\left(A \otimes_R B\right) \otimes_S C
        \end{equation*}
        such that $\beta[a \otimes(b \otimes c)]=(a \otimes b) \otimes c$ for all $a \in A, b \in B, c \in C$. Therefore, $\alpha$ and $\beta$ are isomorphisms.

        2.
        Let $\iota_k, \pi_k$ be the canonical injections and projections of $\bigoplus_{i \in I} A_i$.
        The family of homomorphisms
        \begin{equation*}
            \iota_k \otimes 1_B: A_k \otimes_R B \rightarrow\left(\bigoplus_{i \in I} A_i\right) \otimes_R B
        \end{equation*}
        induce a homomorphism
        \begin{equation*}
            \alpha:
            \bigoplus_{i \in I}\left(A_i \otimes_R B\right)
            \rightarrow
            \left(\bigoplus_{i \in I} A_i\right) \otimes_R B
        \end{equation*}
        such that
        $\alpha\left(\left\{a_i \otimes b\right\}\right)
            =
            \bigoplus\left(\iota_i\left(a_i\right) \otimes b\right)=
            \left(\bigoplus \iota_i\left(a_i\right)\right) \otimes b$ (Note that the summation here is finite.)
        The assignment $(u, b) \mapsto\left\{\pi_i(u) \otimes b\right\}$ defines a middle linear map $\left(\bigoplus_{i \in I} A_i\right) \times B \rightarrow$ $\bigoplus_{i \in I}\left(A_i \otimes_R B\right)$ and thus induces a homomorphism $\beta:\left(\bigoplus A_i\right) \otimes_R B \rightarrow \bigoplus\left(A_i \otimes_R B\right)$ such that $\beta(u \otimes b)=\left\{\pi_i(u) \otimes b\right\}$.
        We can show that $\alpha \beta$ and $\beta \alpha$ are the respective identity maps, whence $\alpha$ is an isomorphism.
    \end{proof}
\end{theorem}



\begin{theorem}[Adjoint Associativity]
    Let $R$ and $S$ be rings and $A_{R},{ }_R B_{S}, C_{S}$ modules. Then there is an isomorphism of abelian groups
    \begin{equation*}
        \alpha: \operatorname{Hom}_{S}\left(A \otimes_R B, C\right)
        \cong
        \operatorname{Hom}_{R}\left(A, \operatorname{Hom}_{S}(B, C)\right),
    \end{equation*}
    defined for each $S$-module homomorphism $f: A \otimes_R B \rightarrow C$ by
    \begin{equation*}
        f\mapsto f(- \otimes -)
    \end{equation*}
    where $f(- \otimes -): A \rightarrow \operatorname{Hom}_S(B, C)$ is the map defined by $a\mapsto f(a\otimes -)$
\end{theorem}





\subsection{}


\begin{theorem}
    If $R$ is a ring and ${A}_{{R}}, \tensor[_R]{B}{}$ are $R$-modules, then there are right $R$-module isomorphisms
    \begin{equation*}
        A \otimes_R R \cong A
    \end{equation*}
    and left $R$-module isomorphisms
    \begin{equation*}
        R \otimes_R B \cong B
    \end{equation*}
\end{theorem}





\begin{theorem}
    Let $R$ be a ring. If $A$ is a right $R$-module and $F$ is a free left $R$-module with basis $Y$.
    Then every element $u$ of ${A} \otimes_{{R}} {F}$ may be written uniquely in the form ${u}=\sum_{i=1}^n {a}_i \otimes {y}_i$, where ${a}_i \in A$ and the ${y}_i$ are distinct elements of $Y$ .
    \begin{proof}
        For each $y \in Y$, let $A_y$ be a copy of $A$ and consider the direct sum $\bigoplus_{y \in Y} A_y$. We first construct an isomorphism
        \begin{equation*}
            \theta: A \otimes_R F \cong \bigoplus_{y \in Y} A_y
        \end{equation*}
        as follows. Since $Y$ is a basis, $\{y\}$ is a linearly independent set for each $y \in Y$. Consequently, the $R$-module epimorphism  \begin{equation*}
            \varphi: R \rightarrow Ry \quad \text{ given by } r \mapsto r y
        \end{equation*}
        is actually an isomorphism. Therefore there is for each $y \in Y$ an isomorphism
        \begin{equation*}
            A \otimes_R R y \xrightarrow{1_A \otimes \varphi^{-1}} A \otimes_R R \cong A=A_y .
        \end{equation*}
        Thus by \ref{thm: operation laws of tensor products} and I.8.10 there is an isomorphism $\theta$ :
        \begin{equation*}
            A \otimes_R F
            =
            A \otimes_R\left(\bigoplus_{y \in Y} R y\right) \cong \bigoplus_{y \in Y} A \otimes_R R y \cong \bigoplus_{y \in Y} A_y
        \end{equation*}

        Verify that for every $a \in A, z \in Y$,
        \begin{equation*}
            \theta(a \otimes z)
            =
            \left\{u_y\right\}
            \in
            \bigoplus A_y
        \end{equation*}
        where $u_z=a$ and $u_y=0$ for $y \neq z$; in other words, $\theta(a \otimes z)=\iota_z(a)$, with $\iota_z: A_z \rightarrow \bigoplus A_y$ the canonical inection. Now every nonzero $v \in \bigoplus A_y$ is a finite sum $v=\iota_{y_1}\left(a_1\right)+\cdots+\iota_{y_n}\left(a_n\right)$ $=\theta\left(a_1 \otimes y_1\right)+\cdots+\theta\left(a_n \otimes y_n\right)$ with $y_1, \ldots, y_n$ distinct elements of $Y$ and $a_i$ uniquely determined nonzero elements of $A$. It follows that every element of $A \otimes_R F$ (which is necessarily $\theta^{-1}(v)$ for some $v$ ) may be written uniquely as $\bigoplus^n_{i=1} a_i \otimes y_i$.
    \end{proof}
\end{theorem}



\begin{corollary}
    If $R$ is a ring with identity and ${A}_{{R}}$ and ${ }_{{R}} B$ are free $R$-modules with bases $X$ and $Y$ respectively, then ${A} \otimes_{{R}} B$ is a free (right) $R$-module with basis ${W}=\{{x} \otimes {y} \mid {x} \in {X}, {y} \in {Y}\}$ of cardinality $|{X} \| {Y}|$.

    REMARKS. Since $R$ is an $R-R$ bimodule, so is every direct sum of copies of $R$. In particular, every free left $R$-module is also a free right $R$-module and vice versa. However, it is not true in general that a free (left) $R$-module is a free object in the category of $R-R$ bimodules (Exercise 12).

    \begin{proof}
        By the proof of Theorem 5.11 and by Theorem 2.1 (for right $R$-modules) there is a group isomorphism
        \begin{equation*}
            \theta: A \otimes_R B \cong \bigoplus_{y \in Y} A_y=\sum_{y \in Y} A=\sum_{y \in Y}\left(\sum_{x \in X} x R\right) .
        \end{equation*}

        Since $B$ is an $R-R$ bimodule by the remark preceding the proof, $A \otimes_R B$ is a right $R$-module by Theorem 5.5. Verify that $\theta$ is an isomorphism of right $R$-modules such that $\theta(W)$ is a basis of the free right $R$-module $\sum_Y\left(\sum_X x R\right)$. Therefore, $A \otimes_R B$ is a free right $R$-module with basis $W$. Since the elements of $W$ are all distinct by Theorem 5.11, $|W|=|X||Y|$.
    \end{proof}
\end{corollary}







































\section{Algebra} % Algebra

\begin{definition}
    Let $K$ be a commutative ring.
    The abelian group $\left(A,+\right)$ is a \textbf{$R$-algebra} if
    \begin{enumerate}[label=(\roman*)]
        \item
              $(A,+,\cdot)$ is a $R$-module
        \item
              $\left(A,+,\times\right)$ is a ring
        \item
              $r\cdot({a\times b})=(ra) {b}=a({rb})$ for all $r \in R$ and ${a}, {b} \in A$.
    \end{enumerate}
    If $A$ which, as a ring, is a division ring, is called a \textbf{division algebra}.
    \begin{remark}
        Condition (ii) may fail to hold if $A$ does not have an identity of noassociativity.
        In this case, we say that $A$ is a \textbf{nonunital $R$-algebra} or \textbf{noassociativity $R$-algebra} respectively.
    \end{remark}

    An algebra over a field $F$ that is finite dimensional as a vector space over $K$ is called a \textbf{finite dimensional algebra} over $F$.
\end{definition}


\begin{definition}
    Let $K$ be a commutative ring with identity $1$ and $A$, $B$ $K$-algebras.
    \begin{enumerate}
        \item
              A \textbf{subalgebra} of $A$ is a subring of $A$ that is also a $K$-submodule of $A$.

        \item
              A (left, right, two-sided) \textbf{algebra ideal} of $A$ is a (left, right, two-sided) ideal of the ring $A$ (in this case it is also a $K$-submodule of $A$).

        \item
              A homomorphism [resp. isomorphism] of $K$-algebras ${f}: A \rightarrow B$ is a ring homomorphism [isomorphism] that is also a $K$-module homomorphism [isomorphism].
    \end{enumerate}
\end{definition}






\begin{theorem}[Tensor product of algebras]
    Let $A$ and $B$ be algebras over a commutative ring $K$.
    Let $\pi$ be the composition
    \begin{equation*}
        \left(A \otimes_K B\right) \otimes_K
        \left(A\otimes_K B\right)
        \xrightarrow{1_A \otimes \alpha \otimes 1_B}
        \left(A \otimes_K A\right) \otimes_K \left(B \otimes_K B\right)
        \xrightarrow{\pi_A \otimes \pi_B} A \otimes_K B
    \end{equation*}
    where $\pi_{{A}}, \pi_{{B}}$ are the product maps of $A$ and $B$ respectively.
    Then ${A} \otimes_{{K}} B$ is a $K$-algebra with product map $\pi=\pi_A\otimes \pi_B$.
    The $K$-algebra $A \otimes_K B$ is called the \textbf{tensor product of the $K$-algebras} $A$ and $B$.
    \begin{proof}
        note that for generators $a \otimes b$ and $a_1 \otimes b_1$ of $A \otimes_K B$ the product is defined to be
        \begin{equation*}
            (a \otimes b)\left(a_1 \otimes b_1\right)=\pi\left(a \otimes b \otimes a_1 \otimes b_1\right)=a a_1 \otimes b b_1
        \end{equation*}
        Thus if $A$ and $B$ have identities $1_A, 1_B$ respectively, then $1_A \otimes 1_B$ is the identity in $A \otimes_K B$.
    \end{proof}

\end{theorem}


\section{Modules over Principal Ideal Domains}
Rings are P.I.D and $R$-module $M$.
\subsection{Preparatory Lemmas} % Preparatory Lemmas

\begin{theorem}
    \label{thm: submodule of a module over P.I.D}
    Let $F$ be a free module over $R$ and $N$ a submodule of $F$.
    Then $N$ is a free $R$-module and $\operatorname{rank} N \leq \operatorname{rank} M$.
    \begin{proof}
        Let $\left\{x_1, x_2, \ldots, x_n\right\}$ be a basis of $F$.
        We proceed by induction on $n$.

        If $n=1$, then $F \cong R$ and $N \cong I$ for some ideal $I$ of $R$. Since $R$ is a principal ideal domain, $I=(a)$ for some $a \in R$. Therefore, $N \cong R$ or $N=0$ is free and $\operatorname{rank} N \leq \operatorname{rank} F$.

        Suppose the theorem is true for $n-1$ and let $F$ be generated by $n$ elements.
        Let $\pi: F \rightarrow R$ be the projection defined by
        \begin{equation*}
            \pi\left(r_1 x_1+r_2 x_2+\cdots+r_n x_n\right)=r_n
        \end{equation*}
        for all $r_i \in R$.
        Then $\pi(N)$ is an ideal of $R$, whence $\pi(N)=(a)$ for some $a \in R$ since $R$ is a principal ideal domain.

        If $\pi(N)=0$, then $N \subseteq \Ker \pi=\langle x_1, x_2, \ldots, x_{n-1}\rangle$. By the induction hypothesis, $N$ is free and $\operatorname{rank} N \leq n-1<\operatorname{rank} F$.

        If $\pi(N) \neq 0$, let $z \in N$ such that $\pi(z)=a$. For each $y \in N$, there exists $r \in R$ such that $\pi(y)=r a=\pi(r z)$. Thus $y-r z \in \Ker \pi \cap N$. It follows that
        \begin{equation*}
            N=R z \oplus(\Ker \pi \cap N)
        \end{equation*}
        Since $\Ker \pi \cap N$ is a submodule of the free module $\Ker \pi= x_1, x_2, \ldots, x_{n-1}$, $\Ker \pi \cap N$ is free and $\operatorname{rank}(\Ker \pi \cap N) \leq n-1$ by the induction hypothesis. Therefore, $N$ is free and
        \begin{equation*}
            \operatorname{rank} N
            =
            1+\operatorname{rank}(\Ker \pi \cap N)
            \leq
            1+(n-1)
            =
            n
            =
            \operatorname{rank} F
        \end{equation*}
    \end{proof}
\end{theorem}

\begin{corollary}
    If $M$ is a finitely generated by $n$ elements, then every submodule of $M$ may be generated by $m$ elements with $m \leq n$.
\end{corollary}

\begin{corollary}
    A module $M$ over a principal ideal domain $R$ is free if and only if $M$ is projective.
\end{corollary}


\subsection{}
\begin{theorem}
    A finitely generated torsion-free module $M$ over a principal ideal domain $R$ is free.
    \begin{proof}
        Let $M$ be generated by $n$ elements.
        By Corollary 6.2, every submodule of $M$ may be generated by $m$ elements with $m \leq n$.
        We proceed by induction on $n$.

        If $n=1$, then $M \cong R / \Ann\left(M\right)$ is torsion-free, whence $\Ann\left(M\right)=0$ and $M \cong R$ is free.

        Suppose the theorem is true for $n-1$ and let $M$ be generated by $n$ elements.
        Let $N$ be a maximal submodule of $M$; then $N$ may be generated by $m$ elements with $m \leq n$.
        Since $M / N$ is cyclic, there exists an epimorphism
        \begin{equation*}
            f: R \rightarrow M / N
        \end{equation*}
        with $\Ker f=\Ann\left(M / N\right)$.
        If $\Ann\left(M / N\right) \neq 0$, then $M / N$ has torsion, which contradicts the fact that $M$ is torsion-free. Therefore, $\Ann\left(M / N\right)=0$ and $M / N \cong R$ is free.

        By the induction hypothesis, $N$ is free since it is generated by at most $n-1$ elements.
        Consequently, the exact sequence
        \begin{equation*}
            0 \rightarrow N \xrightarrow{\subset} M \rightarrow M / N \rightarrow 0
        \end{equation*}
        is split exact and $M \cong N \oplus M / N$ by Theorem 2.6 and Proposition 2.7.
        Therefore, $M$ is free.
    \end{proof}
\end{theorem}

\begin{theorem}
    If $M$ is a finitely generated module over $R$, then
    \begin{equation*}
        M=\Tor\left(M\right) \oplus F
    \end{equation*}
    where $F$ is a free left $R$-module of finite rank and $F \cong M / \Tor\left(M\right)$.
    \begin{proof}
        The quotient module $M / M_t$ is torsion-free since for each $r \neq 0$,
        \begin{equation*}
            r\left(a+M_t\right)=M_t \Rightarrow r a \in M_t \Rightarrow r_1(r a)=0 \text { for some } r_1 \neq 0 \Rightarrow a \in M_t
        \end{equation*}
        Furthermore, $M / M_t$ is finitely generated since $M$ is. Therefore, $M / M_t$ is free of finite rank by Theorem 6.5.
        Consequently, the exact sequence
        \begin{equation*}
            0 \rightarrow M_t \xrightarrow{\subset} M \rightarrow M / M_t \rightarrow 0
        \end{equation*}
        is split exact and $M \cong M_t \oplus \left(M / M_t\right)$ by \ref{thm: Free unitary R-module} and \ref{thm: Split exact sequence}.

        Under the isomorphism $M_t \oplus M / M_t \cong M$ of Theorem 3.4 the image of $M_t$ is $M_t$ and the image of $M / M_t$ is a submodule $F$ of $M$, which is necessarily free of finite rank. It follows that $M$ is the internal direct sum $M=M_t \oplus F$ (see Theorem 1.15).
    \end{proof}
\end{theorem}

\subsection{Torsion module decomposition}

\begin{theorem}
    Let $M$ be a torsion module over a principal ideal domain $R$ and for each prime $p \in R$ let $M_p=\left\{m \in M \mid m \text{ has order a power of } p\right\}$.
    \begin{enumerate}
        \item
              $M_p$ is a submodule of $M$ for each prime $p \in R$.

        \item
              $M=\bigoplus M_p$ where the sum is over all primes $p \in R$. If A is finitely generated, only finitely many of the $M_p$ are nonzero.
    \end{enumerate}
    \begin{proof}
        Let $0 \neq a \in M$ with $\Ann\left(a\right)=(r)$.
        By Theorem III.3.7 $r=p_1{ }^{n_1} \cdots p_k{ }^{n k}$ with $p_i$ distinct primes in $R$ and each $n_i>0$.
        For each $i$, let $r_i=p_1^{n_1} \cdots p_{i-1}^{n_{i-1}} p_{i+1}^{n_{i+1}} \cdots p_k^{n_k}$.
        Then $r_1, \ldots, r_k$ are relatively prime and there exist $s_1, \ldots, s_k \in R$ such that $s_1 r_1+\cdots+ s_k r_k=1_R$ by \ref{pro: unique factorization domain}.
        Consequently, $a=1_R a=s_1 r_1 a+\cdots+s_k r_k a$, and we have proved that the submodules $M_p$ generate the module $M$.

        Let $p \in R$ be prime and let $M_1$ be the submodule of $M$ generated by all $M_q$ with $q \neq p$.
        Suppose $a \in M(p) \cap A_1$. Then $p^m a=0$ for some $m \geq 0$ and $a=a_1+\cdots+a_t$ with $a_i \in A\left(q_i\right)$ for some primes $q_1, \ldots, q_t$ all distinct from $p$. Since $a_i \in A\left(q_i\right)$, there are integers $m_i$ such that $q_i{ }^{m_i} a_i=0$, whence $\left(q_1{ }^{m_1} \cdots q_t{ }^{m_t}\right) a=0$. If $d=q_1{ }^{m_1} \cdots q_t{ }^{m_t}$, then $p^m$ and $d$ are relatively prime and $r p^m+s d=1_R$ for some $r, s \in R$. Consequently, $a=1_R a=r p^m a+s d a=0$. Therefore, $A(p) \cap A_1=0$ and $A=\sum A(p)$ by Theorem 1.15. The last statement of the Theorem is a consequence of the easily verified fact that a direct sum of modules with infinitely many nonzero summands cannot be finitely generated. For each generator has only finitely many nonzero coordinates.
    \end{proof}
\end{theorem}
\begin{lemma}
    Let $M$ be a unitary module over a principal ideal domain $R$ such that $p^{n} M=0$ and $p^{n-1} M \neq 0$ for some prime $p \in R$ and positive integer $n$.
    Let $a$ be an element of $M$ of order $p^{n}$.
    \begin{enumerate}
        \item
              If $M \neq Ra$, then there exists a nonzero $b \in M$ such that $Ra \cap Rb=0$.
        \item
              There is a submodule $N$ of $M$ such that $M=Ra \oplus N$.
    \end{enumerate}
\end{lemma}

\begin{theorem}
    Let $M$ be a finitely generated unitary module over a principal ideal domain $R$ such that every element of $M$ has order a power of some prime $p\in R$.
    Then $M$ is a direct sum of cyclic $R$ -modules of orders $p^{n_1}, \ldots, p^{n_k}$ respectively, where $n_1 \geq n_2 \geq \cdots \geq n_k \geq 1$.
\end{theorem}

\subsection{}
\begin{theorem}
    Let $M$ be a finitely generated unitary module over a principal ideal domain $R$.
    \begin{enumerate}
        \item
              $M$ is the direct sum of a free  submodule $F$ of finite rank and a finite number of cyclic torsion modules.
              The cyclic torsion summands (if any) are of orders $r_1, \ldots, r_{t}$, where $r_1, \ldots, r_{t}$ are (not necessarily distinct) nonzero nonunit elements of R such that $r_1\left|r_2\right| \cdots \mid r_{t}$.
              The rank of $F$ and the list of ideals $\left(r_1\right), \ldots,\left(r_{t}\right)$ are uniquely determined by $M$.
        \item
              $M$ is the direct sum of a free submodule E of finite rank and a finite number of cyclic torsion modules. The cyclic torsion summands (if any) are of orders $p_1^{s_1}, \ldots, p_k^{s_k}$, where $p_1, \ldots, p_k$ are (not necessarily distinct) primes in $R$ and $s_1, \ldots, s_k$ are (not necessarily distinct) positive integers. The rank of E and the list of ideals $\left(p_1^{s_1}\right), \ldots,\left(p_k^{s_k}\right)$ are uniquely determined by $M$.
    \end{enumerate}
    The elements $r_1, \ldots, r_t$ are called the \textbf{invariant factors} of the module $M$ just as in the special case of abelian groups. Similarly $p_1^{s_1}, \ldots, p_k^{s_k}$ are called the \textbf{elementary divisors} of $M$.
\end{theorem}

