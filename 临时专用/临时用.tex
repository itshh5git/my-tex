\documentclass[12pt]{article}
\usepackage{amsmath, amssymb, amsthm, stmaryrd}
\usepackage{mathrsfs}
\usepackage{enumitem}
\usepackage{hyperref}

\title{A Functorial and G-Set Theoretic Formulation of the Fundamental Theorem of Galois Theory}
\author{}
\date{}

\theoremstyle{definition}
\newtheorem{definition}{Definition}[section]
\newtheorem{theorem}[definition]{Theorem}
\newtheorem{proposition}[definition]{Proposition}
\newtheorem{lemma}[definition]{Lemma}
\newtheorem{corollary}[definition]{Corollary}

\begin{document}

\maketitle

\tableofcontents

\section{Introduction}

This note formulates the classical Fundamental Theorem of Galois Theory in 
a modern categorical language, and further describes the Galois correspondence 
via $G$-sets.  
We assume throughout that $L/K$ is a finite Galois extension with Galois group
\[
G = \operatorname{Gal}(L/K).
\]

The goal is to reinterpret:
\[
\{ K \subseteq F \subseteq L \}
\quad \longleftrightarrow \quad
\{ H \le G \}
\]
as an equivalence between two categories and then as a correspondence 
between $G$-equivariant quotients of a canonical $G$-set.

\section{Intermediate Fields and Subgroups as Categories}

\subsection{The category of intermediate fields}

\begin{definition}
Let $\mathbf{Int}(L/K)$ be the category whose objects are intermediate fields
\[
K \subseteq F \subseteq L,
\]
and where there is a unique morphism $F_1 \to F_2$ iff $F_1 \subseteq F_2$.
Thus $\mathbf{Int}(L/K)$ is a poset category.
\end{definition}

\subsection{The category of subgroups}

\begin{definition}
Let $\mathbf{Sub}(G)$ be the category whose objects are subgroups $H \le G$
and where there is a unique morphism $H_1 \to H_2$ iff $H_1 \subseteq H_2$.
This is also a poset category.
\end{definition}

\section{Two Fundamental Contravariant Functors}

\subsection{The fixed field functor}

\begin{definition}
Define the functor
\[
\operatorname{Fix} : \mathbf{Sub}(G)^{op} \to \mathbf{Int}(L/K)
\]
by sending a subgroup $H \le G$ to its fixed field
\[
L^{H} = \{ x \in L : \sigma(x)=x \;\text{for all }\sigma\in H \}.
\]
If $H_1 \subseteq H_2$, then the morphism in $\mathbf{Sub}(G)$
is reversed in $\mathbf{Sub}(G)^{op}$, yielding the inclusion
\[
L^{H_1} \subseteq L^{H_2}.
\]
\end{definition}

\subsection{The Galois group functor}

\begin{definition}
Define the functor
\[
\operatorname{Gal} : \mathbf{Int}(L/K)^{op} \to \mathbf{Sub}(G)
\]
by
\[
\operatorname{Gal}(F) = \operatorname{Gal}(L/F).
\]
If $F_1 \subseteq F_2$, then restriction gives
\[
\operatorname{Gal}(L/F_1) \subseteq \operatorname{Gal}(L/F_2).
\]
\end{definition}

\section{The Categorical Fundamental Theorem of Galois Theory}

\begin{theorem}[Categorical Galois Correspondence]
There is a contravariant equivalence of categories
\[
\operatorname{Fix} : \mathbf{Sub}(G)^{op}
   \rightleftarrows 
\mathbf{Int}(L/K) : \operatorname{Gal}.
\]
Moreover,
\[
\operatorname{Fix} \circ \operatorname{Gal}
   \cong \mathrm{id}_{\mathbf{Int}(L/K)}, 
\qquad
\operatorname{Gal} \circ \operatorname{Fix}
   \cong \mathrm{id}_{\mathbf{Sub}(G)}.
\]
\end{theorem}

\section{Galois Correspondence via G-Sets}

\subsection{The canonical $G$-set}

Let
\[
X = \operatorname{Hom}_{K\text{-alg}}(L, \overline{K})
\]
where $\overline{K}$ is a fixed algebraic closure.

\begin{proposition}
The group $G = \operatorname{Gal}(L/K)$ acts on $X$ by
\[
(\sigma \cdot \varphi) = \sigma \circ \varphi.
\]
This makes $X$ into a finite transitive $G$-set.
\end{proposition}

\subsection{Intermediate fields correspond to quotients of this $G$-set}

For every intermediate field $F$, restriction gives a natural $G$-equivariant map:
\[
\operatorname{res}_{L/F} : X \to \operatorname{Hom}_K(F,\overline{K}).
\]

\begin{proposition}
There is a $G$-equivariant bijection
\[
\operatorname{Hom}_K(F,\overline{K}) 
\;\cong\;
X / H_F,
\qquad
H_F = \operatorname{Gal}(L/F).
\]
\end{proposition}

Thus:

\begin{theorem}[G-set formulation of Galois correspondence]
The category of intermediate fields satisfies
\[
\mathbf{Int}(L/K)
\;\simeq\;
\mathbf{G\text{-}Set}_X,
\]
the category of $G$-equivariant quotient maps $X \to Y$.
Every such quotient is isomorphic to $X/H$ for a unique subgroup $H \le G$.
\end{theorem}

\subsection{Conclusion}

Because $X$ is a transitive $G$-set, all its equivariant quotients are of the form 
$G/H$. Therefore,

\[
\mathbf{G\text{-}Set}_X
\;\simeq\;
\mathbf{Sub}(G)^{op}.
\]

Combining everything we obtain:

\begin{theorem}[Galois Correspondence via G-sets]
\[
\mathbf{Int}(L/K)
\;\simeq\;
\mathbf{Sub}(G)^{op}
\;\simeq\;
\mathbf{G\text{-}Set}_X.
\]

Each intermediate field $F$ corresponds to the subgroup
\[
H_F = \operatorname{Gal}(L/F)
\]
and to the $G$-set quotient
\[
\operatorname{Hom}_K(F,\overline{K})
  \cong X/H_F.
\]
\end{theorem}

\end{document}
