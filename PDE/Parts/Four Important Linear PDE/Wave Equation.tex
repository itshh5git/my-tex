\chapter{Wave Equation}
\section{Solution for n=1, d'Alembert's formula}
\begin{theorem}
    Assume $g \in C^2(\mathbb{R})$, $h \in C^1(\mathbb{R})$,
    \begin{equation*}
    \left\{\begin{aligned}
    u_{t t}-u_{x x}=0 & \text { in } \mathbb{R} \times(0, \infty) \\
    u=g, u_t=h & \text { on } \mathbb{R} \times\{t=0\}
    \end{aligned}\right.
    \end{equation*}
    We define $u$ by \textbf{d'Alembert's formula} 
    \begin{equation*}
    u(x, t)=\frac{1}{2}[g(x+t)+g(x-t)]+\frac{1}{2} \int_{x-t}^{x+t} h(y) d y \quad(x \in \mathbb{R}, t \geq 0)
    \end{equation*}
    Then

    (i) $u \in C^2(\mathbb{R} \times[0, \infty))$

    (ii) $u_{t t}-u_{x x}=0$ in $\mathbb{R} \times(0, \infty)$

    (iii) $\lim\limits_{\substack{(x, t) \rightarrow\left(x^0, 0\right) \\ t>0}} u(x, t)=g\left(x^0\right), \lim\limits_{\substack{(x, t) \rightarrow\left(x^0, 0\right) \\ t>0}} u_t(x, t)=h\left(x^0\right)$
    for each point $x^0 \in \mathbb{R}$.

    \textbf{Remark}  (i) Our solution $u$ has the form
    \begin{equation*}
    u(x, t)=F(x+t)+G(x-t)
    \end{equation*}
    for appropriate functions $F$ and $G$. Conversely any function of this form solves $u_{t t}-u_{x x}=0$.

    (ii) If $g \in C^k$ and $h \in C^{k-1}$, then $u \in C^k$ but is not in general smoother. 
\end{theorem}
\begin{theorem}[Reflection Method]
    Let us consider this initial/boundary-value problem on the half-line $\mathbb{R}_{+}=\{x>0\}$ :
    \begin{equation*}
    \left\{\begin{aligned}
    u_{t t}-u_{x x}=0 & \text { in } \mathbb{R}_{+} \times(0, \infty) \\
    u=g, u_t=h & \text { on } \mathbb{R}_{+} \times\{t=0\} \\
    u=0 & \text { on }\{x=0\} \times(0, \infty)
    \end{aligned}\right.
    \end{equation*}
    where $g, h$ are given, with $g(0)=h(0)=0$ and $g^{\prime \prime}(0)=0$.

    Proof:
    We set
    \begin{equation*}
    \begin{aligned}
    \tilde{u}(x, t) & := \begin{cases}u(x, t) & (x \geq 0, t \geq 0) \\
    -u(-x, t) & (x \leq 0, t \geq 0)\end{cases} \\
    \tilde{g}(x) & := \begin{cases}g(x) & (x \geq 0) \\
    -g(-x) & (x \leq 0)\end{cases} \\
    \tilde{h}(x) & := \begin{cases}h(x) & (x \geq 0) \\
    -h(-x) & (x \leq 0)\end{cases}
    \end{aligned}
    \end{equation*}
    Then (9) becomes
    \begin{equation*}
    \begin{cases}\tilde{u}_{t t}=\tilde{u}_{x x} & \text { in } \mathbb{R} \times(0, \infty) \\ \tilde{u}=\tilde{g}, \tilde{u}_t=\tilde{h} & \text { on } \mathbb{R} \times\{t=0\}\end{cases}
    \end{equation*}
    and $g\in C^2(\mathbb{R})$, $f\in C^1(\mathbb{R})$.
    Hence d'Alembert's formula implies
    \begin{equation*}
    \tilde{u}(x, t)=\frac{1}{2}[\tilde{g}(x+t)+\tilde{g}(x-t)]+\frac{1}{2} \int_{x-t}^{x+t} \tilde{h}(y) d y
    \end{equation*}
\end{theorem}

\section{}
\begin{theorem}[Spherical means; Euler-Poisson-Darboux equation]
    Now suppose $n \geq 2, m \geq 2$, and $u \in C^m\left(\mathbb{R}^n \times [0, \infty) \right) $ solves the initial-value problem
    \begin{equation*}
    \begin{cases}u_{t t}-\Delta u=0 & \text { in } \mathbb{R}^n \times(0, \infty) \\ u=g, u_t=h & \text { on } \mathbb{R}^n \times\{t=0\}\end{cases}
    \end{equation*}
    Let $x \in \mathbb{R}^n, t>0, r>0$. Define
    \begin{equation*}
    \left\{\begin{aligned}
        U(x ; r, t)& :=\fint_{\partial B(x, r)} u(y, t) d S(y)\\
        G(x ; r) & :=\fint_{\partial B(x, r)} g(y) d S(y) \\
        H(x ; r) & :=\fint_{\partial B(x, r)} h(y) d S(y)
    \end{aligned}\right.
    \end{equation*}
    Fix $x \in \mathbb{R}^n$, then $U \in C^m\left(\overline{\mathbb{R}}_{+} \times[0, \infty)\right)$ and
    \begin{equation*}
    \left\{\begin{aligned}
    U_{t t}-U_{r r}-\frac{n-1}{r} U_r=0 & \text { in } \mathbb{R}_{+} \times(0, \infty) \\
    U=G, U_t=H & \text { on } \mathbb{R}_{+} \times\{t=0\}
    \end{aligned}\right.
    \end{equation*}

    Proof:
    1. First, we compute for $r>0$
    \begin{equation*}
    U_r(x ; r, t)=\frac{r}{n} \fint_{B(x, r)} \Delta_y u(y, t) d y
    \end{equation*}
    From this equality we deduce $\lim _{r \rightarrow 0^{+}} U_r(x ; r, t)=0$. We next differentiate it
    \begin{equation*}
    U_{r r}(x ; r, t)=\fint_{\partial B(x, r)} \Delta u d S+\left(\frac{1}{n}-1\right) \fint_{B(x, r)} \Delta u d y
    \end{equation*}
    Thus $\lim _{r \rightarrow 0^{+}} U_{r r}(x ; r, t)=\frac{1}{n} \Delta u(x, t)$. Using this formula, we can similarly compute $U_{r r r}$, etc., and so verify that $U \in C^m\left(\overline{\mathbb{R}}_{+} \times[0, \infty)\right)$.

    2. Continuing the calculation above, we see from that
    \begin{equation*}
    \begin{aligned}
    U_r & =\frac{r}{n} \fint_{B(x, r)} u_{t t}\d y\\
    & =\frac{1}{n \alpha(n)} \frac{1}{r^{n-1}} \int_{B(x, r)} u_{t t} d y
    \end{aligned}
    \end{equation*}
    Thus
    \begin{equation*}
    r^{n-1} U_r=\frac{1}{n \alpha(n)} \int_{B(x, r)} u_{t t} d y
    \end{equation*}
    and so
    \begin{equation*}
    \left(r^{n-1} U_r\right)_r  
    =
    \frac{1}{n \alpha(n)} \int_{\partial B(x, r)} u_{t t} d S 
    =r^{n-1} \fint_{\partial B(x, r)} u_{t t} d S=r^{n-1} U_{t t}
    \end{equation*}
\end{theorem}

\section{Solution for even n=2k+1}
\begin{lemma}\
    Let $\phi: \mathbb{R} \rightarrow \mathbb{R}$ be $C^{k+1}$. Then for $k=1,2, \ldots$

    (i) 
    \begin{equation*}
    \left(\frac{d^2}{d r^2}\right)\left(\frac{1}{r} \frac{d}{d r}\right)^{k-1}\left(r^{2 k-1} \phi(r)\right)=\left(\frac{1}{r} \frac{d}{d r}\right)^k\left(r^{2 k} \frac{d \phi}{d r}(r)\right)
    \end{equation*}

    (ii) 
    \begin{equation*}
    \left(\frac{1}{r} \frac{d}{d r}\right)^{k-1}\left(r^{2 k-1} \phi(r)\right)=\sum_{j=0}^{k-1} \beta_j^k r^{j+1} \frac{d^j \phi}{d r^j}(r)
    \end{equation*}
    where the constants $\beta_j^k(j=0, \ldots, k-1)$ are independent of $\phi$.
    
    (iii) $\beta_0^k=1 \cdot 3 \cdot 5 \cdots(2 k-1)$.
\end{lemma}
\begin{theorem}
    Now assume
    $n \geq 3$ is an odd integer
    and set
    \begin{equation*}
    n=2 k+1 \quad(k \geq 1)
    \end{equation*}
    Henceforth suppose $u \in C^{k+1}\left(\mathbb{R}^n \times[0, \infty)\right)$ solves the initial-value problem
    \begin{equation*}
    \begin{cases}u_{t t}-\Delta u=0 & \text { in } \mathbb{R}^n \times(0, \infty) \\ u=g, u_t=h & \text { on } \mathbb{R}^n \times\{t=0\}\end{cases}
    \end{equation*}
    Then the function $U$ defined is $C^{k+1}$.
    
    We write
    \begin{equation*}
    \left\{\begin{array}{l}
    \tilde{U}(r, t):=\left(\frac{1}{r} \frac{\partial}{\partial r}\right)^{k-1}\left(r^{2 k-1} U(x ; r, t)\right) \\
    \tilde{G}(r):=\left(\frac{1}{r} \frac{\partial}{\partial r}\right)^{k-1}\left(r^{2 k-1} G(x ; r)\right) \quad(r>0, t \geq 0) . \\
    \tilde{H}(r):=\left(\frac{1}{r} \frac{\partial}{\partial r}\right)^{k-1}\left(r^{2 k-1} H(x ; r)\right)
    \end{array}\right.
    \end{equation*}
    Then $\tilde{U}(r,t)$ solves the wave equation on the half-line. We have
    \begin{equation*}
    \left\{\begin{aligned}
    \tilde{U}_{t t}-\tilde{U}_{r r} & =0 & & \text { in } \mathbb{R}_{+} \times(0, \infty) \\
    \tilde{U}=\tilde{G}, \tilde{U}_t & =\tilde{H} & & \text { on } \mathbb{R}_{+} \times\{t=0\} \\
    \tilde{U} & =0 & & \text { on }\{r=0\} \times(0, \infty)
    \end{aligned}\right.
    \end{equation*}
    We conclude for $0 \leq r \leq t$ that
    \begin{equation*}
    \tilde{U}(r, t)=\frac{1}{2}[\tilde{G}(r+t)-\tilde{G}(t-r)]+\frac{1}{2} \int_{t-r}^{t+r} \tilde{H}(y) d y
    \end{equation*}
    
    But recall $u(x, t)=\lim _{r \rightarrow 0} U(x ; r, t)$. Furthermore Lemma 2(ii) asserts
    \begin{equation*}
    \begin{aligned}
    \tilde{U}(r, t) & =\left(\frac{1}{r} \frac{\partial}{\partial r}\right)^{k-1}\left(r^{2 k-1} U(x ; r, t)\right) \\
    & =\sum_{j=0}^{k-1} \beta_j^k r^{j+1} \frac{\partial^j}{\partial r^j} U(x ; r, t)
    \end{aligned}
    \end{equation*}
    and so
    \begin{equation*}
    \lim _{r \rightarrow 0} \frac{\tilde{U}(r, t)}{\beta_0^k r}=\lim _{r \rightarrow 0} U(x ; r, t)=u(x, t)
    \end{equation*}
    
    Thus (30) implies
    \begin{equation*}
    \begin{aligned}
    u(x, t) & =\frac{1}{\beta_0^k} \lim _{r \rightarrow 0}\left[\frac{\tilde{G}(t+r)-\tilde{G}(t-r)}{2 r}+\frac{1}{2 r} \int_{t-r}^{t+r} \tilde{H}(y) d y\right] \\
    & =\frac{1}{\beta_0^k}\left[\tilde{G}^{\prime}(t)+\tilde{H}(t)\right]
    \end{aligned}
    \end{equation*}

    Proof:
    If $r>0$,
    \begin{equation*}
    \begin{aligned}
    \tilde{U}_{r r} & =\left(\frac{\partial^2}{\partial r^2}\right)\left(\frac{1}{r} \frac{\partial}{\partial r}\right)^{k-1}\left(r^{2 k-1} U\right) \\
    & =\left(\frac{1}{r} \frac{\partial}{\partial r}\right)^k\left(r^{2 k} U_r\right) \\
    & =\left(\frac{1}{r} \frac{\partial}{\partial r}\right)^{k-1}\left[r^{2 k-1} U_{r r}+2 k r^{2 k-2} U_r\right] \\
    & =\left(\frac{1}{r} \frac{\partial}{\partial r}\right)^{k-1}\left[r^{2 k-1}\left(U_{r r}+\frac{n-1}{r} U_r\right)\right] \\
    & =\left(\frac{1}{r} \frac{\partial}{\partial r}\right)^{k-1}\left(r^{2 k-1} U_{t t}\right)=\tilde{U}_{t t}
    \end{aligned}
    \end{equation*}
    Using previous lemma (ii) conclude as well that $\tilde{U}=0$ on $\left\{r=0 \right\}$.
\end{theorem}

\section{Nonhomogenous Problem}
\begin{theorem}[Solution of nonhomogeneous wave equation]
    Assume that $n \geq 2$ and $f \in C^{[\frac{n}{2}]+1}\left(\mathbb{R}^n \times[0, \infty)\right)$. 
    \begin{equation*}
    \begin{cases}
        u_{t t}-\Delta u=f(x,t) & \text { in } \mathbb{R}^n \times(0, \infty) \\ u=0, u_t=0 & \text { on } \mathbb{R}^n \times\{t=0\}\end{cases}
    \end{equation*}
    Motivated by Duhamel's principle, we define $u=u(x, t ; s)$ to be the solution of
    \begin{equation*}
    \left\{\begin{aligned}
    u_{t t}(\cdot ; s)-\Delta u(\cdot ; s) & =0 & & \text { in } \mathbb{R}^n \times(s, \infty) \\
    u(\cdot ; s)=0, u_t(\cdot ; s) & =f(\cdot, s) & & \text { on } \mathbb{R}^n \times\{t=s\}
    \end{aligned}\right.
    \end{equation*}
    Now set
    \begin{equation*}
    u(x, t):=\int_0^t u(x, t ; s) d s \quad\left(x \in \mathbb{R}^n, t \geq 0\right)
    \end{equation*}
    Duhamel's principle asserts this is a solution of
    \begin{equation*}
    \left\{\begin{array}{cl}
    u_{t t}-\Delta u=f & \text { in } \mathbb{R}^n \times(0, \infty) \\
    u=0, u_t=0 & \text { on } \mathbb{R}^n \times\{t=0\}
    \end{array}\right.
    \end{equation*}
    and we have

    (i) $u \in C^2\left(\mathbb{R}^n \times[0, \infty)\right)$,

    (ii) $u_{t t}-\Delta u=f$ in $\mathbb{R}^n \times (0, \infty)$

    (iii) $\lim\limits_{\substack{(x, t) \rightarrow\left(x^0, 0\right) \\ x \in \mathbb{R}^n, t>0}} u(x, t)=0$, $\lim\limits_{\substack{(x, t) \rightarrow\left(x^0, 0\right) \\ x \in \mathbb{R}^n, t>0}} u_t(x, t)=0$ for each point $x^0 \in \mathbb{R}^n$.
    
    Proof. 
    1. If $n$ is odd, $\left[\frac{n}{2}\right]+1=\frac{n+1}{2}$. According to Theorem 2, we have $u(\cdot, \cdot ; s) \in C^2\left(\mathbb{R}^n \times[s, \infty)\right)$ for each $s \geq 0$, and so $u \in C^2\left(\mathbb{R}^n \times[0, \infty)\right)$. If $n$ is even, $\left[\frac{n}{2}\right]+1=\frac{n+2}{2}$. Hence $u \in C^2\left(\mathbb{R}^n \times[0, \infty)\right)$, according to Theorem 3 .

    2. We then compute respectively
    \begin{equation*}
    u_t(x, t)
    =u
    (x, t ; t)+\int_0^t u_t(x, t ; s) d s=\int_0^t u_t(x, t ; s) ds
    \end{equation*}
    and
    \begin{equation*}
    u_{t t}(x, t)=u_t(x, t ; t)+\int_0^t u_{t t}(x, t ; s) d s=f(x, t)+\int_0^t u_{t t}(x, t ; s) d s
    \end{equation*}
    Furthermore
    \begin{equation*}
    \Delta u(x, t)=\int_0^t \Delta u(x, t ; s) d s=\int_0^t u_{t t}(x, t ; s) d s
    \end{equation*}
    Thus
    \begin{equation*}
    u_{t t}(x, t)-\Delta u(x, t)=f(x, t) \quad\left(x \in \mathbb{R}^n, t>0\right)
    \end{equation*}
    and clearly $u(x, 0)=u_t(x, 0)=0$ for $x \in \mathbb{R}^n$.
\end{theorem}

\section{Energy Methods}
\begin{theorem}[Uniqueness for wave equation] 
    Let $U \subset \mathbb{R}^n$ be a bounded, open set with a smooth boundary $\partial U$. There exists at most one function $u \in C^2\left(\bar{U}_T\right)$ solving the initial/boundary-value problem
    \begin{equation*}
    \left\{\begin{aligned}
    u_{t t}-\Delta u=f & \text { in } U_T \\
    u=g & \text { on } \Gamma_T \\
    u_t=h & \text { on } U \times\{t=0\}
    \end{aligned}\right.
    \end{equation*}

    Proof:
    If $\tilde{u}$ is another such solution, then $w:=u-\tilde{u}$ solves
    \begin{equation*}
    \left\{\begin{aligned}
    w_{t t}-\Delta w=0 & \text { in } U_T \\
    w=0 & \text { on } \Gamma_T \\
    w_t=0 & \text { on } U \times\{t=0\}
    \end{aligned}\right.
    \end{equation*}
    Define the energy
    \begin{equation*}
    E(t):=\frac{1}{2} \int_U w_t^2(x, t)+|D w(x, t)|^2 \d x \quad(0 \leq t \leq T)
    \end{equation*}
    We compute
    \begin{equation*}
    \begin{aligned}
    \dot{E}(t) & =\int_U w_t w_{t t}+D w \cdot D w_t \d x \\
    & =\int_U w_t\left(w_{t t}-\Delta w\right) \d x=0
    \end{aligned}
    \end{equation*}
    There is no boundary term since $w=0$, and hence $w_t=0$, on $\partial U \times[0, T]$.
\end{theorem}
\begin{theorem}[Finite propagation speed]
    Fix $x_0 \in \mathbb{R}^n, t_0>0$ and consider the backwards wave cone with apex $\left(x_0, t_0\right)$
    \begin{equation*}
    K\left(x_0, t_0\right)
    :=
    \left\{
        (x, t):0 \leq t \leq t_0, \left|x-x_0\right| \leq t_0-t 
    \right\} 
    \end{equation*}
    If $u \equiv u_t \equiv 0$ on $B\left(x_0, t_0\right) \times$ $\{t=0\}$, then $u \equiv 0$ within the cone $K\left(x_0, t_0\right)$.
    
    Proof: 
    Define the local energy
    \begin{equation*}
    e(t):=\frac{1}{2} \int_{B\left(x_0, t_0-t\right)} u_t^2(x, t)+|D u(x, t)|^2 \d x \quad\left(0 \leq t \leq t_0\right) .
    \end{equation*}
    Then
    \begin{equation*}
    \begin{aligned}
    \dot{e}(t)= & \int_{B\left(x_0, t_0-t\right)} u_t u_{t t}+D u \cdot D u_t \d x-\frac{1}{2} \int_{\partial B\left(x_0, t_0-t\right)} u_t^2+|D u|^2 d S \\
    = & \int_{B\left(x_0, t_0-t\right)} u_t\left(u_{t t}-\Delta u\right) \d x \\
    & +\int_{\partial B\left(x_0, t_0-t\right)} \frac{\partial u}{\partial \nu} u_t d S-\frac{1}{2} \int_{\partial B\left(x_0, t_0-t\right)} u_t^2+|D u|^2 d S \\
    = & \int_{\partial B\left(x_0, t_0-t\right)} \frac{\partial u}{\partial \nu} u_t-\frac{1}{2} u_t^2-\frac{1}{2}|D u|^2 d S \leq 0
    \end{aligned}
    \end{equation*}
    by the Cauchy-Schwarz and Cauchy inequalities.  We find $\dot{e}(t) \leq 0$; and so $e(t) \leq e(0)=0$ for all $0 \leq t \leq t_0$. Thus $u_t$, $D u \equiv 0$, and consequently $u \equiv 0$ within the cone $K\left(x_0, t_0\right)$.
\end{theorem}
