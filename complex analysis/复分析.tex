\documentclass[12pt, a4paper, oneside]{book}


\usepackage{../mypackages}


\begin{document}
\frontmatter
\title{{\Huge{\textbf{Complex Analysis}}}}
\maketitle
\dominitoc % 初始化minitoc
\pagenumbering{Roman}
\tableofcontents % 主目录


\mainmatter
\pagenumbering{arabic} % 正文编页码字体 







\part{Basic Theory}
\chapter{Preliminaries to Complex Analysis}
\section{Power series}
\subsection{Formal Power Series}
\begin{definition}
    A \textbf{formal power series} of a neutral letter $T$ over $\mathbb{C}$ is an expansion of the form
    \begin{equation*}
        f(T)=\sum_{n=0}^{\infty}a_n T^n
    \end{equation*}
    where $a_n\in\mathbb{C}$.
\end{definition}
\begin{definition}
    Suppose a power series is of the form
    \begin{equation*}
        f=a_r T^r+a_{r+1} T^{r+1}+\cdots
    \end{equation*}
    and $a_r \neq 0$. Thus $r$ is the smallest integer $n$ such that $a_n \neq 0$. Then we call $r$ the \textbf{order} of $f$, and write $r=\operatorname{ord} f$.
\end{definition}

\begin{proposition}
    Suppose that $f$ and $g\in \mathbb{C}[[T]]$.
    Then
    $\operatorname{ord}f g=\operatorname{ord} f+\operatorname{ord} g$.
\end{proposition}
\begin{corollary}
    A formal power series $f\in \mathbb{C}[[T]]$ has an inverse iff $\operatorname{ord} f=0$.
\end{corollary}



\begin{theorem}\
    Given a power series $\sum\limits_{n=0}^{\infty }a_nz^n$, there exists $0\leq R\leq \infty$ such that:

    (i) If $\left|z\right|<R$ the series converges absolutely.

    (ii) If $\left|z\right|>R$ the series diverges. Moreover, $R$ is given by Hadamard's formula
    $$1/R=\lim\sup|a_n|^{1/n}$$
    The number $R$ is called the radius of convergence of the power series, and the region $B(0.R)$ the disc of convergence.
\end{theorem}
\begin{theorem}\
    The power series $\sum\limits_{n=0}^{\infty }a_nz^n$ defines a holomorphic function in its disc of convergence. The derivative of f is also a
    power series obtained by differentiating term by term the series for $f$, that is,
    $$\sum\limits_{n=0}^{\infty }na_nz^{n-1}$$
    Moreover, $f$ has the same radius of convergence as $f$.
\end{theorem}
\begin{corollary}
    A power series $f$ is infinitely complex differentiable in its disc of convergence, and the higher derivatives are also power series obtained by termwise differentiation.
\end{corollary}
\begin{definition}
    A function $f$ defined on an open set $\Omega$ is said to be analytic (or have a power series expansion) at a point $z_0\in\Omega$ if there exists a power series $\sum a_n(z-z_0)^n$ centered at $z_0$, with positive radius of convergence $\delta$, such that
    $$f(z)=\sum_{n=0}^{\infty}a_n (z-z_0)^n,\quad z\in B(z_0,\delta)\subset\Omega$$

    If $f$ has a power series expansion at every point in $\Omega$, we say that $f$ is analytic on $\Omega$.
\end{definition}


\section{Differential Functions}

The letter $\Omega$ will from now on denote a plane open set.
\begin{definition}
    Suppose $f$ is a complex function defined in $\Omega$. If $z_0 \in \Omega$ and if
    \begin{equation*}
        \lim_{z \rightarrow z_0} \frac{f(z)-f\left(z_0\right)}{z-z_0}
    \end{equation*}
    exists, we denote this limit by $f^{\prime}\left(z_0\right)$ and call it the \textbf{derivative} of $f$ at $z_0$. If $f^{\prime}\left(z_0\right)$ exists for every $z_0 \in \Omega$, we say that $f$ is \textbf{holomorphic} (or analytic) in $\Omega$. The class of all holomorphic functions in $\Omega$ will be denoted by $H(\Omega)$.
\end{definition}

\begin{theorem}
    If $f$ is representable by power series in $\Omega$, then $f \in H(\Omega)$ and $f^{\prime}$ is also representable by power series in $\Omega$. In fact, if
    \begin{equation*}
        f(z)=\sum_{n=0}^{\infty} c_n(z-a)^n
    \end{equation*}
    for $z \in D(a ; r)$, then for these $z$ we also have
    \begin{equation*}
        f^{\prime}(z)=\sum_{n=1}^{\infty} n c_n(z-a)^{n-1}
    \end{equation*}
\end{theorem}

\begin{theorem}
    Suppose $\left(X,\mu\right)$ is a complex (finite) measure space, $\varphi$ is a complex measurable function on $X$, $\Omega$ is an open set in the plane which does not intersect $\varphi(X)$, and
    \begin{equation*}
        f(z)
        =
        \int_X \frac{\d \mu(\zeta)}{\varphi(\zeta)-z} \quad(z \in \Omega)
    \end{equation*}
    Then $f$ is representable by power series in $\Omega$.
    \begin{proof}
        Suppose $D(a ; r) \subset \Omega$. Since
        \begin{equation*}
            \left|\frac{z-a}{\varphi(\zeta)-a}\right| \leq \frac{|z-a|}{r}<1
        \end{equation*}
        for every $z \in D(a ; r)$ and every $\zeta \in X$, the geometric series
        \begin{equation*}
            \sum_{n=0}^{\infty} \frac{(z-a)^n}{(\varphi(\zeta)-a)^{n+1}}=\frac{1}{\varphi(\zeta)-z}
        \end{equation*}
        converges uniformly on $X$, for every fixed $z \in D(a ; r)$. Hence the series (3) may be substituted into (1), and $f(z)$ may be computed by interchanging summation and integration. It follows that
        \begin{equation*}
            f(z)=\sum_0^{\infty} c_n(z-a)^n \quad(z \in D(a ; r))
        \end{equation*}
        where
        \begin{equation*}
            c_n=\int_X \frac{d \mu(\zeta)}{(\varphi(\zeta)-a)^{n+1}} \quad(n=0,1,2, \ldots)
        \end{equation*}

        Note: The convergence of the series (4) in $D(a ; r)$ is a consequence of the proof. We can also derive it from (5), since (5) shows that
        \begin{equation*}
            \left|c_n\right| \leq \frac{|\mu|(X)}{r^{n+1}} \quad(n=0,1,2, \ldots) .
        \end{equation*}
    \end{proof}
\end{theorem}











\section{Integration along curve}

\begin{definition}
    Let 
\end{definition}













\section{Integration along curve}
\begin{definition}
    Suppose $f$ is a function on the open set $\Omega$. A \textbf{primitive} for $f$ on $\Omega$ is a function $F$ that is holomorphic on $\Omega$ and such that $F'(z)=f(z)$ for all $z\in\Omega.$
\end{definition}
\begin{theorem}
    If a continuous function $f$ has a primitive $F$ in $\Omega$, and
    $\gamma$ is a curve in $\Omega$ that begins at $w_1$ and ends at $w_2$, then
    $$\int_\gamma\: f(z)\d z=F(w_2)-F(w_1)$$

    Proof: If $\gamma$ is smooth, let  $z(t):[a,b]\longrightarrow C$ is a parametrization for $\gamma$, then $z(a)=w_1$ and $z(b)=w_2$, and we have
    $$\begin{aligned}
            \int_{\gamma}\:f(z) \d z & =\int_a^bf(z(t))z'(t)\d t                  \\
                                     & =\int_a^bF^{\prime}(z(t))z^{\prime}(t)\d t \\
                                     & =F(z(b))-F(z(a))                           \\
                                     & =F(w_2)-F(w_1)
        \end{aligned}$$
\end{theorem}
\begin{corollary}
    If $\gamma$ is a closed curve in an open set $\Omega$, and $f$ is continuous and has a primitive in $\Omega$, then
    $$\int_\gamma f(z)\d z=0$$
\end{corollary}
\begin{corollary}
    If $f$ is holomorphic in a region and $ f'=0 $, then  $f$ is constant.

    Proof: Since $\Omega$ is connected, for any $w\in\Omega$, there exists a curve $\gamma$ which joins
    $w_0$ to $w$. Since $f$ is clearly a primitive for $f'$, we have
    $$f(w)-f(w_0)=\int_{\gamma}f'(z)\d z=0$$
    we conclude that $f(w)=f(w_0)$ foe any $w\in\Omega$ as desired.
\end{corollary}

\section{Cauchy Theorems}
\begin{theorem}[Goursat's]
    If $\Omega$ is an open set in $\mathbb{C}$, and $T\subset\Omega$ a triangle whose interior is also contained in $\Omega$, then
    $$\int_T f(z)\d z=0$$
    whenever $f$ is holomorphic in $\Omega$
\end{theorem}
\begin{theorem}
    A holomorphic function in an open disc has a primitive.

    Proof:
    After a translation, we may assume without loss of generality that the disc, say $D$, is centered at the origin.

    Define
    \[
        F(z)=\int_{\gamma_z}f(\zeta)\d \zeta
    \]
    We contend that $F$ is holomorphic in $D$ and $F'=f$. To prove this, fix $z\in D$ and let $h\in\mathbb{C}$ be so small that $z + h$ also belongs to the disc. Now consider the difference
    $$\begin{aligned}
            \left|\frac{F(z+h)-F(z)-f(z)h}{h}\right| &
            =\left|\frac{\int_{\gamma_{z+h}}f(\zeta)\d \zeta-\int_{\gamma_z}f(\zeta)\d \zeta-f(z)h}{h}\right|                               \\
                                                     & =\left|\frac{\int_{z\to z+h}f(\zeta+z)\d \zeta-f(z)h}{h}\right|\quad\text{(Goursat)} \\
                                                     & =\left|\frac{\int_{z\to z+h}f(\zeta+z)-f(z)\d \zeta}{h}\right|                       \\
                                                     & \to 0
        \end{aligned}$$
    as $h\to 0$, thereby proving that $F$ is a primitive for $f$ on the disc.
\end{theorem}
\begin{theorem}[Cauchy's Theorem for a Disk]
    If $f$ is holomorphic in a simply connected region, then
    $$\int_\gamma\: f(z)\d z=0$$
    for any closed curve $\gamma$ in that disk.
\end{theorem}

\section{Cauchy'S Integral Formula}
\begin{theorem}[Cauchy integral formula]
    Suppose $f$ is holomorphic in an open set that contains the closure of a disc $D$, If $C$ denotes $\partial D$ withthe positive orientation, then
    $$
        f(z)=\frac{1}{2\pi i}\int_{C}\frac{f(\zeta)}{\zeta-z}\d \zeta
    $$
    for any point $z\in D$.

    Proof:
    By Cauchy's theorem, we claim that
    $$\frac{1}{2\pi i}\int_{C}\frac{f(\zeta)}{\zeta-z}\d \zeta=\frac{1}{2\pi i}\int_{\partial B(z,\delta)}\frac{f(\zeta)}{\zeta-z}\d \zeta$$
    for any small $\delta$ that $B(z,\delta)$ contained in $D$. Then
    $$\begin{aligned}
            \frac{1}{2\pi i}\int_{\partial B}\frac{f(\zeta)}{\zeta-z}\d \zeta & =\frac{1}{2\pi i}\int_{\partial B}\frac{f(\zeta)-f(z)-f'(z)(\zeta-z)}{\zeta-z}+f'(z)+\frac{f(z)}{\zeta-z}\d \zeta \\
                                                                              & =f(z)+\frac{1}{2\pi i}\int_{\partial B}\frac{f(\zeta)-f(z)-f'(z)(\zeta-z)}{\zeta-z}\d \zeta                       \\
                                                                              & \to f(z)
        \end{aligned}$$
    as $\delta\to 0$.
\end{theorem}
\begin{corollary}[Regularity theorem]
    If $f$ is holomorphic in an open set $\Omega$, then $f$ has infnitely many complex derivatives in $\Omega$. Moreover, if $C\subset \Omega$ is a circle whose interior is also contained in $\Omega$, then
    $$f^{(n)}(z)=\frac{n!}{2\pi i}\int_C\frac{f(\zeta)}{(\zeta-z)^{n+1}}\d \zeta $$
    for all z in the interior of $C$.
\end{corollary}

\section{Application}
\begin{theorem}[Morera]
    Suppose $f$ is a continuous function in the open disc $D$ such that for any triangle $T$ contained in $D$
    $$\int_T\:f(z)\d z=0$$
    then $f$ is holomorphic in $D$

    Proof: The function $f$ has a primitive $F$ in $D$ that satisfies $F'=f$. By the regularity theorem, we know that $F$ is indefinitely complex differentiable, and therefore $f$ is holomorphic.
\end{theorem}
\begin{theorem}[Taylor's Theorem]
    Suppose $f$ is holomorphic in an open set $\Omega$. If $D$ is a disc centered at $z_0$ and $D \subset \subset \Omega$, then $f$ has a power series expansion at $z_0$
    $$f(z)=\sum_{n=0}^\infty a_n(z-z_0)^n\quad \text{ for all }z\in D$$
    and the coefficients are given by
    $$a_n=\frac{f^{(n)}(z_0)}{n!}$$
    for all $n\geq 0$.

    Proof:
    Fix $z\in D$. By the Cauchy integral formula, we have
    $$\begin{aligned}
            f(z) & =\frac{1}{2\pi i}\int_C\frac{f(\zeta)}{\zeta-z}\d \zeta                                                     \\
                 & =\frac{1}{2\pi i}\int_C\frac{f(\zeta)}{\zeta-z_0}\frac{1}{1-\left(\frac{z-z_0}{\zeta-z_0}\right)}\d \zeta   \\
                 & =\frac{1}{2\pi i}\int_C\frac{f(\zeta)}{\zeta-z_0}\sum_{n=0}^\infty\left(\frac{z-z_0}{\zeta-z_0}\right)^n    \\
                 & =\sum_{n=0}^\infty\left(\frac{1}{2\pi i}\int_C\frac{f(\zeta)}{(\zeta-z_0)^{n+1}}d\zeta\right)\cdot(z-z_0)^n
        \end{aligned}$$
    Since $\zeta\in C$ and $z\in D$ is fixed, there exists $0 <r< 1$ such that $\left|\frac{z-z_0}{\zeta-z_0}\right|<r<1$, therefore $\sum_{n=0}^\infty\left(\frac{z-z_0}{\zeta-z_0}\right)^n$ converges uniformly. This allows us to interchange the infinite sum with the integral.
\end{theorem}
\begin{corollary}[Cauchy inequalities]
    If $f$ is holomorphic in an open set that contains $\overline{B(z_0,R)}$, then
    $$\left|f^{(n)}(z_0)\right| \leq \frac{n!}{R^n}\Vert f\Vert_{L^{\infty}(\partial B(z_0,R))}$$
\end{corollary}
\begin{corollary}[Liouville's theorem]
    If $f$ is entire and bounded, then $f$ is constant.
\end{corollary}
\begin{corollary}
    Let non-constant polynomial $P(z)$ in $\mathbb{C}[z]$ of degreee $n\geq 1$.\\
    (i) $P(z)$ has a roots in $\mathbb{C}$.\\
    (ii) $p(z)$ has precisely $n$ roots in $\mathbb{C}$.\\
    (ii) If these roots are denoted by $w_1,\ldots,w_n$, then  $P$ can be factored as
    $$P(z)=a_n(z-w_1)(z-w_2)\cdots(z-w_n)$$

\end{corollary}

\subsection{On compact set}
\begin{theorem}
    If $f$ is holomorphic in a open set $\Omega$, let
    $$\Omega_{\delta}
        =\{z\in \Omega:\overline{B(z,\delta)}\subset\Omega\}
        =\{d(z,\partial\Omega)>\delta\}$$
    then
    $$\sup_{z\in\Omega_\delta}\left|F^{\prime}(z)\right|\leq\frac{1}{\delta}\sup_{\zeta\in\Omega}\left|F(\zeta)\right|$$

    Proof: Since for every $z\in\Omega_\delta$, $\overline{B(z,\delta)}$ is contained in $\Omega$. Then
    $$F'(z)=\frac{1}{2\pi i}\int_{\partial B(z,\delta)}\:\frac{F(\zeta)}{(\zeta-z)^2}\d \zeta$$
    Hence
    $$\begin{aligned}
            \left|F'(z)\right| & \leq\frac{1}{2\pi}\int_{\partial B(z,\delta)}\frac{\left|F'(z)\right|}{\left|\zeta-z\right|^2}\left|d\zeta\right| \\
                               & \leq\frac{1}{2\pi}\sup_{\zeta\in\Omega}\left|F(z)\right|\frac{1}{\delta^2}\:2\pi\delta                            \\
                               & =\frac{1}{\delta}\sup_{\zeta\in\Omega}\left|F(z)\right|                                                           \\
        \end{aligned}$$
    as was to be shown.
\end{theorem}
\begin{theorem}[Weierstrass's]
    If $\{f_n\}_{n=1}^\infty$ is a sequence of holomorphic functions in $\Omega$ that converges uniformly to a function $f$ in every compact subset of $\Omega$, then $f$ is holomorphic in $\Omega$.

    Proof: Let $D$ be any disc whose closure is contained in $\Omega$ and $T$ any triangle in that disc. Then, since each $f_n$ is holomorphic, Goursat's theorem implies
    $$\int_T\:f_n(z)\d z=0\quad \text{for all } n$$
    By assumption $f_n\to f$ uniformly in the $\overline{D}$, so $f$ is continuous and
    $$\int_T\:f_n(z)\d z\to\int_T\:f(z)\d z$$
    As a result, we find $\int_T\:f(z)\d z=0$, and by Morera's theorem, we conclude that $f$ is holomorphic in $D$ .

    Since this conclusion is true for every $D$ whose closure is contained in $\Omega$, we find that $f$ is holomorphic in all of $\Omega$.
\end{theorem}

\begin{theorem}
    If $\{f_n\}_{n=1}^\infty$ is a sequence of holomorphic functions in $\Omega$ that converges uniformly to a function $f$ in every compact subset of $\Omega$, then the sequence of $k$-th derivatives $\{f_n^{(k)}\}$ converges uniformly to $f^{(k)}$ on every compact set of  $\Omega$ and $f^{(k)}$ is holomorphic in $\Omega$.

    Proof: We only need to prove $k=1$.

    Suppose $K$ is any compact subset of $\Omega$, then let $\varepsilon=d(K,\partial\Omega)-\delta$ that $\varepsilon>0$ and $\delta>0$. Since $K$ is compact and $K \subset \bigcup\limits_{z\in K} B(z,\varepsilon)\subset \Omega$, we can obtain an open set $V=B(z_1,\varepsilon)\cup B(z_2,\varepsilon)\cdots\cup B(z_t,\varepsilon)$ and $V\subset \Omega_\delta$. Then
    $$\sup_{z\in\Omega_\delta}\left|f_n'-f'\right|\leq\sup_{z\in\Omega}\frac{1}{\delta}\left|f_n-f\right|$$

\end{theorem}
\begin{theorem}
    Let $F(z, s)$ be defined for $(z, s)\in\Omega\times[0, 1]$ where $\Omega$ is an open set in $\mathbb{C}$. Suppose $F$ satisfies the following properties:\\
    (a) $F(z, s)$ is holomorphic in $z$ for each $s$.\\
    (b) $F$ is continuous on $\Omega\times[0, 1]$.\\
    Then the function $f$ defined on $\Omega$ by
    $$f(z)=\int_0^1 F(z,s)\d s$$
    is holomorphic.

    Proof: To prove this result, it suffices to prove that $f$ is holomorphic in any disc $D$ contained in $\Omega$, and by Morera's theorem this could be achieved
    by showing that for any triangle $T$ contained in $D$ we have
    $$\int_T\int_0^1F(z,s)\d sdz=0$$
    Interchanging the order of integration (Fubini theorem), and using property (a) would then yield the desired result.

    Abother proof: For each $n \geq 1$, we consider the Riemann sum
    $$f_n(z)=\frac{1}{n}\sum_{k=1}^nF(z,\frac{k}{n})$$
    Then $f_n$ is holomorphic in all of $\Omega$, and we claim that on any disc $D$ whose closure is contained in $\Omega$, the sequence $\{f_n\}_{n=1}^{\infty}$ converges uniformly to $f(z)=\int_0^1 F(z,s)\d s$. Then $f(z)$ is holomorphic in $D$. As a consequence, $f$ is holomorphic in $\Omega$, as was to be shown.
\end{theorem}
\begin{theorem}[Symmetry principle]
    If $f^+$ and $f^-$ are holomorphicfunctions in $\Omega^+=\Omega\cap\{z:Re(z)>0\}$ and $\Omega^-=\Omega\cap\{z:Re(z)<0\}$ respectively, that extend continuously to $I=\Omega\cap\{z:Re(z)=0\}$ and
    $$f^+(x)=f^-(x)\quad\textit{ for all }x\in I$$
    then the function $f$ defined on $\Omega$ by
    $$f=
        \begin{cases}
            f^+(z)        & ,z\in \Omega^+ \\
            f^+(z)=f^-(z) & ,z\in I        \\
            f^-(z)        & ,z\in \Omega^-
        \end{cases}$$
    is holomorphic on all of $\Omega$

    Proof:
\end{theorem}

\section{Local Properties of Analytic Function} % 2.5

\subsection{Zeros and Poles}
\begin{theorem}
    Suppose that $f$ is holomorphic in a region $\Omega$,  and does not vanish identically in $\Omega$. If $f(z_0)=0$, then there exists a neighborhood $U \subset \Omega$ of $z_0$,  a non-vanishing holomorphic functiong $g$ on $U$ and a unique positive integer $n$ such that
    $$
        f(z)=(z-z_0)^ng(z)
    $$
    for all $z\in U$. We say that $f$ has a \textbf{zero of order $n$} (or multiplicity $n)$ at $z_0$. If a zero is of order $1$, we say that it is \textbf{simple zero}.

        Proof:
        Since $\Omega$ is connected and $f$ is not identically zero, we conclude that $f$ is not identically zero in a neighborhood of $z_0$. (If not, then $f^{-1}(0)$ is open and closed)

        In a small disc centered at $z_0$ the function $f$ has a power series expansion by Taylor's theorem
        $$f(z)=\sum_{k=0}^\infty a_k(z-z_0)^k$$
        Since $f$ is not identically zero near $z_0$, there exists a smallest integer $n$ such that $a_n\neq 0$. Then, we can write
        $$f(z)=(z-z_0)^n[a_n+a_{n+1}(z-z_0)+\cdots]=(z-z_0)^ng(z)\:,$$
        where $g$ is defned by the series in brackets, and hence is holomorphic in this small disk, and is nowhere vanishing for all $z$ close to $z_0$.

        To prove the uniqueness of the integer $n$, suppose that we can also write
        $$f(z)=(z-z_0)^ng(z)=(z-z_0)^mh(z)$$
        where $h(z_0)\neq0.$ If $m>n$, then we may divide by $(z-z_0)^n$ to see that
        $$g(z)=(z-z_0)^{m-n}h(z)$$
        and letting $z\to z_0$ yields $g(z_0)=0$, a contradiction. If $m<n$ a similar argument gives $h(z_0)=0$, which is also a contradiction. We conclude that $m=n$, thus $h=g$, and the theorem is proved.
\end{theorem}

\begin{corollary}
    We can see that the zeros of analytic funcion which does not vanish identically are isolated.
\end{corollary}
\begin{corollary}[Uniqueness]
    If $f$ and $g$ are analytic in $\Omega$, and if $f(z)=g(z)$ on a set which has an convergence point in $\Omega$, then $f$ is identically equal to $g(z)$.
\end{corollary}

\subsection{Pole}

\begin{definition}
    We say that a function $f$ defned in a deleted neighborhood of $z_0$ has a \textbf{pole} at $z_0$, if the function
    \begin{equation*}
        g(z)
        =
        \begin{cases}
            \frac{1}{f(z)} & , \quad z\neq z_0 \\
            0              & ,\quad z=z_0      \\
        \end{cases}
    \end{equation*}
    is holomorphic in a full neighborhood of $z_0$.
\end{definition}

\begin{theorem}
    If $f$ has a pole at $z_0\in \Omega$ then in a neighborhood of $z_0$ there exist a non-vanishing holomorphic function $h$ and a unique positive integer n such that
    $$f(z)=(z-z_0)^{-n}h(z)$$
    The integer $n$ is called the \textbf{order (or multiplicity) of the pole}. If the pole is of order $1$, we say that it is \textbf{simple}.

    Proof:
    By the previous theorem we have $g(z)=(z-z_0)^nh_1(z)$, where $h_1$ is holomorphic and non-vanishing in a neighborhood $U$ of $z_0$, so the result follows with $h(z)=1/h_1(z)$ that is holomorphic in $U$.
\end{theorem}

\begin{theorem}
    If $f$ has a pole of order $n$ at $z_0$, then
    $$f(z)=\frac{a_{-n}}{(z-z_0)^n}+\frac{a_{-n+1}}{(z-z_0)^{n-1}}+\cdots+\frac{a_{-1}}{z-z_0}+G(z)$$
    where $G$ is a holomorphic function in a neighborhood of $z_0$

    Proof: The proof follows from the multiplicative statement in the previous theorem. Indeed, the function $h$ has a power series expansion in a neighbourhood of $z_0$ with $h(0)=A_0\neq 0$
    $$h(z)=A_0+A_1(z-z_0)+\cdots $$
    so that
    $$\begin{aligned}f(z)&=(z-z_0)^{-n}(A_0+A_1(z-z_0)+\cdots)\\&=\frac{a_{-n}}{(z-z_0)^n}+\frac{a_{-n+1}}{(z-z_0)^{n-1}}+\cdots+\frac{a_{-1}}{(z-z_0)}+G(z).\end{aligned}$$

    The sum
    $$\frac{a_{-n}}{(z-z_0)^n}+\frac{a_{-n+1}}{(z-z_0)^{n-1}}+\cdots+\frac{a_{-1}}{(z-z_0)}$$
    is called the \textbf{principal part of $f$ at the pole $z_0$}, and the coefficient $a_{-1}$ is the \textbf{residue of $f$ at that pole}. We write $Res_{z_0}f=a_{-1}$.
\end{theorem}

\begin{theorem}
    If $f$ has a pole of order $n$ at $z_0$, then
    $$
        \mathrm{Res}_{z_0}f =\lim_{z\to z_0}\frac{1}{(n-1)!}\left(\frac{d}{dz}\right)^{n-1}(z-z_0)^nf(z)
    $$
\end{theorem}

\subsection{Singularities}

\subsubsection{Removable Singularity}
\begin{definition}
    Let $f$ be a function holomorphic in an open set $\Omega$ except possibly at one point $z_0$\ in $\Omega$. If we can define $f$ at $z_0$ in such a way that $f$ becomes holomorphic in all of $\Omega$, we say that $z_0$ is a \textbf{removable singularity} for $f$
\end{definition}
\begin{theorem}[Riemann's theorem on removable singularities]
    Suppose that $f$ is holomorphic in an open set $\Omega$ except possibly at a point $z_0$ in $\Omega$. If $f$ is bounded on some deleted neighbourhood of $z_0$, then $z_0$ is a removable singularity.

    Proof: Since the problem is local we may consider a small disc $D$ centered at $z_0$ and whose closure is contained in $\Omega$. We shall prove that if $z
        \in D$ and $z\neq z_0$, then under the assumptions of the
    theorem we have
    $$f(z)=\frac{1}{2\pi i}\int_C\frac{f(\zeta)}{\zeta-z}\d \zeta\quad z\in D-\{z_0\}$$

    Since the right-hand side defines a holomorphic function on all of $D$ that agrees with $f(z)$ when $z\neq z_0$, this give us a (unique) desired extension of $f$.

    Fix $z\in D$ with $z\neq z_0$, we have
    $$\frac{1}{2\pi i}\int_C\frac{f(\zeta)}{\zeta-z}\d \zeta=\frac{1}{2\pi i}\int_{\partial B(z_0,\varepsilon)}\frac{f(\zeta)}{\zeta-z}\d \zeta+\frac{1}{2\pi i}\int_{\partial B(z_0,\varepsilon)}\frac{f(\zeta)}{\zeta-z}\d \zeta$$
    By Cauchy integral formula. We find that
    $$\frac{1}{2\pi i}\int_{\partial B(z,\varepsilon)}\frac{f(\zeta)}{\zeta-z}\d \zeta=f(z)$$
    and
    $$\left|\frac{1}{2\pi i}\int_{\partial B(z_0,\varepsilon)}\frac{f(\zeta)}{\zeta-z}\d \zeta\right|<C\delta$$
\end{theorem}
\begin{corollary}
    Suppose that $f$ has an isolated singularity at the point
    $z_0$. Then $z_0$ is a pole of $f$ if and only if $\left|f(z)\right| \to \infty$  as $z\to z_0$.

    Proof: If $z_0$ is a pole, then we know that $1/f$ has a zero at $z_0$, and therefore $\left|f(z)\right|\to\infty$ as $z\to z_0$.

    Conversely, suppose that this condition holds. Then $1/f$ is bounded near $z_0$. Therefore, $1/f$ has a removable singularity at $z_0$ and must vanish there. This proves the converse, namely that $z_0$ is a pole of $f$.
\end{corollary}

\subsubsection{Essential Singularity}

\begin{definition}
    Any singularity that is not removable or a pole is defined to be an \textbf{essential singularity}.
\end{definition}

\begin{theorem}[Casorati-Weierstrass]
    Suppose $f$ is holomorphic in the punctured disc $B(z_0,r)-\{z_0\}$ and has an essential singularity at $z_0$. Then, the image of $B(z_0,r)-\{z_0\}$ under $f$ is dense in the complex plane.

    Proof: We argue by contradiction. Assume that the range of $f$ is not dense, so that there exists $w\in\mathbb{C}$ and $\delta>0$ such that
    $$|f(z)-w|>\delta\quad\text{for all }z\in D_r(z_0)-\{z_0\}$$
    We may therefore define a new function on $D_r(z_0)-\{z_0\}$ by
    $$g(z)=\frac1{f(z)-w}$$
    which is holomorphic on the punctured disc and bounded by $1/\delta$. Hence $g$ has a removable singularity at $z_0$.

    If $g(z_0)\neq0$, then $f(z)-w=1/g(z)$ is holomorphic at $z_0$, which contradicts the assumption that $z_0$ is an essential singularity. In the case that $g(z_0)=0$, then $f(z)-w$ has a pole at $z_0$ also contradicting the nature of the singularity at $z_0.$ The proof is complete.
\end{theorem}

\begin{theorem}[Picard]

\end{theorem}


\subsection{}
\begin{definition}
    A function $f$ on an open set $\Omega$ is \textbf{meromorphic} if there exists a sequence of points $\{z_0,z_1,z_2,\ldots\}$ that has no limit points in $\Omega$, and such that\\
    (a) the function $f$ is holomorphic in $\Omega-\{z_0,z_1,z_2,\ldots\}$ \\
    (b) $f$ has poles at the points $\{z_0,z_1,z_2,\ldots\}$.
\end{definition}
\begin{definition}
    If $f$ is holomorphic in $B(\infty,R)=\{z:\left|z\right|\geq R\}$, we consider
    $$F(z)=f(1/z)$$
    which is holomorphic in a deleted neighborhood of the origin. We say that \textbf{$f$ has a pole(essential singularity, removable singularity) at infinity} if $F$ has a pole(essential singularity, removable singularity) at the origin.
\end{definition}
\begin{definition}
    A meromorphic function in the complex plane $\mathbb{C}$ that is either holomorphic at infnity or has a pole at infinity is said to be meromorphic in the extended complex plane $\overline{\mathbb{C}}$.
\end{definition}
\begin{theorem}
    The meromorphic functions in the extended complex plane
    are the rational functions.

    Proof: Suppose that $f$ is meromorphic in the extended plane, so $f$ can have only finitely many poles in the plane, say at $z_1,\cdots,z_n$. Near each pole $z_k\in \mathbb{C}$ we can write
    $$f=f_k+g_k$$
    where $f_k(z)$ is the principal part of $f$ at $z_k$ and $g_k$ is holomorphic in a neighborhood of $z_k$. In particular, $f_k$ is a polynomial in $1/(z-z_k)$. Similarly, we can write
    $$f(1/z)=\tilde{f}_\infty(z)+\tilde{g}_\infty(z)$$
    where $\tilde{g}_\infty$ is holomorphic in a neighborhood of the origin and $\tilde{f}_\infty$ is the principal part of $f(1/z)$ at $0$, that is, a polynomial in $1/z$. Finally, let $f_\infty(z)=\tilde{f}_\infty(z)$.

    We contend that the function $H=f-f_\infty-\sum f_k$ is entire and bounded. Indeed, near the pole $z_k$ we subtracted the principal part of $f$ so that the function $H$ has a removable singularity there. Also, $H(1/z)$ is bounded for $z$ near $0$ since we subtracted the principal part of the pole at $\infty$. This proves our contention, and by Liouville's theorem we
    conclude that $H$ is constant. From the definition of $H$, we find that f is a rational function, as was to be shown.
\end{theorem}

\section{The argument principle and applications}
\subsection{Argument Principle}


\begin{theorem}[Argument Principle]
    Suppose $f$ is meromorphic in $\Omega$, then
    \begin{equation*}
        n\left( f(\gamma),w \right)
        =
        \frac{1}{2\pi i}\int_\gamma \frac{f'(z)}{f(z)-w}\d z
        =
        \sum_j n(\gamma,z_j(w))-\sum_k n(\gamma,p_k)
    \end{equation*}
    for every closed curve $\gamma$ which is homologous to $0$ in $\Omega$ and does not pass  through any of zeros of $f(z)-w$ an poles.
\end{theorem}

\begin{corollary}
    Suppoes $f$ is analytic in a disk $\Delta $, then the number of roots of equation $f(z)=a$ in $\Delta$ is
    \begin{equation*}
        \frac{1}{2\pi i}\int_\gamma \frac{f'(z)}{f(z)-a}\d z
    \end{equation*}
\end{corollary}

\begin{corollary}
    Suppoes that $f(z)$ is analytic in sufficiently small  $B\left(z_0,\varepsilon\right)$ such that
    \begin{enumerate}[label=(\roman*)]
        \item $f(z)-f(z_0)$ has a only zero of order $n$ at $z_0$
        \item $f'(z)$ has a only zero at $z_0$
    \end{enumerate}
    Then there exists $\delta>0 $ such that for all $w \in B\left(f(z_0),\delta\right)$ the equation $f(z)-w$ has exactly $n$ different simple roots in $B\left(z_0,\varepsilon\right)$
\end{corollary}


\begin{corollary}[Open Maps]
    A nonconstant analytic funcion is a open map.
\end{corollary}

\begin{corollary}
    If $f(z)$ is analytic at $z_0$ with $f'(z_0)\neq 0$,
    it maps a neighborhood of $z_0$ conformally and homeomorphic onto a region
\end{corollary}



\begin{theorem}[Rouche' theorem ]
    Suppose that $f$ and $g$ are holomorphic in an open set containing a circle $C$ and its interior. If
    $$
        \left|f(z)-g(z)\right|
        <
        \left|f(z)\right|\quad \text{ for all }z\in C
    $$
    then $f$ and $g$ have the same number of zeros inside the circle $C$
\end{theorem}


\begin{theorem}[A.Hurwitz]
    If $f_n$ are analytic and $\neq 0$ in a region $\Omega$, and if $f_n$ converges to $f$ uniformly on every compact subset of $\Omega$. Then $f$ is either identically zero or never equal to zero in $\Omega$.

    Proof:
    Suppoes that $f$ is not identically zero.
    For any $z_0 \in \Omega$ there is therefore a $r$ such that $f$ is defined $\neq 0$ for $0 < \left|z-z_0\right| \leq r$, thus $\left|f\right|$ has a positive minimum on circle $\left|z-z_0\right|=r$.
    It follows that $\frac{1}{f_n}$ converges unifformly to $\frac{1}{f}$ on C. And since $f'$ converges unifformly to $f'$ on C.

    We may conclude that
    \begin{equation*}
        \frac{1}{2 \pi \i} \int_C \frac{f'(z)}{f(z)} \d z
        =
        \lim_{n \to \infty}\frac{1}{2 \pi \i} \int_C \frac{f_n'(z)}{f_n(z)} \d z
        =
        0
    \end{equation*}
    Consequently, $f(z_0)\neq 0$.
\end{theorem}




\section{The General Form of Cauchy's Theorem}
\begin{theorem}
    If $f$ is holomorphic in $\Omega$ , then
    $$\int_{\gamma_0}f(z)\d z=\int_{\gamma_1}f(z)\d z$$
    whenever the two curves $\gamma _0$ and $\gamma _1$ are homotopic in $\Omega$.
\end{theorem}
\begin{theorem}
    Any holomorphic function in a simply connected domain has a primitive.
\end{theorem}






\section{The Calculus of Residues}
\begin{lemma}[Jordan]
    If $f$ is continuous and any $\alpha >0 $
    \begin{equation*}
        \lim_{\substack{z\to \infty \\ \Im(z)>0}} f(z) =0
    \end{equation*}
    Then
    \begin{equation*}
        \lim_{R \to \infty}
        \int_{\gamma_R} e^{\i \alpha z}f(z) \d z
        =
        0
    \end{equation*}
    where $\gamma_R = \left\{z: z=R e^{\i \theta}, 0 \leq \theta \leq \pi\right\}$
\end{lemma}
\begin{theorem}
    Suppose $f$ is meromorphic in $\Im z >0$ and
    For any $\alpha>0$,
    \begin{equation*}
        \int_{-\infty}^\infty e^{\i \alpha x} f(x) \d x
        =
        2 \pi \i \sum \res\left(e^{\i \alpha x} f(x), a_k\right)
    \end{equation*}
\end{theorem}








































\chapter{Entire Function}

\section{Infinite Products}
\begin{definition}
    An finite product of complex numbers
    $$p_1p_2\cdots p_n=\prod_{n=1}^\infty p_n$$
    is evaluated by taking the limit of partial product $P_n=p_1p_2\cdots p_n$. It said to converge to the $P=\lim P_n$ is this limit sxists and is different from zero.
\end{definition}
\begin{theorem}
    The infinite product $\prod\limits_1^{\infty}\left(1+a_n\right)$ with $1+a_n \neq 0$ converges simultaneously with the series $\sum\limits_1^{\infty} \log \left(1+a_n\right)$ whose terms represent the values of the principal branch of the logarithm.
\end{theorem}
\begin{definition}
    An infinite product $\prod\limits_1^{\infty}\left(1+a_n\right)$ is said to be absolutely convergent if the corresponding series $\sum\limits_1^{\infty}\log\left(1+a_n\right)$ converges absolutely.
\end{definition}
\begin{theorem}
    A necessary and sufficient condition for the absolute convergence of the product $\prod\limits_1^{\infty}\left(1+a_n\right)$ is the convergence of the series $\sum\limits_1^{\infty}\left|a_n\right|$ (if and only if $\sum \log(1+\left|a_n\right|)$ converge).
\end{theorem}
\begin{theorem}
    The value of an absolutely convergent product does not change if the factors are reordered
\end{theorem}
\begin{theorem}
    Suppose $\left\{g_n=1+f_n\right\}$ is a sequence of holomorphic functions on the open set $\Omega$. If there exist constants $c_n>0$ such that
    $$
        \sum c_n<\infty \quad \text { and } \quad\left|f_n(z)\right| \leq c_n \quad \text { for all } z \in \Omega
    $$
    then:

    (i) The product $\prod\limits_{n=1}^{\infty} (1+f_n)$ converges uniformly in $\Omega$ to a holomorphic function $G(z)$.

    (ii) If $g_n=1+f_n(z)$ does not vanish for any $n$, then
    $$
        \frac{G^{\prime}(z)}{G(z)}=\sum_{n=1}^{\infty} \frac{g_n^{\prime}(z)}{g_n(z)}
    $$

    Proof:
    $$G_n(z)=\prod_{i=1}^n g_i(z)=e^{\sum\limits^n\log(1+f_i(z))}$$
    converges uniformly to a holomorphic function

    To establish the second part of the theorem, suppose that $K$ is a compact subset of $\Omega$. We have just proved that $G_n \rightarrow F$ uniformly in $\Omega$, so the sequence $\left\{G_n^{\prime}\right\}$ converges uniformly to $F^{\prime}$ in $K$. Since $G_N$ is uniformly bounded from below on $K$ (it cannot be omitted), we conclude that $G_n^{\prime} / G_n \rightarrow$ $F^{\prime} / F$ uniformly on $K$.

    And because $K$ is an arbitrary compact subset of $\Omega$, the limit holds for every point of $\Omega$. Moreover, as we saw
    $$\frac{G_n^{\prime}}{G_n}=\sum_{i=1}^n \frac{g_i^{\prime}}{g_i}$$
    so part (ii) of the theorem is also proved.
\end{theorem}

\section{Jensen's formula}
\begin{lemma}
    Let $a\in \mathbb{C}$ and $\left|a\right| = 1$
    $$\int_0^{2\pi } \log \left|a-e^{i\theta}\right|\d \theta=\int_0^{2\pi } \log \left|1-e^{i\theta}\right|\d \theta=0$$
\end{lemma}

\begin{theorem}
    If $f$ is an analytic function, then $\log\left|f(z)\right|$ is harmonic except the zeros $f$. Therefore, if  $f(z)$ is analytic and free from zeros in $B(0,R)$ then
    $$\log\left|f(0)\right|
        =\frac1{2\pi}\int_0^{2\pi}\log\left|f(Re^{i\theta})\right|\d \theta$$
    and $\log\left|f(z)\right|$ can be expressed by Poission's formula.

    Proof:
    If $f$ has zeros the circle $\partial B(0,R)$. Denotes these zeros by $Re^{i\theta_k}$ ($k=1,2\cdots m$), multiple zeros being repeated, then define
    $$F(z)=f(z)\prod_{k=1}^m\frac{1}{z-Re^{i\theta_k}}$$
    is analytic and has on zeros in $\partial B(0,R)$.
    $$\log\left|f(0)\right|-m\log R=\frac1{2\pi}\int_0^{2\pi}\log\left|f(Re^{i\theta})\right|-\sum\log\left|Re^{i\theta}-Re^{i\theta_k}\right|\d \theta$$
\end{theorem}
\begin{theorem}[Jensen's formula]
    Let $\Omega$ be an open set that contains closed disc $\overline{B(0, \rho )}$ the
    and suppose that f is holomorphic in Ω, and vanish at $z_1,z_2,\cdots,z_n$ in $B(0,\rho)$, mutiple zeros being repeated, and assume that $f(0)\neq 0$. Then
    $$
        \log\left|f(0)\right|
        =
        -\sum_{k=1}^n\log\left(\frac{\rho}{\left|z_k\right|}\right)
        +
        \frac1{2\pi}\int_0^{2\pi}\log\left|f(\rho e^{i\theta})\right|\d \theta
    $$

    Proof:
    1. Let
    $$
        F(z)=f(z)\prod_{i=1}^n\frac{\rho^2-\overline{z_i} z} {\rho(z-z_i)}
    $$
    is free fron zeros in the disk $B(0,\rho)$, and $\left|F\right|=\left|f\right|$ on $\partial B(0,\rho)$. Consequently we obtain
    $$
        \log\left|F(0)\right|
        =\frac1{2\pi}\int_0^{2\pi}\log\left|F(\rho e^{i\theta})\right|\d \theta
    $$
    and, substituting the value
    $$
        \log \left|f(0)\right|+\sum\log\left(\frac{\rho}{\left|z_i\right|}\right)
        =
        \frac1{2\pi}\int_0^{2\pi}\log\left|f(\rho e^{i\theta})\right|\d \theta$$

    2. If $f(0)=0$ we write $g(z)=f(z)/z^h$, then
    $$
        \log\left|a_h\right|+\sum\log\left(\frac{\rho}{\left|z_i\right|}\right)
        =
        \frac1{2\pi}\int_0^{2\pi}\log\left|f(\rho e^{i\theta})\right|\d \theta -h\log\rho
    $$
\end{theorem}
\begin{theorem}[Poission-Jensen formula]
    Let $\Omega$ be an open set that contains closed disc $\overline{B(0, \rho)}$ the
    and suppose that f is holomorphic in Ω, and vanish at $z_1,z_2,\cdots,z_n$ in $B(0,\rho)$, mutiple zeros being repeated. Then

    Proof:
    If $\left|z_0\right|< \rho $ and $f(z_0)\neq 0$ we write
    $$
        \varphi_{\rho,z_0}(z)=\frac{\rho^2(z-z_0)}{\overline{z_0}z-\rho^2}
    $$
    then let
    $$h(z)=f\circ\varphi_{\rho,z_0}(z)$$
    that $h(0)=f(z_0)$ and $h$ vanish at $\varphi_{\rho,z_0}(z_i)$, then
    $$\begin{aligned}
            \log \left|f(z_0)\right|+\sum\log\left(\frac{\rho}{\left|\varphi_{\rho,z_0}(z_i)\right|}\right)
             & = \frac1{2\pi}\int_0^{2\pi}\log\left|f\circ\varphi_{R,z_0}(Re^{i\theta})\right|\d \theta \\
             & =\frac1{2\pi}\int_0^{2\pi}\log\left|f\circ\varphi_{R,z_0}(Re^{i\theta})\right|\d \theta
        \end{aligned} $$
    We obtain
    $$\log \left|f(z)\right|=-\sum\log\left(\frac{z_iz-\rho^2}{\rho(z-z_i)}\right)
        +
        \frac1{2\pi}\int_0^{2\pi}\frac{\rho^2-z^2}{\left|\rho e^{i\theta}-z\right|^2}\log\left|f(\rho e^{i\theta})\right|\d \theta$$
    Provided that $f(z)\neq 0$.

\end{theorem}
\begin{lemma}\
    If $f$ is a holomorphic function in $\Omega$, we denote by $\mathfrak{n}(r)$  the number of zeros of $f$ (counted with their multiplicities) inside the disc $B(0,r) \subset \Omega$. If $z_1,\ldots,z_N$ are the zeros of f inside the disc $B(0,\rho)$, then
    $$\int_0^\rho \mathfrak{n}(r)\:\frac{dr}r=\sum_{k=1}^N\log\left|\frac \rho {z_k}\right|.$$
\end{lemma}

\section{Weierstrass infnite product}
\begin{definition}\
    For each integer $k\geq0$ we define \textbf{canonical factors} by
    $$E_0=1-z\quad\quad E_k=(1-z)e^{z+\frac{z^2}{2}+\cdots\frac{z^k}{k}}$$
    the integer $k$ is called the degreee of the canonical factors
\end{definition}
\begin{lemma}\
    If $\left|z\right|\leq 1/2$, then $\left|1-E_k\right|\leq C\left|z\right|^{k+1}$ for some $C>0$ (independent of $k$)
\end{lemma}
\begin{theorem}[Weierstrass]
    Given any sequence $\{ a_n\}$ of complex numbers with $\left|a_n\right|\to\infty$ as $n\to \infty$. Every entire function with these and no other zeros can be written in the form
    $$f(z)=z^m e^{g(z)} \prod_{n=1}^{\infty}\left(1-\frac{z}{a_n}\right) e^{\frac{z}{a_n}+\frac{1}{2}\left(\frac{z}{a_n}\right)^2+\cdots+\frac{1}{m_n}\left(\frac{z}{a_n}\right)^{m_n}}$$
    $a_n \neq 0$
    where the product is taken over all $a_n \neq 0$, the $m_n$ are certain integers, and $g(z)$ is an entire function.

    Proof: The product converges together with the series with the gengral term
    $$r_n(z)=\log\left(1-\frac{z}{a_n}\right)+p_n(z)$$
    where the branch of logarithm shall

    For a given $R$ we consider only the terms with $\left|a_n\right|>R$. In the disk $\overline{B(0,R)}$ the principal branch of $\log(1-z/z_n)$ can be developed in Taylor series
    $$\begin{aligned}
            r_n(z) & =\log E_{m_n}\left(\frac{z}{a_n}\right)                                                                               \\
                   & =\log\left(1-\frac{z}{a_n}\right)+p_n(z)                                                                              \\
                   & =\sum_{n=1}^\infty -\frac{1}{n}\left(\frac{z}{a_n}\right)^n+ \sum_{n=1}^{m_n} \frac{1}{n}\left(\frac{z}{a_n}\right)^n \\
                   & =\sum_{n=m_n+1}^\infty -\frac{1}{n}\left(\frac{z}{a_n}\right)^n
        \end{aligned}$$
    and we obtain the estimate
    $$\left|r_n(z)\right|\leq \frac{1}{m_n+1}\left(\frac{R}{a_n}\right)^{m_n+1}\left(1-\frac{R}{\left|a_n\right|}\right)^{-1}$$
    where $R/\left|a_n\right|\leq \delta <1$ for all $n\in \{n:\left|a_n\right|>R\}$. Suppoes now that the series
    $$\sum_{n=1}^\infty \frac{1}{m_n+1}\left(\frac{R}{a_n}\right)^{m_n+1}$$
    converges, the comparison shows that the series $\sum r_n(z)=\sum\limits_{\left|a_n\right|>R}r_n+\sum\limits_{\text{else}}r_k$ is absolutely and uniformly convergent for $|z| \leqq R$, and thus the product represents an analytic function in $B(0,R)$.
    It remains only to show that the series can be made convergent for all $R$. But this is obvious, for if we take $m_n=n$.
\end{theorem}
\begin{corollary}
    Every meromorphic function in the whole plane $\mathbb{C}$ is the quotient of two entire function
\end{corollary}
\begin{corollary}[Interpolation]
    Suppose that $a_n \rightarrow \infty$ and that the $A_n$ are arbitrary complex numbers. Then there exists an entire function $f(z)$ which satisfies $f\left(a_n\right)=A_n$.

    Proof:
    Let $g(z)$ be a function with simple zeros at the $a_n$. We show that
    $$\sum_{n=1}^{\infty}  \frac{g(z)A_n}{(z-a_n)g^{\prime}(a_n)} \cdot e^{\gamma_n\left(z-a_n\right)}$$
    converges for some choice of the numbers $\gamma_n$.

    For any $R>0$, we consider
\end{corollary}
\begin{definition}\
    Assume that $h$ is the smallest integer for which $\sum\frac{1}{\left|a_n\right|^{h+1}}$ converges; the expression
    $$\prod_{n=1}^{\infty}E_h\left(\frac{z}{a_n}\right)$$
    is then called the \textbf{canonical product associated with sequence $\{a_n\}$}, and $h$ is the \textbf{genus of the canonical product}

    whenever possible we use the canonical product in the representation of $f$, which is thereby uniquely determined
    $$f=z^me^{g(z)}\prod_{n=1}^{\infty}E_h\left(\frac{z}{a_n}\right)$$
    that
    $$\begin{aligned}
            \left|r_n(z)\right|=\left|\log E_h\left(\frac{z}{a_n}\right)\right| & \leq\frac{1}{h+1}\left(\frac{R}{a_n}\right)^{h+1}\left(1-\frac{R}{\left|a_n\right|}\right)^{-1} \\
                                                                                & \leq \frac{C_R}{(h+1)\left|a_n\right|^{h+1}}
        \end{aligned}
    $$
    then product converges uniformly for $B(0,R)$.

    If in this representation $g(z)$ reduces to a polynomial, the function $f(z)$ is said to be of finite genus, and \textbf{genus of $f(z)$} is by definition equal to $\max\{h,\text{deg } g\}$.
\end{definition}
\begin{theorem}\
    $$
        \sin \pi z=z \pi \prod_{n \neq 0}\left(1-\frac{z}{n}\right) e^{z / n}
        =\pi z \prod_1^{\infty}\left(1-\frac{z^2}{n^2}\right)
    $$
    In order to determine $g(z)$ we form the logarithmic derivatives on both sides. We find
    $$\pi \cot \pi z=\frac{1}{z}+g^{\prime}(z)+\sum_{n \neq 0}\left(\frac{1}{z-n}+\frac{1}{n}\right)$$
    where the procedure is easy to justify by uniform convergence on any compact set which does not contain the points $z=n$. By comparison with the previous formula (10) we conclude that $g^{\prime}(z)=0$. Hence $g(z)$ is a constant, and since $\lim\limits_{z \rightarrow 0} \sin \pi z / z=\pi$ we must have $e^{g(z)}=\pi$.
\end{theorem}

\section{Hadamard's factorization theorem}
\begin{definition}
    Let $f$ be an entire function. If there exist a positive number $\rho$ and constants $A,B>0$ such that
    $$\left|f(z)\right|
        \leq
        Ae^{B\left|z\right|^\rho}\quad\text{for all }z\in\mathbb{C}
    $$
    then we say that $f$ has an order of growth $\leq\rho$. We define the \textbf{order of growth of $f$} as
    $$
        \lambda = \inf\rho
    $$

    Denote by $M(r)$ the maximum of $\left|f(z)\right|$ on $\left|z\right|=r$.
    For any given $\varepsilon>0$ as sonn as $r$ is sufficiently large, we have
    $$
        M(r)\leq e^{r^{\lambda+\varepsilon}}
    $$
    then $M(r)=o(e^{r^{\lambda+\varepsilon}})$ and $\log M(r)=o(r^{\lambda+\varepsilon})$ for large $r$. Actually,
    $$\lambda=\lim\sup\frac{\log\log M(r)}{\log r}$$


\end{definition}
\begin{lemma}
    For all $u\in C$ and $h\in N$
    $$\log\left|E_h(u)\right|\leq(2h+1)\left|u\right|^{h+1}$$
    Furthermore, $E_h(z)$ has order of growth $h$.

    Proof: If $\left|u\right|<1$ we have power series development
    $$\log\left|E_{h}(u)\right|\leq\frac{\left|u\right|^{h+1}}{h+1}+\frac{\left|u\right|^{h+2}}{h+2}+\cdots\leq \frac{1}{h+1}\frac{\left|u\right|^{h+1}}{1-\left|u\right|}$$
    and thus
    $$(1-\left|u\right|)\log\left|E_h(u)\right|\leq\left|u\right|^{h+1}$$
    For arbitrary $u$ and $h\geq 1$ it is also clear that
    $$\log\left|E_h(u)\right|\leq\log\left|E_{h-1}(u)\right|+\left|u\right|^h$$
    since $E_h(u)=E_{h-1}(u)e^{u^h/h}$.

    We assume that with $h-1$ in the place of $h$, that is to say
    $$\log\left|E_{h-1}(u)\right|\leq(2h-1)\left|u\right|^{h}$$
    If $\left|u \right|\geq 1$, this imply
    $$\begin{aligned}
            \log\left|E_{h}(u)\right| &
            \leq\log\left|E_{h-1}(u)\right|+\left|u\right|^h            \\
                                      & \leq2h\left|u\right|^h          \\
                                      & \leq (2h+1)\left|u\right|^{h+1}
        \end{aligned}
    $$
    But if $\left|u\right|<1$ we can also obtain
    $$\begin{aligned}
            \log\left|E_{h}(u)\right| &
            =(1-\left|u\right|)\log\left|E_h(u)\right|+\left|u\right|\log\left|E_h(u)\right|                             \\
                                      & \leq\left|u\right|^{h+1}+\left|u\right|((2h-1)\left|u\right|^h+\left|u\right|^h) \\
                                      & =(2h+1)\left|u\right|^{h+1}
        \end{aligned}
    $$
\end{lemma}
\begin{theorem}
    If $f$ is an entire function that has an order of growth $\lambda$, then for every $\varepsilon >0$

    (i) $\mathfrak{n}(r)=o(r^{\lambda+\varepsilon})$

    (ii) If $a_1, a_2,\ldots$ denote the zeros of $f$, with $a_k\neq 0$, then we have
    $$
        \sum_{k=1}^\infty\frac1{\left|a_k\right|^{\lambda+\varepsilon}}<\infty
    $$

    Proof: It suffices to prove the estimate for $\mathfrak{n}(r)$ when $f(0)\neq0$. Indeed, consider the function $F(z)=f(z)/z^\ell$ where $\ell$ is the order of the zero of $f$ at the origin. Then $\mathfrak{n} _f( r)$ and $\mathfrak{n}_F(r)$ differ only by a constant, and $F$ also has an of order of growth $\leq\rho$.

    1. If $f(0)\neq0$, then for $R=2r>0$ that $f$ vanish nowhere on $\partial B(0,R)$
    $$
        \begin{aligned}
            \mathfrak{n}(r)\log2\leq\int_r^{2r}\frac{\mathfrak{n}(x)}{x}\d x & \leq\int_0^R\frac{\mathfrak{n}(x)}{x}\d x                                                 \\
                                                                             & =\frac1{2\pi}\int_0^{2\pi}\log\left|f(Re^{i\theta})\right|\d \theta-\log\left|f(0)\right| \\
                                                                             & \leq\log\left|M(2r)\right|-\log\left|f(0)\right|                                          \\
                                                                             & \leq (2r)^{\lambda+\varepsilon}-\log\left|f(0)\right|
        \end{aligned}$$
    Consequently, $\lim\mathfrak{n}(r)r^{-\lambda-\varepsilon}=0$

    2. Since $\mathfrak{n}(x)$ vanish near the $0$ and $\mathfrak{n}(x)=o(x^{\lambda+\frac{1}{2}\varepsilon})$, we have
    $$
        \sum\frac{1}{\left|a_n\right|^{\lambda+\varepsilon}}
        =
        \frac{1}{\lambda+\varepsilon}
        \int^\infty_0\frac{\mathfrak{n}(x)}{x^{\lambda+\varepsilon + 1}}\d x<\infty$$

\end{theorem}
\begin{theorem}[Hadamard Theorem]
    The genus and the order of an entire function satisfiy the double inequality
    $$h\leq \lambda\leq h+1$$

    Proof:
    1. Suppose that $f(z)$ is of genus $h$, then the previous lemma gives the estimate
    $$\log\left|P(z)\right|=\sum_n\log\left|E_h\left(\frac{z}{a_n}\right)\right|\leq(2h+1)\left|z\right|^{h+1}\sum_n\frac{1}{\left|a_n\right|^{h+1}}$$
    an it follows that $P(z)$ is at most of order $h+1$

    2. For the opposite inequality assume $f(z)$ is of finite order $\lambda$ and $h_1=[\lambda]$. Then $h_1+1>\lambda$, and we have to prove that $\sum1/\left|a_n\right|^{h_1+1}$ converges. It is obvious by the previous lemma. It remains to prove that $g(z)$ is a polynomial of degreee $\leq h_1=[\lambda]$. If the operation $\frac{\partial}{\partial x}-i\frac{\partial}{\partial y}$ is applied to both sides of the Poission Jensen formula, we obtain
    $$$$
\end{theorem}
\begin{corollary}
    An entire function of fractional ordered assemues
    every finite valus infinity many times

    Proof: It is clear that $f$ and $f-a$ have the same order for any constant $a$. Therefore we need only show that $f$ has infinitely many zeros.

    If $f$ has only a finite number of zeros we can divide by a polynomial and obtain a function of the same order without zeros.
    $$\frac{f(z)}{\prod\limits_{k=1}^nz-a_k}=e^{g(z)}$$
    By the theorem $g(z)$ must be a polynomial. But it is evident that the order of $e^{g(z)}$ is exactly the degree of $g$, and hence an integer. The contradiction proves the corollary.
\end{corollary}




\chapter{Special Function}

\section{The Gamma Function}
\subsection{The Gamma Function}
\begin{definition}
    $$\Gamma(z) = \frac{e^{- \gamma z}}{z} \prod_{n=1}^\infty \left(1 +\frac{z}{n}\right)^{-1} e^{\frac{z}{n}}$$
    We observe that $\gamma(z)$ is meromorphic function with simole poles at $z = 0,-1 \cdots$ but without zeros



    Proof:
    $$G(z)=\prod_{n=1}^\infty \left( 1 + \frac{z}{n}\right) e^{-\frac{z}{n}}$$
    We observe that $G(z-1)$ has the same zeros as $G(z)$, and in addition a zero at the origin. It is therefore clear that we can write
    $$
        G(z-1)=z e^{\gamma(z)} G(z),
    $$
    where $\gamma(z)$ is an entire function. In order to determine $\gamma(z)$ we take the logarithmic derivatives on both sides. This gives the equation
    $$
        \sum_{n=1}^{\infty}\left(\frac{1}{z-1+n}-\frac{1}{n}\right)=\frac{1}{z}+\gamma^{\prime}(z)+\sum_{n=1}^{\infty}\left(\frac{1}{z+n}-\frac{1}{n}\right)
    $$
    In the series to the left we can replace $n$ by $n+1$. By this change we obtain
    $$
        \begin{aligned}
            \sum_{n=1}^{\infty}\left(\frac{1}{z-1+n}-\frac{1}{n}\right) & =\frac{1}{z}-1+\sum_{n=1}^{\infty}\left(\frac{1}{z+n}-\frac{1}{n+1}\right)                                                         \\
                                                                        & =\frac{1}{z}-1+\sum_{n=1}^{\infty}\left(\frac{1}{z+n}-\frac{1}{n}\right)+\sum_{n=1}^{\infty}\left(\frac{1}{n}-\frac{1}{n+1}\right) \\
                                                                        & =\frac{1}{z}+\sum_{n=1}^{\infty}\left(\frac{1}{z+n}-\frac{1}{n}\right)
        \end{aligned}
    $$
    Hence $\gamma^{\prime}(z)=0$ and $\gamma(z)$ is a constant, which we denote by $\gamma$, and $G(z-1) = e^\gamma G(z)$. Taking $z=1$ we have
    $$1=G(1)=e^\gamma G(1)$$
    an hence
    $$\gamma=\lim\left(1 + \frac{1}{2} + \frac{1}{3}+ \cdots + \frac{1}{n} - \log n \right)$$

    Let Euler's gamma function
    $$\Gamma(z) = \frac{1}{z e^{\gamma z} G(z)}$$
    satisfies
    $$\Gamma(z+1) = z \Gamma (z)$$
\end{definition}
\begin{proposition}
    The function $\Gamma$ has the following properties:

    (i)
    \[
        \Gamma(z+1)= z\Gamma(z)
    \]

    (ii) $\frac{1}{\Gamma(s)}$ is an entire function of $s$ with simple zeros at $s=0,-1,-2, \ldots$ and it vanishes nowhere else.
\end{proposition}
\begin{theorem}
    $1 / \Gamma(s)$ has growth
    $$
        \left|\frac{1}{\Gamma(s)}\right| \leq c_1 e^{c_2|s| \log |s|}
    $$
    Therefore, $1 / \Gamma$ is of order 1.


    Proof. By the theorem we may write
    $$\frac{1}{\Gamma(s)}=\Gamma(1-s) \frac{\sin \pi s}{\pi}$$
    and therefore $1 / \Gamma$ is entire with simple zeros at $s=0,-1,-2,-3, \ldots$.

    To prove the estimate, we begin by showing that
    $$\int_1^{\infty} e^{-t} t^\sigma\d  t \leq e^{(\sigma+1) \log (\sigma+1)}$$
    whenever $\sigma=Re(s)$ is positive. Choose $n$ so that $\sigma \leq n \leq \sigma+1$. Then
    $$\begin{aligned}
            \int_1^{\infty} e^{-t} t^\sigma d t & \leq \int_0^{\infty} e^{-t} t^n d t \\
                                                & =n!                                 \\
                                                & \leq n^n                            \\
                                                & =e^{n \log n}                       \\
                                                & \leq e^{(\sigma+1) \log (\sigma+1)}
        \end{aligned}$$
    Since the relation (3) holds on all of $\mathbb{C}$, we see from (5) that
    $$\frac{1}{\Gamma(s)}=\left(\sum_{n=0}^{\infty} \frac{(-1)^n}{n!(n+1-s)}\right) \frac{\sin \pi s}{\pi}+\left(\int_1^{\infty} e^{-t} t^{-s} d t\right) \frac{\sin \pi s}{\pi}$$
    However, from our previous observation,
    $$\left|\int_1^{\infty} e^{-t} t^{-s} d t\right|\leq e^{(|\sigma|+1) \log (|\sigma|+1)}$$
    and because $|\sin \pi s| \leq e^{\pi|s|}$ (by Euler's formula for the sine function) we find that the second term in the formula is dominated by $c e^{(|s|+1) \log (|s|+1)} e^{\pi|s|}$, which is itself majorized by $c_1 e^{c_2|s| \log |s|}$. Next, we consider the term
    $$\sum_{n=0}^{\infty} \frac{(-1)^n}{n!(n+1-s)} \frac{\sin \pi s}{\pi}$$
    There are two cases: $|\operatorname{Im}(s)|>1$ and $|\operatorname{Im}(s)| \leq 1$. In the first case, this expression is dominated in absolute value by $c e^{\pi|s|}$. If $|\operatorname{Im}(s)| \leq 1$, we choose $k$ to be the integer so that $k-1 / 2 \leq \operatorname{Re}(s)<k+1 / 2$. Then if $k \geq 1$,
    $$
        \begin{aligned}
            \sum_{n=0}^{\infty} \frac{(-1)^n}{n!(n+1-s)} \frac{\sin \pi s}{\pi} & =(-1)^{k-1} \frac{\sin \pi s}{(k-1)!(k-s) \pi}+           \\
                                                                                & +\sum_{n \neq k-1}(-1)^n \frac{\sin \pi s}{n!(n+1-s) \pi}
        \end{aligned}
    $$

    Both terms on the right are bounded; the first because $\sin \pi s$ vanishes at $s=k$, and the second because the sum is majorized by $c \sum 1 / n$ !.

    When $k \leq 0$, then $\operatorname{Re}(s)<1 / 2$ by our supposition, and $\sum_{n=0}^{\infty} \frac{(-1)^n}{n!(n+1-s)}$ is again bounded by $c \sum 1 / n$ !. This concludes the proof of the theorem.

    The fact that $1 / \Gamma$ satisfies the type of growth conditions discussed in Chapter 5 leads naturally to the product formula for the function $1 / \Gamma$, which we treat next.
\end{theorem}
\subsection{Properties}
\begin{theorem}
    The residue of $\Gamma$ at $s=-n$ is $(-1)^n / n!$.

    Proof:
    $$\Gamma(z)= \frac{\Gamma(z+n+1)}{(z+1) (z+2)\cdots (z+n+1)} $$
    then
    $$\begin{aligned}
            \mathrm{Res}_{z=-n}\Gamma(z) &
            =(z+n)\Gamma(z)|_{z=-n}                                                            \\
                                         & =\frac{\Gamma(1)}{(-n+1) \cdots (-n+n-1) (-n +n+1)} \\
                                         & = \frac{(-1)^n}{n!}
        \end{aligned}
    $$
\end{theorem}
\begin{theorem}[Gauss]
    $$(2\pi)^{\frac{n-1}{2}} \ \Gamma(nz) = n^{ nz-\frac{1}{2}} \Gamma(z) \Gamma\left(z+\frac{1}{n}\right) \cdots \Gamma\left(z+\frac{n-1}{n}\right) $$

    Proof: Considering the second derivative of $\log \Gamma(z)$
    $$\frac{d}{d z} \left(\frac{\Gamma'(z)}{\Gamma (z)}\right) = \sum_{k=0}^{\infty} \frac{1}{(z+k)^2}$$
    then
    $$\begin{aligned}
            \sum_{m=0}^{n-1} \frac{d}{dz} \left( \frac{\Gamma(z+\frac{m}{n})}{\Gamma'(z+\frac{m}{n})}\right) &
            =\sum_{m=0}^{n-1}\sum_{k=0}^\infty \frac{1}{(z+\frac{m}{n}+k)^2}                                                                                               \\
                                                                                                             & = n^2 \sum_{s=0}^\infty \frac{1}{(nz+s)^2}                  \\
                                                                                                             & =n \frac{d}{dz} \left(\frac{\Gamma(nz)}{\Gamma'(nz)}\right)
        \end{aligned}
    $$
    By integration we obtain

    $$\Gamma(nz) = e^{az +b} \Gamma(z) \Gamma\left(z+\frac{1}{n}\right) \cdots \Gamma\left(z+\frac{n-1}{n}\right) $$
    where the constants $a$ and $b$ have yet to be determined. Considering the residues of two sides at the poles $z=0$, we have
    $$ \frac{1}{n}= e^b \prod_{k=1}^{n-1}\Gamma \left(\frac{k}{n}\right)=e^b \sqrt{\prod_{k=1}^{n-1}\frac{\pi}{\sin \frac{\pi k}{n} }}=e^b \sqrt{\frac{\pi^{n-1}}{\frac{n}{2^{n-1}}}}$$
    $e^b=\frac{1}{n^{1/2}(2\pi)^{\frac{n-1}{2}}}$
    . Next, substituting $z=1$
    $$
        \begin{aligned}
            (n-1)! & = e^{a+b}\prod_{k=1}^{n-1} \Gamma\left(1+ \frac{k}{n}\right)                \\
                   & = e^{a+b} \prod_{k=1}^{n-1} \frac{k}{n}\cdot \Gamma\left(\frac{k}{n}\right) \\
                   & = e^{a+b} \frac{(n-1)!}{n^(n-1) } \frac{1}{n e^b }
        \end{aligned}
    $$
    hence $e^a=n^{n}$. The final result is thus
    $$(2\pi)^{\frac{n-1}{2}} \ \Gamma(nz) = n^{nz-\frac{1}{2}} \Gamma(z) \Gamma\left(z+\frac{1}{n}\right) \cdots \Gamma\left(z+\frac{n-1}{n}\right) $$
\end{theorem}
\begin{corollary}[Legendre's duplication formula]
    $$\sqrt{\pi} \Gamma (2 z) = 2^{2z-1} \Gamma(z) \Gamma(z+\frac{1}{2}) $$
\end{corollary}
\begin{theorem}
    For all $z\in\mathbb{C}$
    $$\Gamma(z) \Gamma(1-z)=\frac{\pi}{\sin \pi z}$$
\end{theorem}
\subsection{The Integral Form}
\begin{lemma}
    The funcion
    $$
        F(s)
        =
        \int_0^\infty e^{-t}t^{s-1}\d t
    $$
    is an analytic function in the half-plane $Re(s)>0$.

    Proof:
    We first observe that the integral is exist for every
    $\mathrm{Re}(s) > 0$.
    It suffess to show that the integral defines a holomorphic function in every strip
    $$S_{\delta,M}=\{s:\delta<{Re}(s)<M\}$$
    where $0<\delta<M<\infty$.

    For $\epsilon>0$, let
    $$F_{\epsilon}(s)=\int_{\varepsilon}^{1/\epsilon}e^{-t}t^{s-1}\d t$$
    the function $F_{\epsilon}$ is holomorphic in the strip $S_{\delta,M}$, it suffices to show the
    $F_{\epsilon}$ converges unifformly to $\Gamma$ on that the strip $S_{\delta,M}$. To see this, we first

    $$\begin{aligned}
            \left|F(s)-F_\varepsilon(s)\right| &
            =\left|\left(\int_0^\epsilon+\int_{1/\epsilon}^\infty\right)e^{-t}t^{s-1}\d t\right|                                          \\
                                               & \leq\int_0^\epsilon e^{-t}t^{\sigma-1}dt+\int_{1/\epsilon}^\infty e^{-t}t^{\sigma-1}\d t \\
                                               & \leq\int_0^\epsilon e^{-t}t^{\delta-1}dt+\int_{1/\epsilon}^\infty e^{-t}t^{M-1}\d t      \\
        \end{aligned}$$
    converges uniformly to $0$ in $S_{\delta,M}$.
\end{lemma}
\begin{lemma}
    The function $F$ initially defined for $\mathrm{Re}(s) > 0$ has an analytic continuation to a meromorphic function on $\mathbb{C}$ whose only singularities are simple poles at the negative integers $s=0,-1,\cdots$ The
    residue of $F$ at $s=-n$
    \[\mathrm{Res}_{z=-n} F (z)= \frac{(-1)^n}{n!}\]

    Proof: It suffices to extend $F$ to each half-plane $\mathrm{Re}(s)>-m$, where $m\geq 1$ is an integer. For $\mathrm{Re}(s)>-1$, we define
    $$F_1(s)=\frac{F(s+1)}{s}$$
    Since $\Gamma(s+1)$ is holomorphic in $\mathrm{Re}(s)>-1$, we see that $F_1$ is meromorphic in that half-plane, with the only possible singularity a simple pole at $s=0$ with residue 1. Moreover, if $\mathrm{Re}(s)>0$, then
    $$F_1(s)=\frac{F(s+1)}{s}=F(s)$$
    So $F_1$ extends $F$ to a meromorphic function on $\mathrm{Re}(s)>-1$.

    We can now continue in this fashion by defining a meromorphic $F_m$ for $\operatorname{Re}(s)>-m$ that agrees with $F$ on $\mathrm{Re}(s)>0$. For $\operatorname{Re}(s)>-m$, where $m$ is an integer $\geq 1$, define
    $$F_m(s)=\frac{\Gamma(s+m)}{(s+m-1)(s+m-2) \cdots s} .$$
    The function $F_m$ is meromorphic in $\operatorname{Re}(s)>-m$ and has simple poles at $s=0,-1,-2, \ldots,-m+1$ with residues
    $$\begin{aligned}
            res_{s=-n} F_m(s) & =\frac{\Gamma(-n+m)}{(m-1-n)!(-1)(-2) \cdots(-n)} \\
                              & =\frac{(m-n-1)!}{(m-1-n)!(-1)(-2) \cdots(-n)}     \\
                              & =\frac{(-1)^n}{n!}
        \end{aligned}$$
    Successive applications of the lemma show that $F_m(s)=F(s)$ for $\operatorname{Re}(s)>0$.
    By uniqueness, this also means that $F_m=F_k$ for $1 \leq k \leq m$ on the domain of definition of $F_k$. Therefore, we have obtained the desired continuation of $F$.
\end{lemma}
\begin{theorem}
    For $\mathrm{Re}(z) >0$, we have
    \[
        \Gamma(z)
        =
        \int_0^\infty e^{-t} t^{z-1} dt
    \]
\end{theorem}

\section{The Zeta Function}
\subsection{The Zeta Function}
\begin{theorem}
    The Riemann's Zeta function
    $$\zeta (s) = \sum_{n=1}^\infty \frac{1}{n^s}$$
    represents an analytic funcion of $s$ in the half plane $\mathrm{Re} (s) > 1$.
\end{theorem}
\begin{theorem}
    For $\sigma=\operatorname{Re} (s) >1$,
    $$
        \frac{1}{\zeta(s)}=\prod_{n=1}^{\infty}\left(1-p_n^{-s}\right)
    $$
\end{theorem}
\subsection{Extension to the Whole Plane}
\begin{theorem}
    For $\sigma >1$
    $$\zeta (s) = -\frac{\Gamma (1-s)}{2 \pi i} \int_C \frac{(-z)^{s-1}}{e^z-1} \d z$$
    where $(-z)^{s-1}$ is defined on the $\mathbf{C} - (0,\infty)$ as $e^{(s-1) \log (-z)}$ with $-\pi < \mathrm{Im} \left(\log (-z)\right) <\pi$

    Proof:
    The integral is obviously convergent. By Cauchy's theorem its value does not depend on the shape of $C$ as long as $C$ does not enclose any limit we are left with an integral back and forth along the positive real axis.

    On the upper edge $(-z)^{s-1} = e^{(s-1)(\log z -\pi i) } = x^{s-1} e^{-(s-1) \pi i}$ and on the lower edge $(-z)^{s-1} = e^{(s-1)(\log z + \pi i) } $ $x^{s-1} e^{(s-1) \pi i}$. We obtain
    $$ \begin{aligned}
            \int_C \frac{(-z)^{s-1}}{e^z-1} d z &
            =-\int_0^{\infty} \frac{x^{s-1} e^{-(s-1) \pi i}}{e^x-1} d x+\int_0^{\infty} \frac{x^{s-1} e^{(s-1) \pi i}}{e^x-1} d x \\
                                                & =2  i \sin ((s-1) \pi)  \zeta(s) \Gamma(s)                                       \\
                                                & = -2 \pi i \frac{ \zeta(s)}{\Gamma(1-s)}
        \end{aligned}$$
\end{theorem}
\begin{theorem}\
    The $\zeta(s)$ can be extended to a meromorphic funcion in the whole plan whose only pole is a simple pole at $s=1$ with the residue 1.

    Proof:
    It is indeed quite obvious that the integral $\int_C \frac{(-z)^{s-1}}{e^z-1} \d z $
    is an entire function of $s$, while $\Gamma(1-s)$ is meromorphic with poles at $s=1,2, \ldots$. Because $\zeta(s)$ is already known to be analytic for $\sigma>1$, the poles at the integers $n \geq 2$ must cancel against zeros of the integral.

    At $s=1,-\Gamma(1-s)$ has a simple pole with the residue 1. On the other hand,
    $$
        \frac{1}{2 \pi i} \int_C \frac{d z}{e^z-1}=1
    $$
    by residues, so $\zeta(s)$ has the residue 1.
\end{theorem}
\begin{theorem}\
    The values $\zeta(-n)$ at the negative integers and zero can be evaluated explicitly. Recall the expansion
    $$
        \frac{1}{e^z-1} = \frac{1}{z} - \frac{1}{2}+ \sum_{k=1}^{\infty}(-1)^{k-1} \frac{B_k}{(2 k)!} z^{2 k-1}
    $$

    From (59)
    $$
        \zeta(-n)=(-1)^n \frac{n!}{2 \pi i} \int_C \frac{z^{-n-1}}{e^z-1} d z
    $$

    Hence $\zeta(-n)$ is equal to $(-1)^n n!$ times the coefficient of $z^n$ in (60), and
\end{theorem}

\subsection{Functional Equation}
\begin{theorem}
    We have
    \[
        \zeta(s)
        =
        2^s \pi^{s-1} \sin\frac{\pi s}{2} \Gamma(1-s) \zeta(1-s)
    \]
    It is equivalent that
    \[
        \zeta(1-s)
        =
        2^{1-s} \pi^{-s} \cos\frac{\pi s}{2} \Gamma(s) \zeta(s)
    \]
\end{theorem}
\begin{corollary}
    The function
    \[\xi(s)
        =\frac{1}{2} s(1-s) \pi^{-\frac{s}{2}} \Gamma \left( \frac{s}{2}\right) \zeta(s)
    \]
    is entire and satisfies
    \[
        \xi(s) =\xi(1-s)
    \]
\end{corollary}



\chapter{Conformal Mappings}

\section{Conformal Mappings}
\begin{definition}
    A bijective holomorphic function $f:U\longrightarrow V$ is called a \textbf{conformal map} or biholomorphism. Given such a mapping $f$, we say that $U$ and $V$ are \textbf{conformally equivalent} or simply biholomorphic.
\end{definition}

\begin{definition}
    A conformal map from an open set $\Omega$ to itself is called an automorphism of $\Omega$.
    The set of all automorphisms of $\Omega$ is denoted by $\Aut(\Omega)$,and carries the structure of a group.
\end{definition}
\begin{theorem}
    If $f:U\longrightarrow V$ is holomorphic and injective, then $f'\neq0$ for all $z\in U$. In particular, the inverse of f defined on its range is holomorphic, and thus the inverse of a conformal map is also holomorphic.

    In particular, the inverse of $f$ defined on its range is holomorphic, and thus the inverse of a conformal map is also holomorphic (conformal).

    Proof: We argue by contradiction, and suppose that $f'(z_0)= 0$ for some $z_0\in U$. Then
    $$f(z)-f(z_0)=\frac{f^{k}(z_0)}{k!}(z-z_0)^k+G(z)\quad\text{ for all }z\mathrm{~near~}z_0$$
    with $f^{k}(z_0)\neq 0, k\geq2$ and $G$ vanishing to order $k + 1$ at $z_0$. For sufficiently small $\left|w\right|<\left|\frac{f^{k}(z_0)}{k!}\right|^{\frac{1}{k}}$, we write
    $$f(z)-f(z_0)- w=F(z)+G(z)$$
    where $F(z)=\frac{f^{k}(z_0)}{k!}(z-z_0)^k-w$. Let $M\left|w\right|^{1/k}>r>\left|\frac{wk!}{f^{(k)}(z_0)}\right|^{1/k}$, then $\left|G(z)\right|<\varepsilon_1<\left|w\right| -\varepsilon_2<\left|F(z)\right|$ on circle $\partial B(z_0,r)$ (For sufficiently small $\left|w\right|$), and $F$ has at least two zeros in $B(z_0,r)$.  Rouche's theorem implies that $f(z)-f(z_0)-w=F+G$ has at least two zeros in $B(z_0,r)$. Since $f'(z)\neq 0$ for all $z \in B(z_0,r)-\{z_0\}$, it follows that the roots of $f(z)-f(z_0)- w$ are distinct, hence $f$ is not injective, a contradiction.

    Now let $g = f^{-1}$ denote the inverse of $f$ on its range $f(U)$ . Suppose $f(z_0)=w_0\in f(U)$ and $f(z)=w$ is close to $w_0$. If $w\neq w_0$, we have
    $$\frac{g(w)-g(w_0)}{w-w_0}=\frac{1}{\frac{w-w_0}{g(w)-g(w_0)}}=\frac{1}{\frac{f(z)-f(z_0)}{z-z_0}}$$
    Since $f'(z_0)\neq 0$, we may let $w\to w_0$ that implies $z\to z_0$ and conclude that $g$ is holomorphic at $w_0$ with $g'(w_0)=1/f'(g(w_0))$.
\end{theorem}

\section{The Schwarz Lemma}
\begin{theorem}[Schwarz lemma]
    Let $f:B(0,1)\longrightarrow B(0,1)$ be holomorphic with $f(0) = 0$. Then
    $$\left|g(z)\right|=\left|\frac{f(z)}{z}\right|\leq 1\quad \text{ for all } z\in \mathbb{D}$$
    If for some $z_0\in\mathbb{D}$ we have $\left|f(z_0)\right| = \left|z_0\right|$, then $f$ is a rotation.
\end{theorem}

\begin{theorem}[$\Aut\left(\mathbb{D}\right)$]
    If $f$ is an automorphism of the unit disc, then there exist  $\theta \in\mathbb{R}$ and $\alpha \in \mathbb{D}$ such that
    \begin{equation*}
        f(z)
        =
        e^{\i \theta}\frac{z-\alpha}{\overline{\alpha}z - 1}
    \end{equation*}
\end{theorem}

\begin{theorem}[$\Aut \left(\mathbb{H}\right)$]
    Every automorphism of upper half-plane $\mathbb{H}$ takes the form
    \begin{equation*}
        f_M(z)=\frac{az+b}{cz+d}
    \end{equation*}
    for some
    \begin{equation*}
        M=
        \left(\begin{matrix}
            a & b \\
            c & d
        \end{matrix}\right)\in SL_2(R)
    \end{equation*}
    Conversely, every map of this form is an automorphism of $H$.

    Proof: Step 1. If $M\in SL_2(R)$, then $f_M$ maps $H$ to itself. This is clear from the observation that

    Step 2. If $M$ and $M'$ are two matrices in $G$, then $f_M\circ f_{M'} = f_{MM'}$. As a consequence, we can prove the first half of the theorem. Each $f_M$ is an
    automorphism because it has a holomorphic inverse $f_{M^{-1}}$.

    Step 3. Given any two points $w_1$ and $w_2$ in $\mathbb{H}$, there exists $M\in SL_2(R)$ such that $f_M(w_1) = w_2$, and therefore G acts transitively on $\mathbb{H}$. (Indeed, let $f(z)=\frac{z+Re(w_2)-Re(w_1)}{Im(w_1)/Im(w_2)}$)

    Step 4.

    Step 5. We can now complete the proof of the theorem. We suppose $f$ is an automorphism of $\mathbb{H}$ with $f(\beta) = i$, and consider a matrix $N\in G$ such that $f_N (i) =\beta$. Then $g=f\circ f_N$ satisfies $g(i) = i$, and therefore $F\circ g\circ F^{-1}$ is an automorphism of the disc that fixes the origin. So $F\circ g\circ F^{-1}$ is a rotation, and by Step 4 there exists  $R$ such that



    Therefore, if we identify the two matrices $M$ and $-M$, then we obtain a new group $PSL_2(R)$ calledthe projective special linear group; this group is isomorphic with $Aut(\mathbb{H})$.
\end{theorem}



\section{The Riemann mapping theorem}
\begin{theorem}
    Given any simply connected region $\Omega$ which is not the whole palne, and a point $z_0 \in \Omega$, there exists a unique analytic funcion $f(z)$ in $\Omega$, normalized by the conditions $f(z_0)=0$, $f'(z_0) >0$, such that $f$ defines  a one-to-one mapping of $\Omega$ to the disk $\left|w\right| <1$
\end{theorem}


\section{The Schwarz-Christoffel Formula}

\subsection{Mapping unit disk to polygons}
\begin{theorem}
    The funcion $w=f(z)$ which maps $\left|z\right| < 1$ conformally onto polygons with angles $\alpha_k\pi$ are of the form
    \[
        f(z)
        =
        C\int_0^z \prod_{k=1}^n \left(z-z_k\right)^{\alpha_k -1} dz
        +C'
    \]
    where $z_k$ are points on the $\left|z\right| =1$, and $C,C'$ are complex constants.
\end{theorem}

\subsection{Mapping Upper-half Plane to Polygons}

\begin{theorem}
    If funcion $w=f(z)$ which maps $\mathbb{H}=\{z: \mathrm{Im}z >0\}$ conformally onto the inside of a polygons $G$ with vertices $w_1,w_2,\ldots,w_n$ and interior angles $\alpha_k\pi$ of vertex $w_k$.
    Suppoes that
    $z_k \in \mathbb{R}$
    correspond to $w_k$ that; $f(z_k) =w_k$, and $ -\infty<z_1 < z_2  < \ldots <z_n \leq \infty$.
    Then $f$ is of the form

    (1)
    \begin{equation*}
        f(z)
        =
        C\int_0^z \prod_{k=1}^n \left(\zeta-z_k\right)^{\alpha_k -1} d\zeta
        +C'
    \end{equation*}
    if $\left|z_k\right| < \infty $.

    (2) Otherwise, if $z_n=\infty$, the form of $f$ is
    \begin{equation*}
        f(z)
        =
        C\int_0^z \prod_{k=1}^{n-1} \left(\zeta-z_k\right)^{\alpha_k -1} d\zeta
        +C'
    \end{equation*}
\end{theorem}

\begin{theorem}[Uniqueness in a way]
    Let $z_1,z_2,z_3$ belong to $\mathbb{R}$ and polygons with vertexs $w_k$ (regardless of the order). Then there exists a unique $f\in C(\overline{\mathbb{H}})  \bigcap H(\mathbb{\mathbb{H}})$ that maps $\mathbb{H}$ conformally onto the inside of a polygons with $f(z_i)=w_{i}$ ($i=1,2,3$).
\end{theorem}

\section{}


\begin{theorem}[]
    \[\varphi_a(z) = \frac{z-a}{\bar{a}z-1}\]
    which maps $a$ to $0$, $0$ to $a$ and $\varphi\circ \varphi(z)=z$.
\end{theorem}


\begin{theorem}
    Map the complement of a line segment onto the inside (of outside) of a circle.
    \[\overline{\mathbb{C}}-(a,b)\longrightarrow\overline{\mathbb{C}}-(-\infty,0)\]
    where $a,b\in \mathbb{R}$ by
    \[f(z)=\frac{z-a}{b-z} \]
\end{theorem}
\begin{theorem}
    Let
    \[f(z)=\frac{1}{2}\left(z+\frac{1}{z}\right)\]
    maps $z=\rho e^{i\theta}$ to
    \[
        \begin{aligned}
            x=\frac{1}{2}\left(\rho +\frac{1}{\rho}\right)\cos\theta \\
            y=\frac{1}{2}\left(\rho -\frac{1}{\rho}\right)\sin\theta
        \end{aligned}\]
    Elimination of $\theta$ yields
    \[
        \frac{x^2}{\left[\frac{1}{2}(\rho +\rho^{-1})\right]^2}    +
        \frac{y^2}{\left[\frac{1}{2}(\rho -\rho^{-1})\right]^2}
        =1
    \]
\end{theorem}





\chapter{Elliptic Functions}

\section{Elliptic functions}
\begin{definition}
    There are two non-zero complex numbers $\omega_1$ and $\omega_2$ such that
    $$f(z+w_1)=f(z)\quad \text{and }\quad f(z+w_2)=f(z)$$
    for all $z\in\mathbb{C}$. A function with two periods is said to be doubly periodic.

    If the periods $w_1$ and $w_2$ are linearly independent over $\mathbb{R}$, we now describe a normalization. Let $\tau =\omega_2/\omega_1$ and assume (after possibly interchanging the roles of $\omega_1$ and $\omega_2$) that $Im(\tau)>0$.

    It is therefore natural to consider the lattice in $\mathbb{C}$ defined by
    $$\Lambda=\{n\omega_1+m\omega_2:\: n,m\in\mathbb{Z}\}$$
    We say that 1 and $\tau$ generate $Lambda$.

    Associated to the lattice $\Lambda$ is the fundamental parallelogram defined by
    $$P_0=\{z\in\mathbb{C}:z=a+b\tau\text{ where }0\leq a<1\text{ and }0\leq b<1\}$$
    A period parallelogram $P$ is any translate of the
    fundamental parallelogram, $P = P_0 + h$ with $h\in\mathbb{C}$

    Two complex numbers $z$ and $w$ are congruent modulo
    $\Lambda$ if $z-w\in \Lambda$, and we write $z\sim w$
\end{definition}
\begin{proposition}\
    Suppose $f$ is a meromorphic function with two periods
    $1$ and $\tau$ which generate the lattice $\Lambda$. Then:

    (i) Every point in $\mathbb{C}$ is congruent to a unique point in any given period parallelogram (fundamental
    parallelogram).

    (ii) The lattice $\Lambda$ provides a disjoint covering of the complex plane, in the sense of
    $$\mathbb{C}=\bigcup_{h\in\Lambda}P_0+h$$

    (iii) The function $f$ is completely determined by its values in any period parallelogram.

    (iv) The number of poles of $f$ is same in all period parallelograms.
\end{proposition}
\begin{theorem}\
    An entire doubly periodic function is constant.
\end{theorem}
\begin{definition}\
    A non-constant doubly periodic meromorphic function is called an \textbf{elliptic funnction}.
\end{definition}
\begin{theorem}\
    The total number of poles of an elliptic function in $P_0$ is always $\geq 2$.

    Proof: Suppose first that $f$ has no poles on the boundary $\partial P_0$. By the residue theorem and period of $f$ we have
    $$2\pi i\sum\mathrm{res}f=\int_{\partial P_0}f(z)\d z=0$$
    Therefore $f$ must have at least two poles in $P_0$.

    If $f$ has a pole on $\partial P_0$ choose a small $h\in \mathbb{C}$ so that if $P= h +P_0$, then $f$ has no poles on $\partial P$. Arguing as before, we find that $f$ must have at least two poles in $P$, and therefore the same conclusion holds for $P_0$.
\end{theorem}
\begin{theorem}\
    The total number of poles (counted according to their multiplicities) of an elliptic function in ($P_0$) is called its order.
\end{theorem}
\begin{theorem}\
    Every elliptic function of order $m$ has $m$ zeros in $P_0$.

    Proof: Assuming first that $f$ has no zeros or poles on the boundary $\partial P$ of $P$, we know by the argument principle and periodicity of $f$ that
    $$0=\int_{\partial P_0}\frac{f'(z)}{f(z)}\d z=2\pi i(N_z-N_p)$$

    In the case when a pole or zero of $f$ lies on $\partial P_0$ it suffices to apply the argument to a translate of $P$.
\end{theorem}
\begin{corollary}\
    If $f$ is elliptic then the equation $f(z) = c$ has as
    many solutions as the order of $f$ for every $c\in\mathbb{C}$.
\end{corollary}

\section{The Weierstrass \texorpdfstring{$\wp$}{} function} % 6.2

\begin{definition}
    Let $\Lambda^*$ denote the lattice $\Lambda=\omega_1\mathbb{Z}\oplus \omega_2\mathbb{Z}$
    minus the origin.
    The \textbf{Weierstrass $\wp(z;\Lambda)$ function}, which is given by the series
    \begin{equation*}
        \wp(z;\Lambda)
        =
        \frac{1}{z^2}
        +
        \sum_{\omega\in\Lambda^*}\left[\frac{1}{(z-\omega)^2}-\frac{1}{\omega^2}\right]
    \end{equation*}
\end{definition}

\begin{theorem}
    The function $\wp$ is an elliptic function that has periods $\omega_1$ and $\omega_1$, and double poles at the lattice points.

    Proof:
    Step 1. To see this, suppose that $\left|z\right|<R$, and write
    \begin{equation*}
        \wp(z;\Lambda)
        =
        \frac{1}{z^2}
        +
        \sum_{|\omega|\leq2R}\left[\frac{1}{(z-\omega)^2}-\frac{1}{\omega^2}\right]+\sum_{|\omega|>2R}\left[\frac{1}{(z-\omega)^2}-\frac{1}{\omega^2}\right]
    \end{equation*}
    The term in the second sum is $O(1/\left|\omega\right|^3)$ uniformly for $\left|z\right|< R$, so by Lemma 1.5 this second sum defines a holomorphic function in $B(0,R)$.
    Finally, note that the first sum exhibits double poles at the lattice points in the disc $B(0,R)$.

    Step 2. To prove that $\wp$ is periodic with the correct
    periods, note that the derivative is given by differentiating the
    series for $\wp$ termwise so
    $$
        \wp'(z)=-2\sum \frac{1}{(z-\omega)^3}
    $$
    This accomplishes two things for us. First, the differentiated series converges absolutely whenever $z$ is not a lattice point, by the case r = 3 of Lemma 1.5.
    Second, the differentiation also eliminates the subtraction term $1/\omega^2$; therefore the series for $\wp'$ is clearly periodic with periods $1$
    and $\tau$, that is, $\wp'(z+w_1)=\wp'(z)$ and $\wp'(z+w_2)=\wp'(z)$.
    Hence, there are two constants $a$ and $b$ such that
    \begin{equation*}
        \wp(z+w_1)=\wp(z)+a
        \quad\text{ and }\quad
        \wp(z+w_2)=\wp(z)+b
    \end{equation*}
    It is clear from the definition, however, that $\wp$ is even, since the sum over $\omega\in\Lambda$ can be replaced by the sum over $-\omega\in\Lambda$.
    Therefore $\wp(-w_1/2)=\wp(w_1/2)$ and $\wp(-w_2/2)=\wp(w_2/2)$, respectively, in the two expressions above proves that $a = b = 0$.
\end{theorem}

\subsection{}

\begin{theorem}[Legendre's relation]
    Since $\wp$ has zero residues, it is the derivative of a single-valued function denote $-\zeta$
    \begin{equation*}
        \zeta(z)=\frac{1}{z}+\sum_{\omega\neq 0} \frac{1}{z-\omega} +\frac{1}{\omega} + \frac{z}{\omega^2}
    \end{equation*}
    It is clear that $\zeta$ satisfies conditions
    \begin{equation*}
        \zeta(z+\omega_1)=\zeta(z)+\eta_1,\
        \zeta(z+\omega_2)=\zeta(z)+\eta_2
    \end{equation*}
    since $\zeta'= - \wp$.
    We choose any $a\neq 0 $ and observe that
    \begin{equation*}
        \frac{1}{2\pi i} \int_{\partial P_a} \zeta(z) \d z
        =
        1
    \end{equation*}
    by residue theorem, and obtain
    \begin{equation*}
        \eta_1 \omega_2 -\eta_2 \omega_1 =2\pi i
    \end{equation*}
    known as \textbf{Legendre's relation}.
\end{theorem}

\begin{theorem}
    The canonical product associated with $\Lambda$
    \begin{equation*}
        \sigma(z)
        =
        z\prod_{\omega\neq 0} \left(1-\frac{z}{\omega}\right) e^{\frac{z}{\omega} + \frac{1}{2}\left(\frac{z}{\omega}\right)^2 }
    \end{equation*}
    converges and represent an entire function which satisfies
    \begin{equation*}
        \frac{\sigma'(z)}{\sigma(z)} =\frac{d \log \sigma(z)}{dz} =\zeta(z)
    \end{equation*}
    and
    \begin{equation*}
        \sigma(z+\omega_1)=-\sigma(z) e^{\eta_1 (z+\frac{\omega_1}{2})},\
        \sigma(z+\omega_2)=-\sigma(z) e^{\eta_2 (z+\frac{\omega_2}{2})}
    \end{equation*}

    Proof:
    Then we have
    \[
        \frac{\sigma'(z+\omega_1)}{\sigma(z+\omega_1)}
        =
        \frac{\sigma'(z)}{\sigma(z)} + \eta_1
    \]
    it is follows at once that
    \[
        \sigma(z+\omega_1)=C\sigma(z) e^{\eta_1 z}
    \]
    On setting $z=-\omega_1/2$ the value of $C$ can be determined, and we find that
    \[
        \sigma(z+\omega_1)=-\sigma(z) e^{\eta_1 (z+\frac{\omega_1}{2})}
    \]
    Similarly, it is also that
    \[
        \sigma(z+\omega_2)=-\sigma(z) e^{\eta_2 (z+\frac{\omega_2}{2})}
    \]
\end{theorem}

\subsection{}

\begin{proposition}
    Let $\wp(z;\Lambda)$

    (1) $\wp$ is even and $\wp'$ is old.


    (2) $\wp'$ vanish at  $\frac{1}{2}\Lambda$ and

    (3)
    \begin{equation*}
        \wp\left(\frac{1}{2}\Lambda\right)
        =
        \left\{
        \wp\left(\frac{\omega_1}{2}\right),
        \wp\left(\frac{\omega_2}{2}\right),
        \wp\left(\frac{\omega_1+\omega_2}{2}\right)\right\}
    \end{equation*}

    (4) If we define
    \begin{equation*}
        \wp\left(\frac{\omega_1}{2}\right)=e_1,
        \quad
        \wp\left(\frac{\omega_2}{2}\right)=e_2
        \quad\text{ and }\quad
        \wp\left(\frac{\omega_1+\omega_2}{2}\right)=e_3
    \end{equation*}
    we conclude that the equation $\wp(z) = e_i$ has a double root $(\wp'(e_i)=0)$. Since $\wp$ has order $2$, there are no other solutions to the equation $\wp(z) = e_i$ in the fundamental parallelogram. In particular, the three numbers $e_1, e_2$ and $e_3$ are distinct.
\end{proposition}

\begin{theorem}
    The function $\left(\wp'\right)^2$ is the cubic polynomial in $\wp$
    $$(\wp')^2=4(\wp-e_1)(\wp-e_2)(\wp-e_3)$$
\end{theorem}

\section{The Representation of Elliptic Function} %6.3

\begin{theorem}
    Any elliptic function with period $\omega_1,\omega_2$ can be written as
    \begin{equation*}
        C\prod_{k=1}^n \frac{\sigma(z-a_k)}{\sigma(z-b_k)}
    \end{equation*}
    where $n$ is the order and $a_k$ are congruence class of all zeros, $b_k$ are congruence class of all poles
    that satisfies $\sum a_k =\sum b_k$.
\end{theorem}
\begin{lemma}\
    Every even elliptic function $F$ with periods $\omega_1$ and $\omega_1$ is a  rational funcion of $\wp$.

    Proof:
    If $F$ has a zero or pole at the origin it must be of even order, since $F$ is an even function. As a consequence, there exists an integer $m$ so that $F\wp^m$ has no zero or pole at the lattice points. We may therefore assume that $F$ itself has no zero or pole on $\Lambda$.

    If $a$ is a zero of $F$, then so is $-a$, since F is even.
    If the points $a_1, -a_1\cdots,a_m, -a_m$ counted with multiplicities (modulo $\Lambda$) describe all the zeros of $F$
    has precisely the same roots as $F$. A similar argument, where $b_1, -b_1,\cdots,b_m, -b_m$ (with multiplicities) describe all the poles of $F$, then shows that
    $$G(z)=\frac{[\wp(z)-\wp(a_1)]\cdots[\wp(z)-\wp(a_m)]}{[\wp(z)-\wp(b_1)]\cdots[\wp(z)-\wp(b_m)]}$$
    is periodic and has the same zeros and poles as $F$. Therefore, $F/G$ is holomorphic and doubly-periodic, hence constant. This concludes the proof of the lemma.
\end{lemma}
\begin{theorem}\
    Every elliptic function f with periods $\omega_1$ and $\omega_2$ is a rational function of $\wp$ and $\wp'$.

    Proof: We first recall that $\wp$ is even while $\wp'$ odd. We then write $f$ as a sum of an even and an odd function
    $$f=f_{even}+f_{ood}$$
    Then, since $f_{ood}/\wp'$ is even, it is clear from the lemma applied to $f_{even}$ and $f_{ood}/\wp'$ that f is a rational function of $\wp$ and $\wp'$.
\end{theorem}
\begin{theorem}
    \[
        \wp(z) - \wp(u) = -\frac{\sigma(z-u)\sigma(z+u)}{\sigma(z)^2\sigma(u)^2}
    \]
    Taking logarithmic derivatives, then
    \[
        \frac{\wp'(z)}{\wp(z)-\wp(u)}
        =
        \zeta(z-u) + \zeta(z+u) - \zeta(2z)
    \]
    If we change $z$ and $u$ and add them
    \[
        \zeta(z+u)=\zeta(z)+ \zeta(z) + \frac{1}{2}\frac{\wp'(z)-\wp'(zu)}{\wp(z)-\wp(u)}
    \]
\end{theorem}
\begin{theorem}[Second derivative]
    \begin{equation*}
        \wp'' = 6 \wp^2 - \frac{1}{2} g_2
    \end{equation*}
\end{theorem}
\begin{theorem}[Addition theorem for $\wp$]
    \[
        \wp(z+u) = -\wp (z) - \wp(u) + \frac{1}{4}\left(\frac{\wp'(z)-\wp'(zu)}{\wp(z)-\wp(u)}\right)^2
    \]

    Proof:
    We recall that
    \[
        \zeta(z+u)=\zeta(z)+ \zeta(z) + \frac{1}{2}\frac{\wp'(z)-\wp'(zu)}{\wp(z)-\wp(u)}
    \]
    Taking derivative
    \[
        -\wp(z+u) = -\wp(z) + \frac{1}{2} \frac{\wp''(z)(\wp(z)-\wp(u))-\wp'(z)(\wp'(z)-\wp'(u))}{\left(\wp(z)-\wp(u)\right)^2}
    \]
    then change $z$ and $u$, we
    \[
        \begin{aligned}
             & -2\wp(z+u)                                                                                                                                       \\
             & = -\wp(z)-\wp(u) + \frac{1}{2} \frac{\left(\wp''(z)-\wp''(u)\right)(\wp(z)-\wp(u))-\left(\wp'(z)-\wp'(u)\right)^2}{\left(\wp(z)-\wp(u)\right)^2} \\
             & = 2\left(\wp(z)+\wp(u)\right)
            +
            \frac{1}{2} \left(\frac{\wp'(z)-\wp'(u)}{\wp(z)-\wp(u)}\right)^2
        \end{aligned}
    \]
    by $\wp'' = 6 \wp^2 - \frac{1}{2} g_2$.

\end{theorem}
\begin{corollary}
    \[
        \wp(2z) = - 2\wp(z) + \frac{1}{4} \left(\frac{\wp''(z)^2}{\wp'(z)^2}\right)^2
    \]
\end{corollary}

\section{The Differential Equation}

\begin{definition}
    The \textbf{Eisenstein series} of order k is defined by
    \begin{equation*}
        E_k(\tau)=\sum_{\omega\neq 0}\frac{1}{\omega^{2k}}
    \end{equation*}
    whenever $k$ is an integer $\geq 2$.
\end{definition}
\begin{theorem}
    Eisenstein series have the following properties:

    (i) The series $E_k(\tau)$ converges if $k\geq 2$, and is holomorphic in the upper half-plane.

    (ii) $E_k(\tau)$ satisfies the following transformation relations:
    $$E_K(\tau+1)=E_K(\tau)\quad\text{and}\quad E_K(\tau)=\tau^{-k}E_K(-1/\tau)$$
    The last property is sometimes referred to as the modular character of the Eisenstein series.
\end{theorem}
\begin{theorem}\
    For $z$ near $0$, we have
    $$\begin{aligned}
            \wp(z) & = \frac{1}{z^2} + 3E_2z^2 + 5E_3z^4 + \cdots         \\
                   & =\frac{1}{z^2}+\sum_{k=2}^\infty(2k-1) E_{k}z^{2k-2}
        \end{aligned}$$

    Proof: From the definition of $\wp$, if we note that we may replace $\omega$ by $-\omega$ without changing the sum, we have
    $$\wp(z)=\frac{1}{z^2}+\sum_{\omega\in\Lambda^*}\left[\frac{1}{(z+\omega)^2}-\frac{1}{\omega^2}\right]=\frac{1}{z^2}+\sum_{\omega\in\Lambda^*}\left[\frac{1}{(z-\omega)^2}-\frac{1}{\omega^2}\right]$$
    The identity
    $$\frac{1}{(1-w)^2}=\sum_{\ell=0}^\infty(\ell+1)w^\ell,\quad\text{ for }\left|w\right|<1$$
    implies that for all small $z$
    $$\frac{1}{(z-\omega)^2}=\frac{1}{\omega^2}\sum_{\ell=0}^\infty\left(\ell+1\right)\left(\frac{z}{\omega}\right)^\ell=\frac{1}{\omega^2}+\frac{1}{\omega^2}\sum_{\ell=1}^\infty\left(\ell+1\right)\left(\frac{z}{\omega}\right)^\ell$$
    Therefore
    $$\begin{aligned}
            \wp(z) & =\frac{1}{z^2}+\sum_{\omega\in\Lambda^*}\sum_{\ell=1}^\infty(\ell+1)\frac{z^\ell}{\omega^{\ell+2}}                          \\
                   & =\frac{1}{z^2}+\sum_{\ell=1}^\infty\left(\ell+1\right)\left(\sum_{\omega\in\Lambda^*}\frac{1}{\omega^{\ell+2}}\right)z^\ell \\
                   & =\frac{1}{z^2}+\sum_{k=2}^\infty(2k-1) E_{k}z^{2k-2}
        \end{aligned}$$
\end{theorem}

\begin{corollary}[Differential Equation]
    If $g_2 = 60E_2$ and $g_3 = 140E_3$, then
    $$(\wp')^2=4\wp^{3}-g_{2}\wp-g_{3}$$

    Proof: From the previous theorem, we obtain the following three expansions for $z$ near $0$
    $$\begin{aligned}
            \wp^{\prime}(z)     &
            \begin{aligned}
                =\frac{-2}{z^3}+6E_4z+20E_6z^3+\cdots,
            \end{aligned}                                \\
            (\wp^{\prime}(z))^2 & =\frac{4}{z^6}-\frac{24E_4}{z^2}-80E_6+\cdots, \\
            (\wp(z))^3          & =\frac{1}{z^6}+\frac{9E_4}{z^2}+15E_6+\cdots
        \end{aligned}$$
    From these, one sees that the difference $(\wp^{\prime}(z))^2-4(\wp(z))^3+60E_4\wp(z)+140E_6$ is holomorphic near $0$, and in fact equal to $0$ at the origin. Since
    this difference is also doubly periodic, we conclude that it is constant, and hence identically $0$.
\end{corollary}


\section{Modular Function}
\begin{definition}
    Notes that $e_1= \wp \left(\frac{\omega_1}{2}\right),e_2= \wp \left(\frac{\omega_2}{2}\right)$ and $e_3= \wp \left(\frac{\omega_1+\omega_2}{2}\right)$ are homogeneous of order $-2$ in $\omega_1, \omega_2$.
    We conclude taht the quantity
    \begin{equation*}
        \lambda(\tau)
        =
        \frac{e_3 - e_2}{e_1-e_2}
    \end{equation*}
    depends on the $\tau =\frac{\omega_2}{\omega_1}$ is analytic in the half plan $\Im \tau >0$.
\end{definition}

\begin{theorem}
    \begin{equation*}
        \lambda\left(\frac{a \tau +b}{c \tau +d}\right)
        =
        \lambda\left(\tau\right)
    \end{equation*}
    for all
    \begin{equation*}
        \left(\begin{matrix}
            a & b \\
            c & d
        \end{matrix}\right)
        =
        \left(\begin{matrix}
            1 & 0 \\
            0 & 1
        \end{matrix}\right)
        \quad
        \bmod 2
    \end{equation*}
\end{theorem}

\begin{theorem}
    Under a modular transformation

    (1)
    \begin{equation*}
        \left(
        \begin{matrix}
            \omega_2' \\
            \omega_1'
        \end{matrix}
        \right)
        =
        \left(
        \begin{matrix}
            1 & 1 \\
            0 & 1
        \end{matrix}
        \right)
        =
        \left(
        \begin{matrix}
            \omega_2 \\
            \omega_1
        \end{matrix}
        \right)
    \end{equation*}
    we have $e_1'=e_1$, $e_2'=e_3$, $e_3'=e_2$ and $\tau' =\tau +1$
    \begin{equation*}
        \lambda\left(\tau+1\right)
        =
        \lambda\left(\tau'\right)
        =
        \frac{e_3'-e_2'}{e_1'-e_2'}
        =
        \frac{e_2-e_3}{e_1-e_3}
        =
        \frac{\lambda\left(\tau\right)}{ \lambda\left(\tau\right) -1}
    \end{equation*}

    (2)
    \begin{equation*}
        \left(
        \begin{matrix}
            \omega_2' \\
            \omega_1'
        \end{matrix}
        \right)
        =
        \left(
        \begin{matrix}
            0 & 1 \\
            1 & 0
        \end{matrix}
        \right)
        =
        \left(
        \begin{matrix}
            \omega_2 \\
            \omega_1
        \end{matrix}
        \right)
    \end{equation*}
    we have
    \begin{equation*}
        \lambda\left(-\frac{1}{\tau}\right)
        =
        1 - \lambda\left(\tau\right)
    \end{equation*}
\end{theorem}





\chapter{Global Analytic Function}

\section{Picard's Theorem}
\begin{theorem}[Picard's Theorem]
    An entire funcion with more than one finite lacunary value reduces to a conatant.

    Proof:
    Suppoes that $f$ has two finite lacunary values $0$ and $1$. Considering the modular funcion
    \begin{equation*}
        \tau : \mathbb{H}\rightarrow \mathbb{C}-\{0,1\}
    \end{equation*}
    is a covering map and holomorphic, where $\mathbb{H}=\left\{z: \mathrm{Im}(z)>0\right\}$.
    By lifting theorem (since domain $\mathbb{C}$ is simply connected), there is holomorphic funcion
    \[
        \tilde{f}
        :
        \mathbb{C} \rightarrow \mathbb{H}
    \]
    that
    \[
        \tau \circ \tilde{f}
        =f
    \]

    \begin{equation*}
        \begin{tikzcd}
            \mathbb{C} \arrow[rrdd, "f"'] \arrow[rr, "\tilde{f}"] &  & \mathbb{H} \arrow[dd, "\tau"] \\
            &  &                               \\
            &  & {\mathbb {C} \backslash \{0,1\}}
        \end{tikzcd}
    \end{equation*}
    Thus
    \[
        \phi\circ \tilde{f}:
        \mathbb{C}\rightarrow \mathbb{D}
    \]
    is a bounded entire funcion, where
    \begin{equation*}
        \phi(z) =\frac{z-i}{z+i}
        :
        \mathbb{H}\rightarrow
        \mathbb{D}
    \end{equation*}
    Therefore, $\tilde{f}$ must be a constant and $f=\tau \circ \tilde{f}$
\end{theorem}



































\end{document}
