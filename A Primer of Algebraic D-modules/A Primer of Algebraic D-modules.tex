\documentclass[12pt, oneside]{book}

\usepackage{../mypackages}




\begin{document}
\frontmatter
\title{{\Huge{\textbf{A Primer of Algebraic D-modules}}}}
\maketitle

\dominitoc % 初始化minitoc
\pagenumbering{Roman}
\tableofcontents % 主目录


\mainmatter
\pagenumbering{arabic} % 正文编页码字体 


Throughout this book, $K$ denotes a field of characteristic zero and $K[X]$ the ring of polynomials $K\left[x_1, \ldots, x_n\right]$ in $n$ commuting indeterminates over $K$.

\chapter{Weyl Algebra} % Weyl Algebra
\section{Definition}
\begin{definition}
    The ring $K[X]$ is a vector space of infinite dimension over $K$.
    Its algebra of $K$-linear operators is denoted by $\End_K (K[X])$.
    Let $\hat{x}_1, \ldots, \hat{x}_n$ be the operators of $K[X]$ which are defined on a polynomial $f \in K[X]$ by the formulae $\hat{x}_i(f)=x_i \cdot f$.
    Similarly, $\partial_1, \ldots, \partial_n$ are the operators defined by $\partial_i(f)=\partial f / \partial x_i$.
    These are linear operators of $K[X]$.

    The \textbf{$n$-th Weyl algebra} $A_n$ is the $K$-subalgebra of $\End_K(K[X])$ generated by the operators $\hat{x}_1, \ldots, \hat{x}_n$ and $\partial_1, \ldots, \partial_n$. For the sake of consistency, we write $A_0=K$.
\end{definition}



\begin{proposition}
    \begin{equation*}
        \begin{aligned}
             & {\left[\partial_i, \hat{x}_j\right]=\delta_{i j} \cdot 1} \\ & {\left[\partial_i, \partial_j\right]=\left[\hat{x}_i, \hat{x}_j\right]=0}
        \end{aligned}
    \end{equation*}
    \begin{equation*}
        \left[\partial_i,f\right]=\frac{\partial f}{\partial x_i}
    \end{equation*}
\end{proposition}

\begin{proposition}
    \label{pro: proposition of Weyl algebra}
    We have
    \begin{enumerate}
        \item
              \begin{equation*}
                  \partial^\beta x^\alpha
                  =
                  \sum_{\substack{\beta_1+\beta_2=\beta\\ \beta_i\in \mathbb{N}^n}}
                  \binom{\beta}{\beta_1} \frac{\partial x^\alpha}{\partial x^{\beta_1}}\cdot \partial^{\beta_2}
              \end{equation*}
              thus $\partial^\beta x^\alpha =  x^\alpha\partial^\beta+ D^\prime$ where $\deg D^\prime \leq \left|\alpha\right|+\left|\beta\right|-2$

        \item
              The set $\mathbf{B}=\left\{x^\alpha \partial^\beta: \alpha, \beta \in \mathbb{N}^n\right\}$ is a basis of $A_n$ as a vector space over $K$.
    \end{enumerate}
\end{proposition}



\begin{theorem}
    Let the free $K$-algebra $K\left\{z_1, \ldots, z_{2 n}\right\}$ and $J$ be the two-sided ideal generated by $\left[z_{i+n}, z_i\right]-1$ for $i=1,2, \ldots, n$ and $\left[z_i, z_j\right]$ for $j \neq i+n$ and $1 \leq i, j \leq 2 n$.
    We may define a surjective homomorphism $\phi: K\left\{z_1, \ldots, z_{2 n}\right\} \rightarrow A_n$ by $\phi\left(z_i\right)=x_i$ and $\phi\left(z_{i+n}\right)=\partial_i$, for $i=1,2, \ldots, n$.
    \begin{equation*}
        \begin{tikzcd}
            \left\{z_1,\ldots,z_n\right\} \arrow[r,""]\arrow[d,"\iota"]& A_n \\
            K\left\{z_1,\ldots,z_n\right\} \arrow[ru,"\phi"]&
        \end{tikzcd}
    \end{equation*}
    It follows that $J \subseteq \operatorname{ker} \phi$. Thus $\phi$ induces a homomorphism $\bar{\phi}: K\left\{z_1, \ldots, z_{2 n}\right\} / J \longrightarrow A_n$.
    \begin{proof}
        We may use the relations to show that every element of $K\left\{z_1, \ldots, z_{2 n}\right\} / J$ may be written as a linear combination of monomials of the form
        \begin{equation*}
            z_1^{m_1} \ldots z_{2 n}^{m_{2 n}}+J
        \end{equation*}
        by \ref{pro: proposition of Weyl algebra} the images of these monomials under $\bar{\phi}$ form a basis of $A_n$ as a vector space over $K$. In particular, the monomials must be linearly independent in $K\left\{z_1, \ldots, z_{2 n}\right\} / J$. Hence $\bar{\phi}$ is an isomorphism of vector spaces and, a fortiori, an isomorphism of rings.
    \end{proof}
\end{theorem}





\begin{corollary}
    Let $m<n$ be positive integers. Choose polynomials $f_i \in K[X]$, for $1 \leq i \leq n$, as follows: if $i \leq m$, then $f_i$ is a polynomial in the variables $x_{m+1}, \ldots, x_n$; otherwise $f_i=0$.
    The map $\sigma: A_n \longrightarrow A_n$ defined by the formulae
    \begin{equation*}
        \begin{aligned}
             & \sigma\left(x_i\right)=x_i+f_i                                                                 \\
             & \sigma\left(\partial_i\right)=\partial_i-\sum_1^n \frac{\partial f_k}{\partial x_i} \partial_k
        \end{aligned}
    \end{equation*}
    induce an automorphism of $A_n$.
    \begin{proof}
        Define a homomorphism $\phi$ of $K\left\{z_1, \ldots, z_{2 n}\right\}$ to $A_n$ by $\phi\left(z_i\right)=x_i+f_i$ and
        \begin{equation*}
            \phi\left(z_{i+n}\right)=\partial_i-\sum_1^n \frac{\partial f_k}{\partial x_i} \partial_k
        \end{equation*}
        Choose $i, j$ such that $1 \leq i, j \leq n$.
        It is clear that $\phi\left(\left[z_i, z_j\right]\right)=0$.

        Let us calculate $\phi\left(\left[z_{i+n}, z_j\right]\right)$. Since $\phi$ is a ring homomorphism, this is the same as $\left[\phi\left(z_{i+n}\right), \phi\left(z_j\right)\right]$, which equals
        \begin{equation*}
            \begin{aligned}
                \phi\left(\left[z_{i+n}, z_j\right]\right) & =\left[\phi\left(z_{i+n}\right), \phi\left(z_j\right)\right]                                                                                                          \\
                                                           & =\left[\partial_i-\sum_1^n \frac{\partial f_k}{\partial x_i} \partial_k, x_j+f_j\right]                                                                               \\
                                                           & = \delta_{i j}+\frac{\partial f_j}{\partial x_i}-\frac{\partial f_j}{\partial x_i}-\sum_1^n \frac{\partial f_k}{\partial x_i} \cdot \frac{\partial f_j}{\partial x_k} \\
                                                           & = \delta_{ij}
            \end{aligned}
        \end{equation*}
        A similar calculation shows that $\phi\left(\left[z_{i+n}, z_{j+n}\right]\right)=0$. Thus $\phi$ induces an endomorphism $\sigma=\bar{\phi}$ of $A_n$.

        A similar argument shows that the map $\tau$ defined by the formulae
        \begin{equation*}
            \begin{aligned}
                 & \tau\left(x_i\right)=x_i-f_i                                                                 \\
                 & \tau\left(\partial_i\right)=\partial_i+\sum_1^n \frac{\partial f_k}{\partial x_i} \partial_k
            \end{aligned}
        \end{equation*}
        is an endomorphism of $A_n$. It is now easy to check that $\tau$ is the inverse of $\sigma$. Thus $\sigma$ is an automorphism of $A_n$.
    \end{proof}
\end{corollary}


\section{Degree}


\begin{definition}
    Let $D \in A_n$.
    The \textbf{degree} of $D$ is the largest length of the multi-indices $(\alpha, \beta) \in \mathbb{N}^n \times \mathbb{N}^n$ for which $x^\alpha \partial^\beta$ appears with non-zero coefficient in the canonical form of $D$. It is denoted by $\operatorname{deg}(D)$. As with the degree of a polynomial, we use the convention that the zero polynomial has degree $-\infty$.
\end{definition}

\begin{lemma}
    We have
    \begin{equation*}
        \partial^\beta x^\alpha
        =
        \sum_{\substack{\beta_1+\beta_2=\beta\\ \beta_i\in \mathbb{N}^n}}\partial^{\beta_1}(x^\alpha)\partial^{\beta_2}
    \end{equation*}
    thus $\partial^\beta x^\alpha =  x^\alpha\partial^\beta+ D^\prime$ where $\deg D^\prime \leq \left|\alpha\right|+\left|\beta\right|-2$
\end{lemma}

\begin{theorem}
    The degree satisfies the following properties; for $D, D^{\prime} \in A_n$ :
    \begin{enumerate}
        \item
              $\operatorname{deg}\left(D+D^{\prime}\right) \leq \max \left\{\operatorname{deg}(D), \operatorname{deg}\left(D^{\prime}\right)\right\}$.
              if $\deg(D) \neq \deg(D')$ then we have equality in the above formula.
        \item
              $\operatorname{deg}\left(D D^{\prime}\right)=\operatorname{deg}(D)+\operatorname{deg}\left(D^{\prime}\right)$.
        \item
              $\operatorname{deg}\left[D, D^{\prime}\right] \leq \operatorname{deg}(D)+\operatorname{deg}\left(D^{\prime}\right)-2$.
    \end{enumerate}
\end{theorem}

\begin{corollary}
    The algebra $A_n$ is a domain.
\end{corollary}

\section{Ideal Structure}

\begin{theorem}
    The algebra $A_n$ is simple.
    \begin{proof}
        Let $I$ be a non-zero two-sided ideal of $A_n$. Choose an element $D \neq 0$ of smallest degree in $I$. If $D$ has degree 0 , it is a constant, and $I=A_n$. Assume that $D$ has degree $k>0$ and let us aim at a contradiction.

        Suppose that $(\alpha, \beta)$ is a multi-index of length $k$. If $x^\alpha \partial^\beta$ is a summand of $D$ with non-zero coefficient and $\beta_i \neq 0$, then $\left[x_i, D\right]$ is non-zero and has degree $k-1$.
        Since $I$ is a two-sided ideal of $A_n$, we have that $\left[x_i, D\right] \in I$. But this contradicts the minimality of $D$. Thus $\beta=(0, \ldots, 0)$. Since $k>0$, we must have that $\alpha_i \neq 0$, for some $i=1,2, \ldots, n$. Hence $\left[\partial_i, D\right]$ is a non-zero element of $I$ of degree $k-1$, and once again we have a contradiction.
    \end{proof}
\end{theorem}

\begin{corollary}
    Every (ring, algebra) endomorphism of $A_n$ is injective.
\end{corollary}

\begin{theorem}
    Every left ideal of $A_n$ is generated by two elements.
\end{theorem}


\chapter{Rings of Differential Operators} % Rings of Differential Operators

Let $R$ be a commutative $K$-algebra.
\section{Definition} % Definition

\begin{definition}
    Let $R$ be a commutative $K$-algebra.

    \begin{enumerate}[label=(\roman*)]
        \item
              We will identify an element $a \in R$ with the operator of $\End_K(R)$
              defined by the rule $ r \mapsto ar$, for every $r \in R$.
              An operator $P \in \End_K(R)$ has order zero if $[a, P] = 0$, for every $a \in R$.

        \item
              Suppose we have defined operators of order $< n$.
              An operator $P \in \End_K(R)$ has order $n$ if
              it does not have order less than $n$ and $[a, P]$ has order less than $n$ for every
              $a \in R$.
    \end{enumerate}
    Let $D_n(R)$ denote the $K$-vector space of all operators of $\End_K(R)$ of order $<n$.
    The \textbf{ring of differential operators} of $R$ is defined as the subring $D(R)=\bigcup_{n\geq 0} D_n(R)$.
    \begin{remark}
        It follows that $D_n\left(R\right)=\left\{P\in \End_K R :[\cdots[[P,a_1],a_2],\cdots,a_{n+1}]=0 \text{ for any } a_i \in R \right\}$ and the set of order $n$ is $D_n-D_{n-1}$.
    \end{remark}
\end{definition}





\begin{definition}
    A \textbf{derivation} of the $K$-algebra $R$ is a $K$-linear operator $D$ of $R$
    which satisfies Leibniz's rule:
    \begin{equation*}
        D(ab) = aD(b) + bD(a)\quad \text{ for every } a, b \in R
    \end{equation*}
    Let $\mathrm{Der}_K (R)$ denote the $K$-vector space of all derivations of $R$.
    If $D\in  \mathrm{Der}_K (R)$ and $a \in R$, we define a new derivation $aD$ by $(aD)(b) = aD(b)$ for every $b\in R$.
    The $K$-vector space $\mathrm{Der}_K (R)$
    is a left $R$-module for this action.
\end{definition}



\begin{proposition}
    Let $R$ be a commutative $K$-algebra.
    \begin{enumerate}
        \item
              The elements of order zero are the elements of $R$ i.e. $D^0\left(R\right)=R$
        \item
              The operators of order $\leq 1$ correspond to the elements of $\operatorname{Der}_K(R)+R$ .i.e. \begin{equation*}
                  D^1\left(R\right)
                  \subset \operatorname{Der}_K(R)+R
              \end{equation*}
    \end{enumerate}
    \begin{proof}
        2.
        Let $Q \in D^1(R)$ and put $P=Q-Q(1)$. Note that $P(1)=0$ and that $P$ has order $\leq 1$. Hence $[P, a]$ has order zero for every $a \in R$. Thus for every $b \in R$, we have that $[[P, a], b]=0$.
        Writing the commutators explicitly, one obtains the equality
        \begin{equation*}
            (P a) b-(a P) b-b(P a)+b(a P)=0
        \end{equation*}
        Applying this operator to $1 \in R$, we end up with $P(a b)=a P(b)+b P(a)$, it follows that $P$ is a derivation of $R$. But $Q= P+Q(1) \in \operatorname{Der}_K(R)+R$, as required.
    \end{proof}
\end{proposition}



\begin{proposition}
    Let $P \in D^n(R)$ and $Q \in D^m(R)$, then $P \cdot Q \in D^{n+m}(R)$.
    \begin{proof}
        The proof is by induction on $m+n$. If $m+n=0$ the result is obvious. Suppose the result true whenever $m+n<k$. If $m+n=k$ and $a \in R$, we have that
        \begin{equation*}
            [P Q, a]=P[Q, a]+[P, a] Q
        \end{equation*}
        The definition of order implies that $[Q, a] \in D^{m-1}(R)$ and $[P, a] \in D^{n-1}(R)$. Thus, by the induction hypothesis $P[Q, a],[P, a] Q \in D^{n+m-1}$. Hence $[P Q, a]$ belongs to $D^{n+m-1}$, as required.
    \end{proof}
\end{proposition}




\section{The Weyl Algebra}

\begin{proposition}
    Every derivation of $K[X]$ is of the form $\sum_1^n f_i \partial_i$, for some $f_1, \ldots, f_n \in K[X]$.
\end{proposition}

\begin{lemma}
    Let $P \in D(K[X])$. If $\left[P, x_i\right]=0$ for every $i=1, \ldots, n$, then $P \in K[X]$.
\end{lemma}


\begin{definition}
    Define $C_r$ to be the set of operators in $A_n$ which can be written in the form $\sum_\alpha f_\alpha \partial^\alpha$ with $|\alpha| \leq r$. A simple calculation shows that
    \begin{equation*}
        C_r=C_{r+1} \cap D^r(K[X])
    \end{equation*}
    By Proposition 1.3, we have that $C_1=\operatorname{Der}_K(K[X])+K[X]$ and that $C_0= K[X]$. We will use the convention that if $k<n$ then $\mathbb{N}^k$ is embedded in $\mathbb{N}^n$ as the set of $n$-tuples whose last $n-k$ components are zero.

\end{definition}

\begin{lemma}
    It follows from the identity $\left[\partial^\beta, x_n\right]=\beta_n \partial^{\beta-e_n}$
    that $\left[\partial^\beta, x_n\right]= 0\Leftrightarrow \beta_n=0$.
    Thus $\left[G, x_n\right]=0$ implies that $G$ can be written as a linear combination of monomials of the form $x^\alpha \partial^\beta$ with $\beta \in \mathbb{N}^{n-1}$.
\end{lemma}

\begin{lemma}
    Let $P_1, \ldots, P_n \in C_{r-1}$ and assume that $\left[P_i, x_j\right]=\left[P_j, x_i\right]$ whenever $1 \leq i, j \leq n$. Then there exists $Q \in C_r$ such that $P_i=\left[Q, x_i\right]$, for $i=1, \ldots, n$.

    \begin{proof}
        Suppose, by induction, that we have determined $Q^{\prime} \in C_r$ such that $\left[Q^{\prime}, x_i\right]=P_i$ for $k+1 \leq i \leq n$.
        Write $G=[Q^\prime,x_k]-P_k$, then
        \begin{equation*}
            G
            =
            \sum_{\alpha \in \mathbb{N}^k} f_\alpha \partial^\alpha.
        \end{equation*}
        since $\left[G, x_i\right]=[[Q^\prime,x_k],x_i]-[P_k,x_i]=[[Q^\prime,x_i],x_k]-[P_k,x_i]=0$ for $k+1 \leq i \leq n$.

        Now write
        \begin{equation*}
            Q^{\prime \prime}=\sum_{\alpha \in \mathbb{N}^k}\left(\alpha_k+1\right)^{-1} f_\alpha \partial^{\alpha+e_k} \in C^r
        \end{equation*}
        We have $[Q^{\prime\prime},x_k]=G$ and $\left[Q^{\prime \prime}, x_i\right]=0$.
        Thus $\left[Q^{\prime}-Q^{\prime \prime}, x_i\right]=P_i$, for $k \leq i \leq n$; and the induction is complete.
    \end{proof}
\end{lemma}


\begin{theorem}
    The ring of differential operators of $K[X]$ is $A_n(K)$ i.e. .
    Besides this, .
    \begin{enumerate}
        \item
              $D^k(K[X])=C_k$
        \item
              $D(K[X])=A_n(K)$

    \end{enumerate}
    \begin{proof}
        It is enough to prove that $D^k(K[X]) \subseteq C_k$. Let $P \in D(K[X])$. If $P \in D^1(K[X])$ then by Lemma 1.1, $P \in \operatorname{Der}_K(K[X])+K[X]$. Thus $P \in C_1$ by Proposition 1.3.

        Suppose, by induction, that $D^k(K[X])=C_k$ for $k \leq m-1$. Let $P \in D^m(K[X])$. Write $P_i=\left[P, x_i\right]$. Since $P_i$ has order $k \leq m-1$, it follows that $P_i \in C_{n-1}$. But, for all $1 \leq i, j \leq n$,
        \begin{equation*}
            \left[P_i, x_j\right]=\left[\left[P, x_i\right], x_j\right]=\left[\left[P, x_j\right], x_i\right]=\left[P_j, x_i\right] .
        \end{equation*}
        Thus by Lemma 2.2 there exists $Q \in C_m$ such that $\left[Q, x_i\right]=P_i, 1 \leq i \leq n$. Hence $\left[Q-P, x_i\right]=0$ in $D(K[X])$. Since this holds whenever $1 \leq i \leq n$, we conclude by Lemma 2.1 that $Q-P \in K[X]=C_0$. Therefore $P \in C_m$. Hence $D^m(K[X]) \subseteq C^m$, as we wanted to prove.
    \end{proof}
\end{theorem}

\begin{theorem}
    $\ord\left(PQ\right)=\ord(P)+\ord(Q)$
\end{theorem}






\chapter{Jacobian Conjecture}
\section{Polynomial Maps}
\begin{definition}
    Let $F: K^n \rightarrow K^m$.
    We say that $F$ is \textbf{polynomial} if there exist $F_1, \ldots, F_m \in K\left[x_1, \ldots, x_n\right]$ such that $F(p)= \left(F_1(p), \ldots, F_m(p)\right)$ for all $p\in K^n$.

    A polynomial map is called an isomorphism or a \textbf{polynomial isomorphism} if it has an inverse which is also a polynomial map.

    For the rest of the section we shall write $X, Y$ for the spaces $K^n$ and $K^m$; and $K[X], K[Y]$ for the polynomial rings $K\left[x_1, \ldots, x_n\right]$ and $K\left[y_1, \ldots, y_m\right]$.
\end{definition}

\begin{definition}
    Suppose that $F: X \rightarrow Y$ is a polynomial map. We may define a map,
    \begin{equation*}
        F^{\sharp}: K[Y] \rightarrow  K[X] ,\quad  \text{ given that } g\mapsto g \circ F
    \end{equation*}
    The map $F^{\sharp}$ is called the \textbf{comorphism} of $F$.

    Suppose that a ring homomorphism $\phi: K[Y] \rightarrow K[X]$ is given. Then we may use it to construct a polynomial map from $X$ to $Y$. Now let
    \begin{equation*}
        \phi_{\sharp}: X \rightarrow Y
        ,\quad
        \mathbf{x}
        \mapsto
        \left(\phi\left(y_1\right)\left(\mathbf{x}\right),\ldots,\phi\left(y_m\right)\left(\mathbf{x}\right)\right)
    \end{equation*}
\end{definition}


\begin{theorem}
    Let $F: X \rightarrow Y$ and $G: Y \rightarrow Z$ be polynomial maps, then
    \begin{enumerate}
        \item
              $\left(F^{\sharp}\right)_{\sharp}=F$
        \item $G \cdot F: X \rightarrow Z$ is a polynomial map and $(G \cdot F)^{\sharp}=F^{\sharp} \cdot G^{\sharp}$
    \end{enumerate}
\end{theorem}

\begin{theorem}
    If $\phi: K[Y] \rightarrow K[X]$ and $\psi: K[Z] \rightarrow K[Y]$ are homomorphism of polynomial rings, then
    \begin{enumerate}
        \item
              $\left(\phi_{\sharp}\right)^{\sharp}=\phi$.
        \item
              $(\phi \cdot \psi)_{\sharp}=\psi_{\sharp} \cdot \phi_{\sharp} .$
    \end{enumerate}
\end{theorem}


\begin{corollary}
    A polynomial map $F: X \rightarrow Y$ is an isomorphism if and only if $F^{\sharp}$ is an isomorphism.
\end{corollary}

\section{Jacobian Conjecture}

Jacobian conJecture. Let $F: K^n \rightarrow K^n$ be a polynomial map. If $\Delta F=1$ on $K^n$ then $F$ has a polynomial inverse on the whole of $K^n$.

\begin{lemma}
    Let $F: X \rightarrow X$ be a polynomial map and suppose that $\Delta F \neq$ 0 everywhere in $X$. Then $F^{\sharp}$ is injective.
    \begin{proof}
        Suppose that $F^{\sharp}$ is not injective, and choose the non-constant polynomial $g \in K[X]$ of smallest degree such that $F^{\sharp}(g)=0$. Then $g\circ F=0$. Let $g_i=\partial g / \partial x_i$ and
        \begin{equation*}
            \mathbf{v}=\left(g_1\left(F_1, \ldots, F_n\right), \ldots, g_n\left(F_1, \ldots, F_n\right)\right) .
        \end{equation*}
        Hence, by the chain rule,
        \begin{equation*}
            D \left(g\circ F\right)= \mathbf{v}(p) \cdot J F(p)=0
        \end{equation*}
        for every $p \in X$. Since
        \begin{equation*}
            \Delta F(p)=\operatorname{det} J F(p) \neq 0,
        \end{equation*}
        we conclude that $\mathbf{v}(p)=0$ for every $p \in X$. Thus $g_i\left(F_1, \ldots, F_n\right)=0$ for $1 \leq i \leq n$. Since $g$ is not constant, at least one of the $g_i$ must be non-zero. But $g_1$ has degree smaller than $g$, a contradiction.
    \end{proof}
\end{lemma}
\begin{proposition}
    Denote by $K\left[F_1, \ldots, F_n\right]$ the subalgebra of $K[X]$ generated by the coordinate functions of $F$. This is the image of the homomorphism $F^{\sharp}$. Thus the Jacobian conjecture may be rephrased as follows.

    Let $F: K^n \rightarrow K^n$ be a polynomial map and assume that $\Delta F=1$ in $K^n$. Then $K\left[F_1, \ldots, F_n\right]=K\left[x_1, \ldots, x_n\right]$.
\end{proposition}


\section{Derivation} % Derivation

\begin{definition}
    Let $D$ be a derivation of a $K$-algebra $S$.
    \begin{enumerate}
        \item
              It follows from Leibniz's rule that the kernel of $D$ is a subring (subalgebra) of $S$, it is called the \textbf{ring of constants of $D$}.

        \item
              The derivation $D$ is \textbf{locally nilpotent} if given $a \in S$, there exists $k \in \mathbb{N}$ such that $D^k(a)=0$.

        \item
              Let $S$ be a ring and $D$ a locally nilpotent derivation. Define a map $\phi$ : $S \longrightarrow S[x]$ by the rule
              \begin{equation*}
                  \phi(a)=\sum_0^{\infty} \frac{D^n(a)}{n!} x^n
              \end{equation*}
              for every $a \in S$.
              It is easy to check that $\phi$ is a ring homomorphism which satisfies
              \begin{equation*}
                  \phi \cdot D=\partial \cdot \phi
              \end{equation*}
    \end{enumerate}
\end{definition}

\begin{proposition}
    Let $S$ be a $K$-algebra and $D_1, \ldots, D_n$ be commuting locally nilpotent derivations of $S$. Suppose that there exist $t_1, \ldots, t_n \in S$ such that $D_i\left(t_j\right)=\delta_{i j}$.
    Then
    \begin{enumerate}
        \item
              $S=R\left[t_1, \ldots, t_n\right]$, where $R$ is the ring of constants with respect to $D_1, \ldots, D_n$,
        \item
              $t_1, \ldots, t_n$ are algebraically independent over $R$,
        \item
              $D_i=\partial / \partial t_i$ for $i=1, \ldots, n$.
    \end{enumerate}
    \begin{proof}
        Proof: We firstly prove it whene $n=1$.
        Put $\bar{S}=S / S t$. Let $\rho: S \longrightarrow \bar{S}[x]$ be the composition of $\phi$ defined above and the projection $S[x] \longrightarrow \bar{S}[x]$. We want to show that $\rho$ is an isomorphism. Note that $\rho(t)=x$.

        To prove that $\rho$ is surjective it is enough to prove that its image contains $\bar{S}$. Let $a \in S$. Denote by $\bar{a}$ its image in $\bar{S}$. Since $D$ is locally nilpotent, there exists $n \in \mathbb{N}$ such that $D^k(a)=0$ for $k>n$. Thus,
        \begin{equation*}
            \rho(a)=\sum_0^n \frac{\overline{D^i(a)}}{i!} x^i .
        \end{equation*}
        If $n=0$, then $\rho(a)=\bar{a}$. If $n>0$ put $a_0=a$ and define $a_{j+1}=a_j- D^{n-j}\left(a_j\right) t^{n-j} /(n-j)!$, for $j=1, \ldots, n$. It is easy to show, by induction on $j$, that $D^k\left(a_j\right)=0$ for $k>n-j$ and that
        \begin{equation*}
            \rho\left(a_j\right)=\sum_0^{n-j} \frac{\overline{D^i\left(a_j\right)}}{i!} x^i .
        \end{equation*}
        Thus $\rho\left(a_n\right)=\bar{a}_n$. However, since $\bar{t}=0$, we have that $\rho\left(a_n\right)=\bar{a}$. Thus $\rho$ is surjective.

        Let us prove that $\rho$ is injective. If not, then there exists a non-zero $a \in S$ such that $\rho(a)=0$. Thus $D^k(a) \in t S$, for every $k \in \mathbb{N}$. Hence $a=a_1 \cdot t$, for some $a_1 \in S$. Since $\rho(t)=x$, we have that $\rho\left(a_1\right)=0$. Thus $a_1 \in t S$ and $a=a_2 \cdot t^2$, for some $a_2 \in S$. Continuing this way we conclude that $t^n$ divides $a$ for all $n \geq 0$. But this is impossible, unless $a=0$. Indeed, $\phi$ maps $t$ to $t+x$. Thus if $t^n$ divides $a$, we also have that $\phi\left(t^n\right)=(t+x)^n$ divides $\phi(a)$ in the polynomial ring $S[x]$. Hence, if $a \neq 0$ we have that $\operatorname{deg}(\phi(a)) \geq n$ for every $n>0$, which is clearly impossible. Thus $a=0$, as required.

        We conclude that the homomorphism $\rho: S \longrightarrow \bar{S}[x]$ is an isomorphism. Since $\rho \cdot D=d / d x \cdot \rho$, we have that $R=\rho^{-1}(\bar{S})$. The result now follows if we recall that $\rho(t)=x$.

        We proceed by induction on the number $n$ of derivations. By Lemma $3.2, S=R_1\left[t_1\right]$, where $R_1$ is the ring of constants of $D_1$. But $t_1$ is algebraically independent over $R_1$ and $D_1=d / d t_1$. Since $D_1$ commutes with $D_i$ for $i>1$, we have that $D_i\left(R_1\right) \subseteq R_1$. Thus, by the induction hypothesis, $R_1=R\left[t_2, \ldots t_n\right]$, and the proposition follows.
    \end{proof}
\end{proposition}



\section{Automorphisms}
\begin{definition}
    Let $X=K^n$. The rational function field of $K[X]$ will be denoted by $K(X)$.
    Let $F: X \rightarrow X$ be a polynomial map with coordinate functions $F_1, \ldots, F_n$. Assume that
    \begin{equation*}
        \Delta=\Delta F \neq 0
    \end{equation*}
    everywhere on $X$.

    \begin{enumerate}
        \item
              Define a map $D_i: K(X) \rightarrow K(X)$ by
              \begin{equation*}
                  D_i(g)=\Delta^{-1} \operatorname{det} J\left(F_1, \ldots, F_{i-1}, g, F_{i+1}, \ldots, F_n\right) .
              \end{equation*}
              It is easy to check that $D_i$ is a derivation of $K(X)$.

        \item
              Now let $K\left[X, \Delta^{-1}\right]$ be the $K$-subalgebra of $K(X)$ of all rational functions whose denominator is a power of $\Delta$. Then $D_i$ restricts to a derivation of $K\left[X, \Delta^{-1}\right]$, since $D_i\left(\Delta^{-1}\right)=-\Delta^2 D_i(\Delta)
                  \in
                  K\left[X, \Delta^{-1}\right] $
    \end{enumerate}
\end{definition}

\begin{proposition}
    Let $R$ be a commutative ring and $D, D^{\prime}$ be derivations of $R$, then $\left[D, D^{\prime}\right]$ is a derivation of $R$.


\end{proposition}
\begin{proposition}
    Let $D$ be a $K$-derivation of $K\left[x_1, \ldots, x_n\right]$.
    \begin{enumerate}
        \item
              $D$ can be extended to the power series ring $K\left[\left[x_1, \ldots, x_n\right]\right]$.
        \item
              If $\Delta$ is a power series such that $\Delta(0) \neq 0$ then $\Delta^{-1} \cdot D$ is a derivation of the power series ring $K\left[\left[x_1, \ldots, x_n\right]\right]$.
    \end{enumerate}
\end{proposition}

\begin{lemma}
    As derivations of $K\left[X, \Delta^{-1}\right]$ the $D_i$ satisfy:
    \begin{enumerate}
        \item
              $D_i\left(F_j\right)=\delta_{i j}$.
        \item
              The $D_i$ commute pairwise.
    \end{enumerate}
    \begin{proof}
        Note first that $\Delta(0) \neq 0$. Thus $\Delta$ is invertible as a power series and $K\left[X, \Delta^{-1}\right] \subseteq K[[X]]$. On the other hand, $\Delta \cdot D_i$ is a derivation of $K\left[x_1, \ldots, x_n\right]$ which can be extended to a derivation on the power series ring $K[[X]]=K\left[\left[x_1, \ldots, x_n\right]\right]$.
        Since $\Delta$ is invertible as a power series, then $D_i$ can also be extended to a derivation of $K[[X]]$.

        Put derivation $B=\left[D_i, D_j\right]$.
        We want to show that $B=0$ on $K\left[X, \Delta^{-1}\right]$. It is enough to show that $B=0$ on the power series ring $K[[X]]$.

        Moreover $B\left(F_k\right)=0$, for $1 \leq k \leq n$; and so $B$ is zero in the subalgebra $K\left[F_1, \ldots, F_n\right]$. But $F_1, \ldots, F_n$ are algebraically independent, by Lemma 2.2. Hence we may consider $B$ as a derivation on the power series ring $K\left[\left[F_1, \ldots, F_n\right]\right]$. By (1), $B$ is zero on $K\left[\left[F_1, \ldots, F_n\right]\right]$.

        For $1 \leq i \leq n$ let $a_i=F_i(0)$. The jacobian matrices of $\left(F_1-a_1, \ldots, F_n-a_n\right)$ and $F$ coincide. Since the latter is invertible in $K\left[\left[x_1, \ldots, x_n\right]\right]$, we conclude from the local inversion theorem (see Appendix 2) that
        \begin{equation*}
            K\left[\left[x_1, \ldots, x_n\right]\right]=K\left[\left[F_1-a_1, \ldots, F_n-a_n\right]\right]=K\left[\left[F_1, \ldots, F_n\right]\right] .
        \end{equation*}
        Thus $B$ is zero on $K\left[\left[x_1, \ldots, x_n\right]\right]$, as required.
    \end{proof}
\end{lemma}

\begin{definition}
    Let $a \in A_n$. The map $\operatorname{ad}_a: A_n \rightarrow A_n$ is defined by
    \begin{equation*}
        \operatorname{ad}_a(b)=[a, b] .
    \end{equation*}
    This is a $K$-linear map, but it is not a $K$-algebra homomorphism.
\end{definition}
\begin{theorem}
    Let $F: K^n \rightarrow K^n$ be a polynomial map and assume that $\Delta F=1$ everywhere on $K^n$. If every endomorphism of $A_n$ is an automorphism, then the Jacobian conjecture holds.
    \begin{proof}
        Since $\Delta F=1$, it follows from Lemma 4.1 that $D_1, \ldots, D_n$ are derivations of $K[X]$ which satisfy
        \begin{equation*}
            \left[D_i, F_j\right]=D_i\left(F_j\right)=\delta_{i j} \text { and }\left[D_i, D_j\right]=0
        \end{equation*}
        for $1 \leq i, j \leq n$. By, there exists an endomorphism $\phi: A_n \rightarrow A_n$ such that $\phi\left(x_i\right)=F_i$ and $\phi\left(\partial_i\right)=D_i$, for $1 \leq i \leq n$. Note that for $b \in A_n$,
        \begin{equation*}
            \operatorname{deg}\left(\operatorname{ad}_{\partial_i}(b)\right)=\operatorname{deg}\left[\partial_i, b\right] \leq \operatorname{deg} b-1
        \end{equation*}
        Thus given $b \in A_n$, there exists $k \in \mathbb{N}$ such that $\left(\operatorname{ad}_{\partial_1}\right)^k(b)=0$. Since
        \begin{equation*}
            \phi\left(\operatorname{ad}_{\partial_i}(b)\right)=\operatorname{ad}_{D_i} \phi(b)
        \end{equation*}
        we have that $\left(\operatorname{ad}_{D_i}\right)^k(\phi(b))=0$. Assuming that $\phi$ is an automorphism, we conclude that $D_i$ is locally nilpotent. It then follows by Proposition 3.1 that $K\left[F_1, \ldots, F_n\right]=K\left[x_1, \ldots, x_n\right]$, which is the Jacobian conjecture as stated in 2.3.
    \end{proof}
\end{theorem}






\chapter{Modules Over The Weyl Algebra} % Modules Over The Weyl Algebra
\section{The Polynomial Ring} % The Polynomial Ring     
\begin{proposition}
    Let $R$ be a ring and $M$ an irreducible left $R$-module.
    \begin{enumerate}
        \item $M=Rm$ for every $0 \neq m \in M$.
        \item If $0 \neq u \in M$, then $M \cong R / a n n_R(u)$.
        \item If $R$ is not a division ring, then $M$ is a torsion module.
    \end{enumerate}
\end{proposition}
\begin{proposition}
    The $A_n$-module $K[X]$ is an irreducible, torsion $A_n$-module. Besides this, we have isomorphism of $A_n$-module
    \begin{equation*}
        K[X] \cong A_n / \sum_1^n A_n \partial_i .
    \end{equation*}
    \begin{proof}
        First of all $1$ is clearly a generator of $A_n$-module $K[X]$ and the annihilator of $1$ is the left ideal generated by $\partial_1,\ldots,\partial_n$.
        \begin{equation*}
            \begin{tikzcd}
                A_n\arrow[r,""]\arrow[d,""]&K[X] \\
                A_n / \sum_1^n A_n \partial_i \arrow[ru,""]&
            \end{tikzcd}
        \end{equation*}
    \end{proof}
    \begin{remark}
        Choose $g_1, \ldots, g_n \in K[X]$ and consider the left ideal $J$ of $A_n$ generated by $\partial_1-g_1, \ldots, \partial_n-g_n$.
        It is easy to check that
        the map \begin{equation*}
            \psi: A_n / J \longrightarrow K[X] \text{ defined by } \psi(f+J)=f
        \end{equation*}
        is an isomorphism of $A_n$-modules.
    \end{remark}
\end{proposition}

\begin{proposition}
    Another $A_n$-module is $A_n / \sum_1^n A_n \cdot x_i$.
    \begin{equation*}
        \overline{x^\alpha \partial^\beta}
        =
        \left(-1\right)^{\left|\alpha\right|}\overline{\partial^{\beta-\alpha}}
    \end{equation*}
    As a $K$-vector space it is isomorphic to $K[\partial]=K\left[\partial_1, \ldots, \partial_n\right]$, the set of polynomials in $\partial_1, \ldots, \partial_n$. Using this isomorphism, we may identify the action of $A_n$
    directly on $K[\partial]$ : the $\partial$ 's act by multiplication, whilst $x_i$ acting on $\partial_j$ gives $-\delta_{i j} \cdot 1$.

\end{proposition}


\section{Twisting} % Twisting

\begin{definition}
    Let $R$ be a ring and $M$ a left $R$ module. Suppose that $\sigma$ is an automorphism of $R$. We shall define a new left module $M_\sigma$, as follows. As an abelian group, $M_\sigma=M$. Let $a \in R$ and $u \in M$, define
    \begin{equation*}
        r \cdot m=\sigma(r) m
    \end{equation*}
    It is called the \textbf{twisted module of $M$ by $\sigma$}.
\end{definition}

\begin{proposition}
    \label{pro: Twisting module}
    Let $R$ be a ring, $M$ a left $R$-module and $\sigma$ an automorphism of $R$. Then:
    \begin{enumerate}
        \item
              $M_\sigma$ is irreducible if and only if $M$ is irreducible.
        \item
              $M_\sigma$ is a torsion module if and only if $M$ is a torsion module.
        \item  $M_\sigma\oplus M^\prime_\sigma \cong (M \oplus M^\prime)_\sigma$.
        \item
              If $N$ is a submodule of $M$ then $(M / N)_\sigma \cong M_\sigma / N_\sigma$.
        \item
              Let $J$ be a left ideal of $R$. Set $\sigma(J)=\{\sigma(r): r \in J\}$. Then $\sigma(J)$ is a left ideal of $J$ and $(R / J)_\sigma \cong R / \sigma^{-1}(J)$.
    \end{enumerate}
\end{proposition}


\begin{proposition}
    The Fourier transform of $K[X]$ is $K[\partial]$.
    \begin{proof}
        Let $\mathcal{F}$ be the Fourier transform automorphism of $A_n$. By Proposition 5.2, the twisted module $\left(K[X]\right)_{\mathcal{F}}$ is isomorphic to
        \begin{equation*}
            A_n / \sum_1^n A_n \cdot \mathcal{F}^{-1}\left(\partial_i\right)
            =
            A_n / \sum_1^n A_n \cdot x_i\cong K[\partial]
        \end{equation*}
    \end{proof}
\end{proposition}

\begin{theorem}
    For every positive integer $r$ let $\sigma_r$ be the automorphism of $A_n$ which satisfies $\sigma_r\left(x_i\right)=x_i$ and $\sigma_r\left(\partial_i\right)=\partial_i-x_i^r$.
    The modules $K[X]_{\sigma_r}$ form an infinite family of pairwise nonisomorphic irreducible modules over $A_n$.
    \begin{proof}
        Let $r<t$, and suppose that there exists an isomorphism, $\phi$ : $K[X]_{\sigma_r} \longrightarrow K[X]_{\sigma_t}$. Since $K[X]_{\sigma_r}$ is irreducible, it is generated by 1 . Thus $\phi$ is completely determined by the image of 1 ; say $\phi(1)=f \neq 0$. Now the equation $\phi\left(\partial_i \bullet 1\right)=\partial_i \bullet \phi(1)$ translates as the differential equation
        \begin{center}
            $\phi(\partial_i \bullet 1)=\phi\left(\left(\partial_i-x_i^r\right)\left(1\right)\right)=\phi(-x_i^r)=-\phi(x_i^r)$
            and
            $\partial_i \bullet \phi(1)=\left(\partial_i-x^r\right)(f)$
        \end{center}
        thus
        \begin{equation*}
            \frac{\partial f}{\partial x_i}=\left(x_i^t-x_i^r\right) f .
        \end{equation*}
        The left hand side of the equation has degree $\leq \operatorname{deg} f-1$. Since $f \neq 0$ and $r<t$, the right hand side has degree $\operatorname{deg} f+t$. This is a contradiction, so the theorem is proved.
    \end{proof}
\end{theorem}



\section{Holomorphic Functions} % Holomorphic Functions

\begin{lemma}
    \label{lem: }
    Let $h(z)$ be the holomorphic function $\exp (\exp (z))$. For every positive integer $m$ there exists a polynomial $F_m(x) \in \mathbb{C}[x]$ of degree $m$ such that
    \begin{equation*}
        d^m h / d z^m=F_m\left(e^z\right) h(z) .
    \end{equation*}
    \begin{proof}
        The proof is by induction on $m$. If $m=1$ then $d h / d z=e^z h(z)$, so we may take $F_1(x)=x$. Suppose that the result is true for $m=k$. Then
        \begin{equation*}
            d^{k+1} h / d z^{k+1}=d / d z\left(F_k\left(e^z\right) h(z)\right)=\left(e^z F_k^{\prime}\left(e^z\right)+e^z F_k\left(e^z\right)\right) h(z) .
        \end{equation*}
        Thus we may take
        \begin{equation*}
            F_{k+1}(x)=x F_k^{\prime}(x)+x F_k(x)
        \end{equation*}
        which is a polynomial of degree $k+1$.
    \end{proof}
\end{lemma}

\begin{proposition}
    The function $h(z)=\exp (\exp (z))$ is not a torsion element of the $A_1(\mathbb{C})$-module $\mathcal{H}(U)$.
    \begin{proof}
        Suppose that there exists a non-zero operator $P \in A_1(\mathbb{C})$ such that $P \cdot h=0$. Write
        \begin{equation*}
            P=\sum_{i=0}^m f_i(z) \partial^i
        \end{equation*}
        with $f_m(z) \neq 0$. By Lemma 6.1, we have
        \begin{equation*}
            0=P \cdot h=\sum_{i=0}^m f_i(z) F_i\left(e^z\right) h(z) .
        \end{equation*}
        Since $h(z) \neq 0$, we conclude that
        \begin{equation*}
            \sum_{i=0}^m f_i(z) F_i\left(e^z\right)=0 .
        \end{equation*}
        But this is impossible since $F_m$ has degree $m$ and $f_m(z) \neq 0$.
    \end{proof}
\end{proposition}


\chapter{Differential Equations}
\section{The D-Modules of Equations} % The D-Modules of Equations
\begin{definition}
    Let $P$ be an operator in $A_n$.
    This differential operator gives rise to the equation
    \begin{equation*}
        P(f)=\sum_\alpha g_\alpha \partial_\alpha(f)=0 \text{ in } K[X],
    \end{equation*}
    where $f\in K[X]$.
    More generally, if $P_1, \ldots, P_m$ are differential operators in $A_n$, then we have a system of differential equations
    \begin{equation}
        P_i(f)=0 , \quad i=1, \ldots, m .
    \end{equation}
    The \textbf{$A_n$-module associated to the system (1)} is $A_n / \sum_1^m A_n P_i$.
    A polynomial solution of (1) is a polynomial $f \in K[X]$ which satisfies $P_i(f)=0$, for $i=1, \ldots, m$. The set of all polynomial solutions of (1) forms a $K$-vector space.
\end{definition}



\begin{theorem}
    Let $M$ be the $A_n$-module associated with the system (1).
    The $K$-vector space of polynomial solutions of the system (1) is isomorphic to $\operatorname{Hom}_{A_n}(M, K[X])$.
    \begin{proof}
        Step1.
        Let $f$ be a polynomial solution of (1). Define a map $\phi_f: A_n \rightarrow K[X]$ by the rule
        \begin{equation*}
            \phi_f(D)=D(f)
        \end{equation*}
        It is easy to check that $\phi_f$ is an $A_n$-module homomorphism. Moreover, if $D \in \sum_1^m A_n P_i$, then
        \begin{equation*}
            \phi_f(D)=D(f)=\sum_1^m D_i P_i(f)=0 .
        \end{equation*}
        that is $\sum_{1}^{m}A_n P_i \subseteq \ker \phi_f $. Thus $\phi_f$ induces an $A_n$-module homomorphism $\bar{\phi}_f: M \rightarrow K[X]$.
        \begin{equation*}
            \begin{tikzcd}
                A_n \arrow[r,"\phi_f"] \arrow[d,"\pi"] & K[X] \\
                M \arrow[ru,"\bar{\phi}_f"] &
            \end{tikzcd}
        \end{equation*}

        Step 2.
        Conversely, let $\psi: M \rightarrow K[X]$ be an $A_n$-module homomorphism. Define $f=\psi(\overline{1})$, where $\overline{1}$ is the image of $1$ in $M$. Then for $i=1, \ldots, m$,
        \begin{equation*}
            P_i(f)=P_i(\psi(\overline{1}))=\psi(P_i(\overline{1}))=0 .
        \end{equation*}
        Thus $f$ is a polynomial solution of (1).

        It is easy to check that the maps $f \mapsto \bar{\phi}_f$ and $\psi \mapsto \psi(\overline{1})$ are inverse to each other. Hence we have established the required isomorphism.
    \end{proof}
\end{theorem}

\begin{definition}
    Let $\mathcal{S}$ be a left $A_n$-module; and let $M=\left(P_i\right)$ be a finitely generated left $A_n$-module. We will call $\operatorname{Hom}_{A_n}(M, \mathcal{S})$ the solution space of $M$ in $\mathcal{S}$ which isomorphic to $K$-vector space of solutions  of
    \begin{equation*}
        P_i \left(f\right)=0 \text{ in } \mathcal{S} , \quad i=1, \ldots, m .
    \end{equation*}
\end{definition}


\section{Microfunctions}
Let $D(\epsilon)$ be the open disk of $\mathbb{C}$ of centre 0 and radius $\epsilon$, $D^{\prime}(\epsilon)=D(\epsilon) \backslash 0$ and $\mathcal{H}\left(\Omega\right)$ be the set of holomorphic functions in the open set $\Omega \subseteq \mathbb{C}$ viewed as an $A_1(\mathbb{C})$-module.

\begin{proposition}
    Let directed set $I=\left\{D(\epsilon): \epsilon \in \mathbb{R}\right\}$ such that $D(\epsilon) \leq D(\epsilon^\prime)$ iif $D(\epsilon) \supseteq D(\epsilon^\prime)$, and a directed family $\left\{\mathcal{H}(D\left(\epsilon\right)): D(\epsilon) \in I \right\}$.
    The homomorphisms $\tau_{\epsilon \epsilon^{\prime}}: \mathcal{H}(D(\epsilon)) \longrightarrow \mathcal{H}\left(D\left(\epsilon^{\prime}\right)\right)$ are defined by restriction of holomorphic functions.

    The elements of $\mathcal{H}_0=\lim\limits_{\longrightarrow} \mathcal{H}(D(\epsilon))$ are called \textbf{germs of holomorphic functions} at $0$.
\end{proposition}

\begin{proposition}
    The \textbf{universal cover} of $D^{\prime}(\epsilon)$ is the set $\tilde{D}(\epsilon)=\{z \in \mathbb{C}: \operatorname{Re}(z)<\log (\epsilon)\}$. The projection $\pi$ of $\tilde{D}(\epsilon)$ on $D^{\prime}(\epsilon)$ is defined by $\pi(z)=e^z$. We have the commutative diagram
    \begin{equation*}
        \begin{tikzcd}
            & \tilde{D}(\epsilon) \arrow[d,"\pi"] & \\
            & D^{\prime}(\epsilon) \arrow[r,hook] & D(\epsilon)
        \end{tikzcd}
    \end{equation*}

\end{proposition}

\begin{proposition}
    Let $h \in \mathcal{H}(\tilde{D}(\epsilon))$. The action of a polynomial $f \in \mathbb{C}[x]$ on $h$ is given by $f \bullet h=f\left(e^z\right) h(z)$. The operator $\partial=d / d x$ acts on $h$ by the formula $\partial \bullet h=h^{\prime}(z) e^{-z}$. Then the map
    \begin{equation*}
        \pi^*: \mathcal{H}\left(D^{\prime}(\epsilon)\right) \longrightarrow \mathcal{H}(\tilde{D}(\epsilon))
    \end{equation*}
    defined by $\pi^*(h)=h\circ \pi$ is an injective homomorphism of $A_1(\mathbb{C})$ modules.
\end{proposition}


\begin{definition}
    Let $\mathcal{M}_\epsilon$ denote the quotient module $\mathcal{H}(\tilde{D}(\epsilon)) / \pi^*(\mathcal{H}(D(\epsilon)))$.
    If $D(\epsilon) \leq D(\epsilon^{\prime})$, then $\tilde{D}\left(\epsilon^{\prime}\right) \subseteq \tilde{D}(\epsilon)$ and $\mathcal{H}(\tilde{D}(\epsilon)) \subseteq \mathcal{H}\left(\tilde{D}\left(\epsilon^{\prime}\right)\right)$.
    This induces a homomorphism of $A_1(\mathbb{C})$-modules
    \begin{equation*}
        \tau_{\epsilon \epsilon^\prime}: \mathcal{M}_\epsilon \longrightarrow \mathcal{M}_{\epsilon^{\prime}} .
    \end{equation*}
    Hence $\left\{\mathcal{M}_\epsilon: \epsilon \in \mathbb{R}\right\}$ is a directed family of $A_1(\mathbb{C})$-modules, and we may take its direct limit called the \textbf{module of microfunctions}, denoted by $\mathcal{M}$.
\end{definition}


\chapter{}






\section{Filtration and Associated Graded Rings and Modules} % Filtration and Associated Graded Rings and Modules
\subsection{Increasing filtration} % Filtered and Graded Modules

\begin{definition}
    Let $A$ be a ring.
    A \textbf{increasing filtration} of $A$ is a sequence of subgroups $\left\{F_iA\right\}$ of $A$ such that
    \begin{enumerate}[label=(\roman*)]
        \item
              $F_0A \subset F_1A \subset F_2A \subset \cdots \subset F_nA \subset \cdots \subset A$
        \item
              $\bigcup_{i\geq 0} F_iA=A$.
        \item
              $F_iA \cdot F_jA \subseteq F_{i+j}A$ for all $i,j$.
    \end{enumerate}
    $A$ is called a \textbf{filtered ring} if it has an increasing filtration.
\end{definition}

\begin{proposition}
    Let $A$ be a ring.
    \begin{enumerate}
        \item
              If $A=\bigoplus_{i\geq 0} A_i $ is a graded ring, then the sequence of subgroups $\left\{F_kA:=\bigoplus_{i=0}^k A_i\right\}$ is an increasing filtration of $A$.
        \item
              If $\mathcal{F}=\left\{F_iA\right\}$ is a filtration of $A$, then
              \begin{equation*}
                  \operatorname{gr}_{\mathcal{F}}A
                  :=
                  \bigoplus_{i\geq 0} F_{i+1}A / F_iA
              \end{equation*}
              is a graded ring (multiplication follows from $A$), called the \textbf{associated graded ring of $A$ associated with the filtration $\mathcal{F}$} .
    \end{enumerate}
\end{proposition}


\begin{definition}
    Let $A$ be a filtered ring with increasing filtration $\mathcal{F}=\left\{F_iA\right\}$. A left $A$-module $M$ is called a \textbf{filtered left $A$-module} if it has a sequence of subgroups $\Gamma=\left\{\Gamma_iM\right\}$ such that
    \begin{enumerate}[label=(\roman*)]
        \item
              $\Gamma_0M \subset \Gamma_1M \subset \Gamma_2M \subset \cdots \subset \Gamma_nM \subset \cdots \subset M$
        \item
              $\bigcup_{i\geq 0} \Gamma_iM = M$
        \item
              $F_iA  \cdot \Gamma_jM \subseteq \Gamma_{i+j}M$ for all $i,j$.
    \end{enumerate}
\end{definition}



\begin{proposition}
    Let $M$ be a left $A$-module where $A$ is a filtered ring with increasing filtration $\mathcal{F}=\left\{F_iA\right\}$.
    \begin{enumerate}
        \item
              If $M=\bigoplus_{i\geq 0} M_i$ is a graded $A$-module, then the sequence of subgroups $\Gamma_kM:=\bigoplus_{i=0}^k M_i$ is an increasing filtration of $M$.
        \item
              If $\Gamma=\left\{\Gamma_iM\right\}$ is a filtration of $M$, then
              The \textbf{graded module of $M$ associated with the filtration $\Gamma$} is defined by
              \begin{equation*}
                  \operatorname{gr}_{\Gamma}M
                  =
                  \bigoplus_{i\geq 0} \Gamma_{i+1}M / \Gamma_i M
              \end{equation*}
              which is a graded $\operatorname{gr}_{\mathcal{F}}A$-module.
    \end{enumerate}
    \begin{remark}
        In (1), the  $\operatorname{gr}_{\mathcal{F}}M\cong M$ (as graded modules).

        But in (2), let $A=\mathbb{Z}$ with trivial filtration and $M=\mathbb{Z}_{p^2}$ with filtration $F_0M=0, F_1M=p \mathbb{Z}_{p^2}, F_2M=M,\ldots$, then $\operatorname{gr}_{\mathcal{F}}M \cong \mathbb{Z}_p^2\not \cong M$ in ${}_\mathbb{Z}\Mod$.

        The functor $\operatorname{gr}(-):  {}_R \mathbf{FiltMod}\rightarrow {}_R\mathbf{GrMod}$ is not faithful
    \end{remark}
\end{proposition}

\section{Good filtration}
\begin{definition}
    Let $A$ be a filtered ring with an increasing filtration $\left\{F_nA\right\}$, and $M$ a filtered $A$-module with an increasing filtration $\left\{F_nM\right\}$.
    The filtration of $M$ is called a \textbf{good filtration} if $\operatorname{gr}(M)$ is finitely generated over $\operatorname{gr}(A)$.
\end{definition}




\section{Dimension}
\begin{definition}
    Let $M$ be a finitely generated left $A_n$-module. Suppose that $\Gamma$ is a good filtration of $M$ with respect to the Bernstein filtration $\mathcal{B}$.
    \begin{enumerate}
        \item
              Denote by $\chi(t, \Gamma, M)$ the \textbf{Hilbert polynomial} of the graded module $g r^{\Gamma} M$ over $S_n$.
              We have
              \begin{equation*}
                  \chi(t, \Gamma, M)=\sum_0^t \operatorname{dim}_k\left(\Gamma_i / \Gamma_{i-1}\right)=\operatorname{dim}_k\left(\Gamma_t\right) .
              \end{equation*}
              The dimension $d(M)$ of $M$ is defined the $\deg\chi(t, \Gamma, M)$.
        \item
              Let $a_{d(M)}$ be the leading coefficient of $\chi(t, \Gamma, M)$. The \textbf{multiplicity} of $M$ is $m(M)=d!a_{d(M)}$.
    \end{enumerate}
    \begin{remark}
        The dimension and multiplicity of $M$ do not depend on the choice of the good filtration $\Gamma$.
    \end{remark}
\end{definition}


\begin{example}
    First let $M$ be the left $A_n$-module $A_n$.
    The Bernstein filtration $\mathcal{B}$ is a good filtration of $M$ and it is possible to calculate $\chi(t, \mathcal{B}, M)$ explicitly in this case.
    \begin{equation*}
        \chi(t, \mathcal{B}, M)=
        \#\left\{x^\alpha \partial^\beta : \alpha,\beta \in \mathbb{N}^n, \left|\alpha\right|+\left|\beta\right| \leq t\right\}
        =
        \binom{t+2 n}{2 n}
    \end{equation*}
    that is
    \begin{equation*}
        \frac{(t+2n)(t+2n-1)\cdots (t+1)}{(2n)!}
    \end{equation*}
    Thus $d\left(A_n\right)=2 n$ and $m\left(A_n\right)=1$.
\end{example}

\begin{example}
    Another $A_n$-module that we know very well is $K[X]=K\left[x_1, \ldots, x_n\right]$ with good filtration $\Gamma_i:=\left\{f\left(x_1,\ldots,x_n\right):\deg f \leq i\right\}$.
    It is easy to show that
    \begin{equation*}
        \chi(t, \mathcal{B}, k[X])
        =
        \operatorname{dim}_k \Gamma_t=\binom{n+t}{n}
    \end{equation*}
    Hence $d(K[X])=n$ and $m(K[X])=1$.
\end{example}



\subsection{Upper boundness}
\begin{theorem}
    Let $M$ be a finitely generated left $A_n$-module and $N$ a submodule of $M$.
    \begin{enumerate}
        \item
              $d(M)=\max \{d(N), d(M / N)\}$.
        \item  If $d(N)=d(M / N)$ then $m(M)=m(N)+m(M / N)$.
    \end{enumerate}
    \begin{proof}
        Let $M$ be a finitely generated left $A_n$-module and $\Gamma$ a good filtration of $M$ with respect to $\mathcal{B}$. Let $N$ be a submodule of $M$. Denote by $\Gamma^{\prime}$ and $\Gamma^{\prime \prime}$ the filtrations induced by $\Gamma$ in $N$ and $M / N$, we have an exact sequence of $S_n$-modules, namely
        \begin{equation*}
            0 \rightarrow g r_{\Gamma^{\prime}} N \rightarrow g r_{\Gamma} M \rightarrow g r_{\Gamma^{\prime \prime}} M / N \rightarrow 0 .
        \end{equation*}
        Since $\Gamma$ is good, $g r_{\Gamma} M$ is finitely generated. But $S_n$ is a noetherian ring. Hence $g r_{\Gamma^{\prime}} N$ and $g r_{\Gamma^{\prime \prime}}(M / N)$ are also finitely generated. Therefore both $\Gamma^{\prime}$ and $\Gamma^{\prime \prime}$ are good filtrations.
        On the other hand, since the sequence of vector spaces
        \begin{equation*}
            0 \rightarrow \Gamma_m^{\prime} / \Gamma_{m-1}^{\prime} \rightarrow \Gamma_m / \Gamma_{m-1} \rightarrow \Gamma_m^{\prime \prime} / \Gamma_{m-1}^{\prime \prime} \rightarrow 0
        \end{equation*}
        is exact, we have that
        \begin{equation*}
            \operatorname{dim}_k \Gamma_m^{\prime} / \Gamma_{m-1}^{\prime}+\operatorname{dim}_k \Gamma_m^{\prime \prime} / \Gamma_{m-1}^{\prime \prime}=\operatorname{dim}_k \Gamma_m / \Gamma_{m-1} .
        \end{equation*}
        Summing these terms for $m=0,1, \ldots, s$ and assuming that $s \gg 0$ one obtains
        \begin{equation*}
            \chi\left(s, \Gamma^{\prime}, N\right)+\chi\left(s, \Gamma^{\prime \prime}, M / N\right)=\chi(s, \Gamma, M)
        \end{equation*}
        The result now follows from the properties of polynomials.
    \end{proof}
\end{theorem}

\begin{corollary}
    Let $M_1, \ldots, M_m$ be finitely generated left $A_n$-modules, and $M=M_1 \oplus \cdots \oplus M_m$.
    \begin{enumerate}
        \item
              $d(M)=\max \left\{d\left(M_1\right), \ldots, d\left(M_m\right)\right\}$.

        \item
              If $d(M)=d\left(M_i\right)$ for $1 \leq i \leq k$, then $m(M)=\sum_1^k m\left(M_i\right)$.
    \end{enumerate}
\end{corollary}

\begin{corollary}
    Let $M$ be a finitely generated $A_n$-module. Then $d(M) \leq 2 n$.

    \begin{proof}
        Suppose that $M$ is generated by $r$ elements. Then there exists a surjective homomorphism $\phi: {A_n}^{\oplus r} \rightarrow M$. It follows from the theorem that $d\left({A_n}^{\oplus r}\right)=\max \{d(M), d(\operatorname{ker} \phi)\}$, thus $d(M) \leq d\left({A_n}^{\oplus r}\right) = 2 n$.
    \end{proof}
\end{corollary}

\begin{corollary}
    Let $I$ be a non-zero left ideal of $A_n$. Then $d\left(A_n / I\right) \leq 2 n-1$.

    \begin{proof}
        First consider the case of a cyclic left ideal. Let $d \in A_n$, and put $I=A_n d$. Then we have an exact sequence
        \begin{equation*}
            0 \rightarrow A_n \xrightarrow{\theta} A_n \rightarrow A_n / A_n d \rightarrow 0
        \end{equation*}
        where the map $\theta$ is defined by $\theta(a)=a d$, for every $a \in A_n$. Suppose, by contradiction, that $d\left(A_n / A_n d\right)=2 n$. Then
        \begin{equation*}
            m\left(A_n\right)=m\left(A_n\right)+m\left(A_n / A_n d\right) .
        \end{equation*}
        Since $m\left(A_n\right)=1$ and the multiplicity is a positive number, this equation is impossible. Hence $d\left(A_n / A_n d\right) \leq 2 n-1$.

        Now for the general case. Let $I$ be a non-zero left ideal of $A_n$ and choose $0 \neq d \in I$. Since $A_n d \subseteq I$, we have that $A_n / I$ is a quotient of $A_n / A_n d$. Since the latter has dimension $\leq 2 n-1$, so does $A_n / I$.
    \end{proof}
\end{corollary}


\subsection{Lower boundness}
\begin{theorem}[Bernstein's Inequality]
    If $M$ is a finitely generated non-zero left $A_n$-module, then $d(M) \geq n$.
\end{theorem}





\chapter{Holonomic Modules} % Holonomic Modules
\section{Definition}
\begin{definition}
    A finitely generated left $A_n$-module is \textbf{holonomic} if it is zero, or if it has dimension $n$.
\end{definition}

\begin{proposition}
    Let $n$ be a positive integer.
    \begin{enumerate}
        \item
              Submodules and quotients of holonomic $A_n$-modules are holonomic.
        \item
              Finite direct sums of holonomic $A_n$-modules are holonomic.
    \end{enumerate}
\end{proposition}









\begin{corollary}
    Finitely generated torsion $A_1$-modules are holonomic.
\end{corollary}





\begin{proposition}
    Holonomic $A_n$-modules are torsion modules.
\end{proposition}


\section{Basic properties}


\begin{theorem}
    Holonomic modules are artinian.
    \begin{proof}
        Let $M$ be a holonomic left $A_n$-module. Suppose that $M$ has a descending chain of submodules

        \begin{equation*}
            M=N_0 \supseteq N_1 \supseteq N_2 \supseteq \cdots \supseteq N_r
        \end{equation*}
        it follows that $m\left(N_i\right)=m\left(N_{i+1}\right)+ m\left(N_i / N_{i+1}\right)$. Putting these together, we get that
        \begin{equation*}
            m(M)=\sum_0^{r-1} m\left(N_i / N_{i+1}\right)+m\left(N_r\right) \geq r
        \end{equation*}
        Hence $M$ cannot have a descending chain of more than $r$ submodules. In particular $M$ cannot have an infinite descending chain.
    \end{proof}

    \begin{remark}
        Noted that the regular left module $A_n$ is not artinian (ring $A_n$ is not left Artinian).
        It is easy to construct an infinite descending chain; take for instance
        \begin{equation*}
            A_n x_n \supseteq A_n x_n^2 \supseteq A_n x_n^3 \supseteq \ldots .
        \end{equation*}
    \end{remark}
\end{theorem}

\begin{corollary}
    Every holonomic $A_n$-module has finite length that cannot exceed its multiplicity.
\end{corollary}

\begin{corollary}
    Every irreducible holonomic $A_n$-module has multiplicity $1$.
\end{corollary}







\end{document}
